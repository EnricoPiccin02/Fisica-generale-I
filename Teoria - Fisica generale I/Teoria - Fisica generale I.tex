\documentclass[a4paper]{extarticle}
\usepackage[utf8]{inputenc}
\usepackage[italian]{babel}
\selectlanguage{italian}
\usepackage[table]{xcolor}
\usepackage{xcolor}
\usepackage{circuitikz}
\usepackage{bm}
\usetikzlibrary{patterns,snakes}
\usetikzlibrary{decorations.markings,intersections,calc}
\usepackage{ifthen}
\usetikzlibrary{calc,patterns,angles,quotes}
\usetikzlibrary{positioning, circuits.logic.US}
\usetikzlibrary {shapes.gates.logic.US, shapes.gates.logic.IEC, calc}
\tikzset {branch/.style={fill, shape = circle, minimum size = 3pt, inner sep = 0pt}}
\usetikzlibrary{matrix,calc}
\usepackage{multirow}
\usepackage{float}
\usepackage{geometry}
\usepackage{pgfplots}
\usepackage{tabularx}
\usepackage{pgf-pie}
\usepackage{tikz}
\usepackage{tikz-3dplot}
\usepackage{amsmath}
\usepackage{amssymb}
\usepackage{color, soul}
\usepackage{fancyhdr}
\usepackage{graphicx}
\usepackage{subfig}
\usepackage{physics}
\tikzset{>=latex} % for LaTeX arrow head
\colorlet{pinkskin}{pink!25}
\colorlet{brownskin}{pink!5!brown!45}
\colorlet{myred}{red!90!black}
\colorlet{myblue}{blue!90!black}
\colorlet{mypurple}{blue!50!red!80!black!80}
\colorlet{Bcol}{violet!90}
\colorlet{BFcol}{red!60!black}
\colorlet{veccol}{green!45!black}
\colorlet{Icol}{blue!70!black}
\colorlet{mucol}{red!90!black}
\tikzstyle{BField}=[->,line width=2,Bcol]
\tikzstyle{current}=[->,Icol] %thick,
\tikzstyle{force}=[->,line width=2,BFcol]
\tikzstyle{vector}=[->,line width=2,veccol]
\tikzstyle{thick vector}=[->,line width=2,veccol]
\tikzstyle{mu vector}=[->,line width=2,mucol]
\tikzstyle{velocity}=[->,line width=2,veccol]
\tikzstyle{charge+}=[very thin,draw=black,top color=red!50,bottom color=red!90!black,shading angle=20,circle,inner sep=0.5]

\graphicspath{ {./img/} }
\newtheorem{theorem}{Teorema}[section]
\newtheorem{corollary}{Corollario}[theorem]
\newtheorem{lemma}[theorem]{Lemma}

% Specifiche
\geometry{
 a4paper,
 top=20mm,
 left=30mm,
 right=30mm,
 bottom=30mm
}

\pagestyle{fancy}
\fancyhf{}
\fancyhead[LO]{\nouppercase{\leftmark}}
\fancyfoot[CE, CO]{\thepage}
\addtolength{\headheight}{1em}
\addtolength{\footskip}{-0.5em}

\newcommand{\quotes}[1]{``#1''}
\renewcommand\tabularxcolumn[1]{>{\vspace{\fill}}m{#1}<{\vspace{\fill}}}
\renewcommand\arraystretch{}
\newcolumntype{P}{>{\centering\arraybackslash}X}

\title{\textbf{Università di Trieste\\ \vspace{1em}
Laurea in ingegneria elettronica e informatica}}
\author{Enrico Piccin - Corso di Fisica generale I - Prof. Pierre Thibault}
\date{Anno Accademico 2021/2022 - 1 Marzo 2022}

\begin{document}

\vspace{-10mm}
\maketitle

\tableofcontents
\newpage

\noindent
\begin{center}
  1 Marzo 2022
\end{center}

\section{Introduzione}
La \textbf{Fisica} è lo studio della materia e delle sue interazioni. La \textbf{Fisica classica} è divisa in tre macroaree:
\begin{enumerate}
  \item Meccanica classica
  \item Termodinamica
  \item Elettromagnetismo
\end{enumerate}
La Fisica è organizzata in
\begin{itemize}
  \item \textbf{Leggi}: relazioni fra grandezze fisiche
  \item \textbf{Principi}: affermazioni generali da reputare vere
  \item \textbf{Modelli}: analogie o rappresentazioni pratiche su cui basare il proprio studio
  \item \textbf{Teoria}: insieme di leggi, principi e modelli
\end{itemize}

\vspace{1em}
\subsection{Metodo scientifico}
Il \textbf{metodo scientifico} si basa su \textbf{osservazioni} della realtà circostante, a cui seguono delle \textbf{ipotesi}, ossia delle possibili spiegazioni dei fonmeni osservati, basati sulle osservazioni precedentemente formulate.\\
Dopo aver esposto le proprie ipotesi, esse devono essere verificate, mediante degli \textbf{esperimenti}, a cui seguono delle \textbf{analisi} dei risultati sperimentali ottenuti. Il processo di analisi viene seguito da delle \textbf{conclusioni} che \quotes{concludono} il metodo scientifico.\\
Naturalmente tale circuito non è chiuso, in quanto ciascuna di queste fasi può essere ripetuta più e più volte. La parte più importante di tale \emph{metodo scientifico} è la dimostrazione, così come la verifica tramite \textbf{sperimentazioni} delle proprie ipotesi, in quanto le ipotesi devono essere \textbf{sempre verificate}. Tale processo permette di sviluppare leggi e teorie con un fondamento concreto e solido.

\vspace{1em}
\noindent
\textbf{Osservazione}: Si osservi che \textbf{verificare un'ipotesi} non significa dimostrare che un'ipotesi è vera, ma \textbf{verificare che un'ipotesi può essere contraddetta}, ovvero ci si deve assicurare che una \textbf{teoria deve essere \quotes{falsificabile}}, ossia che può essere dimostrato che essia sia falsa.

\newpage
\section{Unità e vettori}

\vspace{1em}
\subsection{Grandezza fisica}
Alla base della \textbf{Fisica} si pone il concetto di \textbf{grandezza fisica}. Non è facile, per esempio, definire che cosa sia il \emph{tempo}; tuttavia, la soluzione più immediata è quella che prevede di definire la misura del tempo come ciò che si riesce a misurare tramite, per esempio, un orologio.\\
Si parla, in tale caso, di \textbf{definizione operativa}:

% Tabella per le definizione di concetti, etc...
\vspace{1em}
\rowcolors{1}{black!5}{black!5}
\setlength{\tabcolsep}{14pt}
\renewcommand{\arraystretch}{2}
\noindent
\begin{tabularx}{\textwidth}{@{}|P|@{}}
    \hline
    {\textbf{DEFINIZIONE OPERATIVA}}\\
    \parbox{\linewidth}{Una grandezza fisica è definita solo dalle operazioni necessarie per misurarla.
    \vspace{3mm}}\\
    \hline
\end{tabularx}

\vspace{1em}
\noindent
Inoltre, le grandezze fisiche si esprimono in termini di un \textbf{campione}, il quale prende il nome di \textbf{unità}.\\
In Fisica, inoltre, si distinguono due diverse categorie di grandezze fisiche:
\begin{enumerate}
  \item Grandezze fisiche fondamentali
  \item Grandezze fisiche derivate
\end{enumerate}
Le \textbf{grandezze fisiche fondamentali} sono $3$:
\begin{enumerate}
  \item Tempo: il tempo presenta come unità il \textbf{secondo} (s) che, dal $1967$, è stato definito come
  \[9192631170 \text{ volte il periodo di oscillazione di una risonanza dell'atomo di Cesio } ^{133}C\]
  Prima di tale data, il secondo era definito come una suddivisione del giorno, ma tale definizione era imprecisa: la terra non ruota sempre con la stessa velocità.

  \item Lunghezza: la lunghezza presenta come unità il \textbf{metro} (m), il quale viene definito come
  \[\frac{1}{299 782 458} \text{ la distanza percorsa dalla luce in $1$ s}\]
  Prima di tale definizione, il metro era definito come \(\frac{1}{10000}\) la distanza tra equatore e polo.\\
  La nuova definizione, tuttavia, è più precisa, in quanto la velocità della luce è \textbf{costante}, fissata in quanto su tale costante si definisce il metro.

  \item Massa: la massa presenta come unità il \textbf{chilogrammo} (kg), il quale viene definito in funzione della \textbf{costante di Planck} ($h = 6.62607015 \times 10^{-34} \text{ kg} \text{ m}^{2} \text{ s}^{-1}$). Prima di tale definizione, il chilogrammo era definito con riferimento ad un campione presente a Parigi e su cui si faceva riferimento per ogni misura di massa.
\end{enumerate}

\noindent
Le grandezze fisiche fondamentali permettono, poi, di definire le grandezze fisiche derivate, quale il \textbf{Volume}, la \textbf{Forza}, etc.

\vspace{1em}
\subsection{Cifre significative e incertezza}
In Fisica, quando si effettuano delle misurazioni, deve essere sempre specificata la precisione e, dunque, l'incertezza. Infatti, \textbf{tutte le msiure hanno un livello di incertezza}.\\
Per esempio
\begin{flalign*}
  L & = 1.82 \pm 0.02 \text{ m}\\
  m & = 3.5 \pm 0.1 \text{ kg}
\end{flalign*}
Da notare che l'indicazione dell'incertezza è sempre (o quasi) data da una sola cifra: altrimenti si avrebbe incertezza nell'incertezza. L'indicazione dell'incertezza è la base della \textbf{fisica sperimentale}.\\
Nella pratica, tuttavia, l'indicazione dell'incertezza è ridondante e pesante. Per indicare il livello di precisione si ricorre alle cifre significative.\\
Per esempio
\begin{flalign*}
  L & = 1.82 \text{ m} = 1.82 \pm 0.01 \text{ m}\\
  m & = 3.5 \text{ kg} = 3.5 \pm 0.1 \text{ kg}
\end{flalign*}

\vspace{1em}
\subsubsection{Operazioni di base}
Per la gestione delle cifre significative nelle operazioni di calcolo è importante tenere a mente che
\begin{itemize}
  \item Moltiplicazione e Divisione: bisogna considerare come cifre significative del prodotto o del quoziente il più basso numero di cifre significative dei fattori o di dividendo e divisore.\\
  Per esempio
  \[1,1 \text{ m} \times 3.45 \text{ m} = 3.8 \text{ m}^2\]
  in quanto il più basso numero di cifre significative dei fattori è $1$.

  \item Addizione e Sottrazione: bisogna considerare come cifre significative della somma o differenza il più basso numero di decimali degli addendi o del minuendo e sottraendo.\\
  Per esempio
  \[1.1 \text{ m} - 12 \text{ cm} = 1.1 \text{ m} - 0.12 \text{ m} = 0.98 \text{ m} = 1.0 \text{ m}\]
  in quanto il più basso numero di decimali tra minuendo e sottraendo è $1$.
\end{itemize}

\vspace{1em}
\subsection{Ordini di grandezza}
Molto spesso, nelle stime è importante non tanto la precisione delle misure, ma l'ordine di grandezza delle stesse, in modo tale da effettuare un macroconfronto utile per delle valutazioni pratiche e veloci.\\
Lo scopo, quindi, dell'impiego degli ordini di grandezza è quello di effettuare dei calcoli veloci e, quindi, delle stime. Più precisamente:

% Tabella per le definizione di concetti, etc...
\vspace{1em}
\rowcolors{1}{black!5}{black!5}
\setlength{\tabcolsep}{14pt}
\renewcommand{\arraystretch}{2}
\noindent
\begin{tabularx}{\textwidth}{@{}|P|@{}}
    \hline
    {\textbf{ORDINE DI GRANDEZZA}}\\
    \parbox{\linewidth}{L'ordine di grandezza di una misura è la \textbf{potenza di $10$ più vicina}.
    \vspace{3mm}}\\
    \hline
\end{tabularx}

\vspace{1em}
\noindent
\textbf{Esempio}: Un ingegnere deve fabbricare un nuovo pacemaker. Si stimi quanti battiti di cuore deve fare senza malfunzionamento. Per effettuare tale stima è necessario conoscere la \textit{media dei battiti al secondo} e \textit{l'aspettativa di vita del soggetto}. Considerando, quindi, come media dei battiti $m_B = 1 \text{ battito}/\text{s}$ e come aspettativa di vita $a_V = 60 \text{ anni}$. La stima selectlanguage
\[m_B \times a_V \times \pi \times 10^7 \text{ s}/\text{anno} = 1 \text{ battito}/\text{s} \times 60 \text{ anni} \times \times 10^7 \text{ s}/\text{anno} = 2 \times 10^9 \text{ battiti}\]

\newpage
\noindent
\begin{center}
  2 Marzo 2022
\end{center}
Il metodo scientifico permette di \textbf{falsificare una teoria}, quindi non è vero che permette di validare una teoria senza ambiguità.

\vspace{1em}
\subsection{Analisi dimensionale}
Il concetto di \textbf{unità} è estremamente importante per parlare di \textbf{analisi dimensionale}. In particolare
\[A = B\]
non può essere valido e corretto formalmente se $A$ e $B$ hanno unità diverse. Questo è molto intuitivo per le grandezze fisiche fondamentali, ma quando si parla di grandezze derivate diventa un punto cruciale: tale concetto permette di validare anche delle possibili soluzioni di test.\\
Per esempio, l'unità di misura della costante di richiamo di una molla si può facilmente ricavare dalla formula della \emph{forza di richiamo}:
\[F = k \cdot x\]
Da cui è immediato capire che
\[\left[k\right] = \frac{\left[F\right]}{\left[x\right]} = \frac{\text{kg m s}^{-2}}{m} = \text{kg s}^{-2}\]

\vspace{1em}
\subsection{Scalari e vettori}
Di seguito si espone la definizione di \textbf{scalare}:

% Tabella per le definizione di concetti, etc...
\vspace{1em}
\rowcolors{1}{black!5}{black!5}
\setlength{\tabcolsep}{14pt}
\renewcommand{\arraystretch}{2}
\noindent
\begin{tabularx}{\textwidth}{@{}|P|@{}}
    \hline
    {\textbf{SCALARE}}\\
    \parbox{\linewidth}{Uno \textbf{scalare} è una grandezza specificata da un numero + unità.\\
    Per esempio la \emph{lunghezza}, la \emph{massa} o l'\emph{energia}.
    \vspace{3mm}}\\
    \hline
\end{tabularx}

\vspace{1em}
\noindent
Mentre un \textbf{vettore} è:

% Tabella per le definizione di concetti, etc...
\vspace{1em}
\rowcolors{1}{black!5}{black!5}
\setlength{\tabcolsep}{14pt}
\renewcommand{\arraystretch}{2}
\noindent
\begin{tabularx}{\textwidth}{@{}|P|@{}}
    \hline
    {\textbf{VETTORE}}\\
    \parbox{\linewidth}{Un \textbf{vettore} è  una quantità definita da un valore e una direzione (e un verso, che può essere implicito nella definizione di direzione).
    \vspace{3mm}}\\
    \hline
\end{tabularx}

\vspace{1em}
\noindent
Tale definizione, tuttavia, pur essendo molto intuitiva, non risulta particolarmente pratica. Si potrebbe anche considerare un vettore come una \textbf{quantità con più valori associati}, ovvero una \textbf{lista di numeri a cui conferiamo un significato}.\\
Per esempio, in algebra un vettore viene indicato come segue
\[\vec{v} = (1, 2, 3)\]
a cui la fisica attribuisce un significato preciso: $1$, $2$ e $3$ sono le componenti associate alle tre diverse dimensioni $x$, $y$ e $z$. Il vettore di cui sopra, allora, si può scrivere come
\[\vec{v} = (v_x, v_y, v_z)\]

\vspace{1em}
\noindent
\textbf{Osservazione}: Anche se tale definizione sembra identica alla definizione del \textbf{punto}, in realtà tale definizione è differente, in quanto
\begin{itemize}
  \item un punto non ha una lunghezza;
  \item non è possibile eseguire la somma di due punti, etc.
\end{itemize}

\vspace{1em}
\subsection{Prodotto con uno scalare}
Dato un vettore
\[\vec{v} = \left(v_x, v_y, v_z\right)\]
e si considera uno scalare $a \in \mathbb{R}$, allora
\[a \cdot \vec{v} = \left(a \cdot v_x, a \cdot v_y, a \cdot v_x\right)\]
in cui il vettore $a \cdot \vec{v}$ è un vettore che
\begin{itemize}
  \item presenta come lunghezza la lunghezza del vettore $\vec{v}$ moltiplicata per $\vert a \vert$;
  \item presenta come direzione la stessa direzione del vettore $\vec{v}$;
  \item presenta come verso lo stesso verso del vettore $\vec{v}$ se $a \geq 0$, mentre avrà verso opposto se $a \leq 0$.
\end{itemize}

\vspace{1em}
\subsection{Somma vettoriale}
Dati due vettori $\vec{u}$ e $\vec{v}$, la loro somma viene eseguita graficamente tramite la \textbf{regola del parallelogramma}, o il metodo \quotes{punta-coda}:

\begin{figure}[H]
  \centering
  \begin{tikzpicture}
    \draw [-stealth, thick, red]    (0,0) -- coordinate[midway](u1) (-3,2);
    \draw [-stealth, thick, blue]   (0,0) -- coordinate[midway](v)  (5,2);
    \draw [-stealth, thick, dashed] (5,2) -- coordinate[midway](u2) (2,4);
    \draw [-stealth, thick, orange] (0,0) -- coordinate[midway](r)  (2,4);
    \draw [thick, red]    (u1) node[above]{$\vec{u}$};
    \draw [thick, blue]   (v)  node[above]{$\vec{v}$};
    \draw [thick, dashed] (u2) node[above]{$\vec{u}$};
    \draw [thick, orange] (r)  node[left]{$\vec{r}$};
  \end{tikzpicture}
  \caption{Somma vettoriale con il metodo \quotes{punta-coda}}
  \label{fig:somma_vettoriale_metodo_punta_coda}
\end{figure}

\noindent
Se $\vec{u}$ e $\vec{v}$ sono espressi nello stesso sistema di riferimento, allora è chiaro che la loro somma sarà data \textbf{componente per componente}, ovvero
\[\vec{u} + \vec{v} = \left(u_x + v_x, u_y + v_y, u_z + v_z\right)\]

\vspace{1em}
\subsection{Versori}
Si definiscano tre versori come segue
\begin{flalign*}
  \hat{i} & = (1,0,0)\\
  \hat{j} & = (0,1,0)\\
  \hat{k} & = (0,0,1)
\end{flalign*}
Allora qualsiasi vettore può essere scritto come
\[\vec{v} = \left(v_x, v_y, v_z\right) = v_x \cdot \hat{i} + v_y \cdot \hat{j} + v_z \cdot \hat{k}\]
in cui, naturalmente, $v_x$, $v_y$ e $v_z$ sono le componenti di $\vec{v}$ in direzione $\hat{i}$, $\hat{j}$ e $\hat{k}$.\\
Naturalmente si scrive $\hat{i}$ e non $\vec{i}$ in quanto
\[\left \vert \hat{i} \right \vert = \left \vert \hat{j} \right \vert = \left \vert \hat{k} \right \vert = 1\]
essi, infatti, prendono il nome di \textbf{versori} o \textbf{vettori unità}.

\vspace{1em}
\subsection{Modulo e direzione}
Di seguito si espone la definizione di \textbf{modulo di un vettore}:

% Tabella per le definizione di concetti, etc...
\vspace{1em}
\rowcolors{1}{black!5}{black!5}
\setlength{\tabcolsep}{14pt}
\renewcommand{\arraystretch}{2}
\noindent
\begin{tabularx}{\textwidth}{@{}|P|@{}}
    \hline
    {\textbf{MODULO DI UN VETTORE}}\\
    \parbox{\linewidth}{Il modulo di un vettore è la sua \quotes{lunghezza geometrica} e si indica come segue
    \[v = \left \vert \vec{v} \right \vert\]
    È chiaro che il modulo può essere \textbf{positivo o nullo}, mai negativo. In termini di componenti il modulo si calcola come segue
    \[v = \sqrt{v_x^2 + v_y^2 + v_z^2}\]
    \vspace{3mm}}\\
    \hline
\end{tabularx}

\vspace{1em}
\noindent
Per esempio, si calcoli il modulo del vettore
\[\hat{n} = \frac{\vec{v}}{v}\]
ovviamente si procede come segue
\[\left \vert \frac{\vec{v}}{v} \right \vert = \frac{1}{\left \vert v \right \vert} \cdot \left \vert \vec{v} \right \vert = \frac{v}{v} = 1\]
Ecco che allora tale vettore è a tutti gli effetti un versore in direzione $\vec{v}$, in quanto di modulo $1$.\\
Questo fa capire come si possa definire un versore associato a qualunque vettore: basta dividere il vettore per il suo modulo.

\vspace{1em}
\noindent
Mentre di seguito si espone la definizione di \textbf{direzione di un vettore}:

% Tabella per le definizione di concetti, etc...
\vspace{1em}
\rowcolors{1}{black!5}{black!5}
\setlength{\tabcolsep}{14pt}
\renewcommand{\arraystretch}{2}
\noindent
\begin{tabularx}{\textwidth}{@{}|P|@{}}
    \hline
    {\textbf{DIREZIONE DI UN VETTORE}}\\
    \parbox{\linewidth}{La direzione di un vettore (e anche il suo verso) è definita, in due dimensioni, come l'angolo $\theta$ che il vettore descrive con il semiasse positivo delle ascisse.\\
    È immediato osservare che
    \begin{center}
      $\begin{array}{c}
        v_x = v \cdot \cos(\theta)\\
        v_y = v \cdot \sin(\theta)
      \end{array}$
    \end{center}
    e si può verificare che
    \[\left \vert \vec{v} \right \vert = \sqrt{v_x^2 + v_y^2} = \sqrt{v^2 \cdot \cos^2(\theta) + v^2 \cdot \sin^2(\theta)} = v \cdot \sqrt{\cos^2(\theta) + \sin^2(\theta)} = v\]
    \vspace{-1mm}}\\
    \hline
\end{tabularx}

\newpage
\noindent
\subsection{Prodotto scalare}
Di seguito si espone la definizione di \textbf{prodotto scalare}:

% Tabella per le definizione di concetti, etc...
\vspace{1em}
\rowcolors{1}{black!5}{black!5}
\setlength{\tabcolsep}{14pt}
\renewcommand{\arraystretch}{2}
\noindent
\begin{tabularx}{\textwidth}{@{}|P|@{}}
    \hline
    {\textbf{PRODOTTO SCALARE}}\\
    \parbox{\linewidth}{Il prodotto scalare tra due vettori $\vec{v}$ e $\vec{u}$, in termini di componenti si definisce come segue:
    \[\vec{v} \cdot \vec{u} = v_x \cdot u_x + v_y \cdot u_y + v_z \cdot u_z\]
    che è, naturalmente, uno scalare.\\
    Analogamente si può interpretare il prodotto scalare tra due vettori $\vec{v}$ e $\vec{u}$ come il prodotto dei moduli per il \textbf{coseno} dell'angolo $\theta$ compreso tra i vettori stessi, ovvero
    \[\vec{v} \cdot \vec{u} = v \cdot u \cdot \cos(\theta)\]
    \vspace{-1mm}}\\
    \hline
\end{tabularx}

\vspace{1em}
\noindent
\textbf{Osservazione}: Naturalmente, da tale definizione seguono delle importanti osservazioni:
\begin{itemize}
  \item \(\vec{v} \cdot \vec{v} = v_x^2 + v_y^2 + v_z^2 = \left \vert \vec{v} \right \vert ^2\)
  \item \(\hat{i} \cdot \hat{i} = \hat{j} \cdot \hat{j} = \hat{k} \cdot \hat{k} = 1\)
  \item \(\hat{i} \cdot \hat{j} = \hat{i} \cdot \hat{k} = \hat{j} \cdot \hat{k} = 0\). Questo significa che i due vettori considerati sono ortogonali, ovvero i versori $\hat{i}$, $\hat{j}$ e $\hat{k}$ sono a due a due ortogonali.
\end{itemize}
Si consideri, invece, l'esempio seguente:
\[\vec{v} \cdot \hat{i} = \left(v_x \cdot \hat{i} + v_y \cdot \hat{j} + v_z \cdot \hat{k} \right) \cdot \hat{i} = v_x \cdot \hat{i} \cdot \hat{i} + v_y \cdot \hat{i} \cdot \hat{j} + v_z \cdot \hat{i} \cdot \hat{k} = v_x\]
e questo significa che $\vec{v} \cdot \hat{i}$ è la \textbf{proiezione} del vettore $\vec{v}$ in direzione $\hat{i}$. Tale metodo è molto efficace per effettuare un cambio di base: se al posto dei versori $\hat{i}$, $\hat{j}$ e $\hat{k}$, che presuppongono l'origine del sistema di riferimento in $O = (0,0,0)$ si scegliessere degli altri versori, moltiplicando il vettore $\vec{v}$ per taluni versori si otterrebbero le componenti del nuovo vettore in una nuova base.

\vspace{1em}
\noindent
\textbf{Osservazione}: Se si considerando due vettori $\vec{c}$ e $\vec{d}$, allora il loro prodotto scalare può essere interpretato come segue
\[\vec{c} \cdot \vec{d} \cdot \frac{d}{d} = d \cdot \left(\vec{c} \cdot \frac{\vec{d}}{d}\right)\]
per cui, ricordando che
\[\hat{n} = \frac{\vec{d}}{d}\]
è un versore in direzione del vettore $\vec{d}$, allora il prodotto scalare tra $\vec{c}$ e $\vec{d}$ è proprio la proiezione del vettore $\vec{c}$ sul vettore $\vec{d}$, per quanto appena detto a proposito delle \textbf{proiezioni}, moltiplicata per il modulo del vettore $\vec{d}$.

\newpage
\noindent
\begin{center}
  3 Marzo 2022
\end{center}

\section{Cinematica}
La descrizione del moto di un corpo (approssimanto ad un punto) prende il nome di \textbf{cinematica} (mentre la ragione del moto viene studiata dalla \textbf{dinamica}). Com'è noto, inoltre, un vettore è una quantità con \textbf{modulo} e \textbf{direzione} (e \textbf{verso}). La descrizione di un vettore avviene tramite le sue componenti: in particolare, dato un versore $\hat{n}$, la componente di un vettore $\vec{v}$ in direzione del versore $\hat{n}$ è così definita
\[\vec{v} \cdot \hat{n}\]
Per esempio, la componente del vettore $\vec{v}$ lungo l'asse $x$ è
\[\vec{v} \cdot \hat{i} = v_x\]
Inoltre, il \textbf{prodotto scalare} tra due vettori $\vec{v}$ e $\vec{u}$ viene definito come:
\[\vec{v} \cdot \vec{u} = v_x \cdot u_x + v_y \cdot u_y \cdot v_z \cdot u_z = v \cdot z \cdot \cos(\theta)\]
ove $\theta$ è l'angolo compreso tra i due vettori $\vec{v}$ e $\vec{u}$.\\
Grazie a ciò è possibile definire il concetto di \textbf{Cinematica}:

% Tabella per le definizione di concetti, etc...
\vspace{1em}
\rowcolors{1}{black!5}{black!5}
\setlength{\tabcolsep}{14pt}
\renewcommand{\arraystretch}{2}
\noindent
\begin{tabularx}{\textwidth}{@{}|P|@{}}
    \hline
    {\textbf{CINEMATICA}}\\
    \parbox{\linewidth}{La cinematica è lo \textbf{studio del moto}, a differenza della \textbf{dinamica} che studia la \textbf{causa del moto} e della \textbf{statica} che studia l'\textbf{equilibrio meccanico}, ossia la \textbf{causa dell'immobilità}.
    \vspace{3mm}}\\
    \hline
\end{tabularx}

\vspace{1em}
\noindent
È chiaro che lo studio di un corpo complesso e non omogeneo è molto più elaborato dello studio di un solo \textbf{punto}. Pertanto, il primo passo per lo studio del moto è quello di studiare il comportamento di un modello standard a cui può essere ricondotto, tramite approssimazione, un altro corpo, a seconda della necessità.

\vspace{1em}
\subsection{Posizione e spostamento}
Dato un punto nello spazio, la sua posizione viene descritta tramite un \quotes{vettore} posizione $\vec{r}$, di cui è possibile calcolare la lunghezza $\left(\vec{r}\right)$, la quale, tuttavia, non ha molto significato dal momento che dipende dalla posizione dell'origine del sistema scelta: ovverosia dipende dalla posizione iniziale e, quindi, dal \textbf{sistema di riferimento adottato}:

\begin{figure}[H]
  \centering
  \begin{tikzpicture}
      \draw [-stealth] (0,0) -- (0,2);
      \draw [-stealth] (0,0) -- (1.5,-1);
      \draw [-stealth] (0,0) -- (-1.5,-1);
      \draw [-stealth, blue] (0,0) -- (1.5,1);
      \draw [blue] (0.75,0.8) node[]{$\vec{r}$};
  \end{tikzpicture}
  \caption{\quotes{Vettore} posizione}
  \label{fig:vettore_posizione}
\end{figure}

\vspace{1em}
\noindent
Conoscere il sistema di riferimento è fondamentale, in quanto in base a ciò possono essere effettuate diverse valutazioni che, naturalmente, variano a seconda del sistema di riferimento scelto: si pensi ed effettuare una misurazione adottando come sistema di riferimento un treno che si muove oppure un treno immobile, o ancora un treno che sta accelerando: si parla, in tale contesto, di un \textbf{sistema di riferimento non inerziale}.

% Tabella per le definizione di concetti, etc...
\vspace{1em}
\rowcolors{1}{black!5}{black!5}
\setlength{\tabcolsep}{14pt}
\renewcommand{\arraystretch}{2}
\noindent
\begin{tabularx}{\textwidth}{@{}|P|@{}}
    \hline
    {\textbf{SPOSTAMENTO}}\\
    \parbox{\linewidth}{Lo \textbf{spostamento}, invece, è proprio un vettore e, com'é intuibile, taluno è definito come la differenza tra due posizioni, ovvero
    \[\boxed{\Delta \vec{r} = \vec{r_2} - \vec{r_1}}\]
    di cui è possibile calcolare il modulo come segue
    \[\left \vert \Delta \vec{r} \right \vert = \left \vert \vec{r_2} - \vec{r_1} \right \vert = \sqrt{\left(x_2 - x_1\right)^2 + \left(y_2 - y_1\right)^2 + \left(z_2 - z_1\right)^2} = \text{ distanza}\]
    \vspace{1mm}}\\
    \hline
\end{tabularx}

\vspace{1em}
\noindent
Per esempio, la lunghezza sull'asse $x$ è
\[\Delta \vec{r} = \Delta x \cdot \hat{i}\]
di cui
\[\left \vert \Delta \vec{r} \right \vert = \left \vert x_2 - x_1 \right \vert\]

\vspace{1em}
\subsection{Posizione in funzione del tempo}
È particolarmente importante considerare la variazion della posizione nel tempo

\begin{figure}[H]
  \centering
  \begin{tikzpicture}
      \draw (0,0) node[circ]{} to[out=80,in=180] (2,1);
      \draw (2,1) to[out=0,in=-120] (4,2) node[circ]{};
      \draw (0,-0.3) node[]{$\vec{r}_i$};
      \draw (4,2.3) node[]{$\vec{r}_f$};
      \draw [-stealth] (1,0.91) -- (1.8,1.2);
      \draw (1.2,1.3) node[]{$\vec{r}(t)$};
  \end{tikzpicture}
  \caption{\quotes{Vettore} posizione in funzione del tempo}
  \label{fig:vettore_posizione_funzione_tempo}
\end{figure}

Sia data una funzione spostamento, definita in funzione del tempo $t$, quale
\[\vec{r}(t) = x(t) \cdot \hat{i} + y(t) \cdot \hat{j}\]
in cui
\begin{flalign*}
  x(t) & = 2 \text{ m } + \left(2 \text{ m/s} \right) \cdot t\\
  y(t) & = 0 \text{ m } + \left(4 \text{ m/s} \right) \cdot t
\end{flalign*}
Naturalmente si ha

\vspace{2em}
\noindent
\rowcolors{1}{white}{white}
\begin{tabularx}{\textwidth}{P}
  {
      \centering
      \begin{tikzpicture}
        \begin{axis}[
          grid=both,
          axis lines = middle,
          xlabel = \(t\),
          ylabel = {\(x(t)\)},
          legend pos=outer north east,
          ymajorgrids=true,
          xmajorgrids=true,
          grid style=dashed,
        ]

        \addplot [
          domain=-2:10,
          samples=100,
          color=red,
        ]
        {2 + 2*x};
        \addlegendentry{\(x(t) = 2 \text{ m } + \left(2 \text{ m/s} \right) \cdot t\)}
        \end{axis}
    \end{tikzpicture}
  }
\end{tabularx}

\vspace{2em}
\noindent
\rowcolors{1}{white}{white}
\begin{tabularx}{\textwidth}{P}
  {
      \centering
      \begin{tikzpicture}
          \begin{axis}[
            grid=both,
            axis lines = middle,
            xlabel = \(t\),
            ylabel = {\(y(t)\)},
            legend pos=outer north east,
            ymajorgrids=true,
            xmajorgrids=true,
            grid style=dashed,
          ]

          \addplot [
            domain=-2:10,
            samples=100,
            color=blue,
          ]
          {4*x};
          \addlegendentry{\(y(t) = 0 \text{ m } + \left(4 \text{ m/s} \right) \cdot t\)}
          \end{axis}
      \end{tikzpicture}
  }
\end{tabularx}

\vspace{2em}
\noindent
\rowcolors{1}{white}{white}
\begin{tabularx}{\textwidth}{P}
  {
      \centering
      \begin{tikzpicture}
        \begin{axis}[
          grid=both,
          axis lines = middle,
          xlabel = \(t\),
          ylabel = {\(r(t)\)},
          legend pos=outer north east,
          ymajorgrids=true,
          xmajorgrids=true,
          grid style=dashed,
        ]
      \addplot[
        domain=-2:10,
        samples=100,
        color=orange,
      ]
      ({2 + 2*x},
      {4*x});
      \addlegendentry{\(\vec{r}(t) = x(t) \cdot \hat{i} + y(t) \cdot \hat{j}\)}
      \end{axis}
      \end{tikzpicture}
        }
\end{tabularx}

\vspace{1em}
\subsection{Velocità}
La \textbf{velocità} si pone alla base della cinematica. In fisica la velocità si distingue in due tipolgie
\begin{itemize}
  \item Velocità istantanea
  \item Velocità media
\end{itemize}
Si consideri, a tal proposito, il seguente grafico spazio-tempo:

\vspace{2em}
\noindent
\rowcolors{1}{white}{white}
\begin{tabularx}{\textwidth}{P}
  {
      \centering
      \begin{tikzpicture}
        \begin{axis}[
          grid=both,
          axis lines = middle,
          xlabel = \(t\),
          ylabel = {\(x\)},
          legend pos=outer north east,
          ymajorgrids=true,
          xmajorgrids=true,
          grid style=dashed,
          xtick={50,120},
          xticklabels={$t_1$,$t_2$},
          ytick={76,166},
          yticklabels={$x_1$,$x_2$},
        ]
      \addplot[
        domain=0:200,
        samples=100,
        color=orange,
      ]
      {abs(x*(abs(sin(2*x)) + 0.5) + 2)};
      \addplot [color=red,mark=*] coordinates {(50,76)};
      \addplot [color=red,mark=*] coordinates {(120,166)};
      \addplot [color=blue] coordinates {(50,76)(120,166)};
      %\addlegendentry{\(\vec{r}(t) = x(t) \cdot \hat{i} + y(t) \cdot \hat{j}\)}
      \end{axis}
      \end{tikzpicture}
        }
\end{tabularx}

% Tabella per le definizione di concetti, etc...
\vspace{1em}
\rowcolors{1}{black!5}{black!5}
\setlength{\tabcolsep}{14pt}
\renewcommand{\arraystretch}{2}
\noindent
\begin{tabularx}{\textwidth}{@{}|P|@{}}
    \hline
    {\textbf{VELOCITÀ MEDIA}}\\
    \parbox{\linewidth}{Intuitivamente si ha che la velocità media è proprio il rapporto tra uno spostamento e il tempo impiegato per effettuarlo, ovvero
    \[\boxed{\left<v\right> = \frac{\Delta \vec{r}}{\Delta t} = \frac{x_2 - x_1}{t_2 - t_1}}\]
    che, graficamente, può essere interpretata come la pendenza (o coefficiente angolare), della congiungente i punti $(x_1,t_1)$ e $(x_2,t_2)$ nel grafico spazio/tempo.
    \vspace{3mm}}\\
    \hline
\end{tabularx}
\vspace{1em}

% Tabella per le definizione di concetti, etc...
\vspace{1em}
\rowcolors{1}{black!5}{black!5}
\setlength{\tabcolsep}{14pt}
\renewcommand{\arraystretch}{2}
\noindent
\begin{tabularx}{\textwidth}{@{}|P|@{}}
    \hline
    {\textbf{VELOCITÀ ISTANTANEA}}\\
    \parbox{\linewidth}{Mentre la velocità istantanea è, naturalmente, la derivata nel tempo del vettore posizione, ovvero
    \[\boxed{\vec{v}(t) = \frac{d\vec{r}}{dt} = \lim_{t \to 0} \frac{x - x_0}{t - t_0}}\]
    ovvero la retta tangente il grafico della funzione posizione nel punto $(x_0,t_0)$. Naturalmente essendo un vettore la velocità istantanea, è possibile descriverlo tramite \textbf{componenti} come segue:
    \[\vec{v} = \frac{d}{dt} \left(\vec{r}(t)\right) = \underbrace{\frac{dx}{dt} \cdot \hat{i}}_\text{$v_x$} + \underbrace{\frac{dy}{dt} \cdot \hat{j}}_\text{$v_y$} + \underbrace{\frac{dz}{dt} \cdot \hat{k}}_\text{$v_z$}\]
    in quanto $\hat{i}$, $\hat{j}$ e $\hat{k}$ non dipendono dal tempo (cosa che potrebbe accadere, comunque, in determinate circostanze).\\
    Il \textbf{modulo} della velocità si calcola come segue
    \[\left \vert \vec{v} \right \vert = v = \sqrt{v_x^2 + v_y^2 + v_z^2}\]
    \vspace{-1mm}}\\
    \hline
\end{tabularx}

\vspace{1em}
\noindent
\textbf{Esempio}: Si consideri uno spostamento verso l'alto tale per cui $x_1 = 0 \text{ m}$ e $x_2 = 12000 \text{ m}$ e $t_1 = 2600 \text{ s}$ e $t_2 = 4000 \text{ s}$. Allora si ha che
\[\left<v_z\right> = \frac{x_2 - x_1}{t_2 - t_1} = \frac{12000}{4000 - 2600} = 8.6 \text{ m/s} = 308.6 \text{ km/h}\]
Che è una velocità irrisoria; tuttavia, ciò non sorprende, in quanto è opportuno conoscere anche le altre componenti della velocità, ossia $v_x$ e $v_y$.

\vspace{1em}
\subsection{Accelerazione}
Di seguito si espone la definizione di \textbf{accelerazione}:

% Tabella per le definizione di concetti, etc...
\vspace{1em}
\rowcolors{1}{black!5}{black!5}
\setlength{\tabcolsep}{14pt}
\renewcommand{\arraystretch}{2}
\noindent
\begin{tabularx}{\textwidth}{@{}|P|@{}}
    \hline
    {\textbf{ACCELERAZIONE}}\\
    \parbox{\linewidth}{L'accelerazione viene definita come la derivata prima della velocità nel tempo, o la derivata seconda dello spostamento nel tempo:
    \[\vec{a} = \frac{d \vec{v}}{dt} = \frac{d^2 \vec{r}}{dt}\]
    \vspace{-1mm}}\\
    \hline
\end{tabularx}

\newpage
\noindent
\begin{center}
  7 Marzo 2022
\end{center}
Comè noto, l'accelerazione è la derivata prima della velocità in funzione del tempo, o la derivata seconda dello spostamento in funzione del tempo.\\
L'accelerazione è fondamentale in \textbf{meccanica}, in quanto grazie all'accelerazione è possibile definire il concetto di forza: l'accelerazione è la connessione tra cinematica e dimanica.

\vspace{2em}
\noindent
\rowcolors{1}{white}{white}
\begin{tabularx}{\textwidth}{P}
  {
      \centering
      \begin{tikzpicture}
        \begin{axis}[
          grid=both,
          axis lines = middle,
          xlabel = \(t\),
          ylabel = {\(x(t)\)},
          legend pos=outer north east,
          ymajorgrids=true,
          xmajorgrids=true,
          grid style=dashed,
        ]

        \addplot [
          domain=-2:10,
          samples=100,
          color=red,
        ]
        {(x-4)^3 + 500};
        % \addlegendentry{\(x(t) = 2 \text{ m } + \left(2 \text{ m/s} \right) \cdot t\)}
        \end{axis}
    \end{tikzpicture}
  }
\end{tabularx}

\vspace{1em}
\noindent
Avendo a disposizione il grafico che descrive la variazione della posizione nel tempo, è possibile ora studiarne l'andamento per poi descrivere il comportamento della velocità nel secondo grafico. È sufficiente, pertanto, osservare gli intervalli di crescenza e decrescenza e i punti in cui la derivata si annulla nulla:

\vspace{2em}
\noindent
\rowcolors{1}{white}{white}
\begin{tabularx}{\textwidth}{P}
  {
      \centering
      \begin{tikzpicture}
        \begin{axis}[
          grid=both,
          axis lines = middle,
          xlabel = \(t\),
          ylabel = {\(v(t)\)},
          legend pos=outer north east,
          ymajorgrids=true,
          xmajorgrids=true,
          grid style=dashed,
        ]

        \addplot [
          domain=-2:10,
          samples=100,
          color=red,
        ]
        {(x-4)^2+1000};
        % \addlegendentry{\(x(t) = 2 \text{ m } + \left(2 \text{ m/s} \right) \cdot t\)}
        \end{axis}
    \end{tikzpicture}
  }
\end{tabularx}

\vspace{1em}
\noindent
Ancora una volta, studiando l'andamento della velocità nel suo rispettivo grafico è ora possibile descrivere il grafico della derivata della velocità, ovvero dell'accelerazione, sempre analizzando gli intervalli di crescenza e decrescenza:

\vspace{2em}
\noindent
\rowcolors{1}{white}{white}
\begin{tabularx}{\textwidth}{P}
  {
      \centering
      \begin{tikzpicture}
        \begin{axis}[
          grid=both,
          axis lines = middle,
          xlabel = \(t\),
          ylabel = {\(a(t)\)},
          legend pos=outer north east,
          ymajorgrids=true,
          xmajorgrids=true,
          grid style=dashed,
        ]

        \addplot [
          domain=-2:10,
          samples=100,
          color=red,
        ]
        {x - 4};
        % \addlegendentry{\(x(t) = 2 \text{ m } + \left(2 \text{ m/s} \right) \cdot t\)}
        \end{axis}
    \end{tikzpicture}
  }
\end{tabularx}

\vspace{1em}
\noindent
Ecco che il punto in cui la derivata seconda (ovverosia l'accelerazine) cambia segno è un \textbf{punto di flesso}, ovvero il punto in cui il grafico dello spostamento cambia la propria concavità.

\vspace{1em}
\noindent
\textbf{Esempio}: Si consideri la seguente funzione spostamento in funzione del tempo:
\[x(t) = A \cdot \cos(\omega t)\]
questa è l'equazione di oscillazione di un pendolo o di una molla. Per il calcolo della velocità è sufficiente calcolare la derivata prima, ovvero
\[v(t) = \frac{dx}{dt} = -\omega \cdot A \cdot \sin(\omega t)\]
e per l'accelerazione è sufficiente derivare nuovamente la velocità
\[a(t) = \frac{dv}{dt} = -\omega^2 \cdot A \cdot \cos(\omega t) = -\omega^2 \cdot x(t)\]
Tale risultato ha senso e può essere interpretato anche graficamente, grazie al grafico di una molla: quando lo spostamento è positivo, l'accelerazione è negativa, e viceversa.

\begin{figure}[H]
  \centering
  \begin{tikzpicture}[every node/.style={draw,outer sep=0pt,inner sep=0pt,thick}, scale=2]
    \tikzstyle{spring}=[thick,decorate,decoration={aspect=0.5, segment length=1mm, amplitude=2mm,coil}]
    \draw[thick] (0,0) --(0,1);
    \draw[thick] (0,0) --(3,0) node[draw=none,xshift=5pt]{$x$};
    \node at(0,0.25) (a) [draw=none] {};
    \node at (2,0.25)(b) [minimum size=0.5cm,label=$\rightarrow$] {m};
    \draw [spring] (a) -- (b) node[draw=none,pos=.5,right=.25cm] {};
    \node at (2,-0.3)(c) [draw=none,yshift=5pt] {$x=0$};
    \draw[dashed] (b.south) -- (c.north);
  \end{tikzpicture}
  \caption{Fisica di una molla}
  \label{fig:fisica_molla}
\end{figure}

\vspace{1em}
\noindent
\textbf{Osservazione}: Quando la velocità è nulla, la posizione si mantiene costante e non cresce o decresce. Quando si ha un punto di salto della velocità si ha una situazione difficile da riprodurre fisicamente, in quanto si ha un crollo della velocità istantanea, come un impulso (si pensi anche al fatto che, per il teorema del limite della derivata, una funzione con un salto non può essere la derivata di una funzione derivabile).\\
Si può utilizzare anche l'\textbf{integrale} per passare dalla velocità allo spostamento.

\vspace{1em}
\subsection{Moto uniformemente accelerato}
Nel moto uniformemente accelerato si ha che l'\textbf{accelerazione} è \textbf{costante}. Sapenche l'accelerazione è la derivata prima della velocità nel tempo, si può scrivere:
\[\frac{dv}{dt} = a \longrightarrow dv = a \cdot dt\]
Questo, in particolare, è possibile farlo sia per una variazione $\Delta$, ma anche per una variazione infinitesimale $d$. Ciò che si sta facendo, in questo caso, è risolvere un'\textbf{equazione differenziale}. Pertanto, dopo aver ottenuto $dv = a \cdot dt$ si può procedere all'integrazione
\[\int dv = \int a \cdot dt \longrightarrow v = at + c\]
La costante $c$ che compare nella formula, ottenuta grazie alla risoluzione di una equazione differenziale, è la cosiddettà \textbf{velocità inziale}: se $t=0$, infatti, $v = c = v_0$. Quindi l'equazione diviene
\[\boxed{v(t) = v_0 + a \cdot t}\]
Avendo ottenuto l'equazione della velocità nel tempo, è opportuno ottenere l'equazione della posizione in funzione del tempo, integrando nuovamente, sempre partendo da
\[\frac{dx}{dt} = v \longrightarrow dx = v \cdot dt \longrightarrow \int dx = \int v \cdot dt = \int (v_0 + a \cdot t) dt\]
quindi si ottiene
\[x(t) = \int v_0 \cdot dt + \int a \cdot t \cdot dt + c = v_0 \cdot t + a \cdot \frac{t^2}{2} + c\]
ove $c = x_0 = x(t=0)$. Pertanto l'equazione della posizione in funzione del tempo è
\[\boxed{x(t) = x_0 + v_0 \cdot t + \frac{1}{2} \cdot a \cdot t^2}\]
che rappresenta l'equazione di una parabola, come illustrato nell'esempio seguente:

\vspace{2em}
\noindent
\rowcolors{1}{white}{white}
\begin{tabularx}{\textwidth}{P}
  {
      \centering
      \begin{tikzpicture}
        \begin{axis}[
          axis lines = left,
          xlabel = \(t\),
          ylabel = {\(x(t)\)},
          legend pos=outer north east,
          ymajorgrids=true,
          xmajorgrids=true,
          grid style=dashed,
          ytick={-11},
          yticklabels={$x_0$},
          ymax=15,
        ]

        \addplot [
          domain=-2:10,
          samples=100,
          color=red,
        ]
        {-(x-2)^2 + 5};
        \draw [-stealth, blue, thick] (axis cs:-2,-11) -- (axis cs:1,10);
        \draw [blue] (axis cs:-1.2,0) node[]{$\vec{v}_0$};
        \end{axis}
    \end{tikzpicture}
  }
\end{tabularx}

\vspace{1em}
\noindent
In cui, ovviamente, l'intersezione tra il grafico e l'asse $y$ è $x_0$, il vettore designato in blu, ossia la tangente in $(x_0,0)$, rappresenta la pendenza iniziale del grafico della posizione, ovvero la velocità iniziale $v_0$. Essndo il grafico di un moto uniformemente accelerato, è ovvio che la pendenza decresce in modo costante: prima la velocità sarà positiva, ma dcrescente, e poi continuerà a decrescere, ma con valori negativi.

\vspace{1em}
\noindent
\textbf{Osservazione}: Si consideri la seguente equazione
\[v(t) = a \cdot t + v_0\]
e si provi ad isolare $t$ da tale equazione, come segue
\[t = \frac{v - v_0}{a}\]
se ora si considera l'equazione seguente
\[x(t) = \frac{1}{2} \cdot a \cdot t^2 + v_0 \cdot t + x_0\]
e si sostituisce il $t$ calcolato in precedenza a tale equazione si ottiene
\[x(t) = \frac{1}{2} \cdot a \cdot \left(\frac{v - v_0}{a}\right)^2 + v_0 \cdot \left(\frac{v - v_0}{a}\right) + x_0 = \frac{1}{2} \cdot \frac{1}{a} \cdot (v^2 - 2 \cdot v \cdot v_0 + v_0^2) + \frac{1}{a} \cdot (v_0 \cdot v + v_0^2) + x_0\]
ovvero si ottiene
\[\boxed{x = \frac{v^2}{2a} - \frac{v_0^2}{2a} + x_0}\]
quindi
\[\boxed{v^2 - v_0^2 = 2a \cdot (x - x_0)}\]
la quale è \textbf{valida solamente per il moto uniformemente accelerato} ed è estremamente utile per conoscere lo spazio percorso da un corpo che si muove secondo queste legge oraria, senza conoscere il \textbf{tempo}.

\vspace{1em}
\subsubsection{Caduta libera dei gravi}
La \textbf{caduta libera} avviene con la \textbf{medesima accelerazione} per tutti i corpi (ovviamente, l'attrazione gravitazionale non è costante in tutto l'universo, ma se si considerano un punto sulla terra e distanze piccole e prossime a quella del raggio terrestre, si può, senza perdita di generalità, considerare un'accelerazione costante $g$).

\vspace{1em}
\noindent
\textbf{Osservazione}: Naturalmente sulla Luna non c'è aria, si è nel vuoto, per cui tutti i corpi vengono attratti dalla Luna solamente per la forza di attrazione gravitazionale, senza che tale moto sia influenzato dall'attrito dell'aria.

\vspace{1em}
\noindent
L'accelerazione gravitazionale può essere interpretata come segue:

\begin{figure}[H]
  \centering
  \begin{tikzpicture}
    \draw [-stealth] (0,0) -- (0,-5);
    \draw (2,-2.5) node[]{$\vec{a} = -g \cdot \hat{j} \hspace{0.5em} \text{con} \hspace{0.5em} 9.8 \text{ m/s}$};
  \end{tikzpicture}
  \caption{Visualizzazione grafica del moto di caduta libera}
  \label{fig:visualizzazione_grafica_caduta_libera}
\end{figure}


\vspace{1em}
\noindent
Ove, naturalmente si ha
\[\vec{a} = -g \cdot \hat{j} \hspace{0.5em} \text{con} \hspace{0.5em} 9.8 \text{ m/s}\]
Pertanto si ottengono le seguenti equazioni, a partire da quelle generali per un moto uniformemente accelerato. Per quanto riguarda la velocità di caduta si ha:
\[\boxed{v_y = -g \cdot t + v_{0y}}\]
mentre per quanto riguarda la posizione in funzione del tempo si ottiene
\[\boxed{y = - \frac{1}{2} \cdot g \cdot t^2 + v_{0y} + y_0}\]
Pe capire che altezza raggiungerà un corpo quando viene lanciato verso l'alto si deve osservare il grafico seguente

\vspace{2em}
\noindent
\rowcolors{1}{white}{white}
\begin{tabularx}{\textwidth}{P}
  {
      \centering
      \begin{tikzpicture}
        \begin{axis}[
          axis lines = left,
          xlabel = \(t\),
          ylabel = {\(x(t)\)},
          legend pos=outer north east,
          ymajorgrids=true,
          xmajorgrids=true,
          grid style=dashed,
          ytick={-11,5},
          yticklabels={$y_0$,$y_m$},
          ymax=15,
        ]

        \addplot [
          domain=-2:10,
          samples=100,
          color=red,
        ]
        {-(x-2)^2 + 5};
        \draw [thick, red, dashed] (axis cs:0,5) -- (axis cs:4,5);
        \draw [thick, red] (axis cs:2,9) node[]{$v_y=0$};
        \end{axis}
    \end{tikzpicture}
  }
\end{tabularx}

\vspace{1em}
\noindent
Naturalmente si osserva che la velocità nel punto più alto è nulla, in quanto la tangente è orizzontale, mentre si conosce la velocità inziale $v_0$.\\
Per capire l'altezza, allora, si potrebbe calcolare il tempo impiegato per raggiungere il punto più alto e poi sostituire tale valore all'interno dell'equazione dello spostamento.\\
 Alternativamente, si potrebbe considerare la formula seguente
\[v_y^2 - v_{y0}^2 = -2 \cdot g \cdot (y - y_0)\]
e sostitutendo i valori si ha
\[0 - v_{y0}^2 = 2 \cdot g \cdot (y_m - y_0)\]
Per cui l'altezza massima che un corpo raggiunge quando viene lanciato verso l'alto con velocità iniziale $v_{y0}$ e a partire da un'altezza $y_0$ è
\[\boxed{y_m = y_0 + \frac{v_{y0}^2}{2g}}\]

\vspace{1em}
\subsubsection{Moto dei proiettili}
Il moto dei proiettili ha un'importanza storica fondamentale: Aristotele, nel $340$ a.c. parlava di \textbf{moto \quotes{naturale} e \quotes{forzato}}: tuttavia, naturalmente, un oggetto fermo, privo di sollecitazioni, non ha la propensione a muoversi, quindi non ha senso parlare di \emph{forzatura}.\\
Successivamente, Filipono ($490-570$) d.c. ha introdotto il concetto di \textbf{impeto} e, infine, \textbf{Galileo} ($1564-1642$) d.c., basandosi su osservazioni e misurazioni pratiche precedenti (invece che fornire una spiegazioni a priori), ha fornito una \textbf{spiegazione scentifica} a tale fenomeno.

% Tabella per le definizione di concetti, etc...
\vspace{1em}
\rowcolors{1}{black!5}{black!5}
\setlength{\tabcolsep}{14pt}
\renewcommand{\arraystretch}{2}
\noindent
\begin{tabularx}{\textwidth}{@{}|P|@{}}
    \hline
    {\textbf{LEGGE ORARIA DEL MOTO DEI PROIETTILI}}\\
    \parbox{\linewidth}{Per lo studio del moto dei proiettili si considera il vettore accelerazione
    \[\boxed{\vec{a} = a_x \cdot \hat{i} + a_y \cdot \hat{j} = 0 \cdot \hat{i} - g \cdot \hat{j}}\]
    Analogamente per la velocità si ha
    \[\boxed{\vec{v} = v_x \cdot \hat{i} + v_y \cdot \hat{j} = v_{0x} \cdot \hat{i} - (v_{0y} - g t) \cdot \hat{j}}\]
    Per quanto concerne la posizione in funzione del tempo si ha
    \[\boxed{\vec{r} = x \cdot \hat{i} + y \cdot \hat{j} = (v_{0x} t + x_0) \cdot \hat{i} - (y_0 + v_{0y} t - \frac{1}{2} g t^2) \cdot \hat{j}}\]
    \vspace{-1mm}}\\
    \hline
\end{tabularx}

\vspace{2em}
\noindent
Si consideri il seguente grafico della posizione di un proiettile:

\vspace{2em}
\noindent
\rowcolors{1}{white}{white}
\begin{tabularx}{\textwidth}{P}
  {
      \centering
      \begin{tikzpicture}
        \begin{axis}[
          axis lines = left,
          xlabel = \(t\),
          ylabel = {\(x(t)\)},
          legend pos=outer north east,
          ymajorgrids=true,
          xmajorgrids=true,
          grid style=dashed,
          ytick={-11},
          yticklabels={$x_0$},
          ymax=15,
        ]

        \addplot [
          domain=-2:10,
          samples=100,
          color=red,
        ]
        {-(x-2)^2 + 5};
        \draw [-stealth, blue, thick] (axis cs:-2,-11) -- (axis cs:1,10);
        \draw [blue] (axis cs:-1.2,0) node[]{$\vec{v}_0$};
        \draw [blue] (axis cs:-1.6,-11.2) node[]{$\vec{r}_0$};
        \end{axis}
    \end{tikzpicture}
  }
\end{tabularx}

\vspace{1em}
\noindent
Naturalmente si ha che
\[\vec{v_{0}} = v_{x0} \cdot \hat{i} + v_{x0} \cdot \hat{j}\]
\[\vec{r_{0}} = x_0 \cdot \hat{i} + y_0 \cdot \hat{j}\]

\newpage

\noindent
\begin{center}
  8 Marzo 2022
\end{center}
Naturalmente il modulo di un vettore non dipende dal sistema di coordinate scelto.\\
Naturalmente è possibile avere
\[\left \vert \vec{A} + \vec{B} \right \vert = \left \vert \vec{A} - \vec{B} \right \vert\]
e per dimostrarne la veridicità basta considerare due vettori perpendicolari.\\
Ovviamente, un versore ha sempre modulo unitario per essere definito tale: in particolare, dato il vettore $\vec{v} = \hat{i} + \hat{j} + \hat{k}$, il versore
\[\hat{n} = \frac{\vec{v}}{v}\]
è proprio un versore in direzione $\hat{i} + \hat{j} + \hat{k}$.\\
La componente del vettore $\vec{v} = -3 \cdot \hat{i} + 5 \cdot \hat{j} + \hat{k}$ in direzione $\hat{n} = 0.6 \cdot \hat{j} - 0.8 \cdot \hat{k}$ è ovviamente
\[\vec{v} \cdot \hat{n} = -3 \cdot 0 + 5 \cdot 0.6 - 1 \cdot 0.8 = 2.2\]
È chiaro che in questo caso $\hat{n}$ era già un versore, altrimenti, se si avesse avuto un vetttore, si sarebbe dovuto calcolare il versore corrispondente dividendo per il suo modulo.

\vspace{1em}
\noindent
Nel moto dei proiettili è estremamente importante considerare l'\textbf{altezza massima} che esso raggiungerà, ma soprattutto la sua \textbf{gittata}, ovvero la distanza massima che esso raggiungerà.\\
Molto spesso, in questo caso, per lavorare è molto più convieniente operare con le coordinate polari $v_0$ e $\theta$ (ovvero con modulo e angolo), anziché con $v_{x0}$ e $v_{y0}$, sempre ricordando che
\[v = \left(
  \begin{array}{l}
    v_{x0} = v_0 \cdot \cos(\theta)\\
    v_{y0} = v_0 \cdot \sin(\theta)\\
  \end{array}
\right)\]
In questo caso, per calcolare l'altezza massima raggiunta è sufficiente considerare solamente la componente della velocità verticale, come per la caduta libera dei gravi, ovvero
\[y_m = y_0 + \frac{v_{y0}^2}{2g} = y_0 + \frac{v_0^2 \cdot \sin^2(\theta)}{2g}\]
Analogamente, per calcolare la gittata, ovvero la distanza orizzontale percorsa da un corpo lanciato in aria, si dovrà usare solo la componente della velocità orrizontale.

\vspace{2em}
\noindent
\rowcolors{1}{white}{white}
\begin{tabularx}{\textwidth}{P}
  {
      \centering
      \begin{tikzpicture}
        \begin{axis}[
          axis lines = left,
          xlabel = \(y\),
          ylabel = {\(x\)},
          legend pos=outer north east,
          ymajorgrids=true,
          xmajorgrids=true,
          grid style=dashed,
        ]

        \addplot [
          domain=0:10,
          samples=100,
          color=red,
        ]
        {-(x-2)^2 + 4};
        % \addlegendentry{\(x(t) = 2 \text{ m } + \left(2 \text{ m/s} \right) \cdot t\)}
        \end{axis}
    \end{tikzpicture}
  }
\end{tabularx}

\vspace{1em}
\noindent
Naturalmente, in questo caso, il calcolo della gittata $R$ si effettua come segue: è noto che
\[R = v_{x0} \cdot t_R\]
ove $t_R$ è la \textbf{durata del volo}. Però è noto che al tempo $t_R$ la coordinata $y$ è nulla, ovvero
\[y(t_R) = 0 = v_{y0} \cdot t_R - \frac{1}{2} \cdot g \cdot t_R^2\]
Dal momento che il tempo $t_R$ non è nullo è possibile dividere per $t$, ottenendo
\[v_{y0} - \frac{1}{2} \cdot g \cdot t_R = 0\]
da cui si evince che il tempo $t_R$ cercato è
\[t_R = \frac{2 \cdot v_{y0}}{g}\]
sostituendo, ora, il tempo trovato nella formula di cui sopra si ottiene
\[R = v_{x0} \cdot \frac{2 \cdot v_{y0}}{g} = 2 \cdot \frac{v_{x0} \cdot v_{y0}}{g}\]
ma ricordando come si calcolano le componenti $v_{x0}$ e $v_{y0}$ si ha
\[R = 2 \cdot \frac{v_0^2 \cdot \sin(\theta) \cdot \cos(\theta)}{g}\]
ed essendo $2 \cdot \sin(\theta) \cdot \cos(\theta) = \sin(2 \cdot \theta)$ si ottiene
\[\boxed{R = \frac{v_0^2 \cdot \sin(2\theta)}{g}}\]
Le unità in gioco sono
\[[v^2] = \frac{\text{m}^2}{\text{s}^2}\]
\[[g] = \frac{\text{m}}{\text{s}^2}\]
\[[R] = \frac{\dfrac{\text{m}^2}{\text{s}^2}}{\dfrac{\text{m}}{\text{s}^2}} = \text{m}\]

\vspace{1em}
\noindent
\textbf{Osservazione}: Si osservi che, naturalmente, quando $\theta=0^\circ$, si ha che $\sin(0) = 0$ e quindi $R=0$: ciò ha senso, in quanto lanciando orizzontalmente il corpo cade immediatamente.\\
Quando $\theta=90^\circ$ non si ha gittata, in quanto, lanciando verticalmente il corpo cada verticalmene.\\
Quando $\theta=45^\circ$ si ha la gittata massima, in quanto $\sin(90)=1$.

\vspace{1em}
\subsection{Moto in 2D e 3D}
Si consideri la seguente traiettoria

\begin{figure}[H]
  \centering
  \begin{tikzpicture}
      \draw (0,0) node[circ]{} to[out=80,in=180] (2,1);
      \draw (2,1) to[out=0,in=-120] (4,2) node[circ]{};
      \draw (0,-0.3) node[]{$\vec{r}_i$};
      \draw (4,2.3) node[]{$\vec{r}_f$};
      \draw [-stealth, red] (1,0.91) -- (1.8,1.2);
      \draw (1.2,1.3) node[]{$\vec{v}(t)$};
      \draw [-stealth, blue] (1,0.91) -- (1.2,0.4);
      \draw (0.85,0.6) node[]{$\vec{a}_\perp$};
  \end{tikzpicture}
  \caption{\quotes{Vettore} velocità in funzione del tempo}
  \label{fig:vettore_velocità_funzione_tempo}
\end{figure}

\vspace{1em}
\noindent
Naturalmente $\vec{v}$ è sempre parallelo alla tangente della curva $\vec{r}(t)$.\\
Si osservi, inoltre, che l'accelerazione deve sempre presentare una \textbf{componente lineare} $\vec{a}_\parallel$ (parallela a $\vec{v}$) e una \textbf{componente ortogonale} $\vec{a}_\perp$ (in direzione del cambiamento dell'orientazione della velocità), la quale è fondamentale: infatti, se ci fosse solo una componente lineare, la velocità aumenterebbe il proprio modulo, ma non direzione; ogni qualvolta si ha una variazione della direzione della velocità ci deve essere una componente ortogonale dell'accelerazione.

\vspace{1em}
\noindent
\subsection{Moto circolare uniforme}
Per moto circolare uniforme si intende un moto circolare in cui la \textbf{velocità angolare} si mantiene \textbf{costante}.\\

\begin{figure}[H]
  \centering
  \begin{tikzpicture}[>=Triangle]
    \shade [top color=white, bottom color=gray!50, middle color=white]
      (120:8/3) arc (120:190:8/3) node [black, near end, left] {$\omega$}
      -- (190:25/9) -- (200:15/6) -- (190:20/9) -- (190:7/3)
      arc (190:120:7/3) -- cycle;

    \foreach \i in {90, 210, 330}{
      \draw [->, thick, blue!50!cyan] (\i-65:2) arc (\i-65:\i+60:2);
      \tikzset{shift={(\i:2)}, rotate=\i+180}
      \draw [->, very thick, orange] (0,0) -- (1,0)
        node [black, near end, anchor=\i+90] {$\vec a$};
      \draw [->, very thick, green!50!black] (0,0) -- (0,-2)
        node [black, near end, anchor=\i+180] {$\vec v$};
      \fill circle [radius=1/10];
  }
  \end{tikzpicture}
  \caption{Moto circolare uniforme}
  \label{fig:moto_circolare_uniforme}
\end{figure}

\vspace{1em}
\noindent
Naturalmente, la \textbf{circonferenza} del cerchio è $C = 2\pi R$. Parlando di moto circolare uniforme è possibile introdurre il concetto di \textbf{periodo} $T$, ovvero il tempo impiegato a completare una circonferenza completa.\\
Pertanto, volendo conoscere la velocità del moto si ottiene
\[\boxed{v = \frac{2\pi R}{T} = \omega R}\]
ove $\omega$ prende il nome di \textbf{velocità angolare}, che ha una misura di $\text{RAD}/\text{s}$, per questo prende il nome di velocità angolare, in quanto ha la stessa unità di misura di una frequenza (visto che l'angolo non ha una propria vera misura).\\
Naturalmente, volendo conoscere l'accelerazione che agisce sul punto in movimento, si può immediatamente dire che, essendo la velocità costante, si dovrà solo considerare una componente ortogonale, in quanto se essa non ci fosse, il corpo si muoverebbe in linea retta, senza compiere una traiettoria circolare.\\
Pertanto, dal momento che $\left \vert \vec{v} \right \vert = v =$ costante, si ha che $\vec{a}_\parallel = 0$. L'accelerazione media, naturalmente, può essere calcolata come segue
\[\boxed{\left<\vec{a}\right> = \frac{\Delta \vec{v}}{\Delta t} = \frac{\vec{v}(t_2) - \vec{v}(t_1)}{t_2 - t_1}}\]
Anche graficamente appare evidente come l'accelerazione media sia un vettore ortogonale al vettore velocità e orientato verso il centro della circonferenza.\\

\begin{figure}[H]
  \centering
  \begin{tikzpicture}[scale=2]
    \node[minimum size=4cm,circle,draw,blue!50!cyan] (circle) {};
    \draw [thick] (0,0) -- coordinate[midway](m) (circle.80);
    \draw [thick] (0,0) -- (circle.55);
    \draw [thick, -stealth] (circle.80) -- (circle.55);
    \draw (circle.67.5) ++(0.2,0.2) node[]{$\Delta \vec r$};
    \draw (m) ++(-0.2,0) node[]{$R$};
    \begin{scope}
      \tikzset{shift={(30:2)}, rotate=300}
      \draw [-stealth] (circle.30) -- ++(0.5,0) node [black, near end, anchor=210] {$\vec v(t_1)$};
    \end{scope}
    \begin{scope}
      \tikzset{shift={(30:2)}, rotate=210}
      \draw [-stealth] (circle.300) -- ++(0.5,0) node [black, near end, anchor=90] {\hspace{0.5em} $\vec v(t_2)$};
    \end{scope}
    \begin{scope}
      \tikzset{shift={(30:2)}, rotate=210}
      \draw (circle.300) ++(0.5,0) coordinate(a);
      \tikzset{shift={(30:2)}, rotate=90}
      \draw [-stealth] (a) -- ++(-0.5,0) coordinate(b) node [black, near end, anchor=0] {$-\vec v(t_1)$\hspace{0.5em}};
      \draw [-stealth, orange] (circle.300) -- (b) node [orange, near end, anchor=270] {$\Delta \vec v$\hspace{0.5em}};
    \end{scope}
    \coordinate (i) at (circle.55);
    \coordinate (ctr) at (0,0);
    \coordinate (f) at (circle.80);
    \pic [draw=red, text=red, <->, "$\Delta \theta$", angle eccentricity=1.3, angle radius=1cm] {angle = i--ctr--f};
  \end{tikzpicture}
\end{figure}

\vspace{1em}
\noindent
Lo spostamento tra due punti, naturalmente, è $\Delta \vec{r}$ e i due triangoli che vengono così disegnati sono simili, per cui il rapporto tra i lati corrispondenti deve mantenersi costantte.\\
È facile, pertanto, vedere immediatamente come
\[\boxed{\frac{\left \vert \Delta \vec{r} \right \vert}{R} = \frac{\left \vert \Delta \vec{v} \right \vert}{v}}\]
ossia il rapporto tra lo spostamento e il raggio costante, così come la variazione di velocità e il modulo della velocità costante è uguale. Da ciò si evince che
\[a = \frac{\left \vert \Delta \vec{v} \right \vert}{\Delta t} = \frac{\left \vert \Delta \vec{v} \right \vert}{v} \cdot \frac{v}{\Delta t} = \frac{\left \vert \Delta \vec{r} \right \vert}{R} \cdot \frac{v}{\Delta t} = \frac{v^2}{R}\]
Ovvero si ha che, nel moto circolare uniforme, il modulo dell'accelerazione orientata verso il centro del cerchio è pari a
\[\boxed{a=\frac{v^2}{R}}\]

\vspace{1em}
\noindent
\textbf{Esempio}: Si consideri l'esempio seguente che riguarda un veicolo in movimento a velocità costante su una curva

\begin{figure}[H]
  \centering
  \begin{tikzpicture}
    \draw (0,0) -- (2,0) to[out=0,in=270] coordinate[midway](r) (4,2) -- (4,4);
    \draw (2,2) node[circ](start){} -- coordinate[midway](a) (r) (a) ++(0.3,0.3) node[]{$R$};
    \draw [dotted] (start) -- ++(2,0) node[at end, right]{$t_2$};
    \draw [dotted] (start) -- ++(0,-2) node[at end, below]{$t_1$};
    \draw [-stealth] (2.4,0.05) node[circ]{} -- (2.2,0.8);
    \draw (2.5,0.4) node[]{$\vec{a}$};
    \draw [-stealth, red] (2,0) -- (2,1);
    \draw [-stealth, red] (4,2) -- (3,2);
  \end{tikzpicture}
  \caption{Auto in movimento su una curva}
  \label{fig:auto_movimento_curva}
\end{figure}

\vspace{1em}
\noindent
e si considerino le componenti dell'accelerazione in $x$ e in $y$ in funzione del tempo.

\vspace{2em}
\noindent
\rowcolors{1}{white}{white}
\begin{tabularx}{\textwidth}{P}
  {
      \centering
      \begin{tikzpicture}
        \begin{axis}[
          grid=both,
          axis lines = middle,
          xlabel = \(t\),
          ylabel = {\(a_x\)},
          legend pos=outer north east,
          ymajorgrids=true,
          xmajorgrids=true,
          grid style=dashed,
          xtick={pi/2,pi},
          xticklabels={$t_1$,$t_2$},
          xmin=0,
          xmax=5,
          ymin=-2,
          ymax=2,
        ]

        \addplot [
          domain=pi/2:pi,
          samples=100,
          color=red,
          thick
        ]
        {cos(deg(x)};

        \addplot [
          domain=0:5,
          samples=100,
          color=red,
          dashed
        ]
        {cos(deg(x)};
        \draw [red, thick] (axis cs:0,0) -- (axis cs:pi/2,0);
        \draw [red, thick, dotted] (axis cs:pi,0) -- (axis cs:pi,-1);
        \draw [red, thick] (axis cs:pi,0) -- (axis cs:5,0);
        \end{axis}
    \end{tikzpicture}
  }
\end{tabularx}

\vspace{2em}
\noindent
\rowcolors{1}{white}{white}
\begin{tabularx}{\textwidth}{P}
  {
      \centering
      \begin{tikzpicture}
        \begin{axis}[
          grid=both,
          axis lines = middle,
          xlabel = \(t\),
          ylabel = {\(a_y\)},
          legend pos=outer north east,
          ymajorgrids=true,
          xmajorgrids=true,
          grid style=dashed,
          xtick={pi/2,pi},
          xticklabels={$t_1$,$t_2$},
          xmin=0,
          xmax=5,
          ymin=-2,
          ymax=2,
        ]

        \addplot [
          domain=pi/2:pi,
          samples=100,
          color=blue,
          thick
        ]
        {sin(deg(x))};

        \addplot [
          domain=0:5,
          samples=100,
          color=blue,
          dashed
        ]
        {sin(deg(x))};
        \draw [blue, thick] (axis cs:0,0) -- (axis cs:pi/2,0);
        \draw [blue, thick, dotted] (axis cs:pi/2,0) -- (axis cs:pi/2,1);
        \draw [blue, thick] (axis cs:pi,0) -- (axis cs:5,0);
        \end{axis}
    \end{tikzpicture}
  }
\end{tabularx}

\vspace{1em}
\noindent
per capire la natura delle curve appena disegnate, è sufficiente osservare la Figura \ref{fig:visualizzazione_angoli_coinvolti} seguente:

\begin{figure}[H]
  \centering
  \begin{tikzpicture}[scale=2]
    \draw (0,0) -- (2,0) to[out=0,in=270] coordinate[midway](r) (4,2) -- (4,4);
    \draw (2,2) node[circ](start){} -- coordinate[midway](a) (r) (a) ++(0.3,0.3) node[]{$R$};
    \draw [dotted] (start) -- ++(2,0) node[at end, right]{$t_2$};
    \draw [dotted] (start) -- ++(0,-2) node[at end, below]{$t_1$};
    \draw [dotted] (2.4,0.05) node[circ](b){} -- coordinate[midway](half) (2,2);
    \draw [-stealth] (b) -- (half);
    \draw (b) -- ++(1,0) coordinate(c);
    \draw (2.5,0.6) node[]{$\vec{a}$};
    \draw [-stealth, red] (2,0) -- (2,1);
    \draw [-stealth, red] (4,2) -- (3,2);

    \coordinate (f) at (2.2,0.8);
    \coordinate (ctr) at (b);
    \coordinate (i) at (c);
    \pic [draw=red, text=red, <->, "$\theta$", angle eccentricity=1.5] {angle = i--ctr--f};

    \coordinate (f1) at (half);
    \coordinate (ctr1) at (2,2);
    \coordinate (i1) at (2,1);
    \pic [draw=blue, text=blue, <->, "$\alpha$", angle eccentricity=1.2, angle radius=1.8cm] {angle = i1--ctr1--f1};
  \end{tikzpicture}
  \caption{Visualizzazione degli angoli coinvolti}
  \label{fig:visualizzazione_angoli_coinvolti}
\end{figure}

\vspace{1em}
\noindent
Si può facilmente capire come $\alpha$ sia l'angolo da sommare a $90^\circ$ per ottenere $\theta$, quindi
\[a_x=a \cdot \cos(\theta) = a \cdot \cos(\alpha + 90^\circ) = -a \cdot \sin(\alpha) = -a \cdot \sin(\omega t)\]
pertanto nel caso di $a_x$ si è considerato un ramo di $\sin(\omega t)$, mentre nel caso di $a_y$ si è considerato un ramo di $\cos(\omega t)$.

\newpage
\noindent
\begin{center}
  9 Marzo 2022
\end{center}
Il moto circolare uniforme è un moto semplice: la formula più importante da conoscere è il modulo dell'\textbf{accelereazione centripeta}, ovvero dell'accelerazione diretta ferso il centro della circonferenza.\\
Il raggio $R$ della circonferenza del moto è costante, mentre l'angolo $\theta$ descritto dal punto in movimento all'interno della circonferenza varia linearmente con il tempo, secondo la seguente legge
\[\boxed{\theta(t) = \frac{2\pi}{T} \cdot t = \omega t}\]
ove $\omega$ prende il nome di velocità angolare ed è definita come segue
\[\boxed{\omega = \frac{2\pi}{T}}\]
Per quando concerne la variazione della posizione nel tempo si ha
\[\boxed{\vec{r}(t) = x(t) \cdot \hat{i} + y(t) \cdot \hat{i} = R \cdot \cos(\omega t) \cdot \hat{i} + R \cdot \sin(\omega t) \cdot \hat{j}}\]
A partire da tale risultato si sarebbe potuto determinare il modulo dell'accelerazione, semplicemente procedendo per derivate successive, ottenendo dapprima
\[\boxed{\vec{v}(t) = - \omega R \cdot \sin(\omega t) \cdot \hat{i} + \omega R \cos(\omega t) \cdot \hat{i}}\]
e infine
\[\boxed{\vec{a}(t) = -\omega^2 \cdot \vec{r}(t)}\]
in cui è evidente come il vettore acceerazione é sempre parallelo al vettore posizione, ma con verso opposto: il vettore posizione è sempre diretto verso l'esterno, mentre il vettore acccelerazione è diretto verso il centro della circonferenza.\\
Se ora si procede al calcolo del modulo di tale vettore si ottiene
\[\boxed{\left \vert \vec{a} \right \vert = \omega^2 \cdot R = \frac{v^2}{R}}\]
dal momento che si ha
\[\boxed{v = \omega R}\]
In realtà anche $\omega$ è un vettore, in cui la sua direzione è l'asse di rotazione, mentre il modulo fornirà una stima della velocità alla quale si muove; tale risultato avrà una importante validità in seguito.

\vspace{1em}
\subsection{Moti relativi}
Si consideri il caso di un moto composto da più moti: un sasso che viene lasciato cadere sullo scafo di una barca in movimento.\\
Naturalmente, considerando la caduta di un sasso, fissando un tempo $t$, si ha che
\[\vec{v}_{PB} = -g t \cdot \hat{j}\]
in quanto si tratta di una caduta libera. Questo, tuttavia, osservando il moto dalla barca ($PB$ = punto-barca) in movimento. Se, invece, tale moto viene visto da terra ($PT$ = punto-terra), sarà dotato di due componenti, ovvero
\[\vec{v}_{PT} = \vec{v} + -g t \cdot \hat{j}\]
in cui la componente verticale è la stessa del moto precedente, mentre la componente orizzontale dipende dalla velocità della barca. Questo è proprio quello che ha fatto Galileo: osservare questo tipo di situazioni nella vita reale, fornendovi una spiegazione scientifica e definendo, in questo caso, il concetto di sistema di riferimento e di moto relativo.

\subsubsection{Caso generale}
Per capire come passare da un sistema di riferimento all'altro, è necessario considerare un caso generale, in cui come sistema di riferimento si assume quello definito da:
\begin{itemize}
  \item posizione dell'origine;
  \item assi (posizione e orientamento).
\end{itemize}

\vspace{2em}
\noindent
\rowcolors{1}{white}{white}
\begin{tabularx}{\textwidth}{P}
  {
      \centering
      \begin{tikzpicture}
        \begin{axis}[
          axis lines = left,
          xlabel = \(x_A\),
          ylabel = {\(y_A\)},
          legend pos=outer north east,
          ymajorgrids=true,
          xmajorgrids=true,
          grid style=dashed,
          ymax=10,
          xmax=10,
          ytick={3},
          xtick={3},
        ]

        \addplot [
          domain=0:2,
          samples=100,
          color=green,
        ]
        {x};
        \draw [-stealth] (axis cs:3,3) -- (axis cs:3,10);
        \draw [-stealth] (axis cs:3,3) -- (axis cs:10,3);
        \draw [-stealth, very thick, red] (axis cs:0,0) -- (axis cs:9,5);
        \draw [red] (axis cs:4.5,1.5) node[]{$\vec{r}_{PA}$};
        \draw [-stealth, very thick, blue] (axis cs:3,3) -- (axis cs:9,5);
        \draw [blue] (axis cs:5,4.5) node[]{$\vec{r}_{PB}$};
        \draw [-stealth, very thick, orange] (axis cs:0,0) -- (axis cs:3,3);
        \draw [orange] (axis cs:1,2) node[]{$\vec{r}_{BA}$};
        \node[label={[rotate=90]center:$y_B$}] at (axis cs:1.5,6.5) {};
        \node[label={center:$x_B$}] at (axis cs:6.5,1.5) {};
        \end{axis}
    \end{tikzpicture}
  }
\end{tabularx}

\vspace{1em}
\noindent
Per passare da un sistema di riferimento $A$ all'altro $B$ è necessario definire la posizione relativa tra i due sistemi di riferimento $A-B$.\\
Infatti, definendo un nuovo vettore $\vec{r}_{BA}$ è possibile scrivere la somma di vettori seguente
\[\vec{r}_{PA} = \vec{r}_{PB} + \vec{r}_{BA}\]
Ma non solo, è possibile anche calcolare la derivata nel tempo e considerare, quindi, le velocità
\[\vec{v}_{PA} = \vec{v}_{PB} + \vec{v}_{BA}\]
che corrisponde proprio al caso analizzato per la barca: la differenza tra i due sistemi di osservazione è proprio il moto della barca. Un caso ancora più importante è quello che prevede $\vec{v}_{BA}$ \textbf{costante}, per cui i due sistemi non si muovono l'uno rispetto all'altro e misurare l'accelerazione nell'uno o nell'altro non cambia, in quanto sarà la stessa. Derivando nuovamente nel tempo, infatti, si ottiene
\[\vec{a}_{PA} = \vec{a}_{PB} + \vec{a}_{BA}\]
ma essendo $\vec{v}_{BA}$ costante, ovviamente $\vec{a}_{BA} = 0$. Questo è il caso di un \textbf{sistema di riferimento inerziale}, ovvero di un sistema di riferimento che può muoversi ad una certa velocità, ma non può accelerare. Naturalmente l'accelerazione del sistema ierziale non è relativa, non è da definirsi rispetto ad un altro sistema di riferimento come in questo caso, ma necessita di una definizione molto più rigorosa fornita tramite le leggi della dinamica di Newton. Questo ragionamento, naturamente, si applica sia ad un caso in 2D, ma anche in 3D.

\vspace{1em}
\noindent
\textbf{Osservazione}: Si osservi, ovviamente, che nel moto circolare uniforme velocità e accelerazione.\\
Inoltre, si ha che
\[\frac{d \left \vert \vec{v} \right \vert}{dt} = 0\]
significa che la variazione del modulo della velocità nel tempo è nullo: pertanto, se non c'è variazione del modulo della velocità, si ha che la componnte parallela dell'accelerazione è nulla, ovvero $\vec{a}_\parallel = 0$.

\newpage
\section{Dinamica}
Alla base della dinamica vi sono i $3$ principi della dinamica di Newton, formulati da Newton all'interno del libro \textbf{Philosophiae Naturalis Principia Mathematica}.\\
Esse sono le seguenti:

% Tabella per le definizione di concetti, etc...
\vspace{1em}
\rowcolors{1}{black!5}{black!5}
\setlength{\tabcolsep}{14pt}
\renewcommand{\arraystretch}{2}
\noindent
\begin{tabularx}{\textwidth}{@{}|P|@{}}
    \hline
    {\textbf{LEGGI DELLA DINAMICA}}\\
    \parbox{\linewidth}{Le leggi della dinamica sono le seguenti:
    \begin{enumerate}
      \item \textbf{Prima legge}: \emph{Ciascun corpo persevera nel proprio stato di quiete o di moto rettilineo uniforme, eccetto che sia costretto a mutare quello stato da forze impresse.}
      \item \textbf{Seconda legge}: \emph{Il cambiamento di moto è proporzionale alla forza mmotrice impressa, ed avviene lungo la linea retta secondo la quale la forza è stata impressa.}
      \item \textbf{Terza legge}: \emph{Ad ogni azione corrisponde una reazione uguale e contraria: ossia le azioni di due corpi sono sempre uguai fra loro e dirette verso parti opposte.}
    \end{enumerate}
    \vspace{1mm}}\\
    \hline
\end{tabularx}
\vspace{1em}

\subsection{Massa}
Per parlare della dinamica, si devono introdurre due concetti fondamentali ed interconnessi; prima di tutto si fornisce la definizione di \textbf{massa}, la quale può essere definita in modi diversi a seconda della necessità:

% Tabella per le definizione di concetti, etc...
\vspace{1em}
\rowcolors{1}{black!5}{black!5}
\setlength{\tabcolsep}{14pt}
\renewcommand{\arraystretch}{2}
\noindent
\begin{tabularx}{\textwidth}{@{}|P|@{}}
    \hline
    {\textbf{MASSA INERZIALE}}\\
    \parbox{\linewidth}{La \textbf{massa inerziale} (da \textbf{inerzia}: propensione a non muoversi) viene definita come misura della resistenza alle variazioni di velocità.
    \vspace{3mm}}\\
    \hline
\end{tabularx}
\vspace{1em}

\noindent
che si adatta perfettamente alla seconda legge della dinamica, la quale afferma che l'accelerazione è proporzionale alla forza impressa ed è la massa a rappresentare la \textbf{costante di proporzionalità}: a parità di forza, più il corpo è massivo meno accelera, meno è massivo, più accelera.\\
Di seguito, invece, si definisce il concetto di \textbf{massa gravitazionale}:

% Tabella per le definizione di concetti, etc...
\vspace{1em}
\rowcolors{1}{black!5}{black!5}
\setlength{\tabcolsep}{14pt}
\renewcommand{\arraystretch}{2}
\noindent
\begin{tabularx}{\textwidth}{@{}|P|@{}}
    \hline
    {\textbf{MASSA GRAVITAZIONALE}}\\
    \parbox{\linewidth}{La \textbf{massa gravitazionale} è proporzionale al \textbf{peso}.
    \vspace{3mm}}\\
    \hline
\end{tabularx}
\vspace{1em}

\noindent
Non da ultimo si fornisce una definizione di massa che è approssimabile ad una quantificazione:

% Tabella per le definizione di concetti, etc...
\vspace{1em}
\rowcolors{1}{black!5}{black!5}
\setlength{\tabcolsep}{14pt}
\renewcommand{\arraystretch}{2}
\noindent
\begin{tabularx}{\textwidth}{@{}|P|@{}}
    \hline
    {\textbf{MASSA}}\\
    \parbox{\linewidth}{La \textbf{massa} viene definita come \textbf{quantità di materia} e la sua unità di misura è
    \[[m] = \text{kg}\]
    Inoltre la massa è \textbf{additiva}: dato un corpo, agglomerato compatto di due masse $m_1$ e $m_2$, la massa complessiva è
    \[m = m_1 + m_2\]
    \vspace{-1mm}}\\
    \hline
\end{tabularx}
\vspace{1em}

\newpage
\noindent
\subsection{Forza}
Di seguito si espone il signifiato fisico di \textbf{forza}:

% Tabella per le definizione di concetti, etc...
\vspace{1em}
\rowcolors{1}{black!5}{black!5}
\setlength{\tabcolsep}{14pt}
\renewcommand{\arraystretch}{2}
\noindent
\begin{tabularx}{\textwidth}{@{}|P|@{}}
    \hline
    {\textbf{FORZA}}\\
    \parbox{\linewidth}{Una \textbf{forza} è una spinta che produce un cambiamento di moto di un corpo.\\
    La forza è un \textbf{vettore} (con modulo, direzione e verso) la cui unità di misura è
    \[[F] = \text{N} = \frac{\text{kg m}}{\text{s}^2}\]
    ove N sta per Newton.
    \vspace{3mm}}\\
    \hline
\end{tabularx}

\vspace{1em}
\subsection{Principi della dinamica - Leggi di Newton}
Si descrivano, ora, le leggi di Newton in termini dei due concetti esposti, ossia massa e forza:

\begin{enumerate}
  \item Se la forza risultante che agisce su un corpo è nulla, ovvero
  \[\sum \vec{F} = 0\]
  allora l'accelerazione del corpo è nulla, cioé $\vec{a}=0$, ovvero
  \[\boxed{\sum \vec{F} = 0 \longrightarrow \vec{a}=0}\]
  Tale legge potrebbe sembrare superflua, in quanto un caso particolare della seconda: tuttavia, tale legge assolve al compito di definire un \textbf{sistema di riferimento inerziale}.

  \item La forza risultante su un corpo è direttamente proporzionale all'accelerazione del corpo stesso. L'accelerazione di un corpo, quindi, è proporzionale alla forza risultante
  \[\boxed{\sum \vec{F} = m \vec{a}}\]

  \item La forza esercitata da un corpo $a$ su un corpo $b$ è uguale in modulo e direzione, ma ha verso opposto alla forza esercitata da $b$ su $a$.\\
  Ovvero si ha che
  \[\boxed{\vec{F}_{ab} = -\vec{F}_{ba}}\]
  e ciò è sempre vero.
\end{enumerate}

\vspace{1em}
\noindent
Dopo aver definito tali leggi, è necessario definire diversi tipi di forze, distinguendole in base alla loro tipolgia.

\vspace{1em}
\noindent
\subsection{Forza peso}
Di seguito si espone il significato fisico di \textbf{forza peso}:

% Tabella per le definizione di concetti, etc...
\vspace{1em}
\rowcolors{1}{black!5}{black!5}
\setlength{\tabcolsep}{14pt}
\renewcommand{\arraystretch}{2}
\noindent
\begin{tabularx}{\textwidth}{@{}|P|@{}}
    \hline
    {\textbf{FORZA PESO}}\\
    \parbox{\linewidth}{La forza peso fiene designata con $\vec{F}_t$, ovvero la forza di attrazione esercitata dalla terra su un corpo di massa $m$. In particolare si ha
    \[\boxed{\vec{F}_t = m \vec{g}}\]
    in cui $\vec{g} = -9.8 \text{ m}/\text{s}^2 \cdot \hat{j}$ e prende il nome di \textbf{accelerazione gravitazionale} (o meglio, di \textbf{campo gravitazionale} sulla superficie della terra).
    \vspace{3mm}}\\
    \hline
\end{tabularx}

\vspace{1em}
Di seguito si espone una illustrazione della forza peso:

\vspace{1em}
\begin{figure}[H]
  \centering
  \begin{tikzpicture}
    \draw [fill = purple!30,draw = purple!50] (0,0) rectangle ++(2,1.2);
    \draw [-stealth] (1,0.6) node[circ]{} -- ++(0,-2) node [midway, below right] {$\vec{F}_t = m\vec{g}$};
    \draw (0.5,0.6) node[]{$m$};
  \end{tikzpicture}
  \caption{Forza peso}
  \label{fig:forza_peso}
\end{figure}

\vspace{1em}
\noindent
\textbf{Osservazione}: Si osservi che la formula seguente
\[\vec{F}_t = m \vec{g}\]
potrebbe rassomigliare la formula
\[\vec{F}_t = m \vec{a}\]
Tuttavia, i due concetti sono ben distinti, in quanto $\vec{F}_t = m \vec{g}$ è un caso particolare della \textbf{legge di gravitazione universale}.

\vspace{1em}
\noindent
\textbf{Osservazione}: Si osservi che anche nel caso della forza peso è presente il terzo principio della dinamica: infatti un corpo viene attratto verso il centro della terra ed esercita una forza sulla superficie terrestre, così come la terra esercita una forza uguale e contraria (solamente che è impercettibile, è sempre presente).

\vspace{1em}
\subsection{Forza normale}
Di seguito si espone un altro tipo di forza, una forza i contatto, che prende il nome di \textbf{forza normale}:

% Tabella per le definizione di concetti, etc...
\vspace{1em}
\rowcolors{1}{black!5}{black!5}
\setlength{\tabcolsep}{14pt}
\renewcommand{\arraystretch}{2}
\noindent
\begin{tabularx}{\textwidth}{@{}|P|@{}}
    \hline
    {\textbf{FORZA NORMALE}}\\
    \parbox{\linewidth}{La \textbf{forza normale} $\vec{F}_N$ è un caso di forza di contatto, definita come \textbf{spinta fornita da una superficie (o da un altro corpo)}: quando un oggetto è appoggiato su una superficie e non si muove (ovvero si ha che $\vec{v}=0$ e $\vec{a}=0$), alla \textbf{forza peso} si contrappone la \textbf{forza normale}, uguale e contraria alla forza peso.
    \vspace{3mm}}\\
    \hline
\end{tabularx}

\vspace{1em}
\noindent
Di seguito si una illustrazione della forza normale:

\vspace{1em}
\begin{figure}[H]
  \centering
  \begin{tikzpicture}
    \draw [fill = purple!30,draw = purple!50] (0,0) rectangle ++(2,1.2);
    \draw (-1,0) -- (3,0);
    \foreach \i in {-11,-9,...,27} {
      \draw (\i / 10,-0.3) -- (\i / 10 + 0.3,0);
    }
    \draw [-stealth] (1.5,0) node[circ]{} -- ++(0,2) node [at end, right] {$\vec{F}_N$};
    \draw (1.5,0.2) -- ++(0.2,0) -- ++ (0,-0.2);
    \draw [-stealth] (1,0.6) node[circ]{} -- ++(0,-2) node [midway, below right] {$\vec{F}_t$};
    \draw (0.3,0.6) node[]{$m$};
    \draw (-1,1) node[]{$\vec{a}=0$};
  \end{tikzpicture}
  \caption{Forza normale}
  \label{fig:forza_normale}
\end{figure}

\vspace{1em}
\noindent
Una caratteristica fondamentale della forza normale è che essa è sempre \textbf{ortogonale alla superficie} su cui poggia l'oggetto. Se si osserva che il corpo presenta $\vec{a}=0$, significa che
\[\vec{F}_t + \vec{F}_N\ = 0 = m \vec{a} \longrightarrow \vec{F}_N = - \vec{F}_t\]
Mentre la forza peso presenta un modulo preciso, calcolabile tramite la legge di gravitazione universale, la forza normale, invece, adatta la propria intensità al corpo appoggiato sulla superficie: fintantoché la superficie resiste, il modulo della forza normale coincide con quello della forza peso; se il corpo è eccessivamente massiccio, la superficie si rompe.

\vspace{1em}
\noindent
\textbf{Osservazione}: Una forza presenta sempre un \textbf{punto di applicazione} che, graficamente, è rappresentata dalla \quotes{coda del vettore}:
\begin{enumerate}
  \item Nel caso della forza peso, il punto di applicazione è sempre dato del \textbf{centro di massa} del corpo stesso;
  \item Nel caso della forza normale, il punto di applicazione è la superficie di contatto (anche se vi sono molti punti di applicazione vista l'irregolarità della superficie stessa).
\end{enumerate}
Tuttavia è sempre possibile eseguire la somma di forze per conoscerne la risultante.

\vspace{1em}
\noindent
\textbf{Osservazione}: Quando si deve eseguire la rappresentazione grafica delle forze è necessario introdurre il concetto di \textbf{diagramma di corpo libero}:
\begin{itemize}
  \item ogni corpo è rappresentato da un \textbf{punto} (per cui il punto di applicazione delle forze sul corpo è proprio rappresentato dal punto stesso);
  \item comporta \textbf{solo le forze} che sono applicate sul corpo, e ciò diviene particolarmente utile quando bisogna considerare un sistema di più corpi interagenti.
\end{itemize}

\vspace{1em}
\noindent
\textbf{Esempio}: Si considerino due corpi poggiati uno sopra l'altro e stanti su una superficie fissa. Naturalmente su tali corpi agiscono due forze peso distinte. Inoltre la superficie di contatto tra i due corpi permette di indivduare due forze normali: uno del primo corpo sul secondo e una del secondo corpo sul primo. Infine vi è la forza di contatto dei due corpi con la superficie su cui poggiano, come mostrato di seguito:

\vspace{1em}
\begin{figure}[H]
  \centering
  \begin{tikzpicture}[scale=2]
    \draw [fill = purple!30,draw = purple!50] (0,0) rectangle ++(2,1.2);
    \draw [fill = red!30,draw = red!50] (-0.5,1.2) rectangle ++(3,1.2);
    \draw (-1,0) -- (3,0);
    \foreach \i in {-11,-9,...,27} {
      \draw (\i / 10,-0.3) -- (\i / 10 + 0.3,0);
    }
    \draw [-stealth] (1.2,0) node[circ]{} -- ++(0,2) node [at end, right] {$\vec{F}_N$};
    \draw (1.2,0.2) -- ++(0.2,0) -- ++ (0,-0.2);
    \draw [-stealth] (1,0.6) node[circ]{} -- ++(0,-2) node [midway, below right] {$\vec{F}_{t1}$};
    \draw [-stealth] (1.7,1.8) node[circ]{} -- ++(0,-2) node [midway, right] {$\vec{F}_{t2}$};
    \draw [-stealth] (0.5,1.2) node[circ]{} -- ++(0,-1) node [midway, right] {$\vec{F}_{N1}$};
    \draw [-stealth] (0.5,1.2) node[circ]{} -- ++(0,1) node [midway, right] {$\vec{F}_{N2}$};
    \draw (0.3,0.6) node[]{$m_1$};
    \draw (-0.1,1.8) node[]{$m_2$};
  \end{tikzpicture}
  \caption{Forza normale di due corpi a contatto}
  \label{fig:forza_normale_corpi_contatto}
\end{figure}

\vspace{1em}
\noindent
Dopo aver effettuato la raffigurazione, si procede alla realizzazione del \textbf{diagramma a corpo libero} di ciascuno dei due corpi, come mostrato di seguito:

\vspace{1em}
\begin{figure}[H]
  \centering
  \begin{tikzpicture}[scale=1]
    \draw [-stealth] (0,0) node[circ]{} -- ++(0,2) node [at end, right] {$\vec{F}_N$};
    \draw [-stealth] (0.1,0) -- ++(0,-1) node [midway, right] {$\vec{F}_{t1}$};
    \draw [-stealth] (-0.1,0) -- ++(0,-1) node [midway, left] {$\vec{F}_{N1}$};
  \end{tikzpicture}
  \hspace{2em}
  \begin{tikzpicture}[scale=1]
    \draw [-stealth] (0,0) node[circ]{} -- ++(0,1.5) node [at end, right] {$\vec{F}_{N2}$};
    \draw [-stealth] (0,0) -- ++(0,-1.5) node [midway, right] {$\vec{F}_{t2}$};
  \end{tikzpicture}
  \caption{Diagramma a corpo libero di due corpi a contatto}
  \label{fig:diagramma_corpo_libero_due_corpi_contatto}
\end{figure}

\vspace{1em}
\noindent
Naturalmente in questo caso si possono determinare direttamente le forze coinvolte
\begin{flalign*}
  \vec{F}_{t1} & = -g m_1 \cdot \hat{j} = m_1 \vec{g}\\
  \vec{F}_{t2} & = -g m_2 \cdot \hat{j} = m_2 \vec{g}\\
\end{flalign*}
Applicando la \textbf{$\boldsymbol{2^a}$ legge della dinamica} si perviene al risultato seguente
\begin{flalign*}
  m_1 \vec{a}_1 & = m_1 \vec{g} = \sum \vec{F} = \vec{F}_N + \vec{F}_{N1} + \vec{F}_{t1} = 0\\
  m_2 \vec{a}_2 & = m_2 \vec{g} = \sum \vec{F} = \vec{F}_{N2} + \vec{F}_{t2} = 0
\end{flalign*}
Questo, in quanto l'accelerazione è nulla, un dato noto dal problema. Applicando, ora, la \textbf{$\boldsymbol{3^a}$ legge della dinamica} si perviene al risultato seguente:
\[\vec{F}_{N1} = - \vec{F}_{N2}\]
Da ciò si può concludere il problema, andando a determinare
\begin{flalign*}
  \vec{F}_{N2} & = - \vec{F}_{t2} = g m_2 \cdot \hat{j}\\
  \vec{F}_{N1} & = - \vec{F}_{N2} = \vec{F}_{t2} = - g m_2 \cdot \hat{j}\\
  \vec{F}_{N} & = - \vec{F}_{N1} - \vec{F}_{t1} = g m_2 \cdot \hat{j} + g m_1 \cdot \hat{j} = g \cdot (m_1 + m_2) \cdot \hat{j}\\
\end{flalign*}

\newpage
\noindent
\begin{center}
  10 Marzo 2022
\end{center}
\subsection{Forza di tensione}
Di seguito si espone il significato fisico della \textbf{forza di tensione} che, nel suo comportamente, non è dissimile dalla forza normale:

% Tabella per le definizione di concetti, etc...
\vspace{1em}
\rowcolors{1}{black!5}{black!5}
\setlength{\tabcolsep}{14pt}
\renewcommand{\arraystretch}{2}
\noindent
\begin{tabularx}{\textwidth}{@{}|P|@{}}
    \hline
    {\textbf{FORZA DI TENSIONE}}\\
    \parbox{\linewidth}{La forza di tensione è la forza esercitata, per esempio, da un cavo o una fune, come mostrato di seguito, in cui la forza di tensione va a cancellare la forza peso del corpo appeso alla fune. È importante notare che $\vec{F}_T$ è sempre \textbf{parallela alla direzione della corda} stessa.\vspace{3mm}}\\
    \hline
\end{tabularx}

\vspace{1em}
\noindent
Di seguito si espone una illustrazione della forza di tensione:

\vspace{1em}
\begin{figure}[H]
  \centering
  \begin{tikzpicture}[scale=1]
    \draw [fill = purple!30,draw = purple!50] (0,-3) rectangle ++(2,1.2);
    \draw (-1,-0.3) -- (3,-0.3);
    \foreach \i in {-10,-8,...,26} {
      \draw (\i / 10,-0.3) -- (\i / 10 + 0.3,0);
    }
    \draw (1,-1.8) -- (1,-0.3);
    \draw [-stealth, red] (1,-1.8) -- node[midway, left, red]{$\vec{F}_T$} (1,-0.75);
    \draw [-stealth] (1,-2.4) node[circ]{} -- ++(0,-2) node [midway, below right] {$\vec{F}_{t}$};
    \draw (0.5,-2.4) node[]{$m$};
  \end{tikzpicture}
  \caption{Forza di tensione di un corpo sospeso}
  \label{fig:forza_tensione_corpo_sospeso}
\end{figure}

\vspace{1em}
\noindent
\textbf{Esempio}: Si consideri l'esempio seguente, in cui vi è un corpo che rimane sospeso nel vuoto da due funi che descrivono con il sofftto due angoli, rispettivamente $\theta_1$ e $\theta_2$. Si determinino le tensioni sulle due corde.

\vspace{1em}
\begin{figure}[H]
  \centering
  \begin{tikzpicture}[scale=1]
    \draw [fill = purple!30,draw = purple!50] (0,-3) rectangle ++(2,1.2);
    \draw (-1,-0.3) -- (3,-0.3);
    \foreach \i in {-10,-8,...,26} {
      \draw (\i / 10,-0.3) -- (\i / 10 + 0.3,0);
    }
    \draw (0.5,-1.8) -- coordinate[midway](a) node[midway, below left, red]{$\vec{F}_{T1}$} (-0.5,-0.3);
    \draw (1.5,-1.8) -- coordinate[midway](b) node[midway, below right, red]{$\vec{F}_{T2}$} (2.3,-0.3);
    \draw [-stealth, red] (0.5,-1.8) -- (a);
    \draw [-stealth, red] (1.5,-1.8) -- (b);
    \coordinate (O1) at (-0.5,-0.3);
    \coordinate (O2) at (2.3,-0.3);
    \draw [draw = orange] (O1) ++(.8,0) arc (0:-55:0.8)
    	node [pos=.4, left] {$\theta_1$};
    \draw [draw = violet] (O2) ++(-.8,0) arc (180:242:0.8)
    	node [pos=.4, left] {$\theta_2$};
    \draw [-stealth] (1,-2.4) node[circ]{} -- ++(0,-2) node [midway, below right] {$\vec{F}_{t}$};
    \draw (0.5,-2.4) node[]{$m$};
  \end{tikzpicture}
  \caption{Forza di tensione di un corpo sospeso da due corde}
  \label{fig:forza_tensione_corpo_sospeso_due_corde}
\end{figure}

\noindent
Dopo aver realizzato una figura illustrativa, bisogna sempre procedere alla raffigurazione del \textbf{diagramma a corpo libero}, come mostrato di seguito:

\vspace{1em}
\begin{figure}[H]
  \centering
  \begin{tikzpicture}[scale=1.5]
    \draw [-stealth, red] (0,0) node[circ]{} -- node[midway, below left, red]{$\vec{F}_{T1}$} (-0.5,0.75);
    \draw [-stealth, red] (0,0) -- node[midway, below right, red]{$\vec{F}_{T2}$} (0.4,0.75);
    \draw [-stealth] (0,0) -- node[midway, right]{$\vec{F}_{t1}$} (0,-1);
  \end{tikzpicture}
  \caption{Diagramma a corpo libero di un corpo sospeso da due corde}
  \label{fig:diagramma_corpo_libero_corpo_sospeso_due_corde}
\end{figure}

\vspace{1em}
\noindent
Applicando, ora, la \textbf{$\boldsymbol{2^a}$ legge della dinamica} si perviene al risultato seguente:
\[\sum \vec{F} = m \vec{a} = 0\]
essendo l'accelerazione nulla. Pertanto si ha che
\[\vec{F}_{T1} + \vec{F}_{T2} + \vec{F}_t = 0\]
Sarà ora sufficiente decomporre tale equazione vettoriale nelle sue due componenti ($x$ e $y$), come mostrato di seguito
\begin{flalign*}
    x & = - F_{T1} \cdot \cos(\theta_1) + F_{T2} \cdot \cos(\theta_2) + 0 = 0\\
    y & = F_{T1} \cdot \sin(\theta_1) + F_{T2} \cdot \sin(\theta_2) - mg = 0
\end{flalign*}
Dalla prima equazione si ha che
\[F_{T1} = F_{T2} \cdot \frac{\cos(\theta_2)}{\cos(\theta_1)}\]
che, sostituita nella seconda ecquazione, permette di ottenere
\[F_{T2} \cdot \frac{\cos(\theta_2)}{\cos(\theta_1)} \cdot \sin(\theta_1) + F_{T2} \cdot \sin(\theta_2)= mg\]
che può essere riscritto come segue
\[F_{T2} \cdot \cos(\theta_2) \cdot \left[ \frac{\sin(\theta_1)}{\cos(\theta_1)} + \frac{\sin(\theta_2)}{\cos(\theta_2)} \right] = mg\]
Pertanto si ha che
\[F_{T2} = \frac{mg}{\cos(\theta_2) \cdot \left(\tan(\theta_1) + \tan(\theta_2)\right)} \hspace{1em} \text{e} \hspace{1em} F_{T1} = \frac{mg}{\cos(\theta_1) \cdot \left(\tan(\theta_1) + \tan(\theta_2)\right)}\]
Se si cerca di capire che cosa accade quando gli angoli descriti dalla fune con il soffitto sono prossimi all'angolo limite di $90^\circ$, si rileva immediatamente una instabilità.\\
Intuitivamente si potrebbe pensare che una massa sospesa tramite due cavi come nella configurazione mostrata di seguito sia facilmente in equilibrio:

\vspace{1em}
\begin{figure}[H]
  \centering
  \begin{tikzpicture}[scale=1]
    \draw [fill = purple!30,draw = purple!50] (0,-3) rectangle ++(2,1.2);
    \draw (-1,-0.3) -- (3,-0.3);
    \foreach \i in {-10,-8,...,26} {
      \draw (\i / 10,-0.3) -- (\i / 10 + 0.3,0);
    }
    \draw (0.5,-1.8) -- coordinate[midway](a) node[midway, below left, red]{$\vec{F}_{T1}$} (0.5,-0.3);
    \draw (1.5,-1.8) -- coordinate[midway](b) node[midway, below right, red]{$\vec{F}_{T2}$} (1.5,-0.3);
    \draw (0.5,-1.5) -- ++(0.3,0) -- ++(0,-0.3) (0.5,-0.6) -- ++(-0.3,0) -- ++(0,0.3);
    \draw (1.5,-1.5) -- ++(-0.3,0) -- ++(0,-0.3) (1.5,-0.6) -- ++(0.3,0) -- ++(0,0.3);
    \draw [-stealth, red] (0.5,-1.8) -- (a);
    \draw [-stealth, red] (1.5,-1.8) -- (b);
    \draw [-stealth] (1,-2.4) node[circ]{} -- ++(0,-2) node [midway, below right] {$\vec{F}_{t}$};
    \draw (0.5,-2.4) node[]{$m$};
  \end{tikzpicture}
  \caption{Forza di tensione di un corpo sospeso da due corde parallele}
  \label{fig:forza_tensione_corpo_sospeso_due_corde_parallele}
\end{figure}

\vspace{1em}
\noindent
Tuttavia, è necessario considerare anche un altro aspetto nella determinazione dell'equilibrio: il \textbf{momento di forza}. Il caso precedentemente analizzato riguardava la configurazione per la quale il punto di applicazione della corda alla massa era identico e precisamente al centro della massa stessa:

\vspace{1em}
\begin{figure}[H]
  \centering
  \begin{tikzpicture}[scale=1]
    \draw [fill = purple!30,draw = purple!50] (0,-3) rectangle ++(2,1.2);
    \draw (-1,-0.3) -- (3,-0.3);
    \foreach \i in {-10,-8,...,26} {
      \draw (\i / 10,-0.3) -- (\i / 10 + 0.3,0);
    }
    \draw (1,-1.8) -- coordinate[midway](a) node[midway, below left, red]{$\vec{F}_{T1}$} (-0.5,-0.3);
    \draw (1,-1.8) -- coordinate[midway](b) node[midway, below right, red]{$\vec{F}_{T2}$} (2.3,-0.3);
    \draw [-stealth, red] (1,-1.8) -- (a);
    \draw [-stealth, red] (1,-1.8) -- (b);
    \coordinate (O1) at (-0.5,-0.3);
    \coordinate (O2) at (2.3,-0.3);
    \draw [draw = orange] (O1) ++(.8,0) arc (0:-45:0.8)
    	node [pos=.4, left] {$\theta_1$};
    \draw [draw = violet] (O2) ++(-.8,0) arc (180:230:0.8)
    	node [pos=.4, left] {$\theta_2$};
    \draw [-stealth] (1,-2.4) node[circ]{} -- ++(0,-2) node [midway, below right] {$\vec{F}_{t}$};
    \draw (0.5,-2.4) node[]{$m$};
  \end{tikzpicture}
  \caption{Forza di tensione di un corpo sospeso da due corde con stesso punti di applicazione}
  \label{fig:forza_tensione_corpo_sospeso_due_corde_stesso_punto_applicazione}
\end{figure}

\vspace{1em}
\noindent
Affinché in questa configurazione la massa stia in equilibrio, è necessario che precisamente la lunghezza di ciascun cavo sia identica; se differisse anche di poco, allora tutta la tensione graverebbe sul cavo più corto.

\vspace{2em}
\noindent
\textbf{Esempio}: Si consideri un quadricottero, ovverosia un drone con $4$ eliche e rotori, ciascuna capace do fornire una forza propulsiva verso l'alto di eguale modulo.\\
Considerando il drone stazionario si ha, per la \textbf{$\boldsymbol{2^a}$ legge della dinamica}, la seguente eguaglianza
\[4 \cdot \vec{F}_{s} + \vec{F}_t = 0\]
questo significa che ciascun rotore deve essere in grado di sviluppare una forza propulsiva verso l'alto pari a un quarto del peso del drone; quando, invece, si ha uno sbilanciamento delle forze dei rotori si ottiene un'inclinazione del drone nella direzione delle forze di minore intensità (quello che viene chiamato \textbf{momento di forza}).\\
Pertanto si può concludere che
\[\vec{F}_{s} = - \frac{1}{4} \cdot \vec{F}_t\]

\newpage
\noindent
\textbf{Esempio}: Si consideri una massa su un piano inclinato, come mostrato di seguito (considerando ininfluente l'attrito tra la massa e la superficie del piano):

\vspace{1em}
\noindent
\begin{figure}[H]
  \centering
  \newcommand{\ang}{30}

  \begin{tikzpicture} [font = \small, scale=1.5]

  % triangle:
  \draw [draw = orange, fill = orange!15] (0,0) coordinate (O) -- (\ang:6)
  	coordinate [pos=.45] (M) |- coordinate (B) (O);

  % angles:
  \draw [draw = orange] (O) ++(.8,0) arc (0:\ang:0.8)
  	node [pos=.4, left] {$\theta$};
  \draw [draw = orange] (B) rectangle ++(-0.3,0.3);

  \begin{scope} [-latex,rotate=\ang]

  % Object (rectangle)
  \draw [fill = purple!30,
  	draw = purple!50] (M) rectangle ++ (1,.6);

  % Weight Force and its projections
  \draw [dashed] (M) ++ (.5,.3) coordinate (MM) -- ++ (0,-1.29)
  	node [very near end, right] {$\vec{F}_t \cdot \cos{\theta}$};

  \draw [dashed] (MM) -- ++ (-0.75,0)
  	node [very near end, left] {$\vec{F}_t \cdot \sin{\theta}$};

  \draw (MM) -- ++ (-\ang-90:1.5)
  	node [very near end,below left ] {$\vec{F}_t$};

  % Normal Force
  \draw (MM) -- ++ (0,1.29)
  node [very near end, right] {$\vec{F}_N$};
  \end{scope}
  \end{tikzpicture}
  \caption{Piano inclinato}
  \label{fig:piano_inclinato}
\end{figure}

\vspace{1em}
\noindent
A cui segue il diagramma a corpo libero seguente:

\begin{figure}[H]
  \newcommand{\ang}{30}
  \vspace{-5em}
  \hspace{15em}
  \begin{tikzpicture} [font = \small, scale=1.5]
  % triangle:
  \draw (0,0) coordinate (O)  (\ang:6)
  	coordinate [pos=.45] (M) coordinate (B) (O);

  \begin{scope} [-latex,rotate=\ang]
  % Weight Force and its projections
  \draw [dashed] (M) ++ (.5,.3) coordinate (MM) -- ++ (0,-1.29)
  	node [very near end, right] {$\vec{F}_t \cdot \cos{\theta}$};

  \draw [dashed] (MM) -- ++ (-0.75,0)
  	node [very near end, left] {$\vec{F}_t \cdot \sin{\theta}$};

  \draw (MM) -- ++ (-\ang-90:1.5)
  	node [very near end,below left ] {$\vec{F}_t$};

  % Normal Force
  \draw (MM) -- ++ (0,1.29)
  node [very near end, right] {$\vec{F}_N$};
  \end{scope}
  \end{tikzpicture}
  \caption{Diagramma a corpo libero di un piano inclinato}
  \label{fig:diagramma_corpo_libero_piano_inclinato}
\end{figure}

\vspace{1em}
\noindent
Dopo aver disegnato anche il diagramma a corpo libero si può procedere a ragionare con la \textbf{$\boldsymbol{2^a}$ legge della dinamica}, ottenendo
\[m \cdot \vec{a} = \vec{F}_N + \vec{F}_t\]
e scomponendo tale equazione nelle sue componenti si ottiene
\begin{flalign*}
  m a_x & = - F_N \cdot \sin(\theta)\\
  m a_y & = F_N \cdot \cos(\theta) - mg
\end{flalign*}
Volendo conoscere l'accelerazione del corpo, si osserva che sussiste il seguente vincolo geometrico:
\[\frac{a_y}{a_x} = \tan(\theta) \longrightarrow a_y = \frac{\sin(\theta)}{\cos(\theta)} \cdot a_x\]
da cui
\[a_y \cdot \cos(\theta) = a_x \cdot \sin(\theta) \longrightarrow a_x = a_y \cdot \frac{\cos(\theta)}{\sin(\theta)}\]
Pertanto, procedendo dalla prima equazione si ottiene
\[F_N = -\frac{m a_x}{\sin(\theta)}\]
e quindi, nella seconda equazione si ottiene
\[m a_x \cdot \frac{\sin(\theta)}{\cos(\theta)} = -m a_x \cdot \frac{\cos(\theta)}{\sin(\theta)} - mg \longrightarrow a_x = - g \cdot \sin(\theta) \cdot \cos(\theta)\]
Sfruttando il vincolo geometrico precedente si ottiene anche che
\[a_y = -g \cdot \sin^2(\theta)\]
Avendo determinato ciò è possibile calcolare il modulo dell'accelerazione come segue
\[a = \sqrt{g^2 \cdot \sin^2(\theta) \cdot \cos^2(\theta) + g^2 \cdot \sin^4(\theta)} = g \cdot \sin(\theta) \cdot \sqrt{\cos^2(\theta) + \sin^2(\theta)} = g \cdot \sin(\theta)\]

\vspace{1em}
\noindent
\textbf{Esempio}: Si consideri il caso del piano inclinato precedente, procedendo, ora, alla rotazione del sistema di riferimento dell'angolo $\theta$, dimodoché lo spostamento avvenga lungo l'asse $x$ soltanto, e non vi sia, conseguentemente, acceerazione lungo l'asse $y$. In base a questo nuovo sistema di riferimento si ottiene la seguente decomposizione
\begin{flalign*}
  x & : m a_x = -m g \cdot \sin(\theta)\\
  y & : m a_y = -m g \cdot \cos(\theta) + F_N = 0
\end{flalign*}
per cui si ha che
\[a_x = -g \cdot \sin(\theta)\]

\vspace{1em}
\subsection{Forza di attrito}
Di seguito si espone la definizione generale di \textbf{forza di attrito}:

% Tabella per le definizione di concetti, etc...
\vspace{1em}
\rowcolors{1}{black!5}{black!5}
\setlength{\tabcolsep}{14pt}
\renewcommand{\arraystretch}{2}
\noindent
\begin{tabularx}{\textwidth}{@{}|P|@{}}
    \hline
    {\textbf{FORZA DI ATTRITO}}\\
    \parbox{\linewidth}{La \textbf{forza di attrito} è una forza di contatto, esattamente come la forza normale: quest'ultima è sempre perpendicolare alla superficie e il suo modulo è tale che vincola il moto, al fine di contrastare la forza nell'altra direzione.\\
    La \textbf{forza di attrito}, come la forza normale, è una forza che presenta come punto di applicazione la superficie di contatto con il corpo. Inoltre, la forza di attrito
    \begin{itemize}
      \item è sempre \textbf{parallela alla superficie};
      \item è di \textbf{modulo proporzionale} a $\left \vert F_N \right \vert$.
    \end{itemize}
    Quest'ultima osservazione non è ovvia, in quanto bisogna osservare il comportamento delle particelle a livello microscopico.\vspace{3mm}}\\
    \hline
\end{tabularx}

\vspace{1em}
\noindent
Si consideri l'illustrazione seguente, in cui si espongono le forze di attrito agenti:

\vspace{1em}
\begin{figure}[H]
  \centering
  \begin{tikzpicture}[scale=1]
    \tikzset{rotate=45} {
      \draw [fill = purple!30,draw = purple!50] (0,0) rectangle ++(2,1.2);
      \draw (-1,0) -- (3,0);
      \foreach \i in {-10,-8,...,26} {
        \draw (\i / 10,-0.3) -- (\i / 10 + 0.3,0);
      }
    }
    \draw (1,0.3) -- ++(0.3,0) -- ++(0,-0.3);
    \draw [-stealth] (1,0) -- node[midway, left]{$\vec{F}_N$} (1,3) ;
    \draw [-stealth] (1,0.6) node[circ]{} -- ++(-2,-2) node [midway, below right] {$\vec{F}_{t}$};
    \draw (1.5,0.6) node[]{$m$};
    \draw [-stealth, red] (2,0) node[circ]{} -- ++(2,0) node [midway, above left] {$\vec{F}_{s/k}$};
  \end{tikzpicture}
  \caption{Forza di attrito}
  \label{fig:forza_attrito}
\end{figure}

\vspace{1em}
\noindent
La forza di attrito si distingue in due diverse tipologie:
\begin{enumerate}
  \item L'\textbf{attrito cinetico}, si ha quando il moto relativo tra le superfici in contatto non è nullo, ovvero si ha movimento con $v \neq 0$.\\
  Si ha che
  \[\boxed{F_k = \mu_k \cdot F_N}\]
  in cui $\boldsymbol{\mu_k}$ prende il nome di \textbf{coefficiente di attrito cinetico} (dall'inglese $k$, di \quotes{kinetic}), il quale, per ovvie ragioni, è \textbf{adimensionale}.

  \item L'\textbf{attrito statico}, si ha quando la velocità relativa tra le superfici di contatto è nulla, ovvero non si ha movimento, $v = 0$.\\
  Si ha che
  \[\boxed{F_s \leq \mu_s \cdot F_N}\]
  ovvero il modulo $F_s$ aumenta affinché la risultante delle forze interagenti (e quindi la risultante) sia $0$. Ovviamente, quando la forza che agisce aumenta il proprio modulo fino a superare la forza di attrito statico, il corpo inizia a muoversi e l'attrito si tramuta, a seguito del moto, in attrito cinetico, come si vede di seguito:

  \vspace{2em}
  \noindent
  \rowcolors{1}{white}{white}
  \begin{tabularx}{\textwidth}{P}
    {
        \centering
        \begin{tikzpicture}
          \begin{axis}[
            axis lines = left,
            xlabel = \(F_a\),
            ylabel = {\(F_{a,k}\)},
            legend pos=outer north east,
            ymax=4,
            xmax=10,
            xtick={3},
            xticklabels={$\mu_s \cdot F_N$},
          ]

          \addplot [
            domain=0:3,
            samples=100,
            color=black,
          ]
          {x};
          \draw (axis cs:3,2) -- (axis cs:10,2);
          \draw [dashed] (axis cs:3,0) -- (axis cs:3,3);
          \end{axis}
      \end{tikzpicture}
    }
  \end{tabularx}

  \vspace{1em}
  \noindent
  Il coefficiente moltiplicativo $\boldsymbol{\mu_s}$ prende il nome di \textbf{coefficiente di attrito statico} e generalmente si ha che
  \[\boxed{\mu_k < \mu_s}\]
\end{enumerate}

\newpage
\noindent
\begin{center}
  14 Marzo 2022
\end{center}
Si osservi che quando una massa rimane sospesa tramite due cavi che formano con la superficie di collegamento un angolo di $90^\circ$ ciascuno, non entrano in gioco solamente le forze, ma risulta fondamentale anche conoscere il concetto di \textbf{momento di forza} e di \textbf{punto di applicazione}: se il punto di applicazione delle forze di tensione è lo stesso sul corpo sospeso, basta una leggerissima differenza di lunghezza dei cavi per avere uno sbilanciamento significativo delle forze di tensione.

\vspace{1em}
\noindent
\textbf{Osservazione}: L'attrito statico è come se non fosse un attrito, in quanto non si ha movimento: è come una forza normale che si oppone al tentativo di spostamento, variando il proprio modulo in modo tale da cancellare la forza applicata.\\
L'attrito cinetico, invece, è un vero e proprio attrito che si oppone al moto tramite la dissipazione di energia in calore.

\vspace{1em}
\noindent
\textbf{Esempio}: Si consideri una vettura che, in movimento, procede a frenare e a rallentare fino a fermarsi: il fenomeno che si sta studiando è l'attrito tra le ruote e la strada. in particolare la vettura richiede $70$ m per fermarsi, partendo da una velocità di $100$ km/h (a causa di uno slittamento delle gomme sull'asfalto).\\
A partire da questi dati, si determini il \textbf{coefficiente di attrito cinetico}; si proceda alla realizzazione di un modello grafico del problema:

\vspace{1em}
\begin{figure}[H]
  \centering
  \begin{tikzpicture}[scale=1]
    \draw (0,0) -- ++(10,0);
    \foreach \i in {0,2,...,96} {
      \draw (\i / 10,-0.3) -- (\i / 10 + 0.3,0);
    }
    \node[minimum size=0.5cm,circle,draw] (circle) at (0.5,0.25){};
    \node[minimum size=0.5cm,circle,draw] (circle) at (1.5,0.25){};
    \draw (0.25,0.25) -- ++(-0.3,0) -- ++(0,0.5) -- ++(0.5,0) -- ++(0.3,0.3) -- ++(0.8,0) -- ++(0.3,-0.3) -- ++(0.5,0) -- ++(0,-0.5) -- ++(-0.6,0);

    \node[minimum size=0.5cm,circle,draw] (circle) at (7.5,0.25){};
    \node[minimum size=0.5cm,circle,draw] (circle) at (8.5,0.25){};
    \draw (7.25,0.25) -- ++(-0.3,0) -- ++(0,0.5) -- ++(0.5,0) -- ++(0.3,0.3) -- ++(0.8,0) -- ++(0.3,-0.3) -- ++(0.5,0) -- ++(0,-0.5) -- ++(-0.6,0);

    \draw (1,-0.5) -- ++(0,0.2) ++(0,-0.2) coordinate[midway](mid) -- ++(7,0) -- ++(0,0.2);
    \draw (5,-0.75) node[below]{$d=70$ m};

    \draw [-stealth] (3,2) -- node[midway, above]{$\vec{v}_0$} (4,2) ;
    \draw [-stealth] (0.85,0) -- node[midway, left]{$\vec{F}_N$} (0.85,3) ;
    \draw [-stealth] (1.15,0.65) node[circ]{} -- ++(0,-3) node [midway, below right] {$\vec{F}_{t}$};
    \draw [-stealth, red] (0.5,0) node[circ]{} -- ++(-2,0) node [midway, above left] {$\vec{F}_{k}$};
  \end{tikzpicture}
  \caption{Vettura in decelerazione su una strada}
  \label{fig:vettura_decelerazione_strada}
\end{figure}

\vspace{1em}
\noindent
È molto importante osservare che la forza normale è essenziale per il calcolo della forza di attrito cinetico, in quanto da essa dipende il suo modulo. Si realizzi, ora, il diagramma a corpo libero:

\vspace{1em}
\begin{figure}[H]
  \centering
  \begin{tikzpicture}[scale=1]
    \draw [-stealth] (0,0) node[circ]{} -- ++(0,1) node [at end, right] {$\vec{F}_N$};
    \draw [-stealth] (0,0) -- ++(0,-1) node [midway, right] {$\vec{F}_{t}$};
    \draw [-stealth] (0,0) -- ++(-1,0) node [midway, above] {$\vec{F}_{k}$};
  \end{tikzpicture}
  \caption{Diagramma a corpo libero di una vettura in rallentamento}
  \label{fig:diagramma_corpo_libero_vettura_rallentamento}
\end{figure}

\vspace{1em}
\noindent
Per procedere si richiami la $2^a$ legge della dinamica e si scriva
\[\sum \vec{F} = m \vec{a} = m \cdot \left(a_x \cdot \hat{i} + 0 \cdot \hat{j} \right)\]
È noto, inoltre, che
\[\vec{F}_t + \vec{F}_N = 0 \longrightarrow \vec{F}_t = - \vec{F}_N\]
Per cui si ottiene che
\[\vec{F}_k = m a_x \cdot \hat{i} \longrightarrow a_x = -\frac{F_k}{m}\]
in cui, per convenzione, si pone la componente orizziontale negativa.\\
Inoltre è possibile calcolare anche l'accelerazione con cui la macchina rallenta, impiegando la seguente formula del moto uniformemente accelerato:
\[v^2 - v_0^2 = 2a \cdot (x - x_0)\]
che è possibile applicare al contesto in quanto si parla di moto uniformemente accelerato, giacché l'accelerazione è causata dalla forza di attrito, che è costante in quanto prodotto tra un coefficiente e la forza normale, la quale è costante in quanto si oppone al peso che è costante.\\
Da tale formula si ottiene che
\[a = \frac{1}{2} \cdot \frac{v^2-v_0^2}{x-x_0} = -\frac{1}{2} \cdot \frac{v_0^2}{2d}\]
Da ciò segue che, ovviamente
\[F_k = \mu_k F_N\]
in cui ovviamente
\[F_N = F_t = mg\]
per cui si evince che
\[F_k = \mu_k \cdot m g \longrightarrow a_x = - \frac{F_k}{m} = -\mu_k g\]
Pertanto, dalla uguaglianza appena determinata
\[a_x = -\mu_k g = -\frac{1}{2} \cdot \frac{v_0^2}{2d}\]
si può ricavare
\[\mu_k = \frac{v_0^2}{2 g d} = 0.56\]

\vspace{2em}
\noindent
\textbf{Osservazione}: Si osservi che la $2^a$ legge di Newton è applicabile in un sistema di riferimento inerziale, ovvero in un sistema di riferimento che non ha propensione a muoversi.\\
Il tempo di volo di un proiettile è
\[\boxed{t = \frac{2 v_0 \cdot \sin(\theta)}{g}}\]

\vspace{2em}
\noindent
\textbf{Esempio}: Si consideri un piano su cui poggiano tre masse, l'una collegata all'altra, come mostrato di seguito:

\vspace{1em}
\begin{figure}[H]
  \centering
  \begin{tikzpicture}[scale=1]
    \draw [fill = purple!30,draw = purple!50] (0,0) rectangle ++(2,1.2);
    \draw [fill = orange!30,draw = orange!50] (3,0) rectangle ++(2,1.2);
    \draw [fill = green!30,draw = green!50] (6,0) rectangle ++(2,1.2);
    \draw (-1,0) -- (9,0);
    \foreach \i in {-10,-8,...,86} {
      \draw (\i / 10,-0.3) -- (\i / 10 + 0.3,0);
    }
    \draw (2,0.6) -- ++(1,0) (5,0.6) -- ++(1,0);
    \draw [-stealth] (8,0.6) -- node[midway, above]{$\vec{F}$} ++(2,0);

    \draw (0.5,0.6) node[]{$m_A$};
    \draw [-stealth] (0.85,0) -- node[near end, left]{$\vec{F}_{NA}$} (0.85,3);
    \draw (0.85,0.3) -- ++(-0.3,0) -- ++(0,-0.3);
    \draw [-stealth] (1.15,0.6) node[circ]{} -- ++(0,-2) node [midway, below right] {$\vec{F}_{tA}$};

    \draw (3.5,0.6) node[]{$m_B$};
    \draw [-stealth] (3.85,0) -- node[near end, left]{$\vec{F}_{NB}$} (3.85,3);
    \draw (3.85,0.3) -- ++(-0.3,0) -- ++(0,-0.3);
    \draw [-stealth] (4.15,0.6) node[circ]{} -- ++(0,-2) node [midway, below right] {$\vec{F}_{tB}$};

    \draw (6.5,0.6) node[]{$m_C$};
    \draw [-stealth] (6.85,0) -- node[near end, left]{$\vec{F}_{NC}$} (6.85,3);
    \draw (6.85,0.3) -- ++(-0.3,0) -- ++(0,-0.3);
    \draw [-stealth] (7.15,0.6) node[circ]{} -- ++(0,-2) node [midway, below right] {$\vec{F}_{tC}$};
  \end{tikzpicture}
  \caption{Tre masse trainate}
  \label{fig:tre_masse_trainate}
\end{figure}

\noindent
Tali masse vengono trainate con una forza $F = 200$ N, mentre le massse sono $m_A = 30$ kg, $m_B = 50$ kg e $m_C = 20$ kg; inoltre è noto che il coefficiente di attrito cinetico è $\mu_k = 0.1$. Si determini, allora
\begin{itemize}
  \item l'accelerazione dell'intero sistema;
  \item la tensione delle corde $A-B$ e $B-C$.
\end{itemize}
Si realizzi il diagramma a corpo libero del sistema oggetto di studio

\vspace{1em}
\begin{figure}[H]
  \centering
  \begin{tikzpicture}[scale=1.5]
    \draw [-stealth] (0,0) node[circ]{} -- ++(0,1) node [at end, right] {$\vec{F}_{NA}$};
    \draw [-stealth] (0,0) -- ++(0,-1) node [midway, right] {$\vec{F}_{tA}$};
    \draw [-stealth] (0,0) -- ++(-1,0) node [midway, above] {$\vec{F}_{kA}$};
    \draw [-stealth] (0,0) -- ++(1,0) node [midway, above] {$\vec{F}_{TA}$};
  \end{tikzpicture}
  \hspace{2em}
  \begin{tikzpicture}[scale=1.5]
    \draw [-stealth] (0,0) node[circ]{} -- ++(0,1) node [at end, right] {$\vec{F}_{NB}$};
    \draw [-stealth] (0,0) -- ++(0,-1) node [midway, right] {$\vec{F}_{tB}$};
    \draw [-stealth] (0,0.1) -- ++(-1,0) node [midway, above] {$\vec{F}_{kB}$};
    \draw [-stealth] (0,-0.1) -- ++(-1,0) node [midway, below] {$\vec{F}_{AB}$};
    \draw [-stealth] (0,0) -- ++(1,0) node [midway, above] {$\vec{F}_{TB}$};
  \end{tikzpicture}
  \hspace{2em}
  \begin{tikzpicture}[scale=1.5]
    \draw [-stealth] (0,0) node[circ]{} -- ++(0,1) node [at end, right] {$\vec{F}_{NC}$};
    \draw [-stealth] (0,0) -- ++(0,-1) node [midway, right] {$\vec{F}_{tC}$};
    \draw [-stealth] (0,0.1) -- ++(-1,0) node [midway, above] {$\vec{F}_{kC}$};
    \draw [-stealth] (0,-0.1) -- ++(-1,0) node [midway, below] {$\vec{F}_{BC}$};
    \draw [-stealth] (0,0) -- ++(1,0) node [midway, above] {$\vec{F}$};
  \end{tikzpicture}
  \caption{Diagramma a corpo libero di $3$ masse trainate}
  \label{fig:diagramma_corpo_libero_3_masse_trainate}
\end{figure}

\vspace{1em}
\noindent
Per la risoluzione del primo quesito, si può sfruttare la \textbf{proprietà additiva} della massa, essendo i tre corpi omogenei e costituiti dalla stessa sostanza. Pertanto l'assieme $ABC$ si comporta come un unico corpo, la cui massa complessiva è
\[m = m_A + m_B + m_C = 30 \text{ kg} + 50 \text{ kg} + 20 \text{ kg} = 100 \text{ kg}\]
Pertanto si può realizzare un nuovo diagramam a corpo libero, mostrato di seguito:

\vspace{1em}
\begin{figure}[H]
  \centering
  \begin{tikzpicture}[scale=1.5]
    \draw [-stealth] (0,0) node[circ]{} -- ++(0,1) node [at end, right] {$\vec{F}_{N}$};
    \draw [-stealth] (0,0) -- ++(0,-1) node [midway, right] {$\vec{F}_{t}$};
    \draw [-stealth] (0,0) -- ++(-1,0) node [midway, above] {$\vec{F}_{k}$};
    \draw [-stealth] (0,0) -- ++(1,0) node [midway, above] {$\vec{F}$};
  \end{tikzpicture}
  \caption{Diagramma a corpo libero di un'unica massa}
  \label{fig:diagramma_corpo_libero_unica_massa}
\end{figure}

\vspace{1em}
\noindent
Pertanto si ottiene che, per la $2^a$ legge della dinamica:
\[\sum \vec{F} = m \vec{a}\]
Naturalmente le forze possono essere scomposte nei loro rispettivi componenti, per cui si ottiene
\[
  y : \left\{
  \rowcolors{1}{white}{white}
  \begin{array}{l}
    \vec{F}_N = - \vec{F}_t\\
    F_N = mg
  \end{array}
  \right.
\]
mentre si ottiene che
\[
  x : \left\{
  \rowcolors{1}{white}{white}
  \begin{array}{l}
    m a_x = F - F_k\\
    F_k = \mu_k \cdot F_N
  \end{array}
  \right.
\]
da cui
\[a_x = \frac{F}{m} - \mu_k \cdot g = 2.0 \text{ m/s}^2 - 0.98 \text{ m/s}^2 = 1.0 \text{ m/s}^2\]
Per la risoluzione del secondo quesito, si determini dapprima la tensione della corda $A-B$, ovvero $\vec{F}_{AB}$ che, per la $3^a$ legge di Newton è uguale, in modulo, alla forza $\vec{F}_{TA}$.\\
Dalla $2^a$ legge della dinamica applicata al corpo $A$ si ottiene
\[\vec{F}_{NA} + \vec{F}_{tA} + \vec{F}_{AB} + \vec{F}_{kA} = m_A \cdot \vec{a}_A\]
Naturalmente si ha che, scomponedo tale equazione nelle sue componenti $x$ e $y$ si ottiene
\[
  \left\{
  \rowcolors{1}{white}{white}
  \begin{array}{l}
    F_{AB x} - \mu_k m_A g = m_A a_{A x}\\
    F_N + F_t = m_A a_{A y} = 0 \longrightarrow F_N = - F_t
  \end{array}
  \right.
\]
Si può procedere, ora, al calcolo di $F_{BA x}$ come segue (ricordando che l'accelerazione del sistema è la stessa di ciascuna massa, ovviamente):
\[F_{BA x} = m_A \cdot (a_{Ax} + \mu_k g) = 30 \text{ kg} \cdot \left[1 \text{ m/s}^2 + 0.1 \cdot 9.8 \text{ m/s}^2\right] = 60 \text{ N}\]
Da notare che tale formula poteva anche essere riscritta come segue
\[F_{BA} = m_A \cdot \left(\frac{F}{m} - \mu_k \cdot g + \mu_k \cdot g\right) = \frac{m_A}{m_A + m_B + m_C} \cdot F = 60 \text{ N}\]
Per la determinazione della tensione $B-C$, è sufficiente procedere come già fatto, impiegando la seconda legge della dinamica sul corpo $B$ (o anche sul corpo $C$, visto che sono noti tutti i dati del problema). Si applichi, allora, la $2^a$ legge della dinamica sul corpo $C$, ottenendo
\[\sum \vec{F} = m \cdot \vec{a} \longrightarrow \vec{F} + \vec{F}_{kc} + \vec{F}_{NC} + \vec{F}_{tC} + \vec{F}_{BC} = m_C \cdot \vec{a}_C\]
Naturalmente è possibile scomporre tale equazione nelle sue componenti $x$ e $y$, ottenendo
\[y : F_{NC} - F_{tC} = 0 \longrightarrow F_{NC} = F_{tC} = m_C g\]
\[x : F - F_{kC} - F_{BC} = m_C a_{xC} \longrightarrow F - \mu_k m_C g - F_{BC} = m_C \cdot a_{xC}\]
Da cui si evince che
\[F_{BC} = F - \mu_k g m_C - m_C \cdot a_{xC} = 200 \text{ N} - 20 \text{ N} - 20 \text{ N} = 160 \text{ N}\]

\newpage
\noindent
\begin{center}
  15 Marzo 2022
\end{center}
Quando bisogna risolvere un problema di meccanica, è essenziale visualizzare il problema tramite una rappresentazione grafica; dopodiché è fondamentale procedere alla realizzazione del diagramma a corpo libero del sistema.\\
Solamente a questo punto è possibile procedere alla stesura delle equazioni dinamiche che consentono di isolare le richieste del problema.

\vspace{1em}
\subsection{Attrito dovuto a un fluido (resistenza)}
Un'altra importante forza di attrito è la resistenza di un fluido: l'aria, per esempio, rallenta il moto di un corpo, come accade quando si lascia cadere dalla stessa altezza un martello e una piuma (sulla terra, dove c'è aria, il martello cade prima della piuma, mentre sulla Luna, in assenza di atmosfera, i due corpi cadono alla medesima velocià, impiegando lo stesso tempo).

% Tabella per le definizione di concetti, etc...
\vspace{1em}
\rowcolors{1}{black!5}{black!5}
\setlength{\tabcolsep}{14pt}
\renewcommand{\arraystretch}{2}
\noindent
\begin{tabularx}{\textwidth}{@{}|P|@{}}
    \hline
    {\textbf{ATTRITO DOVUTO A UN FLUIDO (RESISTENZA)}}\\
    \parbox{\linewidth}{La resistenza di un fluido è la forza di attrito che il fluido produce sul corpo in movimento, rallentandone la corsa. La causa di tale resistenza è dovuta alla \textbf{viscosità del fluido} che, naturalmente, esercita una \textbf{forza opposta} $\vec{F}_v$ al moto del fluido (naturalmente le proprietà del fluido sono determinanti).\\
    Il problema della forza esercitata da un fluido viscoso su un corpo in esso immerso è complesso; tuttavia, tale fenomeno vine modellizzato, in maniera approssimativa, attraverso due diversi schemi teorici:
    \begin{enumerate}
      \item George Stokes, nel $1845$, prese in considerazione il problema solo per un caso particolare, quello di un \textbf{oggetto di forma sferica}, \textbf{completamente immerso} in un fluido in \textbf{moto laminare}, di \textbf{densità costante} ed \textbf{incomprimibile}.\\
      Perrtanto, a velocità bassa con densità bassa, si ha un \textbf{flusso laminare} (ovvero si è in \textbf{assenza di turbolenza}) e per determinare la forza resistiva si impiega, appunto, la \textbf{legge di Stokes}
      \[\boxed{\vec{F} = -b \cdot \vec{v}}\]
      in cui $b$ prende il nome di \textbf{coefficiente di viscosità}, che dipende dalle proprietà del fluido ed è un coefficiente di proporzionalità diretta: più aumenta la velocità, maggiore sarà l'intensità della forza d'attrito. Tale forza è molto diversa rispetto alle forze precedentemente analizzate, in quanto dipende strettamente dalla tipologia di moto del corpo considerato. Naturalmente, in questo caso, l'equazione è molto semplice, ma difficilmente applicabile, in quanto nell'equazione di Newton si ha un'accelerazione, mentre la legge di Stokes fornisce una velocità: si tratta, quindi, di risolvere un'\textbf{equazione differenziale}.

      \item Se il \textbf{moto} invece è \textbf{turbolento} le forze inerziali dominano su quelle viscose e la forza di resistenza dipende da diversi fattori:
      \begin{itemize}
        \item \textbf{densità} $\rho$ del fluido;
        \item l'\textbf{area di proiezione} del fluido $A$, ovvero una stima di quando fluido viene spostato nel moto: un corpo lungo e fino non subirà molta resistenza, mentre un corpo grande e largo avrà una resistenza maggiore;
        \item il \textbf{coefficiente di caratterizzazione della forma del corpo} $C_d$ (in cui $D$ sta per \quotes{Drag}), determinato sperimentalmente e, per ovvie ragioni, \textbf{adimensionale}.
      \end{itemize}
      Considerando tali fattori, la forza di resistenza è
      \[\boxed{F_v = \frac{1}{2} \rho A C_d v^2}\]
    \end{enumerate}
    \vspace{1mm}}\\
    \hline
\end{tabularx}

\vspace{1em}
\noindent
\textbf{Esempio}: Si consideri un quadricottero che, inclinandosi orizzontalmente di un angolo $\theta$, produce uno spostamento di velocità $v$ che, naturalmente, viene rallentato dall'aria che oppone una forza di resistenza.\\
Si supponga, per ipotesi, che tale drone esegua un moto che è possibile modellizzare tramite la legge di Stokes, ovvero
\[\vec{F}_v = -b \cdot \vec{v}\]
Si determini, allora, l'angolo $\theta$ tale per cui il drone si muove ad una \textbf{velocità costante} $\vec{v}$. Si realizzi, allora, il diagramma a corpo libero seguente:

\vspace{1em}
\begin{figure}[H]
  \centering
  \begin{tikzpicture}[scale=1.5]
    \draw [-stealth] (0,0) node[circ]{} -- ++(0.5,1) node [at end, right] {$\vec{F}_s$};
    \draw [dotted] (0,0) -- ++(0,1);
    \coordinate (f) at (0,0.8);
    \coordinate (ctr) at (0,0);
    \coordinate (i) at (0.5,1);
    \pic [draw=red, text=red, <->, "$\theta$", angle radius=1cm, angle eccentricity=1.4] {angle = i--ctr--f};
    \draw [-stealth] (0,0) -- ++(0,-1) node [midway, right] {$\vec{F}_{t}$};
    \draw [-stealth] (0,0) -- ++(-1,0) node [midway, above] {$\vec{F}_{v}$};
  \end{tikzpicture}
  \caption{Diagramma a corpo libero di un drone inclinato}
  \label{fig:diagramma_corpo_libero_drone_inclina}
\end{figure}

\vspace{1em}
\noindent
Naturalmente, in questo caso, è possibile applicare la $2^a$ legge della dinamica, in quanto è noto che il drone si muove a velocità costante, quindi l'accelerazione è nulla e quindi la somma delle forze risultanti è nulla.\\
Pertanto si ha che
\[\vec{F}_t + \vec{F}_v + \vec{F}_s = m \cdot \vec{a} = 0\]
Naturalmente è possibile scomporre tale equazione vettoriale nelle sue componenti $x$ e $y$, ottenendo
\[x: F_s \cdot \sin(\theta) - b v = 0\]
\[y: F_s \cdot \cos(\theta) - m g = 0\]
Dalla seconda equazione si ottiene che
\[F_s = \frac{m g}{\cos(\theta)}\]
mentre dalla prima equazione si ha che
\[v = \frac{mg}{b} \cdot \tan(\theta)\]

\vspace{1em}
\noindent
\textbf{Esempio}: Si calcoli la velocità limite di un paracadutista che scente verticalmente, di cui si propone di seguito il diagramma a corpo libero:

\vspace{1em}
\begin{figure}[H]
  \centering
  \begin{tikzpicture}[scale=1]
    \draw [-stealth] (0,0) node[circ]{} -- ++(0,1) node [midway, right] {$\vec{F}_v$};
    \draw [-stealth] (0,0) -- ++(0,-1) node [midway, right] {$\vec{F}_{t}$};
  \end{tikzpicture}
  \caption{Diagramma a corpo libero di un drone inclinato}
  \label{fig:diagramma_corpo_libero_drone_inclinato}
\end{figure}

\vspace{1em}
\noindent
Infatti, ad un certo punto, la forza peso del corpo in caduta libera verrà eguagliata dalla forza di resistenza dell'aria e il paracadutista non aumenterà più la sua velocità, ma la manterrà costante.\\
Naturalmente, in questo caso, è necessario applicare la seconda formula della resistenza, semplicemente osservando che
\[mg = F_v = \frac{1}{2} \cdot \rho A C_d v^2\]
Per isolare la velocità limite, semplicemente si può scrivere
\[v = \sqrt{\frac{2 m g}{\rho A C_d}}\]
Naturalmente, se si suppone che
\begin{itemize}
  \item $m = 70$ kg
  \item $\rho = 1.2$ kg/m$^3$
  \item $C_d = 0.8$
  \item $A = 0.5$ m$^2$
\end{itemize}
Considerando tali dati per il problema si ottiene che
\[v = 55 \text{ m/s}\]
e tale risultato è totalmente ininflente dall'altezza dalla quale ci si paracaduta, in quanto la velocità limite che è possibile raggiungere è $v = 55 \text{ m/s}$.

\vspace{1em}
\subsection{Dinamica del moto circolare uniforme}
Naturalmente, è noto che nel moto circolare uniforme la velocità $v$ è costante, per cui l'accelerazione è soltanto centripeta e diretta verso il centro della circonferenza, ortogonalmente al vettore velocità.\\
È noto che il modulo dell'accelerazione nel moto circolare uniforme è pari a
\[a = \frac{v^2}{R} = \omega^2 R\]
essendo
\[\omega = \frac{v}{R}\]

% Tabella per le definizione di concetti, etc...
\vspace{1em}
\rowcolors{1}{black!5}{black!5}
\setlength{\tabcolsep}{14pt}
\renewcommand{\arraystretch}{2}
\noindent
\begin{tabularx}{\textwidth}{@{}|P|@{}}
    \hline
    {\textbf{FORZA CENTRIPETA}}\\
    \parbox{\linewidth}{La forza necessaria per mantenere il moto circolare è, per la seconda legge della dimanica:
    \[\vec{F} = m \vec{a}\]
    e sostitutendo ad $a$ la formula precedentemente ottenuta si ha
    \[\boxed{F_c = \frac{m v^2}{R}}\]
    orientata verso il centro della circonferenza. Naturalmente quest'ultima è la \textbf{forza centripeta}, così chiamata per descriverne la direzione e il verso, ma non la natura della forza (la forza centripeta non è un tipo di forza, in quanto anche la forza di attrazione gravitazionale è una forza centripeta quando si parla di attrazione tra pianeti, anche la forza d'attrito è una forza centripeta, come quando si gira con la macchina).\vspace{3mm}}\\
    \hline
\end{tabularx}

\vspace{2em}
\noindent
\textbf{Esempio}: Nelle curve delle strade, la carreggiata è inclinata, in quanto la forza normale si scompone in due componenti: una centrifuga e una centripeta; ciò, naturalmente, aiuta la forza di attrito ad evitare che la vettura esca fuori strada.\\
Si determini, allora, l'angolo $\theta$ di inclinazione della strada affinché la componente orizzontale della forza normale fornisca la forza centripeta necessaria per girare (supponendo in assenza di attrito):

\vspace{1em}
\noindent
\begin{figure}[H]
  \centering
  \newcommand{\ang}{30}

  \begin{tikzpicture}[scale=1.5]

  % triangle:
  \draw [draw = orange, fill = orange!15] (0,0) coordinate (O) -- (\ang:6)
  	coordinate [pos=.45] (M) |- coordinate (B) (O);

  % angles:
  \draw [draw = orange] (O) ++(.8,0) arc (0:\ang:0.8)
  	node [pos=.4, left] {$\theta$};
  \draw [draw = orange] (B) rectangle ++(-0.3,0.3);

  \begin{scope} [rotate=\ang]
    \draw (M) -- ++(0,0.25) -- ++(-0.25,0) -- ++(0,0.25) -- ++(0.25,0) -- ++(0,0.5) -- ++(1,0) -- ++(0,-0.5) -- ++(0.25,0) -- ++(0,-0.25) -- ++(-0.25,0) -- ++(0,-0.25) ++(-0.25,0) -- ++(0,0.25) -- ++(-0.5,0) -- ++(0,-0.25);
  \end{scope}

  \begin{scope} [-latex,rotate=\ang]
  % Weight Force and its projections
  \coordinate (MM) at ([xshift=0.5cm,yshift=0.6cm]M);

  \draw (MM) node[circ]{} -- ++ (-\ang-90:1.5)
  	node [very near end, right] {$\vec{F}_t$};

  % Normal Force
  \draw (MM) ++(0,-0.6) -- ++ (0,1.8)
    node [very near end, right] {$\vec{F}_N$};
  \end{scope}
  \end{tikzpicture}
  \caption{Auto in curva su un piano inclinato}
  \label{fig:auto_curva_piano_inclinato}
\end{figure}

\vspace{1em}
\noindent
Si realizzi il diagramma a corpo libero della vettura:

\vspace{1em}
\begin{figure}[H]
  \centering
  \begin{tikzpicture}[scale=1]
    \draw [-stealth] (0,0) node[circ]{} -- ++(-0.8,0.8) node [near end, above right] {$\vec{F}_N$};
    \draw [-stealth] (0,0) -- ++(0,-1) node [midway, right] {$\vec{F}_{t}$};
  \end{tikzpicture}
  \caption{Diagramma a corpo libero di una macchina in curva}
  \label{fig:diagramma_corpo_libero_macchina_curva}
\end{figure}

\vspace{1em}
\noindent
Naturalmente, scomponendo la forza normale nelle sue due componenti (orizzontale e verticale) si ottiene che
\[F_N \cdot \sin(\theta) = F_c = \frac{m v^2}{R}\]
\[F_N \cdot \cos(\theta) - mg = 0 \longrightarrow F_N \cdot \cos(\theta) = mg\]
Allora si ha che
\[\tan(\theta) = \frac{v^2}{R g}\]

\vspace{1em}
\noindent
\textbf{Osservazione}: Anche quando si considera il moto di una corda che viene fatta ruotare nell'aria, la tensione della corda è esattamente la forza centripeta necessaria a mantere il moto circolare uniforme (conoscendo il raggio di rotazione, ossia la lunghezza della corda, nonché la velocità di rotazione).\\
Per verificare che il coefficiente di attrito statico non può essere mai maggiore di $1$ è sufficiente porre una massa su una superficie inclinata e aumentare l'angolo di inclinazione.\\

\newpage
\noindent
\begin{center}
  16 Marzo 2022
\end{center}
\subsection{Sistemi non-inerziali}
Il concetto di sistema non-inerziale è essenziale per lo studio della dinamica dei corpi. Infatti, un sistema non-inerziale è un sistema di riferimento in cui l'accelerazione non é nulla, ovvero $a \neq 0$. Ciò impedisce di applicare la $2^a$ legge della dinamica, in quanto tale legge si applica solo a sistemi di riferimento inerziali.\\
Si consideri il caso di un treno in accelerazione e di una massa sospesa al soffitto; naturalmente, quando il treno accelera (e quindi la sua velocità non è costante), la massa si inclina nella direzione opposta al moto:

\begin{figure}[H]
  \centering
  \begin{tikzpicture}[scale=4]
    \draw [dashed] (0,2.5) node[circ]{} -- ++(0,-1);
    \draw [dashed] (0,2.5) node[circ]{} -- (-0.4,1.5);
    \draw (0,2.5) node[circ]{} -- (-0.2,2) node[circ]{};
    \coordinate (i) at (-0.2,2);
    \coordinate (ctr) at (0,2.5);
    \coordinate (f) at (0,2);
    \pic [draw=red, text=red, <->, "$\theta$", angle radius=1cm, angle eccentricity=1.4] {angle = i--ctr--f};
    \draw [-stealth] (0.2,2) -- ++(0.4,0) node[midway, above]{$\vec a$};
    \draw [-stealth] (-0.2,2) -- ++(0,-0.4) node[at end, right]{$\vec F_t$};
    \draw [-stealth,red] (-0.2,2) -- ++(0.1,0.25) node[left]{$\vec F_T$};
    \draw [-stealth,blue] (-0.2,2) -- ++(-0.2,0) node[above]{$\vec F_a$};
  \end{tikzpicture}
  \caption{Forza apparente}
  \label{fig:forza_apparente}
\end{figure}

\noindent
In cui, ovviamente, applicando la $2^a$ legge della dinamica si ottiene che
\[\vec{F}_T + \vec{F}_t = m \cdot \vec{a}\]
da cui
\[\vec{F}_T + m \vec{g} = m \vec{a} \longrightarrow \vec{F}_T = m \cdot \left(\vec{a} - \vec{g}\right)\]

\vspace{1em}
\noindent
\textbf{Osservazione}: Si osservi che la prima equazione può essere riscritta come segue:
\[\vec{F}_T + \vec{F}_t = m \cdot \vec{a} \longrightarrow \vec{F}_T + \vec{F}_t - m \cdot \vec{a} = 0\]
Ecco che allora la forza $\vec{F}_a = m \vec{a}$ è una nuova forza, la quale prende il nome di \textbf{forza apparente}, o anche \textbf{pseudo-forza apparente} o \textbf{forza inerziale}, che agisce ancora una volta sulla massa $m$, in direzione opposta alla direzione del moto.\\
Naturalmente, in questo caso, l'unico termine da considerare è $m \vec{a}$, in quanto l'accelerazione è costante, mentre se l'accelerazione non lo fosse, all'interno della formula dovrebbero figurare altri termini (si pensi, per esempio, ad una \textbf{pseudo-forza apparente centrifuga}, la quale può essere costante in modulo, ma cambia direzione e verso).

\vspace{1em}
\noindent
\textbf{Esempio}: Si consideri il caso in cui una massa $m$ è posta su un cuneo e l'intero sistema accelera con $\vec{a}$ costante:

\vspace{1em}
\noindent
\begin{figure}[H]
  \centering
  \newcommand{\ang}{30}

  \begin{tikzpicture} [xshift=-12em,font = \small, scale=1.5]
  \reflectbox{\rotatebox[origin=c]{360}{
    % triangle:
    \draw [draw = orange, fill = orange!15] (0,0) coordinate (O) -- (\ang:6)
    	coordinate [pos=.45] (M) |- coordinate (B) (O);

    % angles:
    \draw [draw = orange] (O) ++(.8,0) arc (0:\ang:0.8)
    	node [pos=.4, left] {\reflectbox{\rotatebox[origin=c]{360}{$\theta$}}};
    \draw [draw = orange] (B) rectangle ++(-0.3,0.3);
    \draw [-stealth] (5.5,1.5) ++(1,0) -- ++(-0.75,0) node[midway,above]{\reflectbox{\rotatebox[origin=c]{360}{$a$}}};

    \begin{scope} [-latex,rotate=\ang]
    % Object (rectangle)
    \draw [fill = purple!30,
    	draw = purple!50] (M) rectangle ++ (1,.6);

    % Weight Force and its projections
    \draw [dashed] (M) ++ (.5,.3) coordinate (MM) -- ++ (0,-1.29)
    	node [very near end, right] {\reflectbox{\rotatebox[origin=c]{360}{$\vec{F}_t \cdot \cos{\theta}$}}};

    \draw [dashed] (MM) -- ++ (-0.75,0)
    	node [very near end, left] {\reflectbox{\rotatebox[origin=c]{360}{$\vec{F}_t \cdot \sin{\theta}$}}};

    \draw (MM) -- ++ (-\ang-90:1.5)
    	node [very near end,below left ] {\reflectbox{\rotatebox[origin=c]{360}{$\vec{F}_t$}}};

    % Normal Force
    \draw (MM) -- ++ (0,1.29)
    node [very near end, right] {\reflectbox{\rotatebox[origin=c]{360}{$\vec{F}_N$}}};
    \end{scope}
  }}
  \end{tikzpicture}
  \caption{Piano inclinato in accelerazione}
  \label{fig:piano_inclinato_in_accelerazione}
\end{figure}

\noindent
Si determini, allora, quale deve essere l'accelerazione $\vec{a}$ affinché il blocco rimanga immobile sul cuneo.\\
Naturalmente, osservando il sistema da fuori, le uniche forze che agiscono su tale corpo sono la forza peso e la forza normale. Se, invece, tale corpo viene osservato dall'interno, impiegando un sistema di riferimento non-inserziale, allora a tali forze se ne deve aggiungere una terza, una \textbf{pseudo-forza apparente} che agisce in verso opposto a quello dell'accelerazione $\vec{a}$, ovvero la forza $\vec{F}_a = - m \vec{a}$, come mostrato di seguito:

\vspace{1em}
\begin{figure}[H]
  \centering
  \begin{tikzpicture}[scale=1.5]
    \draw [-stealth] (0,0) -- ++(0,-1) node [midway, right] {$\vec{F}_{t}$};
    \draw [-stealth] (0,0) -- ++(0.5,1) node [midway, above left] {$\vec{F}_{T}$};
    \draw [-stealth] (0,0) -- ++(-1,0) node [midway, above] {$\vec{F}_{a}$};
  \end{tikzpicture}
  \caption{Diagramma a corpo libero di una massa su un piano inclinato in accelerazione}
  \label{fig:diagramma_corpo_libero_massa_piano_inclinato_in_accelerazione}
\end{figure}

\vspace{1em}
\noindent
Da cui si evince che
\[\vec{F}_N + \vec{F}_t + \vec{F}_a = 0\]
e scomponento l'equazione nelle sue componenti orizzontali e verticali si perviene al risultato seguente:
\begin{align*}
  F_N \cdot \cos(\theta) - mg = 0\\
  -F_a + F_N \cdot \sin(\theta) = 0
\end{align*}
per cui si ottiene che
\[a = g \cdot \tan(\theta)\]

\vspace{1em}
\subsection{Prodotto vettoriale}
Si considerino due vettori $\vec{A}$ e $\vec{B}$, allora si ha che
\[\left \vert \vec{A} \times \vec{B} \right \vert = \left \vert A \right \vert \cdot \left \vert B \right \vert \cdot \sin(\theta)\]
ove $\theta$ è l'angolo compreso tra i vettori $\vec{A}$ e $\vec{B}$.

\vspace{1em}
\noindent
\textbf{Osservazione}: Si osservi che il prodotto vettoriale $\vec{A} \times \vec{B}$ è un vettore ortogonale ad $\vec{A}$ e $\vec{B}$:

\begin{figure}[H]
  \centering
  \begin{tikzpicture}
    \draw[-,fill=white!95!red](0,0)--(3,0)--(4,1)--(1,1)--cycle;
    \node at (2,0.5) {$|\textcolor{blue}{a}\times \textcolor{red}{b}|$};
    \draw[ultra thick,-latex,blue](0,0)--(3,0)node[midway,below]{$a$};
    \draw[ultra thick,-latex,red](0,0)--(1,1)node[midway,above]{$b$};
    \draw[ultra thick,-latex,blue!50!red](0,0)--(0,3)node[pos=0.7,right]{$a\times b$};
    \draw (0.6,0) arc [start angle=0,end angle=45,radius=0.6]
    node[pos=0.7,right]{$\theta$};
  \end{tikzpicture}
  \caption{Prodotto vettoriale}
  \label{fig:prodotto_vettoriale}
\end{figure}

\noindent
Inoltre si ha che
\begin{itemize}
  \item se $\vec{A} \perp \vec{B}$, allora si ha che
  \[\left \vert \vec{A} \times \vec{B} \right \vert = \left \vert A \right \vert \cdot \left \vert B \right \vert\]

  \item se $\vec{A} \parallel \vec{B}$, allora si ha che
  \[\left \vert \vec{A} \times \vec{B} \right \vert = 0\]

  \item negli altri casi bisogna impiegare la regola della mano destra. Se si considera il prodotto vettoriale seguente (che non è mai commutativo)
  \[\vec F = q \vec v \times \vec B\]
  la regola della mano destra si applica come segue:

  \begin{figure}[H]
    \centering
    \begin{tikzpicture}[scale=0.5]
      \coordinate (O) at (1.0,0.7);    % ORIGIN
      \coordinate (WT) at ( 2.9,-1.1); % WRIST TOP
      \coordinate (T1) at ( 2.3, 0.7); % THUMB
      \coordinate (T2) at ( 1.75, 2.3);
      \coordinate (T3) at ( 2.0, 3.1);
      \coordinate (T4) at (1.38, 3.15);
      \coordinate (T5) at ( 0.9, 2.3);
      \coordinate (T6) at ( 0.85, 1.2);
      \coordinate (T7) at ( 0.85, 0.2);
      \coordinate (I1) at (-1.1, 2.45); % INDEX
      \coordinate (I2) at (-2.9, 3.45);
      \coordinate (I3) at (-3.3, 2.9);
      \coordinate (I4) at (-1.5, 1.8);
      \coordinate (I5) at (-0.9, 1.1);
      \coordinate (I6) at (-0.9, 0.3);
      \coordinate (M1) at (-2.1, 0.9); % MIDDLE
      \coordinate (M2) at (-3.95,0.55);
      \coordinate (M3) at (-4.0,-0.15);
      \coordinate (M4) at (-2.3, 0.05);
      \coordinate (M5) at (-1.1, 0.20);
      \coordinate (R1) at (-1.9,-0.1); % RING
      \coordinate (R2) at (-1.8,-0.7);
      \coordinate (R3) at (-0.3,-1.5);
      \coordinate (R4) at ( 0.1,-1.7);
      \coordinate (R5) at ( 0.1,-1.0);
      \coordinate (R6) at (-0.5,-0.7);
      \coordinate (R7) at (-1.2,-0.3);
      \coordinate (P1) at (-1.9,-1.3); % PINKY
      \coordinate (P2) at (-0.8,-1.9);
      \coordinate (P3) at (-0.2,-2.1);
      \coordinate (P4) at (-0.05,-1.65);
      \coordinate (W1) at ( 0.4,-2.9); % WRIST BOTTOM
      \coordinate (W2) at ( 1.6,-3.5);

      % HAND
      \fill[pinkskin]
        (WT) -- (T6) -- (I5) -- (M5) -- (R2) -- (P2) -- (W2) to[out=25,in=-90] cycle;
      \draw[fill=pinkskin]
        (WT) to[out=120,in=-60] % THUMB
        (T1) to[out=120,in=-90]
        (T2) to[out=80,in=-110]
        (T3) to[out=80,in=50,looseness=1.5] % tip
        (T4) to[out=-130,in=80]
        (T5) to[out=-100,in=70]
        (T6) to[out=-100,in=100]
        (T7)
        (T6) to[out=150,in=-30] % INDEX
        (I1) to[out=150,in=-30]
        (I2) to[out=150,in=145,looseness=1.7] % tip
        (I3) to[out=-30,in=150]
        (I4) to[out=-30,in=105]
        (I5) to[out=-75,in=90]
        (I6)
        (I5) to[out=-170,in=10] % MIDDLE
        (M1) to[out=-170,in=10]
        (M2) to[out=-170,in=-175,looseness=1.8] % tip
        (M3) to[out=5,in=-170]
        (M4) to[out=10,in=-170] % bottom knuckle
        (M5)
        (M5) to[out=-160,in=50] % RING
        (R1) to[out=-130,in=140,looseness=1.2]
        (R2) to[out=-30,in=160]
        (R3) --
        (R4) to[out=-20,in=-20,looseness=1.5] % tip
        (R5) --
        (R6) to[out=140,in=8,looseness=0.9]
        (R7)
        (R2) to[out=-160,in=155] % PINKY
        (P1) to[out=-35,in=150]
        (P2) to[out=-30,in=160]
        (P3) to[out=-20,in=-30,looseness=1.5] % tip
        %(P4) --
        (R4)
        (P2) to[out=-50,in=140] % WRIST
        (W1) to[out=-40,in=160]
        (W2);

      % FOLDS
      \draw[very thin] (T5)++(-80:0.3) to[out=40,in=180]++ (25:0.45); % THUMB
      \draw[very thin] (I1)++(180:0.2) to[out=-160,in=90]++ (-130:0.6); % INDEX
      \draw[very thin] (I1)++(155:1.3) to[out=-150,in=80]++ (-130:0.55);
      \draw[very thin] (M4)++(30:0.2) to[out=80,in=-65]++ (95:0.5); % MIDDLE FINGER
      \draw[very thin] (M3)++(10:0.8) to[out=80,in=-75]++ (90:0.45);
      \draw[very thin] (M5)++(-140:0.1) to[out=-20,in=90]++ (-54:0.8); % RING
      \draw[very thin] (R6) to[out=160,in=10]++ (180:0.2);
      \draw[very thin] (R3)++(155:0.5) to[out=120,in=-100]++ (100:0.2);
      \draw[very thin] (P2)++(140:0.1) to[out=95,in=-110]++ (80:0.4); % PINKY
      %\draw[very thin] (P1)++( 10:0.04) to[out=95,in=-130]++ (70:0.4);
      \draw[very thin] (I5)++(-40:0.45) to[out=-70,in=90]++ (-70:1.7);    % PALM
      \draw[very thin] (P3)++(-155:0.05) to[out=-120,in=40]++ (-130:0.2); % PALM
      \draw[very thin] (W2)++(70:1.4) to[out=-175,in=-40]++ (160:1.4); % PALM

      % VECTORS
      \def\R{0.32}
      \draw[force]
        (O) --++ (82:3.2)
        node[above,scale=1.5] {$\vb{F} \color{black} = q {\color{veccol}\vb{v}} \times {\color{Bcol}\vb{B}}$};
      \draw[velocity]
        (O) --++ (148:3.3) coordinate (V)
        node[above=2,left=0,scale=1.5] {$\vb{v}$};
      \draw[charge+] (O) circle (\R) node[scale=1.4] {$+$};
      \draw[BField]
        (O)++(-172:0.7*\R) --++ (-172:3.25) coordinate (B)
        node[above=4,left=0,scale=1.5] {$\vb{B}$};
      \draw pic[->,"\huge$\theta$",draw=black,thick,angle radius=28,angle eccentricity=1.26] {angle = V--O--B};
    \end{tikzpicture}
    \caption{Regola della mano destra}
    \label{fig:regola_mano_destra}
  \end{figure}

  \item $\hat{i} \times \hat{j} = \hat{k}$
  \item $\hat{j} \times \hat{k} = \hat{i}$
  \item $\hat{k} \times \hat{i} = \hat{j}$
  \item $\hat{j} \times \hat{i} = -\hat{k}$
  \item $\hat{k} \times \hat{j} = -\hat{i}$
  \item $\hat{i} \times \hat{k} = -\hat{j}$
  \item $\hat{i} \times \hat{i} = \hat{j} \times \hat{j} = \hat{k} \times \hat{k} = 0$
\end{itemize}
Alternativamente, è noto che il prodotto vettoriale $\vec{A} \times \vec{B}$ si calcola come
\[\vec{A} \times \vec{B} = \det \left(
  \rowcolors{1}{white}{white}
  \begin{array}{ccc}
    \hat{i} & \hat{j} & \hat{k}\\
    A_x & A_y & A_z\\
    B_x & B_y & B_z
  \end{array}
\right)\]

\vspace{1em}
\noindent
\textbf{Osservazione}: Mentre il prodotto vettoriale è il determinante della matrice di cui sopra, il prodotto scalare, $\vec{B} \cdot \vec{A}$ si interpretava come $\vert A \vert$ per la componente di $B$ su $A$.\\
Inoltre si ha che il prodotto scalare tra $\vec{A}$ e $\vec{B}$ non viene alterato se ai due vettori vengono aggiunte ulteriori componenti parallele ad $\vec{A}$ e $\vec{B}$.

\vspace{1em}
\noindent
\textbf{Esempio}: Si consideri la forza di Lorentz, la cui formula viene di seguito esposta
\[\boxed{\vec{F} = g \vec{E} + g \vec{v} \times \vec{B}}\]
in cui, senza $\vec{E}$, sarebbe stato
\[\boxed{\vec{F} = g \vec{v} \times \vec{B}}\]
Allora, in questo caso, se il campo magnetico $\vec{B}$ va dentro la pagina e la velocità $\vec{v}$ è parallela alla pagina, la forza $\vec{F}$ è diretta verso il centro della spirale, ovvero $\vec{F}$ è una \textbf{forza centripeta}, calcolata come segue
\[\vec{F} = -e \cdot (\vec{v} \times \vec{B})\]
Ma dalla cinetica è anche noto che la forza centripeta si calcola come
\[F_c = \frac{m v^2}{R}\]
per cui operando una eguaglianza si ottiene
\[\frac{m v^2}{R} = e v B \longrightarrow \frac{m v}{R} = e B \longrightarrow m \omega = e B \longrightarrow \omega = \frac{eB}{m}\]
in cui $\omega$ prende il nome di \textbf{frequenza ciclotronica}, un valore utilizzabile per determinare la traiettoria e la natura del campo magnetico $B$, grazie all'emissione di fotoni da parte delle particelle.\\
Tale fenomeno non si verifica solo sperimentalmente grazie ad un acceleratore di particelle, ma anche nello spazio, dove sono presenti campi magnetici e particelle cariche, le quali ruotano con frequenza angolare corrispondente alla frequenza ciclotronica;

\vspace{1em}
\subsection{Pseudo-forza di Coriolis}
Si consideri il seguente diagramma di un moto circolare

\begin{figure}[H]
  \centering
  \begin{tikzpicture}[>=Triangle]
    \shade [top color=white, bottom color=gray!50, middle color=white]
      (120:8/3) arc (120:190:8/3) node [black, near end, left] {$\omega$}
      -- (190:25/9) -- (200:15/6) -- (190:20/9) -- (190:7/3)
      arc (190:120:7/3) -- cycle;

    \foreach \i in {90, 210, 330}{
      \draw [->, thick, blue!50!cyan] (\i-65:2) arc (\i-65:\i+60:2);
      \tikzset{shift={(\i:2)}, rotate=\i+180}
      \draw [->, very thick, orange] (0,0) -- (1,0)
        node [black, near end, anchor=\i+90] {$\vec a$};
      \draw [->, very thick, green!50!black] (0,0) -- (0,-2)
        node [black, near end, anchor=\i+180] {$\vec v$};
      \fill circle [radius=1/10];
  }
  \end{tikzpicture}
  % Set the Angles of the Axis
	\tdplotsetmaincoords{57}{120}
  \begin{tikzpicture}[scale=2,tdplot_main_coords]
		% Axis
	    \draw[->] (2,0,0) -- (-2,0,0) node[above right]{$y$};
	    \draw[->] (0,-1.3,0) -- (0,1.5,0)coordinate(C) node[right]{$x$};
	    \draw[->] (0,0,0)coordinate(B) -- (0,0,1.5) node[above]{$z$};

  	    % Circural Loop
  		\draw (0,0,0) circle [radius=1];

  		% Node
  		\draw (0,0,0) -- +(0.8,0,0) node [pos=0.6, above left] {\small$R$} ;

  		% Current Direction
  		\draw [blue, ->] (0.3,-1.2,0) arc (280:340:1.1) node [black, pos=0.4, left] {$\omega$};

  		% Vectors
  		\draw[->,very thick] (0,0,0) -- (0,0,1) node [pos=0.5, left] {$\vec \omega$};
      \draw[very thick, ->] (0,0,0) -- (-0.8,0.60,0) coordinate(A) node[above, pos=0.5] {$\vec r$};
      \draw[very thick, ->] (A) -- ++(-0.51,-0.51,0) node[above right, pos=0.5] {$\vec v$};

  		% Angle
  		\pic[draw, angle radius=7mm,"$\phi$", angle eccentricity=1.7, thick] {angle=C--B--A};
	\end{tikzpicture}
  \caption{Moto circolare uniforme e pseudo-forza di Coriolis}
  \label{fig:moto_circolare_uniforme_pseudo_forza_coriolis}
\end{figure}

\vspace{1em}
\noindent
Allora si può scrivere che, naturalmente, la velocità è il prodotto vettoriale tra il vettore velocità angolare e il vettore posizione, ovvero
\[\boxed{\vec v = \vec w \times \vec R}\]
E procedendo a derivare tale espressione rispetto al tempo si ottiene che
\[\boxed{\vec{a} = \vec{\omega} \times \vec{v} = \vec \omega \times \left(\vec \omega \times \vec R \right)}\]
in cui le parentesi sono importanti, essendo il prodotto vettoriale non associativo (infatti $\vec \omega \times \vec \omega = 0$, essendo paralleli).\\
La pseudo-forza centrifuga, pertanto,  si calcola come segue
\[\boxed{\vec F = -m \vec a = -m \vec \omega \times \left(\vec \omega \times \vec R \right)}\]
da cui si evince che se $\vec R = 0$, ovvero il punto si trova sull'asse di rotazione, allora non subisce alcuna forza, ma tanto più grande sarà il raggio, maggiore sarà anche la pseudo-forza che lo spinge in direzione opposta al moto.

\vspace{1em}
\noindent
\textbf{Osservazione}: Si consideri un punto materiale che si muove in linea retta e si supponga che poggi su un piano rotante con velocità angolare $\omega$; allora si osserva che, naturalmente, la pseudo-forza centrifuga è una componente che contribuisce a fare sì che il punto vada verso l'esterno, ma oltre a ciò, è da considerare anche un'altra forza che spinge il punto materiale in direzione opposta al verso di rotazione, ovvero la pseudo-forza di Coriolis, la quale si ha quando il sistema di riferimento non inerziale dal quale si osserva il fenomeno presenta una propria velocità, ossia non si muove con traiettoria circolare.

% Tabella per le definizione di concetti, etc...
\vspace{1em}
\rowcolors{1}{black!5}{black!5}
\setlength{\tabcolsep}{14pt}
\renewcommand{\arraystretch}{2}
\noindent
\begin{tabularx}{\textwidth}{@{}|P|@{}}
    \hline
    {\textbf{PSEUDO-FORZA DI CORIOLIS}}\\
    \parbox{\linewidth}{Tale forza prende il nome di \textbf{pseudo-forza di Coriolis}, la quale è perpendicolare a $\vec \omega$ e a $\vec v$ e si calcola come segue
    \[\boxed{\vec F = -2 m \vec \omega \times \vec v}\]
    \vspace{-1mm}}\\
    \hline
\end{tabularx}

\vspace{1em}
\noindent
\textbf{Esempio}: Si osservi che, in un vortice di bassa pressione, considerando la velocità angolare terrestre con verso uscente dal foglio e la velocità delle particelle di aria dirette verso il centro del vortice, si ottiene che la pseudo-forza di Coriolis, per la regola della mano destra, è diretta ortogonalmente al vettore velocità e giacente sul foglio stesso:

\newcommand\bonusspiral{} % just for safety
\def\bonusspiral[#1](#2)(#3:#4)(#5:#6)[#7]{% \bonusspiral[draw options](placement)(start angle:end angle)(start radius:final radius)[revolutions]
\pgfmathsetmacro{\domain}{#4+#7*360}
\pgfmathsetmacro{\growth}{180*(#6-#5)/(pi*(\domain-#3))}
\draw [#1,
       shift={(#2)},
       domain=#3*pi/180:\domain*pi/180,
       variable=\t,
       smooth,
       samples=int(\domain/5)] plot ({\t r}: {#5+\growth*\t-\growth*#3*pi/180})
}

\begin{figure}[H]
  \centering
  \rowcolors{1}{white}{white}
  \renewcommand{\arraystretch}{4}
  \begin{tabularx}{\textwidth}{|P|}
    \hline
    {
      \vspace{1.5em}
      \begin{tikzpicture}[scale=0.7]
      \node[circle,draw,minimum width=2cm] at (6.5,3.5){};
      \draw (6.5,3.5) node[circ]{} (6.5,4) node[]{$\vec \omega$} (0,0) node[]{\huge$L$};
      \bonusspiral[red](0,0)(60:270)(-1:-5)[2];
      \draw[very thick, -stealth] (2.8,-1) -- ++(-1.5,1) node[at end, above]{$\vec v$};
      \draw[very thick, -stealth] (2.8,-1) -- ++(0.5,1.5) node[at end, right]{$\vec F_c$};
      \end{tikzpicture}
    }\\
    \hline
  \end{tabularx}
  \caption{Vortice di bassa pressione e pseudo-forza di Coriolis}
  \label{fig:vortice_bassa_pressione_pseudo_forza_coriolis}
\end{figure}

\vspace{1em}
\noindent
\textbf{Osservazione}: Si osservi che il calcolo della velocità limite segue la seguente formula
\[v = \sqrt{\frac{2mg}{\rho A C_d}}\]
e sapendo che
\[m = V \cdot \rho\]
si evince che
\[v = \sqrt{\frac{2 V \rho g}{\rho A C_d}} = \sqrt{\frac{2 V g}{A C_d}}\]
pertanto, a parità di accelerazione gravitazione $g$ e coefficiente di resistenza $C_d$, ciò che determina la velocità dell'impatto con il suolo è il rapporto
\[\frac{V}{A}\]
pertanto più un corpo è voluminoso, maggiore sarà la velocità con cui impatta al suolo.

\newpage
\section{Gravità}
\noindent
\begin{center}
  17 Marzo 2022
\end{center}
La \textbf{legge di gravitazione universale} venne formulata da Isaac Newton nell'opera Philosophiae Naturalis Principia Mathematica (\quotes{Principia}) e pubblicata il 5 luglio 1687. Newton, infatti, ha affermato di voler formulare una teoria unica (appunto, universale) che descriva sia la caduta dei corpi sulla terra, sia il movimento degli astri.\\
Al fine di determinare tale formula, Newton si basò sull'osservazione della Luna e del suo moto; l'\textbf{accelerazione centripeta della Luna}, naturalmente, si calcola come segue
\[a_c = \frac{v^2}{R} = \omega^2 R\]
in cui, ovviamente, $\omega$ è la velocità angolare, mentre $R$ è il raggio dell'orbita della Luna, ovvero la distanza Terra-Luna (già nota dai tempi dei Greci); in particolare è noto che $R = 60 \cdot R_T$, in cui $R_T = 6371$ km, mentre per determinare la velocità angolare della Luna, è noto che
\[\omega = \frac{2\pi}{T}\]
ovvero il rapporto tra l'angolo descritto dal moto e il tempo impiegato: anche in questo caso è noto che $T \cong 27.3$ giorni. Ovviamente, ora, con questi dati è possibile calcolare l'accelerazione centripeta della Luna, ossia:
\[a_c = 2.7 \times 10^{-3} \text{ m/s}^2\]
inoltre è noto che l'accelerazione gravitazionale terrestre è $g=9.8$ m/s$^2$ per cui eseguendo il rapporto si ottiene
\[\frac{g}{a_c} \cong 60^2\]
che è un risultato importante, in quanto fa capire come il modulo dell'accelerazione centripeta sia \textbf{inversamente proporzionale al quadrato della distanza} tra i due corpi che si attraggono:
\[F \propto \frac{1}{d^2}\]
Tale proprietà viene espressa tramite la seguente legge di gravitazione universale
\[\boxed{F = G \cdot \frac{m_1 \cdot m_2}{r^2}}\]
in cui $G$ prende il nome di \textbf{costante di gravitazione universale}, la quale venne determinata significativamente dopo Newton.

\vspace{2em}
\noindent
\textbf{Esempio}: Si considerino due punti materiali $m_1$ e $m_2$ ad una certa posizione:

\vspace{1em}
\begin{figure}[H]
  \centering
  \begin{tikzpicture}[scale=1]
    \draw (0,0) node[circ](start){} (-1,2) node[circ, above](m_1){} ++(0,0.4) node[]{$m_1$} (2,3) node[circ](m_2){} ++(0,0.4) node[]{$m_2$};
    \draw [-stealth] (start) -- (m_1) node[midway, below left]{$\vec r_1$};
    \draw [-stealth] (start) -- (m_2) node[midway, below right]{$\vec r_2$};
    \draw [dashed] (m_1) -- coordinate[midway](mid) (m_2);
    \draw [-stealth] (m_1) -- (mid) node[midway, above]{$\vec r_{1,2}$};
    \draw [-stealth] (m_2) ++(-0.5,0.5) -- ++(-0.75,-0.25) node[midway, above]{$\vec F_{1,2}$};
  \end{tikzpicture}
  \caption{Forza di attrazione tra due masse}
  \label{fig:forza_attrazione_gravitazionale}
\end{figure}

\vspace{1em}
\noindent
Allora si ha che, normalmente, il modulo della forza di attrazione gravitazinale dal corpo $1$ al corpo $2$ è
\[\vec F_{1,2} = G \cdot \frac{m_1 \cdot m_2}{\left \vert \vec r_1 - \vec r_2 \right \vert}\]
e volendo determinare il vettore forza di attrazione, semplicemente si può scrivere
\[\boxed{\vec F_{1,2} = -G \cdot \frac{m_1 \cdot m_2}{\left \vert \vec r_1 - \vec r_2 \right \vert} \cdot \hat{r}_{1,2}}\]
in cui cambiando il versore $\hat{r}_{1,2}$ si osserva come tale formula si adatti perfettamente alla terza legge di Newton: infatti il versore cambia segno, ma il modulo della forza rimane lo stesso.

\vspace{1em}
\noindent
\textbf{Osservazione}: Si osservi che, naturalmente, sulla Terra, fissando come origine il centro della Terra:
\[\vec{F}_t = -G \cdot \frac{m_t \cdot m}{r^2} \cdot \hat{r} = - \left(G \cdot \frac{m_2}{R_t^2}\right) \cdot m \cdot \hat{r}\]
in cui, ovviamente
\[g = \left(G \cdot \frac{m_t}{R_t^2}\right)\]
in cui si è fissato come raggio il raggio terrestre, dal momento che la Terra è un corpo sferico e, quindi, tutta la sua massa può essere considerata concentrata nel suo nucleo (un'assunzione che corrisponde alla somma del contributo attrattivo di tutte le infinitesime masse che costiuiscono la Terra).

\vspace{1em}
\subsection{L'esperimento di Cavendish}
Per misurare la forza di attrazione gravitazionale, Cavendish ha impiegato un \textbf{pendolo a torsione} su cui venivano fissate due masse, in corrispondenza delle loro estremità:

\begin{figure}[H]
  \centering
  \begin{tikzpicture}
    \draw (0,0) node[circ]{} -- ++(0,3) -- ++(3,0);
    \draw (3,2) -- (3,4);
    \foreach \i in {22,23,...,39} {
      \draw (3,\i / 10) -- (3.3,\i / 10-0.3);
    }
    \draw (-2,5.5) -- (2,5.5);
    \draw (0,5) -- ++(0,0.5);
    \foreach \i in {-19,-18,...,19} {
      \draw (\i / 10,5.5) -- (\i / 10+0.3,5.8);
    }
    \draw (0,5) node[circ]{} -- ++(1,-1) -- ++(0,-4.7) node[circle,fill=black,minimum width=10pt]{};
    \draw (0,5) node[circ]{} -- ++(-1,-1) -- ++(0,-4.7) node[circle,fill=black,minimum width=10pt]{};
    \draw (-1.5,-0.5) node[circ]{} -- (1.5,0.5) node[circ]{};
    \draw (-1.5,-0.5) ++(0,-0.4) node[]{$m_1$};
    \draw (1.5,0.5) ++(0,0.4) node[]{$m_2$};
  \end{tikzpicture}
  \caption{Esperimento di Cavendish}
  \label{fig:esperimento_cavendish}
\end{figure}

\noindent
Tali piccole masse oscillavano leggermente, a causa dell'attrazione con masse più considenti, in un moto continuativo, ma molto lento (in questo caso il periodo si aggirava intorno a $20$ minuti); l'esperimento doveva svolgersi esattamente sul piano, in modo tale da essere ortogonale alla forza di gravità della terra e ciò richiedeva molta precisione e condizioni sperimentali costanti (quali temperatura, pressioni, vento, etc.).\\
Grazie a ciò Cavendish è riuscito a determinare la \textbf{densità della terra}, che era ciò che gli interessava maggiormente: sapendo, infatti, la densità e il volume della Terra, si sarebbe stati in grado di conoscerne la massa e, sapendo il valore di $g$, si sarebbe, in seguito, stati in grado di isolare la costante gravitazionale $G$, che presenta il valore seguente:
\[\boxed{G = 6.67 \times 10^{-11} \frac{\text{m}^3}{\text{s}^2 \text{ kg}}}\]
A partire da tale risultato è possibile analizzare il moto e le caratteristiche di diversi pianeti.

\vspace{1em}
\subsection{Campo gravitazionale}
A partire dall'equazione seguente
\[\vec{F}_t = -G \cdot \frac{m_t}{R_t^2} \cdot \hat{r} \cdot m\]
si osserva come la forza sia proporzionale alla massa: conoscendo la massa del corpo è possibile conoscere la forza su tale corpo.\\
Per campo è da intendersi una valutazione della \textbf{forza per unità di massa}, come mostrato di seguito:
\[\boxed{\frac{\vec{F}_t}{m} = -G \cdot \frac{m_t}{R_t^2} \cdot \hat{r} = \vec g}\]

\colorlet{myred}{red!65!black}
\colorlet{mygreen}{green!60!black}
\colorlet{mydarkred}{red!50!black}
\colorlet{mydarkblue}{blue!40!black}
\tikzstyle{measure}=[{Latex[length=4,width=3]}-{Latex[length=4,width=3]},line width=0.4,mydarkblue]
\tikzstyle{ground}=[preaction={fill,top color=black!10,bottom color=black!5,shading angle=20},
                    fill,pattern=north east lines,draw=none,minimum width=0.3,minimum height=0.6]
\tikzstyle{mass}=[line width=0.6,red!30!black,fill=red!40!black!10,rounded corners=1,
                  top color=red!40!black!20,bottom color=red!40!black!10,shading angle=20]
\tikzstyle{rope}=[brown!70!black,line width=1.2,line cap=round] %very thick
\tikzstyle{force}=[->,myred,thick,line cap=round]
\tikzstyle{unit}=[->,mygreen,thick,line cap=round]
\tikzset{fieldlines/.style={mydarkred,decoration={markings,mark=at position #1 with {\arrow{latex}}},
                            postaction={decorate},line width=0.7},
         fieldlines/.default=0.55}
\newcommand{\vbF}{\vb{F}}
\begin{figure}[H]
  \centering
  \begin{tikzpicture}[scale=2]
    \def\W{2.4}   % ground width
    \def\D{0.2}   % ground depth
    \def\H{1.5}   % height
    \def\E{0.35}  % Earth's radius
    \def\R{1.3}   % field line max. radius
    \def\N{10}    % number of field lines
    \def\rang{45} % angle or r axis

    % FIELD LINES
    \foreach \i [evaluate={\ang=\i*360/\N;}] in {1,...,\N}{
      \draw[fieldlines=0.6] (\ang:\R) -- (\ang:\E); % field lines
    }
    \node[mydarkred,right] at (-0.6,-1.6) {$\vec g = -G \cdot \dfrac{m_t}{R_t^2} \cdot \hat{r}$};
    \draw[->] (\rang:0.44*\R) --++ (\rang:0.4*\R) node[anchor=\rang+170,inner sep=1] {$r$};

    % EARTH
    \fill[blue!70!black!70] (0,0) circle(\E);
    \begin{scope}[rotate=-11]
      \clip (0,0) circle(\E);
      \fill[white] (0,\E) ellipse({0.6*\E} and {0.15*\E});
      \fill[white] (0,-\E) ellipse({0.8*\E} and {0.08*\E});
      \fill[green!70!black!60,rotate=-30] (160:1.1*\E) ellipse({0.2*\E} and {0.8*\E});
      \fill[green!70!black!60,rotate=40] (-10:1.14*\E) ellipse({0.2*\E} and {0.9*\E});
      \fill[green!60!black!60,very thick,rotate=-20] % Australia
        (230:0.86*\E) ellipse({0.25*\E} and {0.18*\E});
    \end{scope}
    \draw[line width=0.3] (0,0) circle(\E);
  \end{tikzpicture}
  \caption{Campo gravitazionale terrestre}
  \label{fig:campo_gravitazionale_terrestre}
\end{figure}

\noindent
che è esattamente lo stesso concetto di di campo elettrico, ovvero la forza per unità di carica (nel caso di un campo gravitazionale, la carica è proprio la massa): sapendo la carica, e qui la massa, è possibile conoscere la forza associata, in quanto è noto il campo, ossia la forza per unità di carica/massa.

\vspace{1em}
\noindent
\textbf{Osservazione}: Si osservi che il \textbf{campo gravitazionale è additivo}: infatti, date due masse identiche $m_1$ e $m_2$, poste vicine l'una all'altra, le linee di campo dell'una si fondono con quelle dell'altra, dando vita a delle deformazioni delle linee di campo.\\
Si osservi che, ovviamente, le linee di campo sono una convenzione grafica atta a rappresentare qualitativamente il fenomeno attrattivo: in ogni punto dello spazio vi sono dei vettori che dovrebbero essere rappresentati integralmente, ma per ovvie ragioni di comprensione, si preferisce la più elegante soluzione delle linee di campo.\\
Di fatto si ha che il campo gravitazionale risultante è
\[\vec{g} = \vec{g}_1 + \vec{g}_2\]

\vspace{1em}
\subsection{Massa gravitazionale e inerziale}
Dalla $2^a$ legge della dinamica
\[\vec{F} = m_I \cdot \vec{a}\]
si evince come la massa $m_I$ sia da considerarsi com l'\textbf{inerzia}, ovvero la \textbf{resistenza alla variazione di moto} (e quindi di velocità).\\
Mentre dalla legge di gravitazione universale
\[\vec{F} = - G \cdot \frac{m_1 \cdot m_G}{r^2} \cdot \vec{r}\]
si ha che $m_G$ è una sorta di \quotes{carica gravitazionale}; tuttavia $m_I$ e $m_G$ non sono da reputarsi il medesimo concetto dal punto di vista teorico: si consideri, per esempio, il caso di due particelle, una poco massiva, ma molto carica e l'altra più massiva, che presenta una carica elettrica pressoché nulla, tale che la forza che subiscono sia completamente slegata dalla loro massa; in generale, infatti, non vi è correlazione fra le due masse, per cui, nel caso di una caduta libera, si dovrebbe avere che
\[m_I \cdot \vec{a} = - G \cdot \frac{m_1 \cdot m_G}{r^2} \cdot \vec{r}\]
in cui, isolando l'accelerazione si ottiene
\[a_y = -G \cdot \frac{m_1}{r^2} \cdot \left(\frac{m_G}{m_I}\right)\]
in cui non è possibile, in linea teoria, confondere $m_I$ con $m_G$. Tuttavia, dal punto di vista sperimentale, sono state eseguite delle misurazioni con un altissimo livello di precisione che hanno confermato come
\[\frac{m_G}{m_I} = 1\]
per cui le due masse sono da reputarsi identiche. Anche nella teoria della relatività generale tale risultato trova significato: in essa, infatti, non si considera un'attrazione gravitazionale, ma solamente una deformazione spazio-temporale, per cui tutto il sistema è non-inerziale.

\vspace{1em}
\subsection{Corpi estesi}
Il campo gravitazionale è una quantità additiva, per cui per conoscere la forza di attrazione di un corpo non puntiforme, è sufficiente scomporlo in punti e sommare il contributo di ciascuno per pervenire al risultato cercato.\\
In altri termini si ha che
\[\boxed{\vec{g} = \sum_i \vec{g}_i = \sum_i - G \cdot \frac{m_i}{r_i^2} \cdot \hat{r} = -G \cdot \int \frac{dm}{r^2} \cdot \hat{r}}\]
in cui si deve integrare all'infinito il contributo di ciascuna punto.

\vspace{1em}
\noindent
\textbf{Esempio}: Se si considera una \textbf{distribuzione con simmetria sferica} (ovvero una sfera in cui la densità non deve necessariamente essere uniforme), allora tutta la massa è da reputarsi concentrata al centro della sfera, in quanto tale interpretazione è perfettamente equivalente alla somma infinita di tutti i contributi.

\vspace{1em}
\noindent
\textbf{Osservazione}: Si osservi che il numero di giri compiuti nell'unità di tempo è, ovviamente, data da
\[\frac{1}{T}\]
Inoltre, si ha che la forza centripeta descrive la natura della forza, non il tipo di forza, per cui la forza centripeta può essere causata da qualsiasi tipo di forza.\\
Si osservi che un punto materiale si muove con velocità di modulo costante lungo un'ellisse, allora, l'accelerazione punta sempre verso il centro dell'elisse, in quanto il moto considerato è lo stesso di un moto circolare uniforme, ma visto da un sistema di riferimento a velocità costante (per cui l'accelerazione è nulla).\\
Naturalmente, i corpi liberi su una vettura in decelerazione cadono in avanti perché mantengono la stessa velocità che avevano prima della decelerazione, mentre la pseudo-forza di Coriolis ha effetto solamente su grandi distanza, come si può evincere dalla seguente formula:
\[\vec F_c = -2 m  \vec \omega \times \vec v\]
ricordando che
\[\vec \omega = \frac{\vec v}{R}\]

\newpage
\noindent
\begin{center}
  21 Marzo 2022
\end{center}
La forza di attrazione tra due corpi è direttamente proporzionale alla massa dei due corpi e inversamente proporzionale alla distanza fra i due (e tale concetto si trasla anche alla teoria della carica, per la \textbf{Forza di Coulomb}, in cui al posto della massa si deve considerare la carica di una particella e in questo caso la carica può essere positiva o negativa, ottenendo forze attrattive o repulsive). La formula per il calcolo della forza di attrazione gravitazionale presenta un $-$ in quanto si tratta di una forza attrattiva, ovviamente.\\
Il \textbf{campo gravitazionale} è definita come la forza per unità di massa: è, quindi, una vera e propria forza che si applica indistintamente ad ogni corpo, indipendentemente dalla massa di ogni corpo.\\
La differenza tra massa gravitazionale e massa inserziale è puramente concettuale, anche se nella pratica tali masse coincidono; infine, per quanto concerne i corpi estesi, al fine di considerare la forza attrattiva totale, è sufficiente sommare i contributi di tutte le masse; nel caso di un corpo con simmetria sferica, invece, è sufficiente considerare che la massa sia idealmente concentrata nel centro della sfera stessa, un'agevolazione che permette anche di semplificare i calcoli da svolgere.

\vspace{1em}
\subsection{Leggi di Keplero}
Kelpero, impiegando dei dati molto precisi, è riuscito a formulare delle valutazioni estremamente importanti sull'orbita dei pianeti: egli, infatti, ha scoperto che l'orbita dei pianeti è ellettica, in cui il sole si trova in uno dei due fuochi.\\
Non solo, Keplero, osservando il moto degli astri, è riuscito anche a formulare due leggi che spiegano la cinetica del moto dei pianeti, attraverso calcoli molto complessi e basandosi su risultati sperimentali. Di seguito si espongono le tre fondamentali leggi di Keplero:

% Tabella per le definizione di concetti, etc...
\vspace{1em}
\rowcolors{1}{black!5}{black!5}
\setlength{\tabcolsep}{14pt}
\renewcommand{\arraystretch}{2}
\noindent
\begin{tabularx}{\textwidth}{@{}|P|@{}}
    \hline
    {\textbf{LEGGI DI KEPLERO}}\\
    \parbox{\linewidth}{Le leggi di Keplero sono:
    \begin{enumerate}
      \item \emph{Tutti i pianeti si muovono lungo orbite ellittiche di cui il sole occupa uno dei fuochi};
      \item \emph{La congiungente di un pianeta con il Sole spazza aree uguali in intervalli di tempo uguali} (ciò significa che i pianeti, quando sono più lontani dal sole, ruotano più lentamente rispetto a quando si trovano nelle sue prossimità);
      \item \emph{Il quadrato del periodo di un qualunque pianeta è proporzionale al cubo della distanza media del pianeta dal Sole}.
    \end{enumerate}
    \vspace{1mm}}\\
    \hline
\end{tabularx}

\vspace{2em}
\noindent
La terza legge di Keplero può essere facilmente giustificata alla luce della legge di gravitazione universale di Newton.\\
Se, infatti, si considera un'orbita circolare (anziché ellittica) è possibile confondere la forza di attrazione gravitazionale $\vec F_a$ con la forza centripeta $\vec F_c$, per cui si ottiene
\[\vec F_c = \vec F_a \longrightarrow m_P \cdot \frac{v^2}{d} = G \cdot \frac{m_P \cdot m_S}{d^2}\]
o anche
\[\vec F_c = \vec F_a \longrightarrow m_P \cdot \omega^2 \cdot d = G \cdot \frac{m_P \cdot m_S}{d^2}\]
in cui, ovviamente, $m_P$ è la massa del pianeta, mentre $m_S$ è la massa del sole e $d$ è la distanza tra il pianeta e il sole. Da questo si ottiene che:
\[\frac{1}{\omega^2} = \frac{1}{G \cdot m_S} \cdot d^3\]
Ma siccome $\omega$ è la velocità angolare, ovvero il rapporto fra un giro completo e il periodo di rotazione, è facile capire che:
\[\frac{T^2}{(2\pi)^2} = \frac{1}{G \cdot m_S} \cdot d^3\]
Allora isolando il quadrato del periodo si ottiene:
\[\boxed{T^2 = \frac{(2 \pi)^2}{G \cdot m_S} \cdot d^3}\]

\vspace{1em}
\noindent
\textbf{Esercizio}: Un'orbita geostazionaria è l'orbita di un satellite attorno alla Terra che presenta un periodo di rotazione identico a quello della terra, per cui si mantiene sempre sopra lo stesso punto sopra la terra: secondo la terza di legge di Keplero, esiste una sola altezza tale per cui l'orbita di un satellite possa essere geostazionaria.\\
Si calcoli, allora, la velocità alla quale deve ruotare un satellite (idealmente sulla superficie della Terra) affinché la sua orbita sia geostazionaria. È facile capire che tale velocità si calcola come segue
\[v = \frac{\Delta s}{\Delta t} = \frac{2 \pi r}{\sqrt{\dfrac{(2 \pi)^2}{g \cdot r^2} \cdot r^3}} = r \cdot \sqrt{\frac{g}{r}} = \sqrt{g \cdot r}\]
ricordando che
\[g=\frac{G \cdot m_t}{R_t^2} \longrightarrow G \cdot m_t = g \cdot R_t^2\]
Ecco che allora la velocità cercata è
\[\boxed{v = \sqrt{g \cdot r}}\]
Si possono, naturalmente, anche considerare le due uguaglianze seguenti:
\[\boxed{\frac{v^2}{R_t} = g} \hspace{1em} \text{e} \hspace{1em} \boxed{\omega^2 \cdot R_t = g}\]
Ovviamente, per determinare il periodo di rotazione di un satellite attorno alla terra è sufficiente impiegare la terza legge di Kelpero, per cui
\[\boxed{T=\sqrt{\frac{(2\pi)^2}{g} \cdot d}}\]

\vspace{2em}
\noindent
\textbf{Osservazione}: Si osservi che il periodo dell'orbita per diversi pianeti risula essere molto simile; in particolare è noto che, per un qualsiasi pianeta, dalla terza legge di Keplero si ottiene che
\[d^3 = \frac{G \cdot m_S}{(2\pi)^2} \cdot T^2\]
ma più in generale, per un qualsiasi pianeta si ottiene che
\[T = 2\pi \cdot \sqrt{\frac{R_P^3}{G \cdot m_P}}\]
in cui, sotto la radice, compare il rapporto
\[\frac{R_P^3}{m_P}\]
che è esattamente il reciproco della densità del pianeta, essendo il volume di una sfera
\[V_S = \frac{4}{3} \cdot \pi \cdot R^3\]
da cui si evince che
\[T = 2\pi \cdot \sqrt{\frac{3}{4\pi} \cdot \frac{1}{\rho}}\]
in cui, naturalmente
\[\rho = \frac{m}{V}\]
Pertanto si ottiene che tutti i pianeti che sono stati considerati presentano lo stesso periodo di orbita in quanto hanno tutti la \textbf{stessa densità} (in quanto formati da ghiaccio, rocce, acqua, etc.).

\vspace{1em}
\noindent
\textbf{Osservazione}: Se fosse possibile tagliare la terra in due parti uguali e si lasciasse cadere una massa in tale foro, allora la massa accelerebbe fino a raggiungere il centro della terra, dove la forza di attrazione si annullerebbe; pertanto, la massa continuerebbe a scendere venenedo progressivamente decelerata fino a raggiungere velocità nulla al culmine della sua corsa dall'altra parte del pianeta, venendo nuovamente attratta al centro e così via, fino a quando non si viene a creare un moto oscillatorio di periodo esattamente pari al periodo di orbita attorno alla Terra.

\vspace{1em}
\noindent
\textbf{Esercizio}: Si consideri una macchina che cerca di accelerare su diverse superfici e avente diverse configurazioni di trazione.\\
Allora se il coefficiente di attrito statico è $\mu_s=1.0$ e la macchina presenta solo due ruote trainanti, l'accelerazione della vettura deve essere
\[a \leq \frac{\mu_s}{2} \cdot g = 4.9 \text{ m/s}^2\]
e così via, variando il coefficiente di attrito statico e la distribuzione del peso.

\vspace{1em}
\noindent
\textbf{Esercizio}: Si consideri un'ascensore sospeso da un cavo, avente massa $1.7 \times 10^3$ kg.\\
Allora, richiamando la seconda legge della dinamica, si può scrivere:
\[\vec F_T + F_t = m \cdot \vec a\]
e sostiutendo a tali vettori i moduli, si ottiene
\[F_T = m \cdot (a + g)\]
Taluna è la formula che si può adattare a qualsiasi scenario al fine di calcolare la forza di tensione sul cavo: se essso si muove di moto rettilineo uniforme, se accelera o decelera.

\newpage
\noindent
\begin{center}
  22 Marzo 2022
\end{center}
\section{Energia}
L'energia è la grandezza fisica che misura la capacità di un corpo o di un sistema fisico di compiere lavoro, a prescindere dal fatto che tale lavoro sia o possa essere effettivamente svolto.\\
Di seguito si espongono diverse forme di energia e ne viene analizzata nel dettaglio il significato fisico.

\vspace{1em}
\subsection{Energia cinetica}
L'\textbf{energia cinetica} (originariamente chiamata \emph{vis viva}) venne teorizzata da Leibniz e da Decartes: il primo asseriva che l'energia cinetica era direttamente proporzionale alla massa e al quadrato della velocità (ovvero $m v^2$), mentre il secondo considerava soltanto la massa e la velocità e non il suo quadrato (ovvero $m v$, che prende il nome di \emph{quantità di moto}).\\
Grazie a Thomas Young, il concetto fisico teorizzato da Leibniz prese il nome di \textbf{energia} ed infine Gustave Gasparre Coriolis gli attribuì il nome di \textbf{energia cinetica}, definita come segue:

% Tabella per le definizione di concetti, etc...
\vspace{1em}
\rowcolors{1}{black!5}{black!5}
\setlength{\tabcolsep}{14pt}
\renewcommand{\arraystretch}{2}
\noindent
\begin{tabularx}{\textwidth}{@{}|P|@{}}
    \hline
    {\textbf{ENERGIA CINETICA}}\\
    \parbox{\linewidth}{L'\textbf{energia cinetica} viene calcolata come segue:
    \[\boxed{K=\frac{1}{2}mv^2}\]
    per quanto riguarda un solo punto materiale di massa $m$. Se, invece, si considera un agglomerato di punti materiali, allora, essendo l'\textbf{energia additiva}, si ottiene
    \[\boxed{K=\frac{1}{2} \sum_{i} m_i \cdot v_i^2}\]
    Naturalmente, l'energia cinetica è uno scalare (non può avere una direzione), in cui $v^2=\vert \vec v \vert^2 = \vec v \cdot \vec v $.\vspace{3mm}}\\
    \hline
\end{tabularx}

\vspace{2em}
\noindent
\textbf{Esempio}: Si consideri una macchina di massa $m=1000$ kg e avente una velocità di $v_1=50$ km/h. Allora l'energia cinetica si calcola come segue:
\[K_1=\frac{1}{2}m v_1^2 = \frac{1}{2} \cdot 1000 \text{ kg} \cdot \left( \frac{50}{3.6} \text{ m/s} \right)^2 = 96.5 \text{ kJ}\]
in cui
\[\boxed{1 \text{ J} = \frac{\text{kg m}^2}{\text{s}^2}}\]
Mentre se la velocità della vettura è $v_2=60$ km/h allora l'energia cinetica della macchina diventa:
\[K_2=\frac{1}{2}m v_2^2 = \frac{1}{2} \cdot 1000 \text{ kg} \cdot \left( \frac{60}{3.6} \text{ m/s} \right)^2 = 139 \text{ kJ}\]
che significa che l'aumento di velocità di soli $10$ km/h comporta un aumento del $50\%$ dell'energia e, quindi, della dissipazione di energia in calore in caso di frenata (aumento dello spazio di frenata).

\newpage
\noindent
\subsection{Lavoro}
Di seguito si espone la definizione di \textbf{lavoro}, basandosi sulla Figura \ref{fig:lavoro_compiuto_molla_non_parallela_spostamento}:

\begin{figure}[H]
  \colorlet{myred}{red!65!black}
  \colorlet{mydarkblue}{blue!30!black}
  \colorlet{xcol}{blue!70!black}
  \colorlet{vcol}{green!70!black}
  \colorlet{acol}{red!50!blue!80!black!80}
  \tikzstyle{ground}=[preaction={fill,top color=black!10,bottom color=black!5,shading angle=20},
                      fill,pattern=north east lines,draw=none,minimum width=0.3,minimum height=0.6]
  \tikzstyle{mass}=[line width=0.6,red!30!black,fill=red!40!black!10,rounded corners=1,
                    top color=red!40!black!20,bottom color=red!40!black!10,shading angle=20]
  \tikzstyle{vector}=[->,very thick,xcol,line cap=round]
  \tikzstyle{force}=[->,myred,thick,line cap=round]
  \tikzstyle{Fproj}=[force,myred!40]
  \tikzstyle{mydashed}=[dash pattern=on 2pt off 2pt]
  \tikzstyle{smallarrow}=[{Latex[length=2,width=2]}-{Latex[length=2,width=2]}]
  \def\tick#1#2{\draw[thick] (#1) ++ (#2:0.1) --++ (#2-180:0.2)}

  \centering
  \begin{tikzpicture}[scale=2]
    \def\W{2.7}  % ground width
    \def\D{0.2}  % ground depth
    \def\h{0.8}  % mass height
    \def\w{1.0}  % mass width
    \def\F{1.1}  % force magnitude
    \def\ang{30} % angle force
    \coordinate (F0) at (0.4*\w,0.85*\h);
    \coordinate (Fx) at ($(F0)+({\F*cos(\ang)},0)$);
    \coordinate (F)  at ($(F0)+(\ang:\F)$);
    \draw[ground] (-0.3*\W,0) rectangle++ (\W,-\D);
    \draw (-0.3*\W,0) --++ (\W,0);
    \draw[mass] (-\w/2,0) rectangle++ (\w,\h) node[midway] {$m$};
    \draw[force,xcol] (\w/2,0.15*\h) --++ (0.4*\W,0) node[midway,above=0] {$\vb*{\Delta r}$};
    \draw[dashed,myred!80!black!60] (Fx) -- (F);
    \draw[Fproj] (F0) -- (Fx) node[above=1,right=0] {$F\cos\theta$}; %\vu{x}
    \draw[force] (F0) -- (F)  node[above=1,right=-1] {$\vbF$};
    \draw pic["$\theta$",draw=black,angle radius=14,angle eccentricity=1.4] {angle=Fx--F0--F};
  \end{tikzpicture}
  \caption{Lavoro compiuto da una forza non parallela allo spostamento}
  \label{fig:lavoro_compiuto_molla_non_parallela_spostamento}
\end{figure}

% Tabella per le definizione di concetti, etc...
\vspace{1em}
\rowcolors{1}{black!5}{black!5}
\setlength{\tabcolsep}{14pt}
\renewcommand{\arraystretch}{2}
\noindent
\begin{tabularx}{\textwidth}{@{}|P|@{}}
    \hline
    {\textbf{LAVORO}}\\
    \parbox{\linewidth}{Quando una forza $\vec F$ viene applicata su un corpo e produce uno spostamento $\Delta \vec r$, allora si ha che la forza ha compiuto un lavoro, calcolato come
    \[\boxed{W = \vec F \cdot \Delta \vec r \cdot \cos(\theta)}\]
    che significa che
    \begin{itemize}
      \item Se $\vec F$ è parallelo a $\Delta \vec r$, allora il lavoro è massimo.
      \item Se $\vec F$ è perpendicolare a $\Delta \vec r$, allora il lavoro è nullo (come nel moto circolare uniforme).
      \item Se $\vec F$ presenta verso opposto rispetto a $\Delta \vec r$, allora il lavoro è negativo (si ha sottrazione di energia al sistema, come per l'attrito).
    \end{itemize}
    Naturalmente, nel caso in cui vi siano più forze agenti sul sistema, ciascuna delle quali produce uno spostamento, si ottiene che
    \[W_{tot} = \vec F_1 \cdot \Delta \vec r_1 + \vec F_2 \cdot \Delta \vec r_2 + ... = \sum_{i} F_i \cdot \Delta \vec r_i\]
    Infine, se il corpo non ruota e non si deforma, tutti gli spostamenti sono uguali, ovvero $\Delta \vec r_1 = \Delta \vec r_2 = ... + \Delta \vec r_n$, e quindi il lavoro totale sarà dato solo dal prodotto scalare tra la somma delle forze e il singolo spostamento:
    \[W_{tot} = \sum_{i} F_i \cdot \Delta \vec r\]
    \vspace{-1mm}}\\
    \hline
\end{tabularx}

\vspace{2em}
\noindent
\textbf{Osservazione}: Si osservi che un astronauta in orbita attorno alla terra \textbf{ha peso}, in quanto è grazie a tale forza che risce a mantenere la propria orbita, altrimenti continuerebbe il proprio moto in linea retta.\\
È facile, inoltre, determinare il raggio di orbita geostazionaria (partendo dal centro della Terra), calcolabile come:
\[d=\frac{T^2 \cdot g \cdot R_t^2}{(2 \pi)^2} = 42 221 \text{ km}\]

\vspace{1em}
\subsection{Teorema lavoro-energia cinetica}
Si osservi che dalla cinetica, per quanto riguarda il moto uniformemente accelerato, è nota la formula seguente:
\[v^2 - v_0^2 = 2a \cdot (x-x_0)\]
Ma se ambo i membri si moltiplicano per $\frac{1}{2}m$ si ottiene
\[\frac{1}{2} m v^2 - \frac{1}{2} m v_0^2 = ma \cdot (x-x_0)\]
che corrisponde a
\[K - K_0 = F \cdot (x-x_0) \longrightarrow \Delta K = F \cdot \Delta x \longrightarrow \Delta K = W\]
ovvero si è ottenuto il \textbf{lavoro compiuto sul sistema da una forza esterna (trasferimento di energia verso (o da) il sistema)}. In altre parole, quando una forza viene applicata su un sistema, si assiste ad un trasferimento di energia (come quando una forza applicata ad un corpo ne aumenta la velocità, la forza ha compiuto un lavoro sul sistema che si è tramutato in aumento di energia cinetica).\\
Più in generale, è possibile affermare dalla seconda legge della dinamica applicata ad un punto materiale di massa $m$ che
\[\sum \vec F = m \cdot \vec a\]
considerando, ora, uno \textbf{spostamento infinitesimale} $d \vec r$ (che è da interpretarsi come $\Delta \vec r \rightarrow 0$), si può calcolare il \textbf{lavoro infinitesimale} compiuto dalla forza sul punto materiale per determinare tale spostamento:
\[dW = \left(\sum \vec F\right) \cdot d \vec r = m \vec a \cdot d \vec r\]
ma dalla cinetica è anche noto che
\[\vec v = \frac{d \vec r}{d t} \longrightarrow d \vec r = \vec v \cdot dt\]
che significa che in intervallo di tempo infinitesimale $dt$, un punto materiale alla velocità $\vec v$ compie uno spostamento infinitesimale $d \vec r$. Alla luce di tale evidenza, la formula ottenuta precedentemente diviene:
\[m \vec a \cdot d \vec r = m \vec a \cdot \vec v dt\]
Ma calcolando la derivata della velocità al quadrato si ottiene
\[\frac{d}{dt} \vec v^2 = \frac{d}{dt} \left(\vec v \cdot \vec v\right) = \vec a \cdot \vec v + \vec v \cdot \vec a = 2 \cdot \vec a \cdot \vec v\]
Alla luce di tale risultato si può quindi scrivere
\[m \vec a \cdot \vec v dt = \frac{1}{2}m \cdot \frac{d}{dt} (\vec v)^2\]
ovvero
\[\boxed{dW = \frac{1}{2}m \cdot \frac{d}{dt} (\vec v)^2}\]
Avendo ottenuto la formula per il calcolo del lavoro infinitesimale $dW$, è possibile eseguire una somma infinita di tali contributi per determinare il lavoro compiuto per effettuare uno spostamento maggiore, come mostrato di seguito:
\[W_{tot} = \int_i^f dW = \int_{t_i}^{t_f} \left(\sum \vec F\right) \cdot d \vec{r} = \int_{t_i}^{t_f} \frac{1}{2}m \cdot \frac{d}{dt} (\vec v)^2 \cdot dt = \frac{1}{2}m \cdot \left(v_f^2 - v_i^2\right)\]
per cui la formula finale ottenuta è proprio quanto ci si aspettava:
\[\boxed{W_{tot} = \frac{1}{2}m \cdot \left(v_f^2 - v_i^2\right)}\]

\vspace{1em}
\noindent
\textbf{Osservazione}: Il \textbf{teorema lavoro-energia cinetica} è una \textbf{conseguenza diretta} della seconda legge di Newton
\[F = m \vec a\]
e si applica ai punti materiali (ma non ad un sistema più complesso).\\
Si osservi che tale risultato non è equivalente al teorema di \textbf{conservazione dell'energia}, anche se fornisce risultati uguali in importanti casi.\\
È significativo osservare che il lavoro totale $W_{tot}$ così calcolato non corrisponde al \quotes{lavoro} in senso termodinamico. In questo caso si ha che
\[W_{tot} = \Delta K\]
da cui si può capire come il lavoro di una \textbf{forza costante} si possa calcolare come segue:
\[\boxed{W = \int_i^f \vec F \cdot d \vec r = \vec F \cdot (\vec r_f - \vec r_i) = \vec F \cdot \vec l}\]
in cui $\vec l$ indica lo \textbf{spostamento totale}, il quale non dipende assolutamente dal percorso compiuto.

\vspace{1em}
\noindent
\textbf{Esercizio}: Si consideri una vettura di massa $m=1000$ kg che viagga ad una velocità $v_0=50$ km/h. Si calcoli la distanza di arresto, considerando che la frenatura è resa possibile dalla forza di attrito cinetico $\vec F_k$.\\
Naturalmente è ovvio che $F_N=F_t=mg$, mentre $F_k=\mu_k F_N=\mu_k mg$. Allora il lavoro compiuto dalla forza di attrito è
\[W = \int_i^f \vec F_k \cdot d \vec r\]
ed essendo la forza di attrito di verso opposto al moto, il lavoro $W$ avrà segno negativo.\\
Pertanto si ottiene
\[W = \int_i^f \vec F_k \cdot d \vec r = \int_0^d -F_k \cdot dx = -F_k \cdot d = -\mu m g d\]
E richiamando il teorema lavoro-energia cinetica, si ottiene che
\[W=\Delta K \longrightarrow -\mu_k m g d = K_f-K_i=-\frac{1}{2}mv_i^2\longrightarrow d=\frac{v_i^2}{2\mu_k g}\]
e com'era da aspettarsi, lo spazio di frenata aumenta con il quadrato della velocità.

\vspace{1em}
\noindent
\textbf{Esercizio}: Si consideri un proiettile che viene sparato in aria, in direzione verticale, con velocità $\vec v_i$ sufficiente per raggiungere l'altezza di $h$ in cui l'accelerazione gravitazionale non può essere considerata costante.\\
Naturalmente, dalla legge di gravitazione universale si ottiene che
\[\vec F = - G \cdot \frac{m_t \cdot m}{(R_t + h)^2} \cdot \hat{j}\]
Allora per il calcolo del lavoro si può procedere come segue:
\[W=\int \vec F \cdot d \vec r = \int -G \cdot \frac{m_t \cdot m}{(R_t + h)^2} \cdot dh\]
Naturalmente, in questo caso, la forza non è costante, per cui bisogna calcolare l'integrale, posto $h'=R_t+h$
\[W=\int \vec F \cdot d \vec r = \int -G \cdot \frac{m_t \cdot m}{(R_t + h)^2} \cdot dh = \int_{R_t}^{R_t+h} - \frac{G m_t m}{{h'}^2} \cdot dh' = G m_t m \cdot \left(\frac{1}{R_t+h}-\frac{1}{R_t}\right)\]
in cui si ottiene che
\[\Delta K = W_{tot} \longrightarrow 0 - \frac{1}{2}m \cdot v_i^2 = G m_t m \cdot \left(\frac{1}{R_t+h}-\frac{1}{R_t}\right)\]
in cui, isolando $h$ si ottiene
\[h = \frac{1}{2} \cdot \frac{R_t^2 \cdot v_i^2}{G m_t} \cdot \left(1 - \frac{1}{2} \cdot \frac{R_t \cdot v_i^2}{G m_t} \right)^{-1} \longrightarrow h = \frac{1}{2} \cdot \frac{v_i^2}{g} \cdot \left(1 - \frac{1}{2} \cdot \frac{v_i^2}{g R_t} \right)^{-1}\]
in cui appare evidente come l'altezza dipenda dal quadrato della velocità; inoltre se il secondo termine si annulla, con una velocità sufficiente, essendo valutato il regiproco, l'altezza diviene idealmente infinita e quindi il proiettile esce dall'orbita terrestre.

\newpage
\noindent
\begin{center}
  23 Marzo 2022
\end{center}
\subsection{Lavoro compiuto da una forza variabile}
Il lavoro compiuto da una forza variabile si ha quando la forza non è costante, ma dipende dalla distanza: in altre parole, sussiste una \textbf{relazione lineare tra forza e distanza}:


\vspace{2em}
\noindent
\rowcolors{1}{white}{white}
\begin{tabularx}{\textwidth}{P}
  {
      \centering
      \begin{tikzpicture}
        \begin{axis}[
          grid=both,
          axis lines = middle,
          xlabel = \(x\),
          ylabel = {\(F_x\)},
          legend pos=outer north east,
          ymajorgrids=true,
          xmajorgrids=true,
          grid style=dashed,
        ]
      \addplot[
        domain=-5:5,
        samples=100,
        color=orange,
      ]
      {-x};
      \end{axis}
      \end{tikzpicture}
    }
\end{tabularx}

\noindent
Per cui si perviene al risultato seguente:
\[\boxed{F_x=-kx}\]
la quale prende il nome di \textbf{legge di Hooke}, in cui $k$ è la costante di proporzionalità. Tipicamente, tale legge viene impiegata per descrivere la \textbf{forza di una molla}: quando lo spostamento è positivo, la forza è negativa (si chiama, infatti, \textbf{forza di richiamo}), e viceversa.

\begin{figure}[H]
  \centering
  \colorlet{xcol}{blue!70!black}
  \colorlet{darkblue}{blue!40!black}
  \colorlet{myred}{red!65!black}
  \tikzstyle{mydashed}=[xcol,dashed,line width=0.25,dash pattern=on 2.2pt off 2.2pt]
  \tikzstyle{axis}=[->,thick] %line width=0.6
  \tikzstyle{ell}=[{Latex[length=3.3,width=2.2]}-{Latex[length=3.3,width=2.2]},line width=0.3]
  \tikzstyle{dx}=[-{Latex[length=3.3,width=2.2]},darkblue,line width=0.3]
  \tikzstyle{ground}=[preaction={fill,top color=black!10,bottom color=black!5,shading angle=20},
                      fill,pattern=north east lines,draw=none,minimum width=0.3,minimum height=0.6]
  \tikzstyle{mass}=[line width=0.6,red!30!black,fill=red!40!black!10,rounded corners=1,
                    top color=red!40!black!20,bottom color=red!40!black!10,shading angle=20]
  \tikzstyle{spring}=[line width=0.8,blue!7!black!80,snake=coil,segment amplitude=5,segment length=5,line cap=round]
  \tikzset{>=latex} % for LaTeX arrow head
  \tikzstyle{force}=[->,myred,very thick,line cap=round]
  \def\tick#1#2{\draw[thick] (#1)++(#2:0.1) --++ (#2-180:0.2)}

  \begin{tikzpicture}[scale=1.5]
    \def\H{1.1} % wall height
    \def\T{0.3} % wall thickness
    \def\W{3.9} % ground length
    \def\D{0.2} % ground depth
    \def\h{0.7} % mass height
    \def\w{0.8} % mass width
    \def\x{2.0} % mass x position
    \def\dx{0.9} % extension
    \def\y{1.22*\H} % x axis y position
    \def\F{0.8} % force

    % AXIS
    \draw[mydashed] (\x,0) --++ (0,\y) (\x-\dx,0) --++ (0,1.1*\y);
    \draw[axis] (\x-0.4*\W,\y) -- (\x+0.4*\W,\y) node[right] {$x$};
    \tick{\x,\y}{-90} node[scale=0.8,above=0] {$0$};
    \draw[ell] (0,1.3*\h) --++ (\x,0) node[pos=0.4,fill=white,inner sep=0] {$\ell_0$};
    \draw[dx] (\x,1.6*\h) --++ (-\dx,0)
      node[pos=0.45,fill=white,inner sep=0,scale=0.9] {$x$};

    % SPRING & MASS
    \draw[spring,segment length=2.9] (0,\h/2) --++ (\x-\dx,0);
    \draw[ground] (0,0) |-++ (-\T,\H) |-++ (\T+\W,-\H-\D) -- (\W,0) -- cycle;
    \draw (0,\H) -- (0,0) -- (\W,0);
    \draw[mass] (\x-\dx,0) rectangle++ (\w,\h) node[midway] {$m$};
    \draw[force] (\x-\dx+0.8*\w,0.8*\h) --++ (\F,0) node[below=0,right=0] {$\vb{F}$};

  \end{tikzpicture}
  \caption{Fisica di una molla}
  \label{fig:fisica_molla}
\end{figure}

\noindent
Volendo calcolare il lavoro compiuto da tale forza, è sufficiente considerare l'integrale della forza risultante moltiplicata per lo spostamento, come mostrato di seguito:
\[\boxed{W=\int_{x_i}^{x_f}-k x \cdot dx = - \frac{1}{2}k \cdot (x_f^2-x_i^2)}\]
che costiuisce un lavoro che è \textbf{indipendente dal percorso compiuto}, ma dipende solamente dalla posizione iniziale e finale.

\newpage
\noindent
\textbf{Esempio $\boldsymbol{1}$}: Si consideri un blocco sospeso da una molla, in cui la costante di elasticità della molla è $k$, mentre il blocco è immobile:

\begin{figure}[H]
  \centering
  \colorlet{xcol}{blue!70!black}
  \colorlet{darkblue}{blue!40!black}
  \colorlet{myred}{red!65!black}
  \tikzstyle{mydashed}=[xcol,dashed,line width=0.25,dash pattern=on 2.2pt off 2.2pt]
  \tikzstyle{axis}=[->,thick] %line width=0.6
  \tikzstyle{ell}=[{Latex[length=3.3,width=2.2]}-{Latex[length=3.3,width=2.2]},line width=0.3]
  \tikzstyle{dx}=[-{Latex[length=3.3,width=2.2]},darkblue,line width=0.3]
  \tikzstyle{ground}=[preaction={fill,top color=black!10,bottom color=black!5,shading angle=20},
                      fill,pattern=north east lines,draw=none,minimum width=0.3,minimum height=0.6]
  \tikzstyle{mass}=[line width=0.6,red!30!black,fill=red!40!black!10,rounded corners=1,
                    top color=red!40!black!20,bottom color=red!40!black!10,shading angle=20]
  \tikzstyle{spring}=[line width=0.8,blue!7!black!80,snake=coil,segment amplitude=5,segment length=5,line cap=round]
  \tikzset{>=latex} % for LaTeX arrow head
  \tikzstyle{force}=[->,myred,very thick,line cap=round]
  \def\tick#1#2{\draw[thick] (#1)++(#2:0.1) --++ (#2-180:0.2)}

  \begin{tikzpicture}[scale=2]
    \def\H{0.25}     % ceiling height
    \def\W{2.6}      % ceiling width
    \def\h{0.7}      % mass height
    \def\w{0.6}      % mass width
    \def\l{0.5*\y}   % rest length without weight
    \def\dl{0.7*\y}  % rest length with weight
    \def\y{2.4}      % mass y position
    \def\xy{0.38*\W} % mass y position
    \def\F{0.8}      % force magnitude
    \draw[spring,segment length=7.2] (0,0) -- (0,-\y);
    \draw[ground] (-\W/2,0) rectangle++ (\W,\H);
    \draw (-\W/2,0) --++ (\W,0);
    \draw[axis] (-\xy,0) --++ (0,-\y-0.7*\h) node[left] {$y$};
    \draw[axis] ( \xy,0) --++ (0,-\y-0.7*\h) node[right] {$y'$};
    \draw[mydashed] (-\xy,-\l) --++ (2.3*\xy,0);
    \draw[mydashed] (-\xy,-\dl) --++ (2*\xy,0);
    \draw[mydashed] (-0.46*\W,-\y) --++ (0.92*\W,0);
    \tick{-\xy,-\l}{0} node[left] {$0$};
    \tick{-\xy,-\dl}{0} node[left] {$y_0$};
    \tick{ \xy,-\dl}{180} node[right] {$0$};
    \draw[mass] (-\w/2,-\y) rectangle++ (\w,-\h) node[midway] {$m$};
    \draw[force] (0.4*\w,-\y-0.3*\h) --++ (0,1.6*\F) node[pos=0.9,right=0] {$\vb{F}$};
    \draw[force] (0.3*\w,-\y-0.7*\h) --++ (0,-\F) node[above right=0] {$m\vb{g}$};
    \draw[ell] (0.45*\W,0) --++ (0,-\l) node[midway,right=-2] {$\ell_0$};
    %\draw[ell] (-0.4*\W,-0.75*\y) --++ (0,-0.25*\y) node[midway,left=1] {$y_0$};
  \end{tikzpicture}
  \caption{Fisica di una molla verticale}
  \label{fig:fisica_molla_verticale}
\end{figure}

\noindent
Naturalmente, le due forze che agiscono su tale massa sono la forza peso $\vec F_t$ e la forza di richiamo della molla $\vec F_m$. Inoltre, essendo il blocco immobile ($\vec a = 0$), deve essere che
\[\sum \vec F = m \vec a = 0\]
Applicando la seconda legge della dinamica, quindi, si ottiene che
\[k \cdot \Delta y = mg \longrightarrow \Delta y = \frac{mg}{k}\]

\vspace{1em}
\noindent
\textbf{Esempio $\boldsymbol{2}$}: Si consideri un carrello che si muove di velocità $\vec v_i$ e che deve arrestare la sua corsa per mezzo di una molla posta orizzontalmente al moto:

\begin{figure}[H]
  \centering
  \colorlet{xcol}{blue!70!black}
  \colorlet{darkblue}{blue!40!black}
  \colorlet{myred}{red!65!black}
  \tikzstyle{mydashed}=[xcol,dashed,line width=0.25,dash pattern=on 2.2pt off 2.2pt]
  \tikzstyle{axis}=[->,thick] %line width=0.6
  \tikzstyle{ell}=[{Latex[length=3.3,width=2.2]}-{Latex[length=3.3,width=2.2]},line width=0.3]
  \tikzstyle{dx}=[-{Latex[length=3.3,width=2.2]},darkblue,line width=0.3]
  \tikzstyle{ground}=[preaction={fill,top color=black!10,bottom color=black!5,shading angle=20},
                      fill,pattern=north east lines,draw=none,minimum width=0.3,minimum height=0.6]
  \tikzstyle{mass}=[line width=0.6,red!30!black,fill=red!40!black!10,rounded corners=1,
                    top color=red!40!black!20,bottom color=red!40!black!10,shading angle=20]
  \tikzstyle{spring}=[line width=0.8,blue!7!black!80,snake=coil,segment amplitude=5,segment length=5,line cap=round]
  \tikzset{>=latex} % for LaTeX arrow head
  \tikzstyle{force}=[->,myred,very thick,line cap=round]
  \def\tick#1#2{\draw[thick] (#1)++(#2:0.1) --++ (#2-180:0.2)}

  \begin{tikzpicture}[scale=2]
    \def\H{1}    % wall height
    \def\T{0.3}  % wall thickness
    \def\W{2.6}  % ground length
    \def\D{0.25} % ground depth
    \def\h{0.6}  % mass height
    \def\w{0.7}  % mass width
    \def\x{1.6}  % mass x position
    \draw[spring] (0,\h/2) --++ (\x/2,0);
    \draw[ground] (0,0) |-++ (-\T,\H) |-++ (\T+\W,-\H-\D) -- (\W,0) -- cycle;
    \draw (0,\H) -- (0,0) -- (\W,0);
    \draw[mass] (\x/2+0.5,0) rectangle++ (\w,\h) node[midway] {$m$};
    \draw[mass] (\x/2,0) rectangle++ (0.05,\h);
    \draw[-stealth] (\x/2+0.5,\h+0.2) -- (\x/2,\h+0.2) node[midway,above] {$\vec v_i$};
  \end{tikzpicture}
  \caption{Arresto di un carrello tramite freno a molla}
  \label{fig:arresto_carrello_freno_molla}
\end{figure}

\noindent
Si determini, allora, la costante di elasticità della molla $k$ affinché il modulo dell'accelerazione sia al massimo $a_{max}=5g$.\\
Naturalmente, se tale costante è troppo elevata, la molla si comporterà come un muro, mentre se è troppo bassa non sarà sufficiente a rallentare in tempo la corsa del mezzo.\\
Sa la molla è troppo dura, la decelerazione sarà molto forte e rapida, mentre se è troppo morbida, la sua decelerazione non sarà sufficiente ad evitare l'impatto; è chiaro che l'accelerazione sarà massima nel punto di massima compressione, mentre sarà nulla prima del punto di contatto e crescerà in modo lineare secondo la legge di Hooke:

\vspace{2em}
\noindent
\rowcolors{1}{white}{white}
\begin{tabularx}{\textwidth}{P}
  {
      \centering
      \begin{tikzpicture}
        \begin{axis}[
          grid=both,
          axis lines = middle,
          xlabel = \(x\),
          ylabel = {\(a\)},
          legend pos=outer north east,
          ymajorgrids=true,
          xmajorgrids=true,
          grid style=dashed,
          xmin=-6,
          xmax=6,
          ymin=-2,
          ymax=6,
          xtick={0,5},
          xticklabels={O,$x_{max}$},
          ytick={5},
          yticklabels={$a_{max}$},
        ]
      \addplot[
        domain=0:5,
        samples=100,
        color=blue,
        thick,
      ]
      {x};
      \draw[thick,blue] (axis cs:-5,0) -- (axis cs:0,0);
      \end{axis}
      \end{tikzpicture}
    }
\end{tabularx}

\vspace{1em}
\noindent
Dalla seconda legge della dinamica si ottiene che
\[\sum \vec F = m \vec a\]
per cui considerando solamente la componente orizzontale, si ottiene
\[F_x = m a_x \longrightarrow -k x = m a_x \longrightarrow a_x = - \frac{k}{m} x\]
Applicando, ora, il teorema lavoro-energia cinetica si ottiene che
\[W = \Delta K\]
in cui è noto che $\vec v_i = \vec v_i$ mentre $x_i = 0$, e $\vec v_f = 0$, mentre $x_f = x_{max}$ da cui:
\[W = - \frac{1}{2}k \cdot \left(x_f^2 - x_i^2\right) = - \frac{1}{2}k \cdot x_{max}^2\]
invece si ottiene che
\[\Delta K = K_f - K_i = 0 - \frac{1}{2}m v_i^2\]
Applicando infine il teorema lavoro-energia cinetica si ottiene che
\[- \frac{1}{2}k \cdot x_{max}^2 = - \frac{1}{2}m v_i^2 \longrightarrow x_{max} = \sqrt{\frac{m}{k}} \cdot v_i\]
Ecco che avendo isolato $x_{max}$ si può scrivere che
\[a_{max} = \left \vert \frac{k}{m} x_{max} \right \vert \longrightarrow a_{max} = \sqrt{\frac{k}{m}} \cdot v_i\]
Ma sapendo che $a_{max}=5g$, si può facilmente isolare $k$, ottenendo:
\[k \leq \frac{a_{max}^2}{v_{i}^2} \cdot m = \frac{25 g^2}{v_{i}^2} \cdot m\]

\vspace{1em}
\subsection{Lavoro compiuto dalla forza di gravità}
Naturalmente la forza di gravità è una forza costante, per cui il lavoro che essa compie dipende unicamente dalla posizione iniziale e finale scelta.\\
Allora, applicando la formula per il calcolo del lavoro compiuto da una forza costante si ottiene
\[W=\int_i^f \vec F \cdot d \vec r = -mg \cdot (y_f - y_i)\]
pertanto il lavoro compiuto dalla forza di gravità è
\[\boxed{W=-mg \cdot (y_f - y_i)}\]

\vspace{1em}
\noindent
\textbf{Esempio}: Si consideri una massa $m$ che viene lasciata scivolare (senza attrito) su una discesa, partendo con velocità iniziale $\vec v_i = 0$ e giungendo alla base di tale discesa con velocità finale $\vec v_f$. Si calcoli, allora, la velocità finale della massa.\\
Naturalmente, applicando il teorema lavoro-energia cinetica si ottiene che
\[W=\Delta K \longrightarrow -mg \cdot (y_f - y_i) = \frac{1}{2}m \cdot v_f^2 \longrightarrow mgh = \frac{1}{2} m v_f^2\]
Ecco allora che la velocità finale di un corpo che viene lasciato cadere da una altezza $h$, in assenza di attriti è
\[\boxed{v_f=\sqrt{2gh}}\]

\vspace{1em}
\noindent
\textbf{Osservazione}: Si osservi che il lavoro compiuto da una forza può essere sia positivo o negativo (se la forza è diretta nello stesso verso dello spostamento, oppure no).\\
Una forza che non compie mai lavoro è il campo magnetico (in quanto nell'equazione di Lorents, la forza è ortogonale alla velocità, e quindi al vettore spostamento): ciò significa che il campo magnetico non può mai accelerare delle particelle; per farlo è necessario impiegare un campo elettrico.\\
Ovviamente il lavoro è l'integrale della forza per lo spostamento, per cui rappresenta l'area della parte di piano sottesa al grafico della forza in funzione dello spostamento.

\vspace{1em}
\subsection{Forze conservative}
Di seguito si espone la definzione di \textbf{forza conservativa}:

% Tabella per le definizione di concetti, etc...
\vspace{1em}
\rowcolors{1}{black!5}{black!5}
\setlength{\tabcolsep}{14pt}
\renewcommand{\arraystretch}{2}
\noindent
\begin{tabularx}{\textwidth}{@{}|P|@{}}
    \hline
    {\textbf{FORZA CONSERVATIVA}}\\
    \parbox{\linewidth}{Una forza $\vec F$ si definisce \textbf{conservativa} se
    \[\boxed{\int_i^f \vec F \cdot d \vec r}\]
    è \textbf{indipendente dal percorso}. Analogamente si ha che il lavoro compiuto da una forza $\vec F$ conservativa su un \textbf{percorso chiuso è nullo}
    \[\boxed{\oint \vec F \cdot d \vec r = 0}\]
    che corrisponde a dire che
    \[\boxed{\int_i^f \vec F \cdot d \vec r + \int_f^i \vec F \cdot d \vec r = 0}\]
    In altri termini, si potrebbe dire che una forza è definita conservativa se esiste una funzione $\mathcal{U}$ tale che
    \[\boxed{\vec F = \nabla \mathcal{U}}\]
    Ove per $\nabla \mathcal{U}$ è da intendersi il \textbf{gradiente} di $\mathcal{U}$, ossia la derivata coalcolata in tutte le direzioni:
    \[\nabla \mathcal{U} = \frac{d \mathcal{U}}{d x} \cdot \hat{i} + \frac{d \mathcal{U}}{d y} \cdot \hat{j} + \frac{d \mathcal{U}}{d z} \cdot \hat{k}\]
    \vspace{-1mm}}\\
    \hline
\end{tabularx}

\vspace{1em}
\noindent
\textbf{Osservazione}: Si osservi che la funzione $\mathcal{U}$ impiegata nella definizione prende il nome di \textbf{energia potenziale}.

\vspace{1em}
\noindent
\textbf{Esempio}: Si osservi che un esempio molto semplice di \textbf{forza non-conservativa} è l'attrito, in quanto il lavoro compiuto da tale forza dipende necessariamente dal percorso, essendo la forza d'attrito sempre opposta allo spostamento. Pertanto si ottiene che:
\[W_k = \int_i^f F_k \cdot d \vec r = \int_i^f -\mu_k m g \cdot d \vec r = -\mu_k m g \cdot s\]
in cui $s$ è la lunghezza del percorso.

\vspace{1em}
\noindent
\textbf{Osservazione}: Si osservi che Richard Feynman, parlando di fisica fondamentale delle particelle, ha affermato che:\\\\
\quotes{\emph{Abbiamo speso un tempo considerevole per discutere le forze conservative; che cosa diremo delle forze non conservative? Approfondiremo l'argomento più di quanto non si faccia solitamente, e stabiliremo che non esistono forze non conservative! In realtà, tutte le forze fondamentali nella natura appaiono conservative. Questa non è una conseguenza delle leggi di Newton. Infatti, per quanto ne sapeva Newton, le forze avrebbero potuto essere non conservative, come apparentemente è l'attrito. Quando diciamo che l'attrito apparentemente lo è, usiamo un punto di vista moderno, essendo stato scoperto che tutte le forze elementari, le forze fra le particelle a livello fondamentale, sono conservative.}}\\\\
\hspace{10em} Richard Feynman\\\\
Per cui le forze microscopiche sono conservative, ma quando si considera un assieme, la seconda legge della termodinamica stabilirà che deve esserci calore dissipato: pertanto, anche se tutti i meccanismi fondamentali sono conservativi, vi deve essere una aumento di \textbf{entropia}. Si conclude, quindi, che il problema delle forze dissipative (e quindi non conservative), come l'attrito, è un problema prettamente termodinamico.

\newpage
\noindent
\begin{center}
  24 Marzo 2022
\end{center}
Il lavoro viene definito come l'integrale del prodotto tra la forza applicata e lo spostamento ottenuto: naturalmente il lavoro può essere positivo o negativo e può essere eseguito da qualsiasi forza.\\
Le forze conservative sono forze per le quali il lavoro non dipende dal percorso e, quindi, è possibile applicare il principio di conservazione dell'energia (in altre parole, le forze conservative sono forze capaci di immagazzinare dell'energia potenziale, potenziale per svolgere un lavoro). Esempi di forze conservative sono:
\begin{itemize}
  \item Gravità;
  \item Forza elastica (non sempre);
  \item Forza elettrica;
  \item etc.
\end{itemize}
Mentre forze non conservative (e quindi dissipative) sono:
\begin{itemize}
  \item Attrito;
  \item Resistenza dell'aria;
  \item Forza compiuta da una persona;
  \item etc.
\end{itemize}

\vspace{1em}
\subsection{Energia potenziale}
Di seguito si espone la definizione di \textbf{energia potenziale}:

% Tabella per le definizione di concetti, etc...
\vspace{1em}
\rowcolors{1}{black!5}{black!5}
\setlength{\tabcolsep}{14pt}
\renewcommand{\arraystretch}{2}
\noindent
\begin{tabularx}{\textwidth}{@{}|P|@{}}
    \hline
    {\textbf{ENERGIA POTENZIALE}}\\
    \parbox{\linewidth}{Data una forza conservativa $\vec F$, è noto che il lavoro compiuto da tale forza è
    \[W_{i,f}=\int_{i}^f \vec F \cdot d \vec r\]
    il quale, essendo $\vec F$ conservativa, dipende solamente dai vettori posizione iniziale $\vec r_i$ e finale $\vec r_f$. Naturalmente, però, essendo $\vec F$ conservativa, essa può essere definita come il gradiente di una funzione $\mathcal{U}$, ovvero $\vec F = \nabla \mathcal{U}$, per cui eseguendo l'integrale indefinito di $\vec F$, si ottiene solamente l'energia potenziale $\mathcal{U}$, che deve essere valutata da una posizione finale ad una iniziale, come mostrato di seguito:
    \[W_{i,f} = \left(-\mathcal{U}(\vec r_f)\right) - \left(-\mathcal{U}(\vec r_i)\right)\]
    In altre parole si ha che
    \[\boxed{\mathcal{U}(\vec r) = - \int \vec F \cdot d \vec r + c}\]
    in cui $c=$ costante. Da ciò segue che
    \[\boxed{W_{i,f}=-\Delta \mathcal{U}_{i,f}}\]
    per cui il lavoro compiuto da $i$ a $f$ è uguale all'opposto della differenza di energia potenziale.\vspace{3mm}}\\
    \hline
\end{tabularx}

\vspace{2em}
\noindent
\textbf{Osservazione}: Si osservi che ogni qualvolta vi è una forza conservativa vi è energia potenziale.\\
L'energia potenziale, più in generale, può essere interpretata come un modo di immagazzinare energia per essere successivamente impiegata per compiere un lavoro (si parla di \quotes{potenziale di compiere un lavoro}).

\vspace{1em}
\subsubsection{Energia potenziale gravitazionale}
Per quanto concerne la forza di gravità, è noto che $\vec F_t = - mg \cdot \hat{j}$ e $d \vec r = dy \cdot \hat{j}$, per cui l'energia potenziale gravitazionale diviene
\[\mathcal{U}(y)=-\int(-mg \cdot dy) = mg \cdot y + c\]
Per cui l'energia potenziale gravitazionale si calcola come segue:
\[\boxed{\mathcal{U}(h)=mgh}\]

\vspace{1em}
\subsubsection{Energia potenziale elastica}
Per quanto concerne la forza elastica, è noto che $\vec F \cdot d \vec r = -k x \cdot dx$, da cui:
\[\mathcal{U}(x) = -\int - k x \cdot dx = \frac{1}{2}k \cdot x^2 + c\]
Per cui l'energia potenziale elastica si calcola come segue:
\[\boxed{\mathcal{U}(x)=\frac{1}{2}k x^2}\]

\vspace{1em}
\noindent
\textbf{Esempio}: Si consideri il pompaggio idroelettrico che prevede un bacino idrico in rilievo con $h=1050 \text{ m}$, avente una capacità di $V=9000000 \text{ m}^3$. Allora per conoscere l'energia potenziale è sufficiente impiegare la formula precedente:
\[\mathcal{U}=mgh=V \rho g h = 90 TJ\]

\vspace{1em}
\subsection{Potenza}
Di seguito si espone la definizione di \textbf{potenza}:

% Tabella per le definizione di concetti, etc...
\vspace{1em}
\rowcolors{1}{black!5}{black!5}
\setlength{\tabcolsep}{14pt}
\renewcommand{\arraystretch}{2}
\noindent
\begin{tabularx}{\textwidth}{@{}|P|@{}}
    \hline
    {\textbf{POTENZA}}\\
    \parbox{\linewidth}{La \textbf{potenza} viene definita come
    \[\boxed{P=\frac{dW}{dt}}\]
    ovverosia la quantità di energia trasferita per unità di tempo. L'unità di misura è, naturalmente, il \textbf{Watt}, per cui
    \[\boxed{1 \text{ W} = 1 \frac{\text{J}}{\text{s}}}\]
    mentre per l'unità di energia $1 \text{ kWh}$ è da considerarsi
    \[\boxed{1 \text{ kWh} = 1000 \text{ W} \cdot 3600 \text{ s} = 3.6 \text{ MJ}}\]
    \vspace{-3mm}}\\
    \hline
\end{tabularx}

\vspace{1em}
\noindent
\textbf{Esempio}: Se si considera un lavoro di $90 \text{ TJ}$, allora tale lavoro corrisponde ad una potenza di $25 \text{ GWh}$. Se si considera il flusso massimo di una turbina pari a $130 \text{ m}^3/\text{s}$, allora la potenza massima diviene:
\[P=\frac{dW}{dt}=\frac{d\mathcal{U}}{dt}=\frac{d}{dt}(mgh)=\frac{dm}{dt} gh = \rho \cdot \frac{dV}{dt} gh = 1.4 \text{ MW}\]

\vspace{1em}
\noindent
\textbf{Esercizio}: Si consideri una macchina di $10^3 \text{ kg}$ che necessita una potenza di $16 \text{ hp}$ per andare ad una velocità costante di $80 \text{ km/h}$. Si determini, allora, la potenza necessaria per salire una pendenza di $10^\circ$.

\vspace{1em}
\begin{figure}[H]
  \centering
  \begin{tikzpicture}[scale=1]
    \draw (-1.8,0) -- ++(6.2,0);
    \foreach \i in {-20,-18,...,40} {
      \draw (\i / 10,-0.3) -- (\i / 10 + 0.3,0);
    }
    \node[minimum size=0.5cm,circle,draw] (circle) at (0.5,0.25){};
    \node[minimum size=0.5cm,circle,draw] (circle) at (1.5,0.25){};
    \draw (0.25,0.25) -- ++(-0.3,0) -- ++(0,0.5) -- ++(0.5,0) -- ++(0.3,0.3) -- ++(0.8,0) -- ++(0.3,-0.3) -- ++(0.5,0) -- ++(0,-0.5) -- ++(-0.6,0);

    \draw [-stealth] (2,2) -- node[midway, above]{$\vec{v}$} (3,2) ;
    \draw [-stealth] (0.85,0) -- node[midway, left]{$\vec{F}_N$} (0.85,3) ;
    \draw [-stealth] (1.15,0.65) node[circ]{} -- ++(0,-3) node [midway, below right] {$\vec{F}_{t}$};
    \draw [-stealth, red] (2,0.5) node[circ]{} -- ++(2,0) node [midway, above right] {$\vec{F}$};
    \draw [-stealth, red] (0.5,0) node[circ]{} -- ++(-2,0) node [midway, above left] {$\vec{F}_{r}$};
  \end{tikzpicture}
  \caption{Vettura a velocità costante su una strada orizzontale}
  \label{fig:vettura_velocita_costante_strada_orizzontale}
\end{figure}

\noindent
Naturalmente le forze in gioco sono la forza di gravità, la forza normale, la forza di attrito e la forza di trazione:

\vspace{1em}
\begin{figure}[H]
  \centering
  \begin{tikzpicture}[scale=1]
    \draw [-stealth] (0,0) node[circ]{} -- ++(0,1) node [at end, right] {$\vec{F}_N$};
    \draw [-stealth] (0,0) -- ++(0,-1) node [midway, right] {$\vec{F}_{t}$};
    \draw [-stealth] (0,0) -- ++(-1,0) node [midway, above] {$\vec{F}_{r}$};
    \draw [-stealth] (0,0) -- ++(1,0) node [midway, above] {$\vec{F}$};
  \end{tikzpicture}
  \caption{Diagramma a corpo libero di una vettura a velocità costante su una strada orizzontale}
  \label{fig:diagramma_corpo_libero_vettura_velocita_costante_strada_orizzontale}
\end{figure}

\noindent
Giacché la velocità è costante, l'accelerazione della vettura è nulla, per cui
\[\sum \vec F = 0\]
Da ciò si evince la forza di trazione $\vec F$ e la forza di resistenza $\vec F_r$ sono uguali ed opposte, per cui:
\[\vec F = \vec F_r\]
Applicando il teorema lavoro-energia cinetica, si ha che
\[W_{tot}=\Delta K=0\]
in cui è da considerare $W_{tot}=W_m+W_r$, i quali sono uguali e oppposti. Ovviamente si ha che
\[W_m = \int \vec F \cdot d \vec r \hspace{1em} \text{e} \hspace{1em} W_r = \int \vec F_r \cdot d \vec r\]
e sono entrambi uguali a $W=F \cdot d$. Naturalmente, in questo caso, si ottiene che la potenza cercata è proprio:
\[P=\frac{d}{dt}W_m=\frac{d}{dt}F \cdot d = F \cdot v\]
che è un risultato fondamentale:
\[\boxed{P=\vec F \cdot \vec v}\]
Dai dati del problema si ha che tale potenza, ovvero $P=F \cdot v = 16 \text{ hp}$.\\
Se la macchina deve affrontare una salita di $10^\circ$, allora la forza peso deve essere scomposta nelle sue due componenti, e non viene cancellata totalmente dalla forza normale; tuttavia, si ha sempre che:
\[\sum \vec F = 0 = \vec F_r + \vec F_N + \vec F_t + \vec F\]
Ruotando il sistema di $10^\circ$ si ottiene che, scomponendo l'equazione nelle sue due componenti:
\begin{align*}
    \hat{j}:F_N - F_t \cdot \cos(\theta) = 0\\
    \hat{i}:F - F_r  - F_t \cdot \sin(\theta) = 0
\end{align*}
Volendo cercare la potenza della mattina necessaria per affrontare la salita, si deve procedere al calcolo seguente:
\[P=F \cdot v=(F_r + F_t \cdot \sin(\theta)) \cdot v\]
Ma volendo calcolare $F_r \cdot v$, si può impiegare il risultato precedente, sapendo che $F_r \cdot v = 16 \text{ hp}$. Da ciò si evince che:
\[P=F_r \cdot v + mg \cdot \sin(\theta) \cdot v = 16 \text{ hp} + \frac{37.8 \text{ kW}}{743 \text{ W/hp}} = 67 \text{ hp}\]
In cui, ovviamente, $mg \cdot \sin(\theta) \cdot v$ è il tasso di cambio della propria velocità in energia potenziale.

\vspace{1em}
\noindent
\textbf{Esercizio}: Si calcoli la velocità di un motorino elettrico avente una potenza di $3000 \text{ W}$ in salita avente una pendenza del $10^\circ$, in cui la massa complessiva del sistema è di $200 \text{ kg}$ (trascurando la resistenza dell'aria).\\
Naturalmente, giacché $v$ è costante, significa che non vi è accelerazione ($\vec a=0$). Da ciò si ha che la forza di trazione del motorino è $F=mg \cdot \sin(\theta)$, mentre la potenza cercata è
\[P = mg \cdot \sin(\theta) \cdot v = F \cdot v = \frac{d}{dt} \left(\mathcal{U}(y)\right) = mg \cdot \frac{dy}{dt}\]
Da cui si evince che la velocità del sistema cercata è proprio data dal rapporto seguente:
\[v = \frac{P}{mg \cdot \sin(\theta)} = 55 \text{ km/h}\]

\newpage
\noindent
\begin{center}
  28 Marzo 2022
\end{center}
\textbf{Esercizio}: Si consideri una massa $m=500$ g che viene lasciato cadere da un'altezza $h=60$ cm su un ripiano collegato ad una molla di costante elastica $k=120$ N/m.

\begin{figure}[H]
  \centering
  \colorlet{xcol}{blue!70!black}
  \colorlet{darkblue}{blue!40!black}
  \colorlet{myred}{red!65!black}
  \tikzstyle{mydashed}=[xcol,dashed,line width=0.25,dash pattern=on 2.2pt off 2.2pt]
  \tikzstyle{axis}=[->,thick] %line width=0.6
  \tikzstyle{ell}=[{Latex[length=3.3,width=2.2]}-{Latex[length=3.3,width=2.2]},line width=0.3]
  \tikzstyle{dx}=[-{Latex[length=3.3,width=2.2]},darkblue,line width=0.3]
  \tikzstyle{ground}=[preaction={fill,top color=black!10,bottom color=black!5,shading angle=20},
                      fill,pattern=north east lines,draw=none,minimum width=0.3,minimum height=0.6]
  \tikzstyle{mass}=[line width=0.6,red!30!black,fill=red!40!black!10,rounded corners=1,
                    top color=red!40!black!20,bottom color=red!40!black!10,shading angle=20]
  \tikzstyle{spring}=[line width=0.8,blue!7!black!80,snake=coil,segment amplitude=5,segment length=5,line cap=round]
  \tikzset{>=latex} % for LaTeX arrow head
  \tikzstyle{force}=[->,myred,very thick,line cap=round]
  \def\tick#1#2{\draw[thick] (#1)++(#2:0.1) --++ (#2-180:0.2)}

  \begin{tikzpicture}[scale=2]
    \def\H{1}    % wall height
    \def\T{0.3}  % wall thickness
    \def\W{2.6}  % ground length
    \def\D{0.25} % ground depth
    \def\h{0.6}  % mass height
    \def\w{0.7}  % mass width
    \def\x{1.6}  % mass x position
    \draw[spring] (\x/2,0) --++ (0,\h);
    \draw[ground] (0,0) |-++ (-\T,\H+1.5) |-++ (\T+\W,-\H-\D-1.5) -- (\W,0) -- cycle;
    \draw (0,\H+1.5) -- (0,0) -- (\W,0);

    \draw[mass] (\x/4,\h+1) rectangle++ (\w,\h) node[midway] {$m$};
    \draw[mass] (\x/4,\h) rectangle++ (\x/2,0.05);

    \draw[stealth-stealth] (\x,\h/2+0.5) -- ++(0,1) node[midway, right]{$60$ cm};
    \draw[stealth-stealth] (\x+0.75,\h/2) -- ++(0,1.5) node[midway, right]{$l$};
  \end{tikzpicture}
  \caption{Massa lasciata cadere su una molla}
  \label{fig:massa_lasciata_cadere_molla}
\end{figure}

\noindent
Si determini, allora, la lunghezza $l$ di compressione massima della molla fino a quando l'assieme blocco-piattaforma si ferma.\\
Al fine di risolvere tale problema è utile richiamare il teorema lavoro-energia cinetica: naturalmente si ha che il lavoro totale compiuto sul sistema è pari alla variazione di energia cinetica:
\[W_{tot} = \Delta K = \frac{1}{2}m \cdot \left(v_f^2 - v_i^2\right)\]
ma essendo la velocità iniziale e finale del blocco nulle, si deduce che il lavoro totale compiuto sul sistema è nullo. Le uniche forze che compiono lavoro, in questo caso ipotetico, sono la forza di gravità e la forza elastica di richiamo della molla; ciò significa che il lavoro compiuto da ambedue le forze è uguale, ma opposto, da cui:
\[w_{tot}=0=W_g+W_m=mgl-\frac{1}{2}k \cdot (l-d)^2\]
Da ciò si evince che:
\[-\frac{1}{2}kl^2 + (kd+mg) \cdot l - \frac{1}{2}kd^2=0\]
e volendo isolare $l^2$ si ottiene:
\[l^2 - 2 \cdot \left(\frac{mg}{k} + d\right) \cdot l + d^2=0\]
Naturalmente tale equazione presenterà due soluzioni, ma in questo caso si deve chiedere che $l>d$, per cui:
\[l_{1,2}=\frac{\dfrac{mg}{k}+l \pm \sqrt{4 \cdot \left(\dfrac{mg}{k} + d\right)^2 - 4d^2}}{2}\]
ma l'unica soluzione che ha significato è
\[l = \frac{mg}{k} \cdot \left(1+\sqrt{1+\frac{2kd}{mg}}\right)+d = 86.6 \text{ cm}\]

\vspace{1em}
\noindent
\textbf{Osservazione}: Si osservi che l'altra soluzione dell'equazione corrisponde non al caso in cui la molla è compressa, ma quando la molla è stesa.

\newpage
\noindent
\subsection{Conservazione dell'energia}
L'energia può essere interpretata come un fluido, la quale si traforma in una forma o in un'altra, ma conservando sempre la propria quantità. Tale concetto viene condensato dal \textbf{teorema di conservazione dell'energia} esposto di seguito:

% Tabella per le definizione di concetti, etc...
\vspace{1em}
\rowcolors{1}{black!5}{black!5}
\setlength{\tabcolsep}{14pt}
\renewcommand{\arraystretch}{2}
\noindent
\begin{tabularx}{\textwidth}{@{}|P|@{}}
    \hline
    {\textbf{CONSERVAZIONE DELL'ENERGIA}}\\
    \parbox{\linewidth}{Quando si impiega il teorema di conservazione dell'energia, è sempre fondamentale specificare il \textbf{sistema} su cui si sta lavorando, distinguendolo con l'\textbf{ambiente esterno}: tra il sistema e l'ambiente può esserci scambio (o trasferimento) di energia (in ambo le direzioni) oppure no; nel secondo caso si può applicare il \textbf{teorema di conservazione dell'energia} sul sistema considerato:
    \[\boxed{E_{\text{sistema}}=K+\mathcal{U}+U}\]
    in cui $K$ è l'\textbf{energia cinetica}, $\mathcal{U}$ è l'\textbf{energia potenziale}, mentre $U$ prende il nome di \textbf{energia interna}. Tuttavia, in \textbf{meccanica}, si ignora totalmente l'energia del sistema, e si preferisce considerare solamente l'\textbf{energia meccanica} $E_m=K+\mathcal{U}$.\\
    Se si ha conservazione dell'energia meccanica, si ottiene che
    \[\boxed{\Delta E_{\text{sistema}}=0=\sum\text{trasferimenti}=\text{lavoro}+\text{calore}}\]
    per cui su un sistema chiuso è possibile modificare l'energia del sistema compiendo lavoro su di esso, oppure tramite scambi di calore. Esempi di sistema possono essere costituiti da
    \begin{itemize}
      \item un corpo solo;
      \item due (o più corpi) che interagiscono;
      \item un corpo deformabile;
      \item una regione dello spazio (in questo caso si può avere non solo scambi d energia, ma anche di materia);
      \item etc.
    \end{itemize}
    in cui tali sistemi possono essere \textbf{aperti} (per cui si ha scambio sia di energia che di materia), \textbf{chiusi} (in cui si ha scambio solamente di energia) o \textbf{isolati} (in cui non si ha nè scambio di energia nè di materia).\\
    Più limitatamente all'energia meccanica, il teorema di conservazione dell'energia meccanica afferma che
    \[\boxed{\Delta E_{\text{sistema}}=\Delta K + \Delta \mathcal{U}=W_{\text{sistema}}}\]
    ove $W_{\text{sistema}}$ è da intendersi il lavoro totale compiuto sul sistema.\vspace{3mm}}\\
    \hline
\end{tabularx}

\vspace{1em}
\noindent
\textbf{Esempio}: Nel caso precedente, in cui si considerava una massa che cadeva sulla molla, si può considerare come sistema l'assieme blocco-molla-terra, in cui si assiste alla conservazione dell'energia meccanica, in quanto non c'è lavoro compiuto sul sistema dall'esterno: pertanto si assiste solamente alla trasformazione di energia potenziale gravitazionale in energia cinetica e, successivamente, in energia potenziale elastica.

\vspace{1em}
\noindent
\textbf{Osservazione}: Si osservi che l'energia cinetica non può mai essere negativa, mentre l'energia potenziale è totalmente arbitraria: ciò che è fondamentale da valutare è la variazione di energia potenziale.\\
Inoltre, l'energia totale di un sistema può cambiare se una forza esterna compie lavoro sul sistema: il lavoro totale compiuto sul sistema è uguale alla variazione in energia totale del sistema.

\vspace{1em}
\noindent
\textbf{Esercizio}: Se si considera una biglia poggiata su una molla in compressione: nello stadio iniziale vi sarà solamente energia potenziale elastica, che poi si tramuterà in energia cinetica ed energia potenziale gravitazionale, fino a quando la biglia raggiungerà il suo punto di massima altezza, in cui la totalità dell'energia del sistema è data soltanto dalla componente potenziale gravitazionale.

\vspace{1em}
\noindent
\textbf{Osservazione}: Si osservi che l'energia potenziale di una forza conservativa $\vec F$ era stata definita come
\[\mathcal{U} = - \int \vec F \cdot d \vec r\]
per cui si può scrivere che
\[\boxed{F_x = - \frac{d \mathcal{U}}{dx}}\]
ovvero la forza $F$ corrisponde all'opposto della pendenza del grafico energia potenziale-spostamento:

\vspace{2em}
\noindent
\rowcolors{1}{white}{white}
\begin{tabularx}{\textwidth}{P}
  {
      \centering
      \begin{tikzpicture}
        \begin{axis}[
          grid=both,
          axis lines = middle,
          xlabel = \(x\),
          ylabel = {\(\mathcal{U}\)},
          legend pos=outer north east,
          ymajorgrids=true,
          ymin=-5,
          xmajorgrids=true,
          grid style=dashed,
        ]

        \addplot [
          domain=-2:12,
          samples=100,
          color=red,
        ]
        {(x-5)^2+20};
        \draw (axis cs:5,20) node[circ]{} node[above=0.5cm]{$F=0$};
        \draw (axis cs:8,29) node[circ]{} node[below=0.5cm]{$\vec F$};
        \draw (axis cs:2,29) node[circ]{} node[below=0.5cm]{$\vec F$};
        \draw[-stealth, red, very thick] (axis cs:1,15) -- (axis cs:3,15);
        \draw[-stealth, red, very thick] (axis cs:9,15) -- (axis cs:7,15);
        \end{axis}
    \end{tikzpicture}
  }
\end{tabularx}

\noindent
Questo è un sistema che viene definito \textbf{stabile}, in quanto ad ogni spostamento positivo (o negativo) vi è una forza di richiamo negativa (o positiva, rispettivamente) che permette di bilanciare il sistema, portandolo in una condizione di equilibrio.\\
Analogamente, un sistema è \textbf{instabile} quando ad ogni spostamento viene associata una forza che sbilancia ancora di più il sistema nella direzione dello spostamento:

\vspace{2em}
\noindent
\rowcolors{1}{white}{white}
\begin{tabularx}{\textwidth}{P}
  {
      \centering
      \begin{tikzpicture}
        \begin{axis}[
          grid=both,
          axis lines = middle,
          xlabel = \(x\),
          ylabel = {\(\mathcal{U}\)},
          legend pos=outer north east,
          ymin=-5,
          ymax=18,
          xmin=-2,
          xmax=12,
          ymajorgrids=true,
          xmajorgrids=true,
          grid style=dashed,
        ]

        \addplot [
          domain=-2:12,
          samples=100,
          color=red,
        ]
        {-(x-5)^2+12};
        \draw (axis cs:5,12) node[circ]{} node[above=0.5cm]{$F=0$};
        \draw (axis cs:8,3) node[circ]{} node[above=1.5cm]{$\vec F$};
        \draw (axis cs:2,3) node[circ]{} node[above=1.5cm]{$\vec F$};
        \draw[-stealth, red, very thick] (axis cs:3,12) -- (axis cs:1,12);
        \draw[-stealth, red, very thick] (axis cs:7,12) -- (axis cs:9,12);
        \end{axis}
    \end{tikzpicture}
  }
\end{tabularx}

\noindent
Non da ultimo devono essere citati sistemi molto più complessi, in cui vi può essere la combinazione di più configurazioni stabili e instabili, valutabili tramite la derivata seconda, ovvero:
\[\frac{d^2 \mathcal{U}}{dx^2} > 0 \longrightarrow \text{sistema stabile} \hspace{1em} \text{e} \hspace{1em} \frac{d^2 \mathcal{U}}{dx^2} < 0 \longrightarrow \text{sistema instabile}\]

\vspace{1em}
\noindent
\textbf{Osservazione}: Si osservi che l'energia potenziale gravitazione può essere interpretata come
\[\boxed{\mathcal{U}(r) = - G \cdot \frac{m_1 \cdot m_2}{r}}\]
che graficamente corrisponde ad un'iperbole equilatera sul quarto quadrante.

\newpage
\noindent
\begin{center}
  29 Marzo 2022
\end{center}
A proposito dell'energia potenziale gravitazione, è nota la formula seguente, in quanto deriva dall'integrale della forza di attrazione gravitazionale per la distanza:
\[\boxed{\mathcal{U}(r) = - G \cdot \frac{m_1 \cdot m_2}{r}}\]
Che presenta un grafico sull'asse della distanza che rappresenta un'iperbole equilatera

\vspace{2em}
\noindent
\rowcolors{1}{white}{white}
\begin{tabularx}{\textwidth}{P}
  {
      \centering
      \begin{tikzpicture}
        \begin{axis}[
          grid=both,
          axis lines = middle,
          xlabel = \(r\),
          ylabel = {\(\mathcal{U}\)},
          legend pos=outer north east,
          ymax = 4,
          ymajorgrids=true,
          xmajorgrids=true,
          grid style=dashed,
        ]

        \addplot [
          domain=0:3,
          samples=100,
          color=red,
        ]
        {- 1/x};
        \end{axis}
    \end{tikzpicture}
  }
\end{tabularx}

\vspace{1em}
\noindent
\textbf{Osservazione}: Si osservi che in base alla formula di cui sopra, l'energia potenziale gravitazionale è sempre negativa (anche se sarebbe possibile aggiungere a tale calcolo una costante $c$ di integrazione).

\vspace{1em}
\noindent
\textbf{Esempio}: L'energia meccanica totale di un satellite in orbita attorno alla terra è data dalla somma di due componenti: l'energia cinetica e l'energia potenziale
\[E_{tot} = K+\mathcal{U}=\frac{1}{2}mv^2 - G \cdot \frac{m_t \cdot m}{r}\]
Tuttavia, in tali sistemi, si assiste ad un minimo scambio di energia (a causa delle radiazioni), per cui con sufficiente precisione è possibile affermare che si ha conservazione di energia, ovvero
\[\Delta E = 0 \longrightarrow K+\mathcal{U}=\text{costante}\]
Se il satellite si allontasse eccessivamente dalla terra, incrementando la propria energia cinetica e riducento la propria energia potenziale, non starebbe più in orbita attorno alla terna.\\
Pertanto, è possibile definire il concetto di \emph{$E:=$energia di legame}, per cui se $E < 0$ si ha la condizione per avere un'orbita attorno alla terra.\\
Naturalmente, l'energia potenziale dipende solamente dal parametro $r$, per cui più $r$ aummenta, più l'energia potenziale diminuisce, mentre se $r$ tende a $0$, allora l'energia potenziale assume il suo massimo valore.\\
Disegnando l'energia potenziale in tre dimensioni, è possibile visualizzare le linee equipotenziali che permettono di descrivere le possibili orbite di due pianeti.

\vspace{1em}
\subsection{Felocità di fuga}
È noto che la velocità di fuga si ha quando l'energia di legame è pari a $0$, ossia $E=0$: taluna è la condizione che permette al satellite di fuggire dalla sua orbita, ossia di uscire dal campo gravitazionale, ovvero:
\[\frac{1}{2}m v^2 = G \cdot \frac{m_t \cdot m}{r}\]
in cui $v$ è proprio la velocità di fuga, per cui
\[\boxed{v_{\text{fuga}} = \sqrt{\frac{2G \cdot m_t}{r}}}\]
e dipende naturalmente da $r$: se la distanza è molto elevata, è chiaro che l'energia potenziale gravitazionale sarà inferiore e quindi la velocità di fuga sarà inferiore.\\
Nel caso della terra, allora, volendo calcolare la velocità di fuga sulla superficie della terra si ottiene:
\[v_{\text{fuga}} = \sqrt{\frac{2G \cdot m_t}{R_t}} = \sqrt{2} \cdot \sqrt{g \cdot R_t} \cong 11 000 \text{ m/s} \cong 40 000 \text{ km/h}\]

\vspace{1em}
\noindent
\textbf{Osservazione}: Quando si considera un sistema stabile, partendo con una certa energia cinetica, converte molteplici volte tale energia in energia potenziale e viceversa; se vi è dissipazione, ovviamente, parte dell'energia viene dispersa ad ogni oscillazione fino a quando non si raggiunge la configurazione stabile di equilibrio.

\vspace{1em}
\noindent
\textbf{Esempio}: Si consideri una molla verticale sulla quale è attaccata una massa. Naturalmente, l'energia potenziale del sistema in funzione di $y$ è data da
\[\mathcal{U}(y) = mg y + \frac{1}{2} k y^2\]
e per conoscere il minimo valore di energia potenziale (corrispondente proprio al punto di equilibrio nel grafico di una parabola con concavità rivolta verso l'alto), è sufficiente derivare tale espressione e porre la condizione che la derivata sia nulla:
\[\frac{d}{dy} \mathcal{U(y)} = mg + ky \longrightarrow y = \frac{mg}{k}\]
che è il medesimo risultato che sarebbe stato ottenuto ragionando con la $2^a$ legge della dinamica.\\

\vspace{1em}
\noindent
\textbf{Esercizio $\boldsymbol{1}$}: Il raggio dell'orbita di Marte è $1.52$ volte quello dell'orbita terrestre. Utilizzando la terza legge di Keplero si determini il periodo della rivoluzione di Marte in anni.

\vspace{1em}
\noindent
La terza legge di Keplero afferma che il quadrato del periodo di rivoluzione di un pianeta attorno al sole è proporizionale al cubo della distanza tra i due:
\[T^2 \propto R^3\]
Ma naturalmente si ha che la costante di proporizionalità sarà
\[\frac{T^2}{R^3} = \text{costante}\]
e tale costante dipende solamente dalla massa del sole, per cui è la stessa per tutti i pianeti, pertanto si ha che
\[\frac{T_t^2}{R_t^3} = \frac{T_m^2}{R_m^3}\]
ma sapendo che $R_m=1.52 R_t$ è facile isolare $T_m$, ottenendo
\[T_m = T_t \cdot \left(\frac{R_m}{R_t}\right)^{\frac{3}{2}} = T_t \cdot (1.52)^{\frac{3}{2}}=1.87 \text{ anni}\]

\vspace{1em}
\noindent
\textbf{Esercizio $\boldsymbol{2}$}: Si supponga che il disco da hockey abbia una velocità di modulo $v_0$ nella posizione più bassa.\\
Si determini il valore minimo $v_0$ che consente al disco di completare il suo percorso circolare.

\vspace{1em}
\noindent
Naturalmente la tensione del filo nel punto più alto è nulla. Per cui, affinché si abbia un moto circolare, è necessario che la componente diretta nel piano della forza di gravità sia la forza centripeta necessaria affinché la massa compia un moto circolare uniforme.\\
Pertanto appare evindente come si debba imporre che
\[F_{tx} = mg \cdot \sin(\theta) = m \cdot \frac{v^2}{r}\]
da cui si evince come la velocità nel punto più in alto sia
\[v = \sqrt{rg \cdot \sin(\theta)}\]
E per determinare $v_0$ è sufficiente applicare il teorema di conservazione dell'energia, da cui
\[K_i + \mathcal{U}_i = K_f + \mathcal{U}_f\]
e volendo determinare $K_f$, ossia l'energia cinetica al termine della rotazione della massa nel punto più basso si ottiene:
\[K_f = K_i + \mathcal{U}_i - \mathcal{U}_f \longrightarrow \frac{1}{2} m v_0^2 = \frac{1}{2} m v^2 + mg \cdot (2L \cdot \sin(\theta))\]
per cui, volendo isolare $v_0$ si ottiene che
\[v_0 = \sqrt{5L g \cdot \sin(\theta)}\]
Volendo determninare la tensione del filo nella posizione più bassa, sarà sufficiente imporre che la somma della forza di tensione e della componente nel piano della forza peso devono essere pari alla forca centripeta nel punto più bassso, da cui:
\[F_{tx} + F_T = F_c \longrightarrow F_T + F_{tx} = -m \cdot \frac{v_0^2}{L}\]
essendo diretta verso il centro della circonferenza. Da ciò è immediato capire che
\[F_T = -m \cdot \frac{v_0^2}{L} - mg \cdot \sin(\theta) = -6 mg \cdot \sin(\theta)\]



\newpage
\section{Oscillazioni}

\newpage
\section{Solidi e fluidi}

\newpage
\section{Temperatura e calore}

\newpage
\section{Il primo principio della termodinamica}

\newpage
\section{Il secondo principio della termodinamica}


\end{document}
