\documentclass[a4paper]{extarticle}
\usepackage[utf8]{inputenc}
\usepackage[italian]{babel}
\selectlanguage{italian}
\usepackage[table]{xcolor}
\usepackage{xcolor}
\usepackage{circuitikz}
\usepackage{bm}
\usetikzlibrary{patterns,snakes}
\usetikzlibrary{decorations.markings,intersections,calc}
\usepackage{ifthen}
\usetikzlibrary{calc,patterns,angles,quotes}
\usetikzlibrary{positioning, circuits.logic.US}
\usetikzlibrary {shapes.gates.logic.US, shapes.gates.logic.IEC, calc}
\tikzset {branch/.style={fill, shape = circle, minimum size = 3pt, inner sep = 0pt}}
\usetikzlibrary{matrix,calc}
\usepackage{multirow}
\usepackage{float}
\usepackage{geometry}
\usepackage{pgfplots}
\usepackage{tabularx}
\usepackage{pgf-pie}
\usepackage{tikz}
\usepackage{tikz-3dplot}
\usepackage{amsmath}
\usepackage{amssymb}
\usepackage{color, soul}
\usepackage{fancyhdr}
\usepackage{graphicx}
\usepackage{subfig}
\usepackage{physics}
\tikzset{>=latex} % for LaTeX arrow head
\colorlet{pinkskin}{pink!25}
\colorlet{brownskin}{pink!5!brown!45}
\colorlet{myred}{red!90!black}
\colorlet{myblue}{blue!90!black}
\colorlet{mypurple}{blue!50!red!80!black!80}
\colorlet{Bcol}{violet!90}
\colorlet{BFcol}{red!60!black}
\colorlet{veccol}{green!45!black}
\colorlet{Icol}{blue!70!black}
\colorlet{mucol}{red!90!black}
\tikzstyle{BField}=[->,line width=2,Bcol]
\tikzstyle{current}=[->,Icol] %thick,
\tikzstyle{force}=[->,line width=2,BFcol]
\tikzstyle{vector}=[->,line width=2,veccol]
\tikzstyle{thick vector}=[->,line width=2,veccol]
\tikzstyle{mu vector}=[->,line width=2,mucol]
\tikzstyle{velocity}=[->,line width=2,veccol]
\tikzstyle{charge+}=[very thin,draw=black,top color=red!50,bottom color=red!90!black,shading angle=20,circle,inner sep=0.5]

\graphicspath{ {./img/} }
\newtheorem{theorem}{Teorema}[section]
\newtheorem{corollary}{Corollario}[theorem]
\newtheorem{lemma}[theorem]{Lemma}

% Specifiche
\geometry{
 a4paper,
 top=20mm,
 left=30mm,
 right=30mm,
 bottom=30mm
}

\pagestyle{fancy}
\fancyhf{}
\fancyhead[LO]{\nouppercase{\leftmark}}
\fancyfoot[CE, CO]{\thepage}
\addtolength{\headheight}{1em}
\addtolength{\footskip}{-0.5em}

\newcommand{\quotes}[1]{``#1''}
\renewcommand\tabularxcolumn[1]{>{\vspace{\fill}}m{#1}<{\vspace{\fill}}}
\renewcommand\arraystretch{}
\newcolumntype{P}{>{\centering\arraybackslash}X}

\title{\textbf{Università di Trieste\\ \vspace{1em}
Laurea in ingegneria elettronica e informatica}}
\author{Enrico Piccin - Corso di Fisica generale I - Prof. Pierre Thibault}
\date{Anno Accademico 2021/2022 - 1 Marzo 2022}

\begin{document}

\vspace{-10mm}
\maketitle

\tableofcontents
\newpage

\noindent
\begin{center}
  1 Marzo 2022
\end{center}

\section{Introduzione}
La \textbf{Fisica} è lo studio della materia e delle sue interazioni. La \textbf{Fisica classica} è divisa in tre macroaree:
\begin{enumerate}
  \item Meccanica classica
  \item Termodinamica
  \item Elettromagnetismo
\end{enumerate}
La Fisica è organizzata in
\begin{itemize}
  \item \textbf{Leggi}: relazioni fra grandezze fisiche
  \item \textbf{Principi}: affermazioni generali da reputare vere
  \item \textbf{Modelli}: analogie o rappresentazioni pratiche su cui basare il proprio studio
  \item \textbf{Teoria}: insieme di leggi, principi e modelli
\end{itemize}

\vspace{1em}
\subsection{Metodo scientifico}
Il \textbf{metodo scientifico} si basa su \textbf{osservazioni} della realtà circostante, a cui seguono delle \textbf{ipotesi}, ossia delle possibili spiegazioni dei fonmeni osservati, basati sulle osservazioni precedentemente formulate.\\
Dopo aver esposto le proprie ipotesi, esse devono essere verificate, mediante degli \textbf{esperimenti}, a cui seguono delle \textbf{analisi} dei risultati sperimentali ottenuti. Il processo di analisi viene seguito da delle \textbf{conclusioni} che \quotes{concludono} il metodo scientifico.\\
Naturalmente tale circuito non è chiuso, in quanto ciascuna di queste fasi può essere ripetuta più e più volte. La parte più importante di tale \emph{metodo scientifico} è la dimostrazione, così come la verifica tramite \textbf{sperimentazioni} delle proprie ipotesi, in quanto le ipotesi devono essere \textbf{sempre verificate}. Tale processo permette di sviluppare leggi e teorie con un fondamento concreto e solido.

\vspace{1em}
\noindent
\textbf{Osservazione}: Si osservi che \textbf{verificare un'ipotesi} non significa dimostrare che un'ipotesi è vera, ma \textbf{verificare che un'ipotesi può essere contraddetta}, ovvero ci si deve assicurare che una \textbf{teoria deve essere \quotes{falsificabile}}, ossia che può essere dimostrato che essia sia falsa.

\newpage
\section{Unità e vettori}

\vspace{1em}
\subsection{Grandezza fisica}
Alla base della \textbf{Fisica} si pone il concetto di \textbf{grandezza fisica}. Non è facile, per esempio, definire che cosa sia il \emph{tempo}; tuttavia, la soluzione più immediata è quella che prevede di definire la misura del tempo come ciò che si riesce a misurare tramite, per esempio, un orologio.\\
Si parla, in tale caso, di \textbf{definizione operativa}:

% Tabella per le definizione di concetti, etc...
\vspace{1em}
\rowcolors{1}{black!5}{black!5}
\setlength{\tabcolsep}{14pt}
\renewcommand{\arraystretch}{2}
\noindent
\begin{tabularx}{\textwidth}{@{}|P|@{}}
    \hline
    {\textbf{DEFINIZIONE OPERATIVA}}\\
    \parbox{\linewidth}{Una grandezza fisica è definita solo dalle operazioni necessarie per misurarla.
    \vspace{3mm}}\\
    \hline
\end{tabularx}

\vspace{1em}
\noindent
Inoltre, le grandezze fisiche si esprimono in termini di un \textbf{campione}, il quale prende il nome di \textbf{unità}.\\
In Fisica, inoltre, si distinguono due diverse categorie di grandezze fisiche:
\begin{enumerate}
  \item Grandezze fisiche fondamentali
  \item Grandezze fisiche derivate
\end{enumerate}
Le \textbf{grandezze fisiche fondamentali} sono $3$:
\begin{enumerate}
  \item Tempo: il tempo presenta come unità il \textbf{secondo} (s) che, dal $1967$, è stato definito come
  \[9192631170 \text{ volte il periodo di oscillazione di una risonanza dell'atomo di Cesio } ^{133}C\]
  Prima di tale data, il secondo era definito come una suddivisione del giorno, ma tale definizione era imprecisa: la terra non ruota sempre con la stessa velocità.

  \item Lunghezza: la lunghezza presenta come unità il \textbf{metro} (m), il quale viene definito come
  \[\frac{1}{299 782 458} \text{ la distanza percorsa dalla luce in $1$ s}\]
  Prima di tale definizione, il metro era definito come \(\frac{1}{10000}\) la distanza tra equatore e polo.\\
  La nuova definizione, tuttavia, è più precisa, in quanto la velocità della luce è \textbf{costante}, fissata in quanto su tale costante si definisce il metro.

  \item Massa: la massa presenta come unità il \textbf{chilogrammo} (kg), il quale viene definito in funzione della \textbf{costante di Planck} ($h = 6.62607015 \times 10^{-34} \text{ kg} \text{ m}^{2} \text{ s}^{-1}$). Prima di tale definizione, il chilogrammo era definito con riferimento ad un campione presente a Parigi e su cui si faceva riferimento per ogni misura di massa.
\end{enumerate}

\noindent
Le grandezze fisiche fondamentali permettono, poi, di definire le grandezze fisiche derivate, quale il \textbf{Volume}, la \textbf{Forza}, etc.

\vspace{1em}
\subsection{Cifre significative e incertezza}
In Fisica, quando si effettuano delle misurazioni, deve essere sempre specificata la precisione e, dunque, l'incertezza. Infatti, \textbf{tutte le msiure hanno un livello di incertezza}.\\
Per esempio
\begin{flalign*}
  L & = 1.82 \pm 0.02 \text{ m}\\
  m & = 3.5 \pm 0.1 \text{ kg}
\end{flalign*}
Da notare che l'indicazione dell'incertezza è sempre (o quasi) data da una sola cifra: altrimenti si avrebbe incertezza nell'incertezza. L'indicazione dell'incertezza è la base della \textbf{fisica sperimentale}.\\
Nella pratica, tuttavia, l'indicazione dell'incertezza è ridondante e pesante. Per indicare il livello di precisione si ricorre alle cifre significative.\\
Per esempio
\begin{flalign*}
  L & = 1.82 \text{ m} = 1.82 \pm 0.01 \text{ m}\\
  m & = 3.5 \text{ kg} = 3.5 \pm 0.1 \text{ kg}
\end{flalign*}

\vspace{1em}
\subsubsection{Operazioni di base}
Per la gestione delle cifre significative nelle operazioni di calcolo è importante tenere a mente che
\begin{itemize}
  \item Moltiplicazione e Divisione: bisogna considerare come cifre significative del prodotto o del quoziente il più basso numero di cifre significative dei fattori o di dividendo e divisore.\\
  Per esempio
  \[1,1 \text{ m} \times 3.45 \text{ m} = 3.8 \text{ m}^2\]
  in quanto il più basso numero di cifre significative dei fattori è $1$.

  \item Addizione e Sottrazione: bisogna considerare come cifre significative della somma o differenza il più basso numero di decimali degli addendi o del minuendo e sottraendo.\\
  Per esempio
  \[1.1 \text{ m} - 12 \text{ cm} = 1.1 \text{ m} - 0.12 \text{ m} = 0.98 \text{ m} = 1.0 \text{ m}\]
  in quanto il più basso numero di decimali tra minuendo e sottraendo è $1$.
\end{itemize}

\vspace{1em}
\subsection{Ordini di grandezza}
Molto spesso, nelle stime è importante non tanto la precisione delle misure, ma l'ordine di grandezza delle stesse, in modo tale da effettuare un macroconfronto utile per delle valutazioni pratiche e veloci.\\
Lo scopo, quindi, dell'impiego degli ordini di grandezza è quello di effettuare dei calcoli veloci e, quindi, delle stime. Più precisamente:

% Tabella per le definizione di concetti, etc...
\vspace{1em}
\rowcolors{1}{black!5}{black!5}
\setlength{\tabcolsep}{14pt}
\renewcommand{\arraystretch}{2}
\noindent
\begin{tabularx}{\textwidth}{@{}|P|@{}}
    \hline
    {\textbf{ORDINE DI GRANDEZZA}}\\
    \parbox{\linewidth}{L'ordine di grandezza di una misura è la \textbf{potenza di $10$ più vicina}.
    \vspace{3mm}}\\
    \hline
\end{tabularx}

\vspace{1em}
\noindent
\textbf{Esempio}: Un ingegnere deve fabbricare un nuovo pacemaker. Si stimi quanti battiti di cuore deve fare senza malfunzionamento. Per effettuare tale stima è necessario conoscere la \textit{media dei battiti al secondo} e \textit{l'aspettativa di vita del soggetto}. Considerando, quindi, come media dei battiti $m_B = 1 \text{ battito}/\text{s}$ e come aspettativa di vita $a_V = 60 \text{ anni}$. La stima selectlanguage
\[m_B \times a_V \times \pi \times 10^7 \text{ s}/\text{anno} = 1 \text{ battito}/\text{s} \times 60 \text{ anni} \times \times 10^7 \text{ s}/\text{anno} = 2 \times 10^9 \text{ battiti}\]

\newpage
\noindent
\begin{center}
  2 Marzo 2022
\end{center}
Il metodo scientifico permette di \textbf{falsificare una teoria}, quindi non è vero che permette di validare una teoria senza ambiguità.

\vspace{1em}
\subsection{Analisi dimensionale}
Il concetto di \textbf{unità} è estremamente importante per parlare di \textbf{analisi dimensionale}. In particolare
\[A = B\]
non può essere valido e corretto formalmente se $A$ e $B$ hanno unità diverse. Questo è molto intuitivo per le grandezze fisiche fondamentali, ma quando si parla di grandezze derivate diventa un punto cruciale: tale concetto permette di validare anche delle possibili soluzioni di test.\\
Per esempio, l'unità di misura della costante di richiamo di una molla si può facilmente ricavare dalla formula della \emph{forza di richiamo}:
\[F = k \cdot x\]
Da cui è immediato capire che
\[\left[k\right] = \frac{\left[F\right]}{\left[x\right]} = \frac{\text{kg m s}^{-2}}{m} = \text{kg s}^{-2}\]

\vspace{1em}
\subsection{Scalari e vettori}
Di seguito si espone la definizione di \textbf{scalare}:

% Tabella per le definizione di concetti, etc...
\vspace{1em}
\rowcolors{1}{black!5}{black!5}
\setlength{\tabcolsep}{14pt}
\renewcommand{\arraystretch}{2}
\noindent
\begin{tabularx}{\textwidth}{@{}|P|@{}}
    \hline
    {\textbf{SCALARE}}\\
    \parbox{\linewidth}{Uno \textbf{scalare} è una grandezza specificata da un numero + unità.\\
    Per esempio la \emph{lunghezza}, la \emph{massa} o l'\emph{energia}.
    \vspace{3mm}}\\
    \hline
\end{tabularx}

\vspace{1em}
\noindent
Mentre un \textbf{vettore} è:

% Tabella per le definizione di concetti, etc...
\vspace{1em}
\rowcolors{1}{black!5}{black!5}
\setlength{\tabcolsep}{14pt}
\renewcommand{\arraystretch}{2}
\noindent
\begin{tabularx}{\textwidth}{@{}|P|@{}}
    \hline
    {\textbf{VETTORE}}\\
    \parbox{\linewidth}{Un \textbf{vettore} è  una quantità definita da un valore e una direzione (e un verso, che può essere implicito nella definizione di direzione).
    \vspace{3mm}}\\
    \hline
\end{tabularx}

\vspace{1em}
\noindent
Tale definizione, tuttavia, pur essendo molto intuitiva, non risulta particolarmente pratica. Si potrebbe anche considerare un vettore come una \textbf{quantità con più valori associati}, ovvero una \textbf{lista di numeri a cui conferiamo un significato}.\\
Per esempio, in algebra un vettore viene indicato come segue
\[\vec{v} = (1, 2, 3)\]
a cui la fisica attribuisce un significato preciso: $1$, $2$ e $3$ sono le componenti associate alle tre diverse dimensioni $x$, $y$ e $z$. Il vettore di cui sopra, allora, si può scrivere come
\[\vec{v} = (v_x, v_y, v_z)\]

\vspace{1em}
\noindent
\textbf{Osservazione}: Anche se tale definizione sembra identica alla definizione del \textbf{punto}, in realtà tale definizione è differente, in quanto
\begin{itemize}
  \item un punto non ha una lunghezza;
  \item non è possibile eseguire la somma di due punti, etc.
\end{itemize}

\vspace{1em}
\subsection{Prodotto con uno scalare}
Dato un vettore
\[\vec{v} = \left(v_x, v_y, v_z\right)\]
e si considera uno scalare $a \in \mathbb{R}$, allora
\[a \cdot \vec{v} = \left(a \cdot v_x, a \cdot v_y, a \cdot v_x\right)\]
in cui il vettore $a \cdot \vec{v}$ è un vettore che
\begin{itemize}
  \item presenta come lunghezza la lunghezza del vettore $\vec{v}$ moltiplicata per $\vert a \vert$;
  \item presenta come direzione la stessa direzione del vettore $\vec{v}$;
  \item presenta come verso lo stesso verso del vettore $\vec{v}$ se $a \geq 0$, mentre avrà verso opposto se $a \leq 0$.
\end{itemize}

\vspace{1em}
\subsection{Somma vettoriale}
Dati due vettori $\vec{u}$ e $\vec{v}$, la loro somma viene eseguita graficamente tramite la \textbf{regola del parallelogramma}, o il metodo \quotes{punta-coda}:

\begin{figure}[H]
  \centering
  \begin{tikzpicture}
    \draw [-stealth, thick, red]    (0,0) -- coordinate[midway](u1) (-3,2);
    \draw [-stealth, thick, blue]   (0,0) -- coordinate[midway](v)  (5,2);
    \draw [-stealth, thick, dashed] (5,2) -- coordinate[midway](u2) (2,4);
    \draw [-stealth, thick, orange] (0,0) -- coordinate[midway](r)  (2,4);
    \draw [thick, red]    (u1) node[above]{$\vec{u}$};
    \draw [thick, blue]   (v)  node[above]{$\vec{v}$};
    \draw [thick, dashed] (u2) node[above]{$\vec{u}$};
    \draw [thick, orange] (r)  node[left]{$\vec{r}$};
  \end{tikzpicture}
  \caption{Somma vettoriale con il metodo \quotes{punta-coda}}
  \label{fig:somma_vettoriale_metodo_punta_coda}
\end{figure}

\noindent
Se $\vec{u}$ e $\vec{v}$ sono espressi nello stesso sistema di riferimento, allora è chiaro che la loro somma sarà data \textbf{componente per componente}, ovvero
\[\vec{u} + \vec{v} = \left(u_x + v_x, u_y + v_y, u_z + v_z\right)\]

\vspace{1em}
\subsection{Versori}
Si definiscano tre versori come segue
\begin{flalign*}
  \hat{i} & = (1,0,0)\\
  \hat{j} & = (0,1,0)\\
  \hat{k} & = (0,0,1)
\end{flalign*}
Allora qualsiasi vettore può essere scritto come
\[\vec{v} = \left(v_x, v_y, v_z\right) = v_x \cdot \hat{i} + v_y \cdot \hat{j} + v_z \cdot \hat{k}\]
in cui, naturalmente, $v_x$, $v_y$ e $v_z$ sono le componenti di $\vec{v}$ in direzione $\hat{i}$, $\hat{j}$ e $\hat{k}$.\\
Naturalmente si scrive $\hat{i}$ e non $\vec{i}$ in quanto
\[\left \vert \hat{i} \right \vert = \left \vert \hat{j} \right \vert = \left \vert \hat{k} \right \vert = 1\]
essi, infatti, prendono il nome di \textbf{versori} o \textbf{vettori unità}.

\vspace{1em}
\subsection{Modulo e direzione}
Di seguito si espone la definizione di \textbf{modulo di un vettore}:

% Tabella per le definizione di concetti, etc...
\vspace{1em}
\rowcolors{1}{black!5}{black!5}
\setlength{\tabcolsep}{14pt}
\renewcommand{\arraystretch}{2}
\noindent
\begin{tabularx}{\textwidth}{@{}|P|@{}}
    \hline
    {\textbf{MODULO DI UN VETTORE}}\\
    \parbox{\linewidth}{Il modulo di un vettore è la sua \quotes{lunghezza geometrica} e si indica come segue
    \[v = \left \vert \vec{v} \right \vert\]
    È chiaro che il modulo può essere \textbf{positivo o nullo}, mai negativo. In termini di componenti il modulo si calcola come segue
    \[v = \sqrt{v_x^2 + v_y^2 + v_z^2}\]
    \vspace{3mm}}\\
    \hline
\end{tabularx}

\vspace{1em}
\noindent
Per esempio, si calcoli il modulo del vettore
\[\hat{n} = \frac{\vec{v}}{v}\]
ovviamente si procede come segue
\[\left \vert \frac{\vec{v}}{v} \right \vert = \frac{1}{\left \vert v \right \vert} \cdot \left \vert \vec{v} \right \vert = \frac{v}{v} = 1\]
Ecco che allora tale vettore è a tutti gli effetti un versore in direzione $\vec{v}$, in quanto di modulo $1$.\\
Questo fa capire come si possa definire un versore associato a qualunque vettore: basta dividere il vettore per il suo modulo.

\vspace{1em}
\noindent
Mentre di seguito si espone la definizione di \textbf{direzione di un vettore}:

% Tabella per le definizione di concetti, etc...
\vspace{1em}
\rowcolors{1}{black!5}{black!5}
\setlength{\tabcolsep}{14pt}
\renewcommand{\arraystretch}{2}
\noindent
\begin{tabularx}{\textwidth}{@{}|P|@{}}
    \hline
    {\textbf{DIREZIONE DI UN VETTORE}}\\
    \parbox{\linewidth}{La direzione di un vettore (e anche il suo verso) è definita, in due dimensioni, come l'angolo $\theta$ che il vettore descrive con il semiasse positivo delle ascisse.\\
    È immediato osservare che
    \begin{center}
      $\begin{array}{c}
        v_x = v \cdot \cos(\theta)\\
        v_y = v \cdot \sin(\theta)
      \end{array}$
    \end{center}
    e si può verificare che
    \[\left \vert \vec{v} \right \vert = \sqrt{v_x^2 + v_y^2} = \sqrt{v^2 \cdot \cos^2(\theta) + v^2 \cdot \sin^2(\theta)} = v \cdot \sqrt{\cos^2(\theta) + \sin^2(\theta)} = v\]
    \vspace{-1mm}}\\
    \hline
\end{tabularx}

\newpage
\noindent
\subsection{Prodotto scalare}
Di seguito si espone la definizione di \textbf{prodotto scalare}:

% Tabella per le definizione di concetti, etc...
\vspace{1em}
\rowcolors{1}{black!5}{black!5}
\setlength{\tabcolsep}{14pt}
\renewcommand{\arraystretch}{2}
\noindent
\begin{tabularx}{\textwidth}{@{}|P|@{}}
    \hline
    {\textbf{PRODOTTO SCALARE}}\\
    \parbox{\linewidth}{Il prodotto scalare tra due vettori $\vec{v}$ e $\vec{u}$, in termini di componenti si definisce come segue:
    \[\vec{v} \cdot \vec{u} = v_x \cdot u_x + v_y \cdot u_y + v_z \cdot u_z\]
    che è, naturalmente, uno scalare.\\
    Analogamente si può interpretare il prodotto scalare tra due vettori $\vec{v}$ e $\vec{u}$ come il prodotto dei moduli per il \textbf{coseno} dell'angolo $\theta$ compreso tra i vettori stessi, ovvero
    \[\vec{v} \cdot \vec{u} = v \cdot u \cdot \cos(\theta)\]
    \vspace{-1mm}}\\
    \hline
\end{tabularx}

\vspace{1em}
\noindent
\textbf{Osservazione}: Naturalmente, da tale definizione seguono delle importanti osservazioni:
\begin{itemize}
  \item \(\vec{v} \cdot \vec{v} = v_x^2 + v_y^2 + v_z^2 = \left \vert \vec{v} \right \vert ^2\)
  \item \(\hat{i} \cdot \hat{i} = \hat{j} \cdot \hat{j} = \hat{k} \cdot \hat{k} = 1\)
  \item \(\hat{i} \cdot \hat{j} = \hat{i} \cdot \hat{k} = \hat{j} \cdot \hat{k} = 0\). Questo significa che i due vettori considerati sono ortogonali, ovvero i versori $\hat{i}$, $\hat{j}$ e $\hat{k}$ sono a due a due ortogonali.
\end{itemize}
Si consideri, invece, l'esempio seguente:
\[\vec{v} \cdot \hat{i} = \left(v_x \cdot \hat{i} + v_y \cdot \hat{j} + v_z \cdot \hat{k} \right) \cdot \hat{i} = v_x \cdot \hat{i} \cdot \hat{i} + v_y \cdot \hat{i} \cdot \hat{j} + v_z \cdot \hat{i} \cdot \hat{k} = v_x\]
e questo significa che $\vec{v} \cdot \hat{i}$ è la \textbf{proiezione} del vettore $\vec{v}$ in direzione $\hat{i}$. Tale metodo è molto efficace per effettuare un cambio di base: se al posto dei versori $\hat{i}$, $\hat{j}$ e $\hat{k}$, che presuppongono l'origine del sistema di riferimento in $O = (0,0,0)$ si scegliessere degli altri versori, moltiplicando il vettore $\vec{v}$ per taluni versori si otterrebbero le componenti del nuovo vettore in una nuova base.

\vspace{1em}
\noindent
\textbf{Osservazione}: Se si considerando due vettori $\vec{c}$ e $\vec{d}$, allora il loro prodotto scalare può essere interpretato come segue
\[\vec{c} \cdot \vec{d} \cdot \frac{d}{d} = d \cdot \left(\vec{c} \cdot \frac{\vec{d}}{d}\right)\]
per cui, ricordando che
\[\hat{n} = \frac{\vec{d}}{d}\]
è un versore in direzione del vettore $\vec{d}$, allora il prodotto scalare tra $\vec{c}$ e $\vec{d}$ è proprio la proiezione del vettore $\vec{c}$ sul vettore $\vec{d}$, per quanto appena detto a proposito delle \textbf{proiezioni}, moltiplicata per il modulo del vettore $\vec{d}$.

\newpage
\noindent
\begin{center}
  3 Marzo 2022
\end{center}

\section{Cinematica}
La descrizione del moto di un corpo (approssimato ad un punto) prende il nome di \textbf{cinematica} (mentre la ragione del moto viene studiata dalla \textbf{dinamica}). Com'è noto, inoltre, un vettore è una quantità con \textbf{modulo} e \textbf{direzione} (e \textbf{verso}). La descrizione di un vettore avviene tramite le sue componenti: in particolare, dato un versore $\hat{n}$, la componente di un vettore $\vec{v}$ in direzione del versore $\hat{n}$ è così definita
\[\vec{v} \cdot \hat{n}\]
Per esempio, la componente del vettore $\vec{v}$ lungo l'asse $x$ è
\[\vec{v} \cdot \hat{i} = v_x\]
Inoltre, il \textbf{prodotto scalare} tra due vettori $\vec{v}$ e $\vec{u}$ viene definito come:
\[\vec{v} \cdot \vec{u} = v_x \cdot u_x + v_y \cdot u_y \cdot v_z \cdot u_z = v \cdot z \cdot \cos(\theta)\]
ove $\theta$ è l'angolo compreso tra i due vettori $\vec{v}$ e $\vec{u}$.\\
Grazie a ciò è possibile definire il concetto di \textbf{Cinematica}:

% Tabella per le definizione di concetti, etc...
\vspace{1em}
\rowcolors{1}{black!5}{black!5}
\setlength{\tabcolsep}{14pt}
\renewcommand{\arraystretch}{2}
\noindent
\begin{tabularx}{\textwidth}{@{}|P|@{}}
    \hline
    {\textbf{CINEMATICA}}\\
    \parbox{\linewidth}{La cinematica è lo \textbf{studio del moto}, a differenza della \textbf{dinamica} che studia la \textbf{causa del moto} e della \textbf{statica} che studia l'\textbf{equilibrio meccanico}, ossia la \textbf{causa dell'immobilità}.
    \vspace{3mm}}\\
    \hline
\end{tabularx}

\vspace{1em}
\noindent
È chiaro che lo studio di un corpo complesso e non omogeneo è molto più elaborato dello studio di un solo \textbf{punto}. Pertanto, il primo passo per lo studio del moto è quello di studiare il comportamento di un modello standard a cui può essere ricondotto, tramite approssimazione, un altro corpo, a seconda della necessità.

\vspace{1em}
\subsection{Posizione e spostamento}
Dato un punto nello spazio, la sua posizione viene descritta tramite un \quotes{vettore} posizione $\vec{r}$, di cui è possibile calcolare la lunghezza $\left(\vec{r}\right)$, la quale, tuttavia, non ha molto significato dal momento che dipende dalla posizione dell'origine del sistema scelta: ovverosia dipende dalla posizione iniziale e, quindi, dal \textbf{sistema di riferimento adottato}:

\begin{figure}[H]
  \centering
  \begin{tikzpicture}
      \draw [-stealth] (0,0) -- (0,2);
      \draw [-stealth] (0,0) -- (1.5,-1);
      \draw [-stealth] (0,0) -- (-1.5,-1);
      \draw [-stealth, blue] (0,0) -- (1.5,1);
      \draw [blue] (0.75,0.8) node[]{$\vec{r}$};
  \end{tikzpicture}
  \caption{\quotes{Vettore} posizione}
  \label{fig:vettore_posizione}
\end{figure}

\vspace{1em}
\noindent
Conoscere il sistema di riferimento è fondamentale, in quanto in base a ciò possono essere effettuate diverse valutazioni che, naturalmente, variano a seconda del sistema di riferimento scelto: si pensi ed effettuare una misurazione adottando come sistema di riferimento un treno che si muove oppure un treno immobile, o ancora un treno che sta accelerando: si parla, in tale contesto, di un \textbf{sistema di riferimento non inerziale}.

% Tabella per le definizione di concetti, etc...
\vspace{1em}
\rowcolors{1}{black!5}{black!5}
\setlength{\tabcolsep}{14pt}
\renewcommand{\arraystretch}{2}
\noindent
\begin{tabularx}{\textwidth}{@{}|P|@{}}
    \hline
    {\textbf{SPOSTAMENTO}}\\
    \parbox{\linewidth}{Lo \textbf{spostamento}, invece, è proprio un vettore e, com'é intuibile, taluno è definito come la differenza tra due posizioni, ovvero
    \[\boxed{\Delta \vec{r} = \vec{r_2} - \vec{r_1}}\]
    di cui è possibile calcolare il modulo come segue
    \[\left \vert \Delta \vec{r} \right \vert = \left \vert \vec{r_2} - \vec{r_1} \right \vert = \sqrt{\left(x_2 - x_1\right)^2 + \left(y_2 - y_1\right)^2 + \left(z_2 - z_1\right)^2} = \text{ distanza}\]
    \vspace{1mm}}\\
    \hline
\end{tabularx}

\vspace{1em}
\noindent
Per esempio, la lunghezza sull'asse $x$ è
\[\Delta \vec{r} = \Delta x \cdot \hat{i}\]
di cui
\[\left \vert \Delta \vec{r} \right \vert = \left \vert x_2 - x_1 \right \vert\]

\vspace{1em}
\subsection{Posizione in funzione del tempo}
È particolarmente importante considerare la variazion della posizione nel tempo

\begin{figure}[H]
  \centering
  \begin{tikzpicture}
      \draw (0,0) node[circ]{} to[out=80,in=180] (2,1);
      \draw (2,1) to[out=0,in=-120] (4,2) node[circ]{};
      \draw (0,-0.3) node[]{$\vec{r}_i$};
      \draw (4,2.3) node[]{$\vec{r}_f$};
      \draw [-stealth] (1,0.91) -- (1.8,1.2);
      \draw (1.2,1.3) node[]{$\vec{r}(t)$};
  \end{tikzpicture}
  \caption{\quotes{Vettore} posizione in funzione del tempo}
  \label{fig:vettore_posizione_funzione_tempo}
\end{figure}

Sia data una funzione spostamento, definita in funzione del tempo $t$, quale
\[\vec{r}(t) = x(t) \cdot \hat{i} + y(t) \cdot \hat{j}\]
in cui
\begin{flalign*}
  x(t) & = 2 \text{ m } + \left(2 \text{ m/s} \right) \cdot t\\
  y(t) & = 0 \text{ m } + \left(4 \text{ m/s} \right) \cdot t
\end{flalign*}
Naturalmente si ha

\vspace{2em}
\noindent
\rowcolors{1}{white}{white}
\begin{tabularx}{\textwidth}{P}
  {
      \centering
      \begin{tikzpicture}
        \begin{axis}[
          grid=both,
          axis lines = middle,
          xlabel = \(t\),
          ylabel = {\(x(t)\)},
          legend pos=outer north east,
          ymajorgrids=true,
          xmajorgrids=true,
          grid style=dashed,
        ]

        \addplot [
          domain=-2:10,
          samples=100,
          color=red,
        ]
        {2 + 2*x};
        \addlegendentry{\(x(t) = 2 \text{ m } + \left(2 \text{ m/s} \right) \cdot t\)}
        \end{axis}
    \end{tikzpicture}
  }
\end{tabularx}

\vspace{2em}
\noindent
\rowcolors{1}{white}{white}
\begin{tabularx}{\textwidth}{P}
  {
      \centering
      \begin{tikzpicture}
          \begin{axis}[
            grid=both,
            axis lines = middle,
            xlabel = \(t\),
            ylabel = {\(y(t)\)},
            legend pos=outer north east,
            ymajorgrids=true,
            xmajorgrids=true,
            grid style=dashed,
          ]

          \addplot [
            domain=-2:10,
            samples=100,
            color=blue,
          ]
          {4*x};
          \addlegendentry{\(y(t) = 0 \text{ m } + \left(4 \text{ m/s} \right) \cdot t\)}
          \end{axis}
      \end{tikzpicture}
  }
\end{tabularx}

\vspace{2em}
\noindent
\rowcolors{1}{white}{white}
\begin{tabularx}{\textwidth}{P}
  {
      \centering
      \begin{tikzpicture}
        \begin{axis}[
          grid=both,
          axis lines = middle,
          xlabel = \(t\),
          ylabel = {\(r(t)\)},
          legend pos=outer north east,
          ymajorgrids=true,
          xmajorgrids=true,
          grid style=dashed,
        ]
      \addplot[
        domain=-2:10,
        samples=100,
        color=orange,
      ]
      ({2 + 2*x},
      {4*x});
      \addlegendentry{\(\vec{r}(t) = x(t) \cdot \hat{i} + y(t) \cdot \hat{j}\)}
      \end{axis}
      \end{tikzpicture}
        }
\end{tabularx}

\vspace{1em}
\subsection{Velocità}
La \textbf{velocità} si pone alla base della cinematica. In fisica la velocità si distingue in due tipolgie
\begin{itemize}
  \item Velocità istantanea
  \item Velocità media
\end{itemize}
Si consideri, a tal proposito, il seguente grafico spazio-tempo:

\vspace{2em}
\noindent
\rowcolors{1}{white}{white}
\begin{tabularx}{\textwidth}{P}
  {
      \centering
      \begin{tikzpicture}
        \begin{axis}[
          grid=both,
          axis lines = middle,
          xlabel = \(t\),
          ylabel = {\(x\)},
          legend pos=outer north east,
          ymajorgrids=true,
          xmajorgrids=true,
          grid style=dashed,
          xtick={50,120},
          xticklabels={$t_1$,$t_2$},
          ytick={76,166},
          yticklabels={$x_1$,$x_2$},
        ]
      \addplot[
        domain=0:200,
        samples=100,
        color=orange,
      ]
      {abs(x*(abs(sin(2*x)) + 0.5) + 2)};
      \addplot [color=red,mark=*] coordinates {(50,76)};
      \addplot [color=red,mark=*] coordinates {(120,166)};
      \addplot [color=blue] coordinates {(50,76)(120,166)};
      %\addlegendentry{\(\vec{r}(t) = x(t) \cdot \hat{i} + y(t) \cdot \hat{j}\)}
      \end{axis}
      \end{tikzpicture}
        }
\end{tabularx}

% Tabella per le definizione di concetti, etc...
\vspace{1em}
\rowcolors{1}{black!5}{black!5}
\setlength{\tabcolsep}{14pt}
\renewcommand{\arraystretch}{2}
\noindent
\begin{tabularx}{\textwidth}{@{}|P|@{}}
    \hline
    {\textbf{VELOCITÀ MEDIA}}\\
    \parbox{\linewidth}{Intuitivamente si ha che la velocità media è proprio il rapporto tra uno spostamento e il tempo impiegato per effettuarlo, ovvero
    \[\boxed{\left<v\right> = \frac{\Delta \vec{r}}{\Delta t} = \frac{x_2 - x_1}{t_2 - t_1}}\]
    che, graficamente, può essere interpretata come la pendenza (o coefficiente angolare), della congiungente i punti $(x_1,t_1)$ e $(x_2,t_2)$ nel grafico spazio/tempo.
    \vspace{3mm}}\\
    \hline
\end{tabularx}
\vspace{1em}

% Tabella per le definizione di concetti, etc...
\vspace{1em}
\rowcolors{1}{black!5}{black!5}
\setlength{\tabcolsep}{14pt}
\renewcommand{\arraystretch}{2}
\noindent
\begin{tabularx}{\textwidth}{@{}|P|@{}}
    \hline
    {\textbf{VELOCITÀ ISTANTANEA}}\\
    \parbox{\linewidth}{Mentre la velocità istantanea è, naturalmente, la derivata nel tempo del vettore posizione, ovvero
    \[\boxed{\vec{v}(t) = \frac{d\vec{r}}{dt} = \lim_{t \to 0} \frac{x - x_0}{t - t_0}}\]
    ovvero la retta tangente il grafico della funzione posizione nel punto $(x_0,t_0)$. Naturalmente essendo un vettore la velocità istantanea, è possibile descriverlo tramite \textbf{componenti} come segue:
    \[\vec{v} = \frac{d}{dt} \left(\vec{r}(t)\right) = \underbrace{\frac{dx}{dt} \cdot \hat{i}}_\text{$v_x$} + \underbrace{\frac{dy}{dt} \cdot \hat{j}}_\text{$v_y$} + \underbrace{\frac{dz}{dt} \cdot \hat{k}}_\text{$v_z$}\]
    in quanto $\hat{i}$, $\hat{j}$ e $\hat{k}$ non dipendono dal tempo (cosa che potrebbe accadere, comunque, in determinate circostanze).\\
    Il \textbf{modulo} della velocità si calcola come segue
    \[\left \vert \vec{v} \right \vert = v = \sqrt{v_x^2 + v_y^2 + v_z^2}\]
    \vspace{-1mm}}\\
    \hline
\end{tabularx}

\vspace{1em}
\noindent
\textbf{Esempio}: Si consideri uno spostamento verso l'alto tale per cui $x_1 = 0 \text{ m}$ e $x_2 = 12000 \text{ m}$ e $t_1 = 2600 \text{ s}$ e $t_2 = 4000 \text{ s}$. Allora si ha che
\[\left<v_z\right> = \frac{x_2 - x_1}{t_2 - t_1} = \frac{12000}{4000 - 2600} = 8.6 \text{ m/s} = 308.6 \text{ km/h}\]
Che è una velocità irrisoria; tuttavia, ciò non sorprende, in quanto è opportuno conoscere anche le altre componenti della velocità, ossia $v_x$ e $v_y$.

\vspace{1em}
\subsection{Accelerazione}
Di seguito si espone la definizione di \textbf{accelerazione}:

% Tabella per le definizione di concetti, etc...
\vspace{1em}
\rowcolors{1}{black!5}{black!5}
\setlength{\tabcolsep}{14pt}
\renewcommand{\arraystretch}{2}
\noindent
\begin{tabularx}{\textwidth}{@{}|P|@{}}
    \hline
    {\textbf{ACCELERAZIONE}}\\
    \parbox{\linewidth}{L'accelerazione viene definita come la derivata prima della velocità nel tempo, o la derivata seconda dello spostamento nel tempo:
    \[\vec{a} = \frac{d \vec{v}}{dt} = \frac{d^2 \vec{r}}{dt}\]
    \vspace{-1mm}}\\
    \hline
\end{tabularx}

\newpage
\noindent
\begin{center}
  7 Marzo 2022
\end{center}
Comè noto, l'accelerazione è la derivata prima della velocità in funzione del tempo, o la derivata seconda dello spostamento in funzione del tempo.\\
L'accelerazione è fondamentale in \textbf{meccanica}, in quanto grazie all'accelerazione è possibile definire il concetto di forza: l'accelerazione è la connessione tra cinematica e dimanica.

\vspace{2em}
\noindent
\rowcolors{1}{white}{white}
\begin{tabularx}{\textwidth}{P}
  {
      \centering
      \begin{tikzpicture}
        \begin{axis}[
          grid=both,
          axis lines = middle,
          xlabel = \(t\),
          ylabel = {\(x(t)\)},
          legend pos=outer north east,
          ymajorgrids=true,
          xmajorgrids=true,
          grid style=dashed,
        ]

        \addplot [
          domain=-2:10,
          samples=100,
          color=red,
        ]
        {(x-4)^3 + 500};
        % \addlegendentry{\(x(t) = 2 \text{ m } + \left(2 \text{ m/s} \right) \cdot t\)}
        \end{axis}
    \end{tikzpicture}
  }
\end{tabularx}

\vspace{1em}
\noindent
Avendo a disposizione il grafico che descrive la variazione della posizione nel tempo, è possibile ora studiarne l'andamento per poi descrivere il comportamento della velocità nel secondo grafico. È sufficiente, pertanto, osservare gli intervalli di crescenza e decrescenza e i punti in cui la derivata si annulla nulla:

\vspace{2em}
\noindent
\rowcolors{1}{white}{white}
\begin{tabularx}{\textwidth}{P}
  {
      \centering
      \begin{tikzpicture}
        \begin{axis}[
          grid=both,
          axis lines = middle,
          xlabel = \(t\),
          ylabel = {\(v(t)\)},
          legend pos=outer north east,
          ymajorgrids=true,
          xmajorgrids=true,
          grid style=dashed,
        ]

        \addplot [
          domain=-2:10,
          samples=100,
          color=red,
        ]
        {(x-4)^2+1000};
        % \addlegendentry{\(x(t) = 2 \text{ m } + \left(2 \text{ m/s} \right) \cdot t\)}
        \end{axis}
    \end{tikzpicture}
  }
\end{tabularx}

\vspace{1em}
\noindent
Ancora una volta, studiando l'andamento della velocità nel suo rispettivo grafico è ora possibile descrivere il grafico della derivata della velocità, ovvero dell'accelerazione, sempre analizzando gli intervalli di crescenza e decrescenza:

\vspace{2em}
\noindent
\rowcolors{1}{white}{white}
\begin{tabularx}{\textwidth}{P}
  {
      \centering
      \begin{tikzpicture}
        \begin{axis}[
          grid=both,
          axis lines = middle,
          xlabel = \(t\),
          ylabel = {\(a(t)\)},
          legend pos=outer north east,
          ymajorgrids=true,
          xmajorgrids=true,
          grid style=dashed,
        ]

        \addplot [
          domain=-2:10,
          samples=100,
          color=red,
        ]
        {x - 4};
        % \addlegendentry{\(x(t) = 2 \text{ m } + \left(2 \text{ m/s} \right) \cdot t\)}
        \end{axis}
    \end{tikzpicture}
  }
\end{tabularx}

\vspace{1em}
\noindent
Ecco che il punto in cui la derivata seconda (ovverosia l'accelerazine) cambia segno è un \textbf{punto di flesso}, ovvero il punto in cui il grafico dello spostamento cambia la propria concavità.

\vspace{1em}
\noindent
\textbf{Esempio}: Si consideri la seguente funzione spostamento in funzione del tempo:
\[x(t) = A \cdot \cos(\omega t)\]
questa è l'equazione di oscillazione di un pendolo o di una molla. Per il calcolo della velocità è sufficiente calcolare la derivata prima, ovvero
\[v(t) = \frac{dx}{dt} = -\omega \cdot A \cdot \sin(\omega t)\]
e per l'accelerazione è sufficiente derivare nuovamente la velocità
\[a(t) = \frac{dv}{dt} = -\omega^2 \cdot A \cdot \cos(\omega t) = -\omega^2 \cdot x(t)\]
Tale risultato ha senso e può essere interpretato anche graficamente, grazie al grafico di una molla: quando lo spostamento è positivo, l'accelerazione è negativa, e viceversa.

\begin{figure}[H]
  \centering
  \begin{tikzpicture}[every node/.style={draw,outer sep=0pt,inner sep=0pt,thick}, scale=2]
    \tikzstyle{spring}=[thick,decorate,decoration={aspect=0.5, segment length=1mm, amplitude=2mm,coil}]
    \draw[thick] (0,0) --(0,1);
    \draw[thick] (0,0) --(3,0) node[draw=none,xshift=5pt]{$x$};
    \node at(0,0.25) (a) [draw=none] {};
    \node at (2,0.25)(b) [minimum size=0.5cm,label=$\rightarrow$] {m};
    \draw [spring] (a) -- (b) node[draw=none,pos=.5,right=.25cm] {};
    \node at (2,-0.3)(c) [draw=none,yshift=5pt] {$x=0$};
    \draw[dashed] (b.south) -- (c.north);
  \end{tikzpicture}
  \caption{Fisica di una molla}
  \label{fig:fisica_molla}
\end{figure}

\vspace{1em}
\noindent
\textbf{Osservazione}: Quando la velocità è nulla, la posizione si mantiene costante e non cresce o decresce. Quando si ha un punto di salto della velocità si ha una situazione difficile da riprodurre fisicamente, in quanto si ha un crollo della velocità istantanea, come un impulso (si pensi anche al fatto che, per il teorema del limite della derivata, una funzione con un salto non può essere la derivata di una funzione derivabile).\\
Si può utilizzare anche l'\textbf{integrale} per passare dalla velocità allo spostamento.

\vspace{1em}
\subsection{Moto uniformemente accelerato}
Nel moto uniformemente accelerato si ha che l'\textbf{accelerazione} è \textbf{costante}. Sapenche l'accelerazione è la derivata prima della velocità nel tempo, si può scrivere:
\[\frac{dv}{dt} = a \longrightarrow dv = a \cdot dt\]
Questo, in particolare, è possibile farlo sia per una variazione $\Delta$, ma anche per una variazione infinitesimale $d$. Ciò che si sta facendo, in questo caso, è risolvere un'\textbf{equazione differenziale}. Pertanto, dopo aver ottenuto $dv = a \cdot dt$ si può procedere all'integrazione
\[\int dv = \int a \cdot dt \longrightarrow v = at + c\]
La costante $c$ che compare nella formula, ottenuta grazie alla risoluzione di una equazione differenziale, è la cosiddettà \textbf{velocità inziale}: se $t=0$, infatti, $v = c = v_0$. Quindi l'equazione diviene
\[\boxed{v(t) = v_0 + a \cdot t}\]
Avendo ottenuto l'equazione della velocità nel tempo, è opportuno ottenere l'equazione della posizione in funzione del tempo, integrando nuovamente, sempre partendo da
\[\frac{dx}{dt} = v \longrightarrow dx = v \cdot dt \longrightarrow \int dx = \int v \cdot dt = \int (v_0 + a \cdot t) dt\]
quindi si ottiene
\[x(t) = \int v_0 \cdot dt + \int a \cdot t \cdot dt + c = v_0 \cdot t + a \cdot \frac{t^2}{2} + c\]
ove $c = x_0 = x(t=0)$. Pertanto l'equazione della posizione in funzione del tempo è
\[\boxed{x(t) = x_0 + v_0 \cdot t + \frac{1}{2} \cdot a \cdot t^2}\]
che rappresenta l'equazione di una parabola, come illustrato nell'esempio seguente:

\vspace{2em}
\noindent
\rowcolors{1}{white}{white}
\begin{tabularx}{\textwidth}{P}
  {
      \centering
      \begin{tikzpicture}
        \begin{axis}[
          axis lines = left,
          xlabel = \(t\),
          ylabel = {\(x(t)\)},
          legend pos=outer north east,
          ymajorgrids=true,
          xmajorgrids=true,
          grid style=dashed,
          ytick={-11},
          yticklabels={$x_0$},
          ymax=15,
        ]

        \addplot [
          domain=-2:10,
          samples=100,
          color=red,
        ]
        {-(x-2)^2 + 5};
        \draw [-stealth, blue, thick] (axis cs:-2,-11) -- (axis cs:1,10);
        \draw [blue] (axis cs:-1.2,0) node[]{$\vec{v}_0$};
        \end{axis}
    \end{tikzpicture}
  }
\end{tabularx}

\vspace{1em}
\noindent
In cui, ovviamente, l'intersezione tra il grafico e l'asse $y$ è $x_0$, il vettore designato in blu, ossia la tangente in $(x_0,0)$, rappresenta la pendenza iniziale del grafico della posizione, ovvero la velocità iniziale $v_0$. Essndo il grafico di un moto uniformemente accelerato, è ovvio che la pendenza decresce in modo costante: prima la velocità sarà positiva, ma dcrescente, e poi continuerà a decrescere, ma con valori negativi.

\vspace{1em}
\noindent
\textbf{Osservazione}: Si consideri la seguente equazione
\[v(t) = a \cdot t + v_0\]
e si provi ad isolare $t$ da tale equazione, come segue
\[t = \frac{v - v_0}{a}\]
se ora si considera l'equazione seguente
\[x(t) = \frac{1}{2} \cdot a \cdot t^2 + v_0 \cdot t + x_0\]
e si sostituisce il $t$ calcolato in precedenza a tale equazione si ottiene
\[x(t) = \frac{1}{2} \cdot a \cdot \left(\frac{v - v_0}{a}\right)^2 + v_0 \cdot \left(\frac{v - v_0}{a}\right) + x_0 = \frac{1}{2} \cdot \frac{1}{a} \cdot (v^2 - 2 \cdot v \cdot v_0 + v_0^2) + \frac{1}{a} \cdot (v_0 \cdot v + v_0^2) + x_0\]
ovvero si ottiene
\[\boxed{x = \frac{v^2}{2a} - \frac{v_0^2}{2a} + x_0}\]
quindi
\[\boxed{v^2 - v_0^2 = 2a \cdot (x - x_0)}\]
la quale è \textbf{valida solamente per il moto uniformemente accelerato} ed è estremamente utile per conoscere lo spazio percorso da un corpo che si muove secondo queste legge oraria, senza conoscere il \textbf{tempo}.

\vspace{1em}
\subsubsection{Caduta libera dei gravi}
La \textbf{caduta libera} avviene con la \textbf{medesima accelerazione} per tutti i corpi (ovviamente, l'attrazione gravitazionale non è costante in tutto l'universo, ma se si considerano un punto sulla terra e distanze piccole e prossime a quella del raggio terrestre, si può, senza perdita di generalità, considerare un'accelerazione costante $g$).

\vspace{1em}
\noindent
\textbf{Osservazione}: Naturalmente sulla Luna non c'è aria, si è nel vuoto, per cui tutti i corpi vengono attratti dalla Luna solamente per la forza di attrazione gravitazionale, senza che tale moto sia influenzato dall'attrito dell'aria.

\vspace{1em}
\noindent
L'accelerazione gravitazionale può essere interpretata come segue:

\begin{figure}[H]
  \centering
  \begin{tikzpicture}
    \draw [-stealth] (0,0) -- (0,-5);
    \draw (2,-2.5) node[]{$\vec{a} = -g \cdot \hat{j} \hspace{0.5em} \text{con} \hspace{0.5em} 9.8 \text{ m/s}$};
  \end{tikzpicture}
  \caption{Visualizzazione grafica del moto di caduta libera}
  \label{fig:visualizzazione_grafica_caduta_libera}
\end{figure}


\vspace{1em}
\noindent
Ove, naturalmente si ha
\[\vec{a} = -g \cdot \hat{j} \hspace{0.5em} \text{con} \hspace{0.5em} 9.8 \text{ m/s}\]
Pertanto si ottengono le seguenti equazioni, a partire da quelle generali per un moto uniformemente accelerato. Per quanto riguarda la velocità di caduta si ha:
\[\boxed{v_y = -g \cdot t + v_{0y}}\]
mentre per quanto riguarda la posizione in funzione del tempo si ottiene
\[\boxed{y = - \frac{1}{2} \cdot g \cdot t^2 + v_{0y} + y_0}\]
Pe capire che altezza raggiungerà un corpo quando viene lanciato verso l'alto si deve osservare il grafico seguente

\vspace{2em}
\noindent
\rowcolors{1}{white}{white}
\begin{tabularx}{\textwidth}{P}
  {
      \centering
      \begin{tikzpicture}
        \begin{axis}[
          axis lines = left,
          xlabel = \(t\),
          ylabel = {\(x(t)\)},
          legend pos=outer north east,
          ymajorgrids=true,
          xmajorgrids=true,
          grid style=dashed,
          ytick={-11,5},
          yticklabels={$y_0$,$y_m$},
          ymax=15,
        ]

        \addplot [
          domain=-2:10,
          samples=100,
          color=red,
        ]
        {-(x-2)^2 + 5};
        \draw [thick, red, dashed] (axis cs:0,5) -- (axis cs:4,5);
        \draw [thick, red] (axis cs:2,9) node[]{$v_y=0$};
        \end{axis}
    \end{tikzpicture}
  }
\end{tabularx}

\vspace{1em}
\noindent
Naturalmente si osserva che la velocità nel punto più alto è nulla, in quanto la tangente è orizzontale, mentre si conosce la velocità inziale $v_0$.\\
Per capire l'altezza, allora, si potrebbe calcolare il tempo impiegato per raggiungere il punto più alto e poi sostituire tale valore all'interno dell'equazione dello spostamento.\\
 Alternativamente, si potrebbe considerare la formula seguente
\[v_y^2 - v_{y0}^2 = -2 \cdot g \cdot (y - y_0)\]
e sostitutendo i valori si ha
\[0 - v_{y0}^2 = 2 \cdot g \cdot (y_m - y_0)\]
Per cui l'altezza massima che un corpo raggiunge quando viene lanciato verso l'alto con velocità iniziale $v_{y0}$ e a partire da un'altezza $y_0$ è
\[\boxed{y_m = y_0 + \frac{v_{y0}^2}{2g}}\]

\vspace{1em}
\subsubsection{Moto dei proiettili}
Il moto dei proiettili ha un'importanza storica fondamentale: Aristotele, nel $340$ a.c. parlava di \textbf{moto \quotes{naturale} e \quotes{forzato}}: tuttavia, naturalmente, un oggetto fermo, privo di sollecitazioni, non ha la propensione a muoversi, quindi non ha senso parlare di \emph{forzatura}.\\
Successivamente, Filipono ($490-570$) d.c. ha introdotto il concetto di \textbf{impeto} e, infine, \textbf{Galileo} ($1564-1642$) d.c., basandosi su osservazioni e misurazioni pratiche precedenti (invece che fornire una spiegazioni a priori), ha fornito una \textbf{spiegazione scentifica} a tale fenomeno.

% Tabella per le definizione di concetti, etc...
\vspace{1em}
\rowcolors{1}{black!5}{black!5}
\setlength{\tabcolsep}{14pt}
\renewcommand{\arraystretch}{2}
\noindent
\begin{tabularx}{\textwidth}{@{}|P|@{}}
    \hline
    {\textbf{LEGGE ORARIA DEL MOTO DEI PROIETTILI}}\\
    \parbox{\linewidth}{Per lo studio del moto dei proiettili si considera il vettore accelerazione
    \[\boxed{\vec{a} = a_x \cdot \hat{i} + a_y \cdot \hat{j} = 0 \cdot \hat{i} - g \cdot \hat{j}}\]
    Analogamente per la velocità si ha
    \[\boxed{\vec{v} = v_x \cdot \hat{i} + v_y \cdot \hat{j} = v_{0x} \cdot \hat{i} - (v_{0y} - g t) \cdot \hat{j}}\]
    Per quanto concerne la posizione in funzione del tempo si ha
    \[\boxed{\vec{r} = x \cdot \hat{i} + y \cdot \hat{j} = (v_{0x} t + x_0) \cdot \hat{i} - (y_0 + v_{0y} t - \frac{1}{2} g t^2) \cdot \hat{j}}\]
    \vspace{-1mm}}\\
    \hline
\end{tabularx}

\vspace{2em}
\noindent
Si consideri il seguente grafico della posizione di un proiettile:

\vspace{2em}
\noindent
\rowcolors{1}{white}{white}
\begin{tabularx}{\textwidth}{P}
  {
      \centering
      \begin{tikzpicture}
        \begin{axis}[
          axis lines = left,
          xlabel = \(t\),
          ylabel = {\(x(t)\)},
          legend pos=outer north east,
          ymajorgrids=true,
          xmajorgrids=true,
          grid style=dashed,
          ytick={-11},
          yticklabels={$x_0$},
          ymax=15,
        ]

        \addplot [
          domain=-2:10,
          samples=100,
          color=red,
        ]
        {-(x-2)^2 + 5};
        \draw [-stealth, blue, thick] (axis cs:-2,-11) -- (axis cs:1,10);
        \draw [blue] (axis cs:-1.2,0) node[]{$\vec{v}_0$};
        \draw [blue] (axis cs:-1.6,-11.2) node[]{$\vec{r}_0$};
        \end{axis}
    \end{tikzpicture}
  }
\end{tabularx}

\vspace{1em}
\noindent
Naturalmente si ha che
\[\vec{v_{0}} = v_{x0} \cdot \hat{i} + v_{x0} \cdot \hat{j}\]
\[\vec{r_{0}} = x_0 \cdot \hat{i} + y_0 \cdot \hat{j}\]

\newpage

\noindent
\begin{center}
  8 Marzo 2022
\end{center}
Naturalmente il modulo di un vettore non dipende dal sistema di coordinate scelto.\\
Naturalmente è possibile avere
\[\left \vert \vec{A} + \vec{B} \right \vert = \left \vert \vec{A} - \vec{B} \right \vert\]
e per dimostrarne la veridicità basta considerare due vettori perpendicolari.\\
Ovviamente, un versore ha sempre modulo unitario per essere definito tale: in particolare, dato il vettore $\vec{v} = \hat{i} + \hat{j} + \hat{k}$, il versore
\[\hat{n} = \frac{\vec{v}}{v}\]
è proprio un versore in direzione $\hat{i} + \hat{j} + \hat{k}$.\\
La componente del vettore $\vec{v} = -3 \cdot \hat{i} + 5 \cdot \hat{j} + \hat{k}$ in direzione $\hat{n} = 0.6 \cdot \hat{j} - 0.8 \cdot \hat{k}$ è ovviamente
\[\vec{v} \cdot \hat{n} = -3 \cdot 0 + 5 \cdot 0.6 - 1 \cdot 0.8 = 2.2\]
È chiaro che in questo caso $\hat{n}$ era già un versore, altrimenti, se si avesse avuto un vetttore, si sarebbe dovuto calcolare il versore corrispondente dividendo per il suo modulo.

\vspace{1em}
\noindent
Nel moto dei proiettili è estremamente importante considerare l'\textbf{altezza massima} che esso raggiungerà, ma soprattutto la sua \textbf{gittata}, ovvero la distanza massima che esso raggiungerà.\\
Molto spesso, in questo caso, per lavorare è molto più convieniente operare con le coordinate polari $v_0$ e $\theta$ (ovvero con modulo e angolo), anziché con $v_{x0}$ e $v_{y0}$, sempre ricordando che
\[v = \left(
  \begin{array}{l}
    v_{x0} = v_0 \cdot \cos(\theta)\\
    v_{y0} = v_0 \cdot \sin(\theta)\\
  \end{array}
\right)\]
In questo caso, per calcolare l'altezza massima raggiunta è sufficiente considerare solamente la componente della velocità verticale, come per la caduta libera dei gravi, ovvero
\[y_m = y_0 + \frac{v_{y0}^2}{2g} = y_0 + \frac{v_0^2 \cdot \sin^2(\theta)}{2g}\]
Analogamente, per calcolare la gittata, ovvero la distanza orizzontale percorsa da un corpo lanciato in aria, si dovrà usare solo la componente della velocità orrizontale.

\vspace{2em}
\noindent
\rowcolors{1}{white}{white}
\begin{tabularx}{\textwidth}{P}
  {
      \centering
      \begin{tikzpicture}
        \begin{axis}[
          axis lines = left,
          xlabel = \(y\),
          ylabel = {\(x\)},
          legend pos=outer north east,
          ymajorgrids=true,
          xmajorgrids=true,
          grid style=dashed,
        ]

        \addplot [
          domain=0:10,
          samples=100,
          color=red,
        ]
        {-(x-2)^2 + 4};
        % \addlegendentry{\(x(t) = 2 \text{ m } + \left(2 \text{ m/s} \right) \cdot t\)}
        \end{axis}
    \end{tikzpicture}
  }
\end{tabularx}

\vspace{1em}
\noindent
Naturalmente, in questo caso, il calcolo della gittata $R$ si effettua come segue: è noto che
\[R = v_{x0} \cdot t_R\]
ove $t_R$ è la \textbf{durata del volo}. Però è noto che al tempo $t_R$ la coordinata $y$ è nulla, ovvero
\[y(t_R) = 0 = v_{y0} \cdot t_R - \frac{1}{2} \cdot g \cdot t_R^2\]
Dal momento che il tempo $t_R$ non è nullo è possibile dividere per $t$, ottenendo
\[v_{y0} - \frac{1}{2} \cdot g \cdot t_R = 0\]
da cui si evince che il tempo $t_R$ cercato è
\[t_R = \frac{2 \cdot v_{y0}}{g}\]
sostituendo, ora, il tempo trovato nella formula di cui sopra si ottiene
\[R = v_{x0} \cdot \frac{2 \cdot v_{y0}}{g} = 2 \cdot \frac{v_{x0} \cdot v_{y0}}{g}\]
ma ricordando come si calcolano le componenti $v_{x0}$ e $v_{y0}$ si ha
\[R = 2 \cdot \frac{v_0^2 \cdot \sin(\theta) \cdot \cos(\theta)}{g}\]
ed essendo $2 \cdot \sin(\theta) \cdot \cos(\theta) = \sin(2 \cdot \theta)$ si ottiene
\[\boxed{R = \frac{v_0^2 \cdot \sin(2\theta)}{g}}\]
Le unità in gioco sono
\[[v^2] = \frac{\text{m}^2}{\text{s}^2}\]
\[[g] = \frac{\text{m}}{\text{s}^2}\]
\[[R] = \frac{\dfrac{\text{m}^2}{\text{s}^2}}{\dfrac{\text{m}}{\text{s}^2}} = \text{m}\]

\vspace{1em}
\noindent
\textbf{Osservazione}: Si osservi che, naturalmente, quando $\theta=0^\circ$, si ha che $\sin(0) = 0$ e quindi $R=0$: ciò ha senso, in quanto lanciando orizzontalmente il corpo cade immediatamente.\\
Quando $\theta=90^\circ$ non si ha gittata, in quanto, lanciando verticalmente il corpo cada verticalmene.\\
Quando $\theta=45^\circ$ si ha la gittata massima, in quanto $\sin(90)=1$.

\vspace{1em}
\subsection{Moto in 2D e 3D}
Si consideri la seguente traiettoria

\begin{figure}[H]
  \centering
  \begin{tikzpicture}
      \draw (0,0) node[circ]{} to[out=80,in=180] (2,1);
      \draw (2,1) to[out=0,in=-120] (4,2) node[circ]{};
      \draw (0,-0.3) node[]{$\vec{r}_i$};
      \draw (4,2.3) node[]{$\vec{r}_f$};
      \draw [-stealth, red] (1,0.91) -- (1.8,1.2);
      \draw (1.2,1.3) node[]{$\vec{v}(t)$};
      \draw [-stealth, blue] (1,0.91) -- (1.2,0.4);
      \draw (0.85,0.6) node[]{$\vec{a}_\perp$};
  \end{tikzpicture}
  \caption{\quotes{Vettore} velocità in funzione del tempo}
  \label{fig:vettore_velocità_funzione_tempo}
\end{figure}

\vspace{1em}
\noindent
Naturalmente $\vec{v}$ è sempre parallelo alla tangente della curva $\vec{r}(t)$.\\
Si osservi, inoltre, che l'accelerazione deve sempre presentare una \textbf{componente lineare} $\vec{a}_\parallel$ (parallela a $\vec{v}$) e una \textbf{componente ortogonale} $\vec{a}_\perp$ (in direzione del cambiamento dell'orientazione della velocità), la quale è fondamentale: infatti, se ci fosse solo una componente lineare, la velocità aumenterebbe il proprio modulo, ma non direzione; ogni qualvolta si ha una variazione della direzione della velocità ci deve essere una componente ortogonale dell'accelerazione.

\vspace{1em}
\noindent
\subsection{Moto circolare uniforme}
Per moto circolare uniforme si intende un moto circolare in cui la \textbf{velocità angolare} si mantiene \textbf{costante}.\\

\begin{figure}[H]
  \centering
  \begin{tikzpicture}[>=Triangle]
    \shade [top color=white, bottom color=gray!50, middle color=white]
      (120:8/3) arc (120:190:8/3) node [black, near end, left] {$\omega$}
      -- (190:25/9) -- (200:15/6) -- (190:20/9) -- (190:7/3)
      arc (190:120:7/3) -- cycle;

    \foreach \i in {90, 210, 330}{
      \draw [->, thick, blue!50!cyan] (\i-65:2) arc (\i-65:\i+60:2);
      \tikzset{shift={(\i:2)}, rotate=\i+180}
      \draw [->, very thick, orange] (0,0) -- (1,0)
        node [black, near end, anchor=\i+90] {$\vec a$};
      \draw [->, very thick, green!50!black] (0,0) -- (0,-2)
        node [black, near end, anchor=\i+180] {$\vec v$};
      \fill circle [radius=1/10];
  }
  \end{tikzpicture}
  \caption{Moto circolare uniforme}
  \label{fig:moto_circolare_uniforme}
\end{figure}

\vspace{1em}
\noindent
Naturalmente, la \textbf{circonferenza} del cerchio è $C = 2\pi R$. Parlando di moto circolare uniforme è possibile introdurre il concetto di \textbf{periodo} $T$, ovvero il tempo impiegato a completare una circonferenza completa.\\
Pertanto, volendo conoscere la velocità del moto si ottiene
\[\boxed{v = \frac{2\pi R}{T} = \omega R}\]
ove $\omega$ prende il nome di \textbf{velocità angolare}, che ha una misura di $\text{RAD}/\text{s}$, per questo prende il nome di velocità angolare, in quanto ha la stessa unità di misura di una frequenza (visto che l'angolo non ha una propria vera misura).\\
Naturalmente, volendo conoscere l'accelerazione che agisce sul punto in movimento, si può immediatamente dire che, essendo la velocità costante, si dovrà solo considerare una componente ortogonale, in quanto se essa non ci fosse, il corpo si muoverebbe in linea retta, senza compiere una traiettoria circolare.\\
Pertanto, dal momento che $\left \vert \vec{v} \right \vert = v =$ costante, si ha che $\vec{a}_\parallel = 0$. L'accelerazione media, naturalmente, può essere calcolata come segue
\[\boxed{\left<\vec{a}\right> = \frac{\Delta \vec{v}}{\Delta t} = \frac{\vec{v}(t_2) - \vec{v}(t_1)}{t_2 - t_1}}\]
Anche graficamente appare evidente come l'accelerazione media sia un vettore ortogonale al vettore velocità e orientato verso il centro della circonferenza.\\

\begin{figure}[H]
  \centering
  \begin{tikzpicture}[scale=2]
    \node[minimum size=4cm,circle,draw,blue!50!cyan] (circle) {};
    \draw [thick] (0,0) -- coordinate[midway](m) (circle.80);
    \draw [thick] (0,0) -- (circle.55);
    \draw [thick, -stealth] (circle.80) -- (circle.55);
    \draw (circle.67.5) ++(0.2,0.2) node[]{$\Delta \vec r$};
    \draw (m) ++(-0.2,0) node[]{$R$};
    \begin{scope}
      \tikzset{shift={(30:2)}, rotate=300}
      \draw [-stealth] (circle.30) -- ++(0.5,0) node [black, near end, anchor=210] {$\vec v(t_1)$};
    \end{scope}
    \begin{scope}
      \tikzset{shift={(30:2)}, rotate=210}
      \draw [-stealth] (circle.300) -- ++(0.5,0) node [black, near end, anchor=90] {\hspace{0.5em} $\vec v(t_2)$};
    \end{scope}
    \begin{scope}
      \tikzset{shift={(30:2)}, rotate=210}
      \draw (circle.300) ++(0.5,0) coordinate(a);
      \tikzset{shift={(30:2)}, rotate=90}
      \draw [-stealth] (a) -- ++(-0.5,0) coordinate(b) node [black, near end, anchor=0] {$-\vec v(t_1)$\hspace{0.5em}};
      \draw [-stealth, orange] (circle.300) -- (b) node [orange, near end, anchor=270] {$\Delta \vec v$\hspace{0.5em}};
    \end{scope}
    \coordinate (i) at (circle.55);
    \coordinate (ctr) at (0,0);
    \coordinate (f) at (circle.80);
    \pic [draw=red, text=red, <->, "$\Delta \theta$", angle eccentricity=1.3, angle radius=1cm] {angle = i--ctr--f};
  \end{tikzpicture}
\end{figure}

\vspace{1em}
\noindent
Lo spostamento tra due punti, naturalmente, è $\Delta \vec{r}$ e i due triangoli che vengono così disegnati sono simili, per cui il rapporto tra i lati corrispondenti deve mantenersi costantte.\\
È facile, pertanto, vedere immediatamente come
\[\boxed{\frac{\left \vert \Delta \vec{r} \right \vert}{R} = \frac{\left \vert \Delta \vec{v} \right \vert}{v}}\]
ossia il rapporto tra lo spostamento e il raggio costante, così come la variazione di velocità e il modulo della velocità costante è uguale. Da ciò si evince che
\[a = \frac{\left \vert \Delta \vec{v} \right \vert}{\Delta t} = \frac{\left \vert \Delta \vec{v} \right \vert}{v} \cdot \frac{v}{\Delta t} = \frac{\left \vert \Delta \vec{r} \right \vert}{R} \cdot \frac{v}{\Delta t} = \frac{v^2}{R}\]
Ovvero si ha che, nel moto circolare uniforme, il modulo dell'accelerazione orientata verso il centro del cerchio è pari a
\[\boxed{a=\frac{v^2}{R}}\]

\vspace{1em}
\noindent
\textbf{Esempio}: Si consideri l'esempio seguente che riguarda un veicolo in movimento a velocità costante su una curva

\begin{figure}[H]
  \centering
  \begin{tikzpicture}
    \draw (0,0) -- (2,0) to[out=0,in=270] coordinate[midway](r) (4,2) -- (4,4);
    \draw (2,2) node[circ](start){} -- coordinate[midway](a) (r) (a) ++(0.3,0.3) node[]{$R$};
    \draw [dotted] (start) -- ++(2,0) node[at end, right]{$t_2$};
    \draw [dotted] (start) -- ++(0,-2) node[at end, below]{$t_1$};
    \draw [-stealth] (2.4,0.05) node[circ]{} -- (2.2,0.8);
    \draw (2.5,0.4) node[]{$\vec{a}$};
    \draw [-stealth, red] (2,0) -- (2,1);
    \draw [-stealth, red] (4,2) -- (3,2);
  \end{tikzpicture}
  \caption{Auto in movimento su una curva}
  \label{fig:auto_movimento_curva}
\end{figure}

\vspace{1em}
\noindent
e si considerino le componenti dell'accelerazione in $x$ e in $y$ in funzione del tempo.

\vspace{2em}
\noindent
\rowcolors{1}{white}{white}
\begin{tabularx}{\textwidth}{P}
  {
      \centering
      \begin{tikzpicture}
        \begin{axis}[
          grid=both,
          axis lines = middle,
          xlabel = \(t\),
          ylabel = {\(a_x\)},
          legend pos=outer north east,
          ymajorgrids=true,
          xmajorgrids=true,
          grid style=dashed,
          xtick={pi/2,pi},
          xticklabels={$t_1$,$t_2$},
          xmin=0,
          xmax=5,
          ymin=-2,
          ymax=2,
        ]

        \addplot [
          domain=pi/2:pi,
          samples=100,
          color=red,
          thick
        ]
        {cos(deg(x)};

        \addplot [
          domain=0:5,
          samples=100,
          color=red,
          dashed
        ]
        {cos(deg(x)};
        \draw [red, thick] (axis cs:0,0) -- (axis cs:pi/2,0);
        \draw [red, thick, dotted] (axis cs:pi,0) -- (axis cs:pi,-1);
        \draw [red, thick] (axis cs:pi,0) -- (axis cs:5,0);
        \end{axis}
    \end{tikzpicture}
  }
\end{tabularx}

\vspace{2em}
\noindent
\rowcolors{1}{white}{white}
\begin{tabularx}{\textwidth}{P}
  {
      \centering
      \begin{tikzpicture}
        \begin{axis}[
          grid=both,
          axis lines = middle,
          xlabel = \(t\),
          ylabel = {\(a_y\)},
          legend pos=outer north east,
          ymajorgrids=true,
          xmajorgrids=true,
          grid style=dashed,
          xtick={pi/2,pi},
          xticklabels={$t_1$,$t_2$},
          xmin=0,
          xmax=5,
          ymin=-2,
          ymax=2,
        ]

        \addplot [
          domain=pi/2:pi,
          samples=100,
          color=blue,
          thick
        ]
        {sin(deg(x))};

        \addplot [
          domain=0:5,
          samples=100,
          color=blue,
          dashed
        ]
        {sin(deg(x))};
        \draw [blue, thick] (axis cs:0,0) -- (axis cs:pi/2,0);
        \draw [blue, thick, dotted] (axis cs:pi/2,0) -- (axis cs:pi/2,1);
        \draw [blue, thick] (axis cs:pi,0) -- (axis cs:5,0);
        \end{axis}
    \end{tikzpicture}
  }
\end{tabularx}

\vspace{1em}
\noindent
per capire la natura delle curve appena disegnate, è sufficiente osservare la Figura \ref{fig:visualizzazione_angoli_coinvolti} seguente:

\begin{figure}[H]
  \centering
  \begin{tikzpicture}[scale=2]
    \draw (0,0) -- (2,0) to[out=0,in=270] coordinate[midway](r) (4,2) -- (4,4);
    \draw (2,2) node[circ](start){} -- coordinate[midway](a) (r) (a) ++(0.3,0.3) node[]{$R$};
    \draw [dotted] (start) -- ++(2,0) node[at end, right]{$t_2$};
    \draw [dotted] (start) -- ++(0,-2) node[at end, below]{$t_1$};
    \draw [dotted] (2.4,0.05) node[circ](b){} -- coordinate[midway](half) (2,2);
    \draw [-stealth] (b) -- (half);
    \draw (b) -- ++(1,0) coordinate(c);
    \draw (2.5,0.6) node[]{$\vec{a}$};
    \draw [-stealth, red] (2,0) -- (2,1);
    \draw [-stealth, red] (4,2) -- (3,2);

    \coordinate (f) at (2.2,0.8);
    \coordinate (ctr) at (b);
    \coordinate (i) at (c);
    \pic [draw=red, text=red, <->, "$\theta$", angle eccentricity=1.5] {angle = i--ctr--f};

    \coordinate (f1) at (half);
    \coordinate (ctr1) at (2,2);
    \coordinate (i1) at (2,1);
    \pic [draw=blue, text=blue, <->, "$\alpha$", angle eccentricity=1.2, angle radius=1.8cm] {angle = i1--ctr1--f1};
  \end{tikzpicture}
  \caption{Visualizzazione degli angoli coinvolti}
  \label{fig:visualizzazione_angoli_coinvolti}
\end{figure}

\vspace{1em}
\noindent
Si può facilmente capire come $\alpha$ sia l'angolo da sommare a $90^\circ$ per ottenere $\theta$, quindi
\[a_x=a \cdot \cos(\theta) = a \cdot \cos(\alpha + 90^\circ) = -a \cdot \sin(\alpha) = -a \cdot \sin(\omega t)\]
pertanto nel caso di $a_x$ si è considerato un ramo di $\sin(\omega t)$, mentre nel caso di $a_y$ si è considerato un ramo di $\cos(\omega t)$.

\newpage
\noindent
\begin{center}
  9 Marzo 2022
\end{center}
Il moto circolare uniforme è un moto semplice: la formula più importante da conoscere è il modulo dell'\textbf{accelereazione centripeta}, ovvero dell'accelerazione diretta ferso il centro della circonferenza.\\
Il raggio $R$ della circonferenza del moto è costante, mentre l'angolo $\theta$ descritto dal punto in movimento all'interno della circonferenza varia linearmente con il tempo, secondo la seguente legge
\[\boxed{\theta(t) = \frac{2\pi}{T} \cdot t = \omega t}\]
ove $\omega$ prende il nome di velocità angolare ed è definita come segue
\[\boxed{\omega = \frac{2\pi}{T}}\]
Per quando concerne la variazione della posizione nel tempo si ha
\[\boxed{\vec{r}(t) = x(t) \cdot \hat{i} + y(t) \cdot \hat{i} = R \cdot \cos(\omega t) \cdot \hat{i} + R \cdot \sin(\omega t) \cdot \hat{j}}\]
A partire da tale risultato si sarebbe potuto determinare il modulo dell'accelerazione, semplicemente procedendo per derivate successive, ottenendo dapprima
\[\boxed{\vec{v}(t) = - \omega R \cdot \sin(\omega t) \cdot \hat{i} + \omega R \cos(\omega t) \cdot \hat{i}}\]
e infine
\[\boxed{\vec{a}(t) = -\omega^2 \cdot \vec{r}(t)}\]
in cui è evidente come il vettore acceerazione é sempre parallelo al vettore posizione, ma con verso opposto: il vettore posizione è sempre diretto verso l'esterno, mentre il vettore acccelerazione è diretto verso il centro della circonferenza.\\
Se ora si procede al calcolo del modulo di tale vettore si ottiene
\[\boxed{\left \vert \vec{a} \right \vert = \omega^2 \cdot R = \frac{v^2}{R}}\]
dal momento che si ha
\[\boxed{v = \omega R}\]
In realtà anche $\omega$ è un vettore, in cui la sua direzione è l'asse di rotazione, mentre il modulo fornirà una stima della velocità alla quale si muove; tale risultato avrà una importante validità in seguito.

\vspace{1em}
\subsection{Moti relativi}
Si consideri il caso di un moto composto da più moti: un sasso che viene lasciato cadere sullo scafo di una barca in movimento.\\
Naturalmente, considerando la caduta di un sasso, fissando un tempo $t$, si ha che
\[\vec{v}_{PB} = -g t \cdot \hat{j}\]
in quanto si tratta di una caduta libera. Questo, tuttavia, osservando il moto dalla barca ($PB$ = punto-barca) in movimento. Se, invece, tale moto viene visto da terra ($PT$ = punto-terra), sarà dotato di due componenti, ovvero
\[\vec{v}_{PT} = \vec{v} + -g t \cdot \hat{j}\]
in cui la componente verticale è la stessa del moto precedente, mentre la componente orizzontale dipende dalla velocità della barca. Questo è proprio quello che ha fatto Galileo: osservare questo tipo di situazioni nella vita reale, fornendovi una spiegazione scientifica e definendo, in questo caso, il concetto di sistema di riferimento e di moto relativo.

\subsubsection{Caso generale}
Per capire come passare da un sistema di riferimento all'altro, è necessario considerare un caso generale, in cui come sistema di riferimento si assume quello definito da:
\begin{itemize}
  \item posizione dell'origine;
  \item assi (posizione e orientamento).
\end{itemize}

\vspace{2em}
\noindent
\rowcolors{1}{white}{white}
\begin{tabularx}{\textwidth}{P}
  {
      \centering
      \begin{tikzpicture}
        \begin{axis}[
          axis lines = left,
          xlabel = \(x_A\),
          ylabel = {\(y_A\)},
          legend pos=outer north east,
          ymajorgrids=true,
          xmajorgrids=true,
          grid style=dashed,
          ymax=10,
          xmax=10,
          ytick={3},
          xtick={3},
        ]

        \addplot [
          domain=0:2,
          samples=100,
          color=green,
        ]
        {x};
        \draw [-stealth] (axis cs:3,3) -- (axis cs:3,10);
        \draw [-stealth] (axis cs:3,3) -- (axis cs:10,3);
        \draw [-stealth, very thick, red] (axis cs:0,0) -- (axis cs:9,5);
        \draw [red] (axis cs:4.5,1.5) node[]{$\vec{r}_{PA}$};
        \draw [-stealth, very thick, blue] (axis cs:3,3) -- (axis cs:9,5);
        \draw [blue] (axis cs:5,4.5) node[]{$\vec{r}_{PB}$};
        \draw [-stealth, very thick, orange] (axis cs:0,0) -- (axis cs:3,3);
        \draw [orange] (axis cs:1,2) node[]{$\vec{r}_{BA}$};
        \node[label={[rotate=90]center:$y_B$}] at (axis cs:1.5,6.5) {};
        \node[label={center:$x_B$}] at (axis cs:6.5,1.5) {};
        \end{axis}
    \end{tikzpicture}
  }
\end{tabularx}

\vspace{1em}
\noindent
Per passare da un sistema di riferimento $A$ all'altro $B$ è necessario definire la posizione relativa tra i due sistemi di riferimento $A-B$.\\
Infatti, definendo un nuovo vettore $\vec{r}_{BA}$ è possibile scrivere la somma di vettori seguente
\[\vec{r}_{PA} = \vec{r}_{PB} + \vec{r}_{BA}\]
Ma non solo, è possibile anche calcolare la derivata nel tempo e considerare, quindi, le velocità
\[\vec{v}_{PA} = \vec{v}_{PB} + \vec{v}_{BA}\]
che corrisponde proprio al caso analizzato per la barca: la differenza tra i due sistemi di osservazione è proprio il moto della barca. Un caso ancora più importante è quello che prevede $\vec{v}_{BA}$ \textbf{costante}, per cui i due sistemi non si muovono l'uno rispetto all'altro e misurare l'accelerazione nell'uno o nell'altro non cambia, in quanto sarà la stessa. Derivando nuovamente nel tempo, infatti, si ottiene
\[\vec{a}_{PA} = \vec{a}_{PB} + \vec{a}_{BA}\]
ma essendo $\vec{v}_{BA}$ costante, ovviamente $\vec{a}_{BA} = 0$. Questo è il caso di un \textbf{sistema di riferimento inerziale}, ovvero di un sistema di riferimento che può muoversi ad una certa velocità, ma non può accelerare. Naturalmente l'accelerazione del sistema ierziale non è relativa, non è da definirsi rispetto ad un altro sistema di riferimento come in questo caso, ma necessita di una definizione molto più rigorosa fornita tramite le leggi della dinamica di Newton. Questo ragionamento, naturalmente, si applica sia ad un caso in 2D, ma anche in 3D.

\vspace{1em}
\noindent
\textbf{Osservazione}: Si osservi, ovviamente, che nel moto circolare uniforme velocità e accelerazione.\\
Inoltre, si ha che
\[\frac{d \left \vert \vec{v} \right \vert}{dt} = 0\]
significa che la variazione del modulo della velocità nel tempo è nullo: pertanto, se non c'è variazione del modulo della velocità, si ha che la componnte parallela dell'accelerazione è nulla, ovvero $\vec{a}_\parallel = 0$.

\newpage
\section{Dinamica}
Alla base della dinamica vi sono i $3$ principi della dinamica di Newton, formulati da Newton all'interno del libro \textbf{Philosophiae Naturalis Principia Mathematica}.\\
Esse sono le seguenti:

% Tabella per le definizione di concetti, etc...
\vspace{1em}
\rowcolors{1}{black!5}{black!5}
\setlength{\tabcolsep}{14pt}
\renewcommand{\arraystretch}{2}
\noindent
\begin{tabularx}{\textwidth}{@{}|P|@{}}
    \hline
    {\textbf{LEGGI DELLA DINAMICA}}\\
    \parbox{\linewidth}{Le leggi della dinamica sono le seguenti:
    \begin{enumerate}
      \item \textbf{Prima legge}: \emph{Ciascun corpo persevera nel proprio stato di quiete o di moto rettilineo uniforme, eccetto che sia costretto a mutare quello stato da forze impresse.}
      \item \textbf{Seconda legge}: \emph{Il cambiamento di moto è proporzionale alla forza mmotrice impressa, ed avviene lungo la linea retta secondo la quale la forza è stata impressa.}
      \item \textbf{Terza legge}: \emph{Ad ogni azione corrisponde una reazione uguale e contraria: ossia le azioni di due corpi sono sempre uguai fra loro e dirette verso parti opposte.}
    \end{enumerate}
    \vspace{1mm}}\\
    \hline
\end{tabularx}
\vspace{1em}

\subsection{Massa}
Per parlare della dinamica, si devono introdurre due concetti fondamentali ed interconnessi; prima di tutto si fornisce la definizione di \textbf{massa}, la quale può essere definita in modi diversi a seconda della necessità:

% Tabella per le definizione di concetti, etc...
\vspace{1em}
\rowcolors{1}{black!5}{black!5}
\setlength{\tabcolsep}{14pt}
\renewcommand{\arraystretch}{2}
\noindent
\begin{tabularx}{\textwidth}{@{}|P|@{}}
    \hline
    {\textbf{MASSA INERZIALE}}\\
    \parbox{\linewidth}{La \textbf{massa inerziale} (da \textbf{inerzia}: propensione a non muoversi) viene definita come misura della resistenza alle variazioni di velocità.
    \vspace{3mm}}\\
    \hline
\end{tabularx}
\vspace{1em}

\noindent
che si adatta perfettamente alla seconda legge della dinamica, la quale afferma che l'accelerazione è proporzionale alla forza impressa ed è la massa a rappresentare la \textbf{costante di proporzionalità}: a parità di forza, più il corpo è massivo meno accelera, meno è massivo, più accelera.\\
Di seguito, invece, si definisce il concetto di \textbf{massa gravitazionale}:

% Tabella per le definizione di concetti, etc...
\vspace{1em}
\rowcolors{1}{black!5}{black!5}
\setlength{\tabcolsep}{14pt}
\renewcommand{\arraystretch}{2}
\noindent
\begin{tabularx}{\textwidth}{@{}|P|@{}}
    \hline
    {\textbf{MASSA GRAVITAZIONALE}}\\
    \parbox{\linewidth}{La \textbf{massa gravitazionale} è proporzionale al \textbf{peso}.
    \vspace{3mm}}\\
    \hline
\end{tabularx}
\vspace{1em}

\noindent
Non da ultimo si fornisce una definizione di massa che è approssimabile ad una quantificazione:

% Tabella per le definizione di concetti, etc...
\vspace{1em}
\rowcolors{1}{black!5}{black!5}
\setlength{\tabcolsep}{14pt}
\renewcommand{\arraystretch}{2}
\noindent
\begin{tabularx}{\textwidth}{@{}|P|@{}}
    \hline
    {\textbf{MASSA}}\\
    \parbox{\linewidth}{La \textbf{massa} viene definita come \textbf{quantità di materia} e la sua unità di misura è
    \[[m] = \text{kg}\]
    Inoltre la massa è \textbf{additiva}: dato un corpo, agglomerato compatto di due masse $m_1$ e $m_2$, la massa complessiva è
    \[m = m_1 + m_2\]
    \vspace{-1mm}}\\
    \hline
\end{tabularx}
\vspace{1em}

\newpage
\noindent
\subsection{Forza}
Di seguito si espone il signifiato fisico di \textbf{forza}:

% Tabella per le definizione di concetti, etc...
\vspace{1em}
\rowcolors{1}{black!5}{black!5}
\setlength{\tabcolsep}{14pt}
\renewcommand{\arraystretch}{2}
\noindent
\begin{tabularx}{\textwidth}{@{}|P|@{}}
    \hline
    {\textbf{FORZA}}\\
    \parbox{\linewidth}{Una \textbf{forza} è una spinta che produce un cambiamento di moto di un corpo.\\
    La forza è un \textbf{vettore} (con modulo, direzione e verso) la cui unità di misura è
    \[[F] = \text{N} = \frac{\text{kg m}}{\text{s}^2}\]
    ove N sta per Newton.
    \vspace{3mm}}\\
    \hline
\end{tabularx}

\vspace{1em}
\subsection{Principi della dinamica - Leggi di Newton}
Si descrivano, ora, le leggi di Newton in termini dei due concetti esposti, ossia massa e forza:

\begin{enumerate}
  \item Se la forza risultante che agisce su un corpo è nulla, ovvero
  \[\sum \vec{F} = 0\]
  allora l'accelerazione del corpo è nulla, cioé $\vec{a}=0$, ovvero
  \[\boxed{\sum \vec{F} = 0 \longrightarrow \vec{a}=0}\]
  Tale legge potrebbe sembrare superflua, in quanto un caso particolare della seconda: tuttavia, tale legge assolve al compito di definire un \textbf{sistema di riferimento inerziale}.

  \item La forza risultante su un corpo è direttamente proporzionale all'accelerazione del corpo stesso. L'accelerazione di un corpo, quindi, è proporzionale alla forza risultante
  \[\boxed{\sum \vec{F} = m \vec{a}}\]

  \item La forza esercitata da un corpo $a$ su un corpo $b$ è uguale in modulo e direzione, ma ha verso opposto alla forza esercitata da $b$ su $a$.\\
  Ovvero si ha che
  \[\boxed{\vec{F}_{ab} = -\vec{F}_{ba}}\]
  e ciò è sempre vero.
\end{enumerate}

\vspace{1em}
\noindent
Dopo aver definito tali leggi, è necessario definire diversi tipi di forze, distinguendole in base alla loro tipolgia.

\vspace{1em}
\noindent
\subsection{Forza peso}
Di seguito si espone il significato fisico di \textbf{forza peso}:

% Tabella per le definizione di concetti, etc...
\vspace{1em}
\rowcolors{1}{black!5}{black!5}
\setlength{\tabcolsep}{14pt}
\renewcommand{\arraystretch}{2}
\noindent
\begin{tabularx}{\textwidth}{@{}|P|@{}}
    \hline
    {\textbf{FORZA PESO}}\\
    \parbox{\linewidth}{La forza peso fiene designata con $\vec{F}_t$, ovvero la forza di attrazione esercitata dalla terra su un corpo di massa $m$. In particolare si ha
    \[\boxed{\vec{F}_t = m \vec{g}}\]
    in cui $\vec{g} = -9.8 \text{ m}/\text{s}^2 \cdot \hat{j}$ e prende il nome di \textbf{accelerazione gravitazionale} (o meglio, di \textbf{campo gravitazionale} sulla superficie della terra).
    \vspace{3mm}}\\
    \hline
\end{tabularx}

\vspace{1em}
Di seguito si espone una illustrazione della forza peso:

\vspace{1em}
\begin{figure}[H]
  \centering
  \begin{tikzpicture}
    \draw [fill = purple!30,draw = purple!50] (0,0) rectangle ++(2,1.2);
    \draw [-stealth] (1,0.6) node[circ]{} -- ++(0,-2) node [midway, below right] {$\vec{F}_t = m\vec{g}$};
    \draw (0.5,0.6) node[]{$m$};
  \end{tikzpicture}
  \caption{Forza peso}
  \label{fig:forza_peso}
\end{figure}

\vspace{1em}
\noindent
\textbf{Osservazione}: Si osservi che la formula seguente
\[\vec{F}_t = m \vec{g}\]
potrebbe rassomigliare la formula
\[\vec{F}_t = m \vec{a}\]
Tuttavia, i due concetti sono ben distinti, in quanto $\vec{F}_t = m \vec{g}$ è un caso particolare della \textbf{legge di gravitazione universale}.

\vspace{1em}
\noindent
\textbf{Osservazione}: Si osservi che anche nel caso della forza peso è presente il terzo principio della dinamica: infatti un corpo viene attratto verso il centro della terra ed esercita una forza sulla superficie terrestre, così come la terra esercita una forza uguale e contraria (solamente che è impercettibile, è sempre presente).

\vspace{1em}
\subsection{Forza normale}
Di seguito si espone un altro tipo di forza, una forza i contatto, che prende il nome di \textbf{forza normale}:

% Tabella per le definizione di concetti, etc...
\vspace{1em}
\rowcolors{1}{black!5}{black!5}
\setlength{\tabcolsep}{14pt}
\renewcommand{\arraystretch}{2}
\noindent
\begin{tabularx}{\textwidth}{@{}|P|@{}}
    \hline
    {\textbf{FORZA NORMALE}}\\
    \parbox{\linewidth}{La \textbf{forza normale} $\vec{F}_N$ è un caso di forza di contatto, definita come \textbf{spinta fornita da una superficie (o da un altro corpo)}: quando un oggetto è appoggiato su una superficie e non si muove (ovvero si ha che $\vec{v}=0$ e $\vec{a}=0$), alla \textbf{forza peso} si contrappone la \textbf{forza normale}, uguale e contraria alla forza peso.
    \vspace{3mm}}\\
    \hline
\end{tabularx}

\vspace{1em}
\noindent
Di seguito si una illustrazione della forza normale:

\vspace{1em}
\begin{figure}[H]
  \centering
  \begin{tikzpicture}
    \draw [fill = purple!30,draw = purple!50] (0,0) rectangle ++(2,1.2);
    \draw (-1,0) -- (3,0);
    \foreach \i in {-11,-9,...,27} {
      \draw (\i / 10,-0.3) -- (\i / 10 + 0.3,0);
    }
    \draw [-stealth] (1.5,0) node[circ]{} -- ++(0,2) node [at end, right] {$\vec{F}_N$};
    \draw (1.5,0.2) -- ++(0.2,0) -- ++ (0,-0.2);
    \draw [-stealth] (1,0.6) node[circ]{} -- ++(0,-2) node [midway, below right] {$\vec{F}_t$};
    \draw (0.3,0.6) node[]{$m$};
    \draw (-1,1) node[]{$\vec{a}=0$};
  \end{tikzpicture}
  \caption{Forza normale}
  \label{fig:forza_normale}
\end{figure}

\vspace{1em}
\noindent
Una caratteristica fondamentale della forza normale è che essa è sempre \textbf{ortogonale alla superficie} su cui poggia l'oggetto. Se si osserva che il corpo presenta $\vec{a}=0$, significa che
\[\vec{F}_t + \vec{F}_N\ = 0 = m \vec{a} \longrightarrow \vec{F}_N = - \vec{F}_t\]
Mentre la forza peso presenta un modulo preciso, calcolabile tramite la legge di gravitazione universale, la forza normale, invece, adatta la propria intensità al corpo appoggiato sulla superficie: fintantoché la superficie resiste, il modulo della forza normale coincide con quello della forza peso; se il corpo è eccessivamente massiccio, la superficie si rompe.

\vspace{1em}
\noindent
\textbf{Osservazione}: Una forza presenta sempre un \textbf{punto di applicazione} che, graficamente, è rappresentata dalla \quotes{coda del vettore}:
\begin{enumerate}
  \item Nel caso della forza peso, il punto di applicazione è sempre dato del \textbf{centro di massa} del corpo stesso;
  \item Nel caso della forza normale, il punto di applicazione è la superficie di contatto (anche se vi sono molti punti di applicazione vista l'irregolarità della superficie stessa).
\end{enumerate}
Tuttavia è sempre possibile eseguire la somma di forze per conoscerne la risultante.

\vspace{1em}
\noindent
\textbf{Osservazione}: Quando si deve eseguire la rappresentazione grafica delle forze è necessario introdurre il concetto di \textbf{diagramma di corpo libero}:
\begin{itemize}
  \item ogni corpo è rappresentato da un \textbf{punto} (per cui il punto di applicazione delle forze sul corpo è proprio rappresentato dal punto stesso);
  \item comporta \textbf{solo le forze} che sono applicate sul corpo, e ciò diviene particolarmente utile quando bisogna considerare un sistema di più corpi interagenti.
\end{itemize}

\vspace{1em}
\noindent
\textbf{Esempio}: Si considerino due corpi poggiati uno sopra l'altro e stanti su una superficie fissa. Naturalmente su tali corpi agiscono due forze peso distinte. Inoltre la superficie di contatto tra i due corpi permette di indivduare due forze normali: uno del primo corpo sul secondo e una del secondo corpo sul primo. Infine vi è la forza di contatto dei due corpi con la superficie su cui poggiano, come mostrato di seguito:

\vspace{1em}
\begin{figure}[H]
  \centering
  \begin{tikzpicture}[scale=2]
    \draw [fill = purple!30,draw = purple!50] (0,0) rectangle ++(2,1.2);
    \draw [fill = red!30,draw = red!50] (-0.5,1.2) rectangle ++(3,1.2);
    \draw (-1,0) -- (3,0);
    \foreach \i in {-11,-9,...,27} {
      \draw (\i / 10,-0.3) -- (\i / 10 + 0.3,0);
    }
    \draw [-stealth] (1.2,0) node[circ]{} -- ++(0,2) node [at end, right] {$\vec{F}_N$};
    \draw (1.2,0.2) -- ++(0.2,0) -- ++ (0,-0.2);
    \draw [-stealth] (1,0.6) node[circ]{} -- ++(0,-2) node [midway, below right] {$\vec{F}_{t1}$};
    \draw [-stealth] (1.7,1.8) node[circ]{} -- ++(0,-2) node [midway, right] {$\vec{F}_{t2}$};
    \draw [-stealth] (0.5,1.2) node[circ]{} -- ++(0,-1) node [midway, right] {$\vec{F}_{N1}$};
    \draw [-stealth] (0.5,1.2) node[circ]{} -- ++(0,1) node [midway, right] {$\vec{F}_{N2}$};
    \draw (0.3,0.6) node[]{$m_1$};
    \draw (-0.1,1.8) node[]{$m_2$};
  \end{tikzpicture}
  \caption{Forza normale di due corpi a contatto}
  \label{fig:forza_normale_corpi_contatto}
\end{figure}

\vspace{1em}
\noindent
Dopo aver effettuato la raffigurazione, si procede alla realizzazione del \textbf{diagramma a corpo libero} di ciascuno dei due corpi, come mostrato di seguito:

\vspace{1em}
\begin{figure}[H]
  \centering
  \begin{tikzpicture}[scale=1]
    \draw [-stealth] (0,0) node[circ]{} -- ++(0,2) node [at end, right] {$\vec{F}_N$};
    \draw [-stealth] (0.1,0) -- ++(0,-1) node [midway, right] {$\vec{F}_{t1}$};
    \draw [-stealth] (-0.1,0) -- ++(0,-1) node [midway, left] {$\vec{F}_{N1}$};
  \end{tikzpicture}
  \hspace{2em}
  \begin{tikzpicture}[scale=1]
    \draw [-stealth] (0,0) node[circ]{} -- ++(0,1.5) node [at end, right] {$\vec{F}_{N2}$};
    \draw [-stealth] (0,0) -- ++(0,-1.5) node [midway, right] {$\vec{F}_{t2}$};
  \end{tikzpicture}
  \caption{Diagramma a corpo libero di due corpi a contatto}
  \label{fig:diagramma_corpo_libero_due_corpi_contatto}
\end{figure}

\vspace{1em}
\noindent
Naturalmente in questo caso si possono determinare direttamente le forze coinvolte
\begin{flalign*}
  \vec{F}_{t1} & = -g m_1 \cdot \hat{j} = m_1 \vec{g}\\
  \vec{F}_{t2} & = -g m_2 \cdot \hat{j} = m_2 \vec{g}\\
\end{flalign*}
Applicando la \textbf{$\boldsymbol{2^a}$ legge della dinamica} si perviene al risultato seguente
\begin{flalign*}
  m_1 \vec{a}_1 & = m_1 \vec{g} = \sum \vec{F} = \vec{F}_N + \vec{F}_{N1} + \vec{F}_{t1} = 0\\
  m_2 \vec{a}_2 & = m_2 \vec{g} = \sum \vec{F} = \vec{F}_{N2} + \vec{F}_{t2} = 0
\end{flalign*}
Questo, in quanto l'accelerazione è nulla, un dato noto dal problema. Applicando, ora, la \textbf{$\boldsymbol{3^a}$ legge della dinamica} si perviene al risultato seguente:
\[\vec{F}_{N1} = - \vec{F}_{N2}\]
Da ciò si può concludere il problema, andando a determinare
\begin{flalign*}
  \vec{F}_{N2} & = - \vec{F}_{t2} = g m_2 \cdot \hat{j}\\
  \vec{F}_{N1} & = - \vec{F}_{N2} = \vec{F}_{t2} = - g m_2 \cdot \hat{j}\\
  \vec{F}_{N} & = - \vec{F}_{N1} - \vec{F}_{t1} = g m_2 \cdot \hat{j} + g m_1 \cdot \hat{j} = g \cdot (m_1 + m_2) \cdot \hat{j}\\
\end{flalign*}

\newpage
\noindent
\begin{center}
  10 Marzo 2022
\end{center}
\subsection{Forza di tensione}
Di seguito si espone il significato fisico della \textbf{forza di tensione} che, nel suo comportamente, non è dissimile dalla forza normale:

% Tabella per le definizione di concetti, etc...
\vspace{1em}
\rowcolors{1}{black!5}{black!5}
\setlength{\tabcolsep}{14pt}
\renewcommand{\arraystretch}{2}
\noindent
\begin{tabularx}{\textwidth}{@{}|P|@{}}
    \hline
    {\textbf{FORZA DI TENSIONE}}\\
    \parbox{\linewidth}{La forza di tensione è la forza esercitata, per esempio, da un cavo o una fune, come mostrato di seguito, in cui la forza di tensione va a cancellare la forza peso del corpo appeso alla fune. È importante notare che $\vec{F}_T$ è sempre \textbf{parallela alla direzione della corda} stessa.\vspace{3mm}}\\
    \hline
\end{tabularx}

\vspace{1em}
\noindent
Di seguito si espone una illustrazione della forza di tensione:

\vspace{1em}
\begin{figure}[H]
  \centering
  \begin{tikzpicture}[scale=1]
    \draw [fill = purple!30,draw = purple!50] (0,-3) rectangle ++(2,1.2);
    \draw (-1,-0.3) -- (3,-0.3);
    \foreach \i in {-10,-8,...,26} {
      \draw (\i / 10,-0.3) -- (\i / 10 + 0.3,0);
    }
    \draw (1,-1.8) -- (1,-0.3);
    \draw [-stealth, red] (1,-1.8) -- node[midway, left, red]{$\vec{F}_T$} (1,-0.75);
    \draw [-stealth] (1,-2.4) node[circ]{} -- ++(0,-2) node [midway, below right] {$\vec{F}_{t}$};
    \draw (0.5,-2.4) node[]{$m$};
  \end{tikzpicture}
  \caption{Forza di tensione di un corpo sospeso}
  \label{fig:forza_tensione_corpo_sospeso}
\end{figure}

\vspace{1em}
\noindent
\textbf{Esempio}: Si consideri l'esempio seguente, in cui vi è un corpo che rimane sospeso nel vuoto da due funi che descrivono con il sofftto due angoli, rispettivamente $\theta_1$ e $\theta_2$. Si determinino le tensioni sulle due corde.

\vspace{1em}
\begin{figure}[H]
  \centering
  \begin{tikzpicture}[scale=1]
    \draw [fill = purple!30,draw = purple!50] (0,-3) rectangle ++(2,1.2);
    \draw (-1,-0.3) -- (3,-0.3);
    \foreach \i in {-10,-8,...,26} {
      \draw (\i / 10,-0.3) -- (\i / 10 + 0.3,0);
    }
    \draw (0.5,-1.8) -- coordinate[midway](a) node[midway, below left, red]{$\vec{F}_{T1}$} (-0.5,-0.3);
    \draw (1.5,-1.8) -- coordinate[midway](b) node[midway, below right, red]{$\vec{F}_{T2}$} (2.3,-0.3);
    \draw [-stealth, red] (0.5,-1.8) -- (a);
    \draw [-stealth, red] (1.5,-1.8) -- (b);
    \coordinate (O1) at (-0.5,-0.3);
    \coordinate (O2) at (2.3,-0.3);
    \draw [draw = orange] (O1) ++(.8,0) arc (0:-55:0.8)
    	node [pos=.4, left] {$\theta_1$};
    \draw [draw = violet] (O2) ++(-.8,0) arc (180:242:0.8)
    	node [pos=.4, left] {$\theta_2$};
    \draw [-stealth] (1,-2.4) node[circ]{} -- ++(0,-2) node [midway, below right] {$\vec{F}_{t}$};
    \draw (0.5,-2.4) node[]{$m$};
  \end{tikzpicture}
  \caption{Forza di tensione di un corpo sospeso da due corde}
  \label{fig:forza_tensione_corpo_sospeso_due_corde}
\end{figure}

\noindent
Dopo aver realizzato una figura illustrativa, bisogna sempre procedere alla raffigurazione del \textbf{diagramma a corpo libero}, come mostrato di seguito:

\vspace{1em}
\begin{figure}[H]
  \centering
  \begin{tikzpicture}[scale=1.5]
    \draw [-stealth, red] (0,0) node[circ]{} -- node[midway, below left, red]{$\vec{F}_{T1}$} (-0.5,0.75);
    \draw [-stealth, red] (0,0) -- node[midway, below right, red]{$\vec{F}_{T2}$} (0.4,0.75);
    \draw [-stealth] (0,0) -- node[midway, right]{$\vec{F}_{t1}$} (0,-1);
  \end{tikzpicture}
  \caption{Diagramma a corpo libero di un corpo sospeso da due corde}
  \label{fig:diagramma_corpo_libero_corpo_sospeso_due_corde}
\end{figure}

\vspace{1em}
\noindent
Applicando, ora, la \textbf{$\boldsymbol{2^a}$ legge della dinamica} si perviene al risultato seguente:
\[\sum \vec{F} = m \vec{a} = 0\]
essendo l'accelerazione nulla. Pertanto si ha che
\[\vec{F}_{T1} + \vec{F}_{T2} + \vec{F}_t = 0\]
Sarà ora sufficiente decomporre tale equazione vettoriale nelle sue due componenti ($x$ e $y$), come mostrato di seguito
\begin{flalign*}
    x & = - F_{T1} \cdot \cos(\theta_1) + F_{T2} \cdot \cos(\theta_2) + 0 = 0\\
    y & = F_{T1} \cdot \sin(\theta_1) + F_{T2} \cdot \sin(\theta_2) - mg = 0
\end{flalign*}
Dalla prima equazione si ha che
\[F_{T1} = F_{T2} \cdot \frac{\cos(\theta_2)}{\cos(\theta_1)}\]
che, sostituita nella seconda ecquazione, permette di ottenere
\[F_{T2} \cdot \frac{\cos(\theta_2)}{\cos(\theta_1)} \cdot \sin(\theta_1) + F_{T2} \cdot \sin(\theta_2)= mg\]
che può essere riscritto come segue
\[F_{T2} \cdot \cos(\theta_2) \cdot \left[ \frac{\sin(\theta_1)}{\cos(\theta_1)} + \frac{\sin(\theta_2)}{\cos(\theta_2)} \right] = mg\]
Pertanto si ha che
\[F_{T2} = \frac{mg}{\cos(\theta_2) \cdot \left(\tan(\theta_1) + \tan(\theta_2)\right)} \hspace{1em} \text{e} \hspace{1em} F_{T1} = \frac{mg}{\cos(\theta_1) \cdot \left(\tan(\theta_1) + \tan(\theta_2)\right)}\]
Se si cerca di capire che cosa accade quando gli angoli descriti dalla fune con il soffitto sono prossimi all'angolo limite di $90^\circ$, si rileva immediatamente una instabilità.\\
Intuitivamente si potrebbe pensare che una massa sospesa tramite due cavi come nella configurazione mostrata di seguito sia facilmente in equilibrio:

\vspace{1em}
\begin{figure}[H]
  \centering
  \begin{tikzpicture}[scale=1]
    \draw [fill = purple!30,draw = purple!50] (0,-3) rectangle ++(2,1.2);
    \draw (-1,-0.3) -- (3,-0.3);
    \foreach \i in {-10,-8,...,26} {
      \draw (\i / 10,-0.3) -- (\i / 10 + 0.3,0);
    }
    \draw (0.5,-1.8) -- coordinate[midway](a) node[midway, below left, red]{$\vec{F}_{T1}$} (0.5,-0.3);
    \draw (1.5,-1.8) -- coordinate[midway](b) node[midway, below right, red]{$\vec{F}_{T2}$} (1.5,-0.3);
    \draw (0.5,-1.5) -- ++(0.3,0) -- ++(0,-0.3) (0.5,-0.6) -- ++(-0.3,0) -- ++(0,0.3);
    \draw (1.5,-1.5) -- ++(-0.3,0) -- ++(0,-0.3) (1.5,-0.6) -- ++(0.3,0) -- ++(0,0.3);
    \draw [-stealth, red] (0.5,-1.8) -- (a);
    \draw [-stealth, red] (1.5,-1.8) -- (b);
    \draw [-stealth] (1,-2.4) node[circ]{} -- ++(0,-2) node [midway, below right] {$\vec{F}_{t}$};
    \draw (0.5,-2.4) node[]{$m$};
  \end{tikzpicture}
  \caption{Forza di tensione di un corpo sospeso da due corde parallele}
  \label{fig:forza_tensione_corpo_sospeso_due_corde_parallele}
\end{figure}

\vspace{1em}
\noindent
Tuttavia, è necessario considerare anche un altro aspetto nella determinazione dell'equilibrio: il \textbf{momento di forza}. Il caso precedentemente analizzato riguardava la configurazione per la quale il punto di applicazione della corda alla massa era identico e precisamente al centro della massa stessa:

\vspace{1em}
\begin{figure}[H]
  \centering
  \begin{tikzpicture}[scale=1]
    \draw [fill = purple!30,draw = purple!50] (0,-3) rectangle ++(2,1.2);
    \draw (-1,-0.3) -- (3,-0.3);
    \foreach \i in {-10,-8,...,26} {
      \draw (\i / 10,-0.3) -- (\i / 10 + 0.3,0);
    }
    \draw (1,-1.8) -- coordinate[midway](a) node[midway, below left, red]{$\vec{F}_{T1}$} (-0.5,-0.3);
    \draw (1,-1.8) -- coordinate[midway](b) node[midway, below right, red]{$\vec{F}_{T2}$} (2.3,-0.3);
    \draw [-stealth, red] (1,-1.8) -- (a);
    \draw [-stealth, red] (1,-1.8) -- (b);
    \coordinate (O1) at (-0.5,-0.3);
    \coordinate (O2) at (2.3,-0.3);
    \draw [draw = orange] (O1) ++(.8,0) arc (0:-45:0.8)
    	node [pos=.4, left] {$\theta_1$};
    \draw [draw = violet] (O2) ++(-.8,0) arc (180:230:0.8)
    	node [pos=.4, left] {$\theta_2$};
    \draw [-stealth] (1,-2.4) node[circ]{} -- ++(0,-2) node [midway, below right] {$\vec{F}_{t}$};
    \draw (0.5,-2.4) node[]{$m$};
  \end{tikzpicture}
  \caption{Forza di tensione di un corpo sospeso da due corde con stesso punti di applicazione}
  \label{fig:forza_tensione_corpo_sospeso_due_corde_stesso_punto_applicazione}
\end{figure}

\vspace{1em}
\noindent
Affinché in questa configurazione la massa stia in equilibrio, è necessario che precisamente la lunghezza di ciascun cavo sia identica; se differisse anche di poco, allora tutta la tensione graverebbe sul cavo più corto.

\vspace{2em}
\noindent
\textbf{Esempio}: Si consideri un quadricottero, ovverosia un drone con $4$ eliche e rotori, ciascuna capace do fornire una forza propulsiva verso l'alto di eguale modulo.\\
Considerando il drone stazionario si ha, per la \textbf{$\boldsymbol{2^a}$ legge della dinamica}, la seguente eguaglianza
\[4 \cdot \vec{F}_{s} + \vec{F}_t = 0\]
questo significa che ciascun rotore deve essere in grado di sviluppare una forza propulsiva verso l'alto pari a un quarto del peso del drone; quando, invece, si ha uno sbilanciamento delle forze dei rotori si ottiene un'inclinazione del drone nella direzione delle forze di minore intensità (quello che viene chiamato \textbf{momento di forza}).\\
Pertanto si può concludere che
\[\vec{F}_{s} = - \frac{1}{4} \cdot \vec{F}_t\]

\newpage
\noindent
\textbf{Esempio}: Si consideri una massa su un piano inclinato, come mostrato di seguito (considerando ininfluente l'attrito tra la massa e la superficie del piano):

\vspace{1em}
\noindent
\begin{figure}[H]
  \centering
  \newcommand{\ang}{30}

  \begin{tikzpicture} [font = \small, scale=1.5]

  % triangle:
  \draw [draw = orange, fill = orange!15] (0,0) coordinate (O) -- (\ang:6)
  	coordinate [pos=.45] (M) |- coordinate (B) (O);

  % angles:
  \draw [draw = orange] (O) ++(.8,0) arc (0:\ang:0.8)
  	node [pos=.4, left] {$\theta$};
  \draw [draw = orange] (B) rectangle ++(-0.3,0.3);

  \begin{scope} [-latex,rotate=\ang]

  % Object (rectangle)
  \draw [fill = purple!30,
  	draw = purple!50] (M) rectangle ++ (1,.6);

  % Weight Force and its projections
  \draw [dashed] (M) ++ (.5,.3) coordinate (MM) -- ++ (0,-1.29)
  	node [very near end, right] {$\vec{F}_t \cdot \cos{\theta}$};

  \draw [dashed] (MM) -- ++ (-0.75,0)
  	node [very near end, left] {$\vec{F}_t \cdot \sin{\theta}$};

  \draw (MM) -- ++ (-\ang-90:1.5)
  	node [very near end,below left ] {$\vec{F}_t$};

  % Normal Force
  \draw (MM) -- ++ (0,1.29)
  node [very near end, right] {$\vec{F}_N$};
  \end{scope}
  \end{tikzpicture}
  \caption{Piano inclinato}
  \label{fig:piano_inclinato}
\end{figure}

\vspace{1em}
\noindent
A cui segue il diagramma a corpo libero seguente:

\begin{figure}[H]
  \newcommand{\ang}{30}
  \vspace{-5em}
  \hspace{15em}
  \begin{tikzpicture} [font = \small, scale=1.5]
  % triangle:
  \draw (0,0) coordinate (O)  (\ang:6)
  	coordinate [pos=.45] (M) coordinate (B) (O);

  \begin{scope} [-latex,rotate=\ang]
  % Weight Force and its projections
  \draw [dashed] (M) ++ (.5,.3) coordinate (MM) -- ++ (0,-1.29)
  	node [very near end, right] {$\vec{F}_t \cdot \cos{\theta}$};

  \draw [dashed] (MM) -- ++ (-0.75,0)
  	node [very near end, left] {$\vec{F}_t \cdot \sin{\theta}$};

  \draw (MM) -- ++ (-\ang-90:1.5)
  	node [very near end,below left ] {$\vec{F}_t$};

  % Normal Force
  \draw (MM) -- ++ (0,1.29)
  node [very near end, right] {$\vec{F}_N$};
  \end{scope}
  \end{tikzpicture}
  \caption{Diagramma a corpo libero di un piano inclinato}
  \label{fig:diagramma_corpo_libero_piano_inclinato}
\end{figure}

\vspace{1em}
\noindent
Dopo aver disegnato anche il diagramma a corpo libero si può procedere a ragionare con la \textbf{$\boldsymbol{2^a}$ legge della dinamica}, ottenendo
\[m \cdot \vec{a} = \vec{F}_N + \vec{F}_t\]
e scomponendo tale equazione nelle sue componenti si ottiene
\begin{flalign*}
  m a_x & = - F_N \cdot \sin(\theta)\\
  m a_y & = F_N \cdot \cos(\theta) - mg
\end{flalign*}
Volendo conoscere l'accelerazione del corpo, si osserva che sussiste il seguente vincolo geometrico:
\[\frac{a_y}{a_x} = \tan(\theta) \longrightarrow a_y = \frac{\sin(\theta)}{\cos(\theta)} \cdot a_x\]
da cui
\[a_y \cdot \cos(\theta) = a_x \cdot \sin(\theta) \longrightarrow a_x = a_y \cdot \frac{\cos(\theta)}{\sin(\theta)}\]
Pertanto, procedendo dalla prima equazione si ottiene
\[F_N = -\frac{m a_x}{\sin(\theta)}\]
e quindi, nella seconda equazione si ottiene
\[m a_x \cdot \frac{\sin(\theta)}{\cos(\theta)} = -m a_x \cdot \frac{\cos(\theta)}{\sin(\theta)} - mg \longrightarrow a_x = - g \cdot \sin(\theta) \cdot \cos(\theta)\]
Sfruttando il vincolo geometrico precedente si ottiene anche che
\[a_y = -g \cdot \sin^2(\theta)\]
Avendo determinato ciò è possibile calcolare il modulo dell'accelerazione come segue
\[a = \sqrt{g^2 \cdot \sin^2(\theta) \cdot \cos^2(\theta) + g^2 \cdot \sin^4(\theta)} = g \cdot \sin(\theta) \cdot \sqrt{\cos^2(\theta) + \sin^2(\theta)} = g \cdot \sin(\theta)\]

\vspace{1em}
\noindent
\textbf{Esempio}: Si consideri il caso del piano inclinato precedente, procedendo, ora, alla rotazione del sistema di riferimento dell'angolo $\theta$, dimodoché lo spostamento avvenga lungo l'asse $x$ soltanto, e non vi sia, conseguentemente, acceerazione lungo l'asse $y$. In base a questo nuovo sistema di riferimento si ottiene la seguente decomposizione
\begin{flalign*}
  x & : m a_x = -m g \cdot \sin(\theta)\\
  y & : m a_y = -m g \cdot \cos(\theta) + F_N = 0
\end{flalign*}
per cui si ha che
\[a_x = -g \cdot \sin(\theta)\]

\vspace{1em}
\subsection{Forza di attrito}
Di seguito si espone la definizione generale di \textbf{forza di attrito}:

% Tabella per le definizione di concetti, etc...
\vspace{1em}
\rowcolors{1}{black!5}{black!5}
\setlength{\tabcolsep}{14pt}
\renewcommand{\arraystretch}{2}
\noindent
\begin{tabularx}{\textwidth}{@{}|P|@{}}
    \hline
    {\textbf{FORZA DI ATTRITO}}\\
    \parbox{\linewidth}{La \textbf{forza di attrito} è una forza di contatto, esattamente come la forza normale: quest'ultima è sempre perpendicolare alla superficie e il suo modulo è tale che vincola il moto, al fine di contrastare la forza nell'altra direzione.\\
    La \textbf{forza di attrito}, come la forza normale, è una forza che presenta come punto di applicazione la superficie di contatto con il corpo. Inoltre, la forza di attrito
    \begin{itemize}
      \item è sempre \textbf{parallela alla superficie};
      \item è di \textbf{modulo proporzionale} a $\left \vert F_N \right \vert$.
    \end{itemize}
    Quest'ultima osservazione non è ovvia, in quanto bisogna osservare il comportamento delle particelle a livello microscopico.\vspace{3mm}}\\
    \hline
\end{tabularx}

\vspace{1em}
\noindent
Si consideri l'illustrazione seguente, in cui si espongono le forze di attrito agenti:

\vspace{1em}
\begin{figure}[H]
  \centering
  \begin{tikzpicture}[scale=1]
    \tikzset{rotate=45} {
      \draw [fill = purple!30,draw = purple!50] (0,0) rectangle ++(2,1.2);
      \draw (-1,0) -- (3,0);
      \foreach \i in {-10,-8,...,26} {
        \draw (\i / 10,-0.3) -- (\i / 10 + 0.3,0);
      }
    }
    \draw (1,0.3) -- ++(0.3,0) -- ++(0,-0.3);
    \draw [-stealth] (1,0) -- node[midway, left]{$\vec{F}_N$} (1,3) ;
    \draw [-stealth] (1,0.6) node[circ]{} -- ++(-2,-2) node [midway, below right] {$\vec{F}_{t}$};
    \draw (1.5,0.6) node[]{$m$};
    \draw [-stealth, red] (2,0) node[circ]{} -- ++(2,0) node [midway, above left] {$\vec{F}_{s/k}$};
  \end{tikzpicture}
  \caption{Forza di attrito}
  \label{fig:forza_attrito}
\end{figure}

\vspace{1em}
\noindent
La forza di attrito si distingue in due diverse tipologie:
\begin{enumerate}
  \item L'\textbf{attrito cinetico}, si ha quando il moto relativo tra le superfici in contatto non è nullo, ovvero si ha movimento con $v \neq 0$.\\
  Si ha che
  \[\boxed{F_k = \mu_k \cdot F_N}\]
  in cui $\boldsymbol{\mu_k}$ prende il nome di \textbf{coefficiente di attrito cinetico} (dall'inglese $k$, di \quotes{kinetic}), il quale, per ovvie ragioni, è \textbf{adimensionale}.

  \item L'\textbf{attrito statico}, si ha quando la velocità relativa tra le superfici di contatto è nulla, ovvero non si ha movimento, $v = 0$.\\
  Si ha che
  \[\boxed{F_s \leq \mu_s \cdot F_N}\]
  ovvero il modulo $F_s$ aumenta affinché la risultante delle forze interagenti (e quindi la risultante) sia $0$. Ovviamente, quando la forza che agisce aumenta il proprio modulo fino a superare la forza di attrito statico, il corpo inizia a muoversi e l'attrito si tramuta, a seguito del moto, in attrito cinetico, come si vede di seguito:

  \vspace{2em}
  \noindent
  \rowcolors{1}{white}{white}
  \begin{tabularx}{\textwidth}{P}
    {
        \centering
        \begin{tikzpicture}
          \begin{axis}[
            axis lines = left,
            xlabel = \(F_a\),
            ylabel = {\(F_{a,k}\)},
            legend pos=outer north east,
            ymax=4,
            xmax=10,
            xtick={3},
            xticklabels={$\mu_s \cdot F_N$},
          ]

          \addplot [
            domain=0:3,
            samples=100,
            color=black,
          ]
          {x};
          \draw (axis cs:3,2) -- (axis cs:10,2);
          \draw [dashed] (axis cs:3,0) -- (axis cs:3,3);
          \end{axis}
      \end{tikzpicture}
    }
  \end{tabularx}

  \vspace{1em}
  \noindent
  Il coefficiente moltiplicativo $\boldsymbol{\mu_s}$ prende il nome di \textbf{coefficiente di attrito statico} e generalmente si ha che
  \[\boxed{\mu_k < \mu_s}\]
\end{enumerate}

\newpage
\noindent
\begin{center}
  14 Marzo 2022
\end{center}
Si osservi che quando una massa rimane sospesa tramite due cavi che formano con la superficie di collegamento un angolo di $90^\circ$ ciascuno, non entrano in gioco solamente le forze, ma risulta fondamentale anche conoscere il concetto di \textbf{momento di forza} e di \textbf{punto di applicazione}: se il punto di applicazione delle forze di tensione è lo stesso sul corpo sospeso, basta una leggerissima differenza di lunghezza dei cavi per avere uno sbilanciamento significativo delle forze di tensione.

\vspace{1em}
\noindent
\textbf{Osservazione}: L'attrito statico è come se non fosse un attrito, in quanto non si ha movimento: è come una forza normale che si oppone al tentativo di spostamento, variando il proprio modulo in modo tale da cancellare la forza applicata.\\
L'attrito cinetico, invece, è un vero e proprio attrito che si oppone al moto tramite la dissipazione di energia in calore.

\vspace{1em}
\noindent
\textbf{Esempio}: Si consideri una vettura che, in movimento, procede a frenare e a rallentare fino a fermarsi: il fenomeno che si sta studiando è l'attrito tra le ruote e la strada. in particolare la vettura richiede $70$ m per fermarsi, partendo da una velocità di $100$ km/h (a causa di uno slittamento delle gomme sull'asfalto).\\
A partire da questi dati, si determini il \textbf{coefficiente di attrito cinetico}; si proceda alla realizzazione di un modello grafico del problema:

\vspace{1em}
\begin{figure}[H]
  \centering
  \begin{tikzpicture}[scale=1]
    \draw (0,0) -- ++(10,0);
    \foreach \i in {0,2,...,96} {
      \draw (\i / 10,-0.3) -- (\i / 10 + 0.3,0);
    }
    \node[minimum size=0.5cm,circle,draw] (circle) at (0.5,0.25){};
    \node[minimum size=0.5cm,circle,draw] (circle) at (1.5,0.25){};
    \draw (0.25,0.25) -- ++(-0.3,0) -- ++(0,0.5) -- ++(0.5,0) -- ++(0.3,0.3) -- ++(0.8,0) -- ++(0.3,-0.3) -- ++(0.5,0) -- ++(0,-0.5) -- ++(-0.6,0);

    \node[minimum size=0.5cm,circle,draw] (circle) at (7.5,0.25){};
    \node[minimum size=0.5cm,circle,draw] (circle) at (8.5,0.25){};
    \draw (7.25,0.25) -- ++(-0.3,0) -- ++(0,0.5) -- ++(0.5,0) -- ++(0.3,0.3) -- ++(0.8,0) -- ++(0.3,-0.3) -- ++(0.5,0) -- ++(0,-0.5) -- ++(-0.6,0);

    \draw (1,-0.5) -- ++(0,0.2) ++(0,-0.2) coordinate[midway](mid) -- ++(7,0) -- ++(0,0.2);
    \draw (5,-0.75) node[below]{$d=70$ m};

    \draw [-stealth] (3,2) -- node[midway, above]{$\vec{v}_0$} (4,2) ;
    \draw [-stealth] (0.85,0) -- node[midway, left]{$\vec{F}_N$} (0.85,3) ;
    \draw [-stealth] (1.15,0.65) node[circ]{} -- ++(0,-3) node [midway, below right] {$\vec{F}_{t}$};
    \draw [-stealth, red] (0.5,0) node[circ]{} -- ++(-2,0) node [midway, above left] {$\vec{F}_{k}$};
  \end{tikzpicture}
  \caption{Vettura in decelerazione su una strada}
  \label{fig:vettura_decelerazione_strada}
\end{figure}

\vspace{1em}
\noindent
È molto importante osservare che la forza normale è essenziale per il calcolo della forza di attrito cinetico, in quanto da essa dipende il suo modulo. Si realizzi, ora, il diagramma a corpo libero:

\vspace{1em}
\begin{figure}[H]
  \centering
  \begin{tikzpicture}[scale=1]
    \draw [-stealth] (0,0) node[circ]{} -- ++(0,1) node [at end, right] {$\vec{F}_N$};
    \draw [-stealth] (0,0) -- ++(0,-1) node [midway, right] {$\vec{F}_{t}$};
    \draw [-stealth] (0,0) -- ++(-1,0) node [midway, above] {$\vec{F}_{k}$};
  \end{tikzpicture}
  \caption{Diagramma a corpo libero di una vettura in rallentamento}
  \label{fig:diagramma_corpo_libero_vettura_rallentamento}
\end{figure}

\vspace{1em}
\noindent
Per procedere si richiami la $2^a$ legge della dinamica e si scriva
\[\sum \vec{F} = m \vec{a} = m \cdot \left(a_x \cdot \hat{i} + 0 \cdot \hat{j} \right)\]
È noto, inoltre, che
\[\vec{F}_t + \vec{F}_N = 0 \longrightarrow \vec{F}_t = - \vec{F}_N\]
Per cui si ottiene che
\[\vec{F}_k = m a_x \cdot \hat{i} \longrightarrow a_x = -\frac{F_k}{m}\]
in cui, per convenzione, si pone la componente orizziontale negativa.\\
Inoltre è possibile calcolare anche l'accelerazione con cui la macchina rallenta, impiegando la seguente formula del moto uniformemente accelerato:
\[v^2 - v_0^2 = 2a \cdot (x - x_0)\]
che è possibile applicare al contesto in quanto si parla di moto uniformemente accelerato, giacché l'accelerazione è causata dalla forza di attrito, che è costante in quanto prodotto tra un coefficiente e la forza normale, la quale è costante in quanto si oppone al peso che è costante.\\
Da tale formula si ottiene che
\[a = \frac{1}{2} \cdot \frac{v^2-v_0^2}{x-x_0} = -\frac{1}{2} \cdot \frac{v_0^2}{2d}\]
Da ciò segue che, ovviamente
\[F_k = \mu_k F_N\]
in cui ovviamente
\[F_N = F_t = mg\]
per cui si evince che
\[F_k = \mu_k \cdot m g \longrightarrow a_x = - \frac{F_k}{m} = -\mu_k g\]
Pertanto, dalla uguaglianza appena determinata
\[a_x = -\mu_k g = -\frac{1}{2} \cdot \frac{v_0^2}{2d}\]
si può ricavare
\[\mu_k = \frac{v_0^2}{2 g d} = 0.56\]

\vspace{2em}
\noindent
\textbf{Osservazione}: Si osservi che la $2^a$ legge di Newton è applicabile in un sistema di riferimento inerziale, ovvero in un sistema di riferimento che non ha propensione a muoversi.\\
Il tempo di volo di un proiettile è
\[\boxed{t = \frac{2 v_0 \cdot \sin(\theta)}{g}}\]

\vspace{2em}
\noindent
\textbf{Esempio}: Si consideri un piano su cui poggiano tre masse, l'una collegata all'altra, come mostrato di seguito:

\vspace{1em}
\begin{figure}[H]
  \centering
  \begin{tikzpicture}[scale=1]
    \draw [fill = purple!30,draw = purple!50] (0,0) rectangle ++(2,1.2);
    \draw [fill = orange!30,draw = orange!50] (3,0) rectangle ++(2,1.2);
    \draw [fill = green!30,draw = green!50] (6,0) rectangle ++(2,1.2);
    \draw (-1,0) -- (9,0);
    \foreach \i in {-10,-8,...,86} {
      \draw (\i / 10,-0.3) -- (\i / 10 + 0.3,0);
    }
    \draw (2,0.6) -- ++(1,0) (5,0.6) -- ++(1,0);
    \draw [-stealth] (8,0.6) -- node[midway, above]{$\vec{F}$} ++(2,0);

    \draw (0.5,0.6) node[]{$m_A$};
    \draw [-stealth] (0.85,0) -- node[near end, left]{$\vec{F}_{NA}$} (0.85,3);
    \draw (0.85,0.3) -- ++(-0.3,0) -- ++(0,-0.3);
    \draw [-stealth] (1.15,0.6) node[circ]{} -- ++(0,-2) node [midway, below right] {$\vec{F}_{tA}$};

    \draw (3.5,0.6) node[]{$m_B$};
    \draw [-stealth] (3.85,0) -- node[near end, left]{$\vec{F}_{NB}$} (3.85,3);
    \draw (3.85,0.3) -- ++(-0.3,0) -- ++(0,-0.3);
    \draw [-stealth] (4.15,0.6) node[circ]{} -- ++(0,-2) node [midway, below right] {$\vec{F}_{tB}$};

    \draw (6.5,0.6) node[]{$m_C$};
    \draw [-stealth] (6.85,0) -- node[near end, left]{$\vec{F}_{NC}$} (6.85,3);
    \draw (6.85,0.3) -- ++(-0.3,0) -- ++(0,-0.3);
    \draw [-stealth] (7.15,0.6) node[circ]{} -- ++(0,-2) node [midway, below right] {$\vec{F}_{tC}$};
  \end{tikzpicture}
  \caption{Tre masse trainate}
  \label{fig:tre_masse_trainate}
\end{figure}

\noindent
Tali masse vengono trainate con una forza $F = 200$ N, mentre le massse sono $m_A = 30$ kg, $m_B = 50$ kg e $m_C = 20$ kg; inoltre è noto che il coefficiente di attrito cinetico è $\mu_k = 0.1$. Si determini, allora
\begin{itemize}
  \item l'accelerazione dell'intero sistema;
  \item la tensione delle corde $A-B$ e $B-C$.
\end{itemize}
Si realizzi il diagramma a corpo libero del sistema oggetto di studio

\vspace{1em}
\begin{figure}[H]
  \centering
  \begin{tikzpicture}[scale=1.5]
    \draw [-stealth] (0,0) node[circ]{} -- ++(0,1) node [at end, right] {$\vec{F}_{NA}$};
    \draw [-stealth] (0,0) -- ++(0,-1) node [midway, right] {$\vec{F}_{tA}$};
    \draw [-stealth] (0,0) -- ++(-1,0) node [midway, above] {$\vec{F}_{kA}$};
    \draw [-stealth] (0,0) -- ++(1,0) node [midway, above] {$\vec{F}_{TA}$};
  \end{tikzpicture}
  \hspace{2em}
  \begin{tikzpicture}[scale=1.5]
    \draw [-stealth] (0,0) node[circ]{} -- ++(0,1) node [at end, right] {$\vec{F}_{NB}$};
    \draw [-stealth] (0,0) -- ++(0,-1) node [midway, right] {$\vec{F}_{tB}$};
    \draw [-stealth] (0,0.1) -- ++(-1,0) node [midway, above] {$\vec{F}_{kB}$};
    \draw [-stealth] (0,-0.1) -- ++(-1,0) node [midway, below] {$\vec{F}_{AB}$};
    \draw [-stealth] (0,0) -- ++(1,0) node [midway, above] {$\vec{F}_{TB}$};
  \end{tikzpicture}
  \hspace{2em}
  \begin{tikzpicture}[scale=1.5]
    \draw [-stealth] (0,0) node[circ]{} -- ++(0,1) node [at end, right] {$\vec{F}_{NC}$};
    \draw [-stealth] (0,0) -- ++(0,-1) node [midway, right] {$\vec{F}_{tC}$};
    \draw [-stealth] (0,0.1) -- ++(-1,0) node [midway, above] {$\vec{F}_{kC}$};
    \draw [-stealth] (0,-0.1) -- ++(-1,0) node [midway, below] {$\vec{F}_{BC}$};
    \draw [-stealth] (0,0) -- ++(1,0) node [midway, above] {$\vec{F}$};
  \end{tikzpicture}
  \caption{Diagramma a corpo libero di $3$ masse trainate}
  \label{fig:diagramma_corpo_libero_3_masse_trainate}
\end{figure}

\vspace{1em}
\noindent
Per la risoluzione del primo quesito, si può sfruttare la \textbf{proprietà additiva} della massa, essendo i tre corpi omogenei e costituiti dalla stessa sostanza. Pertanto l'assieme $ABC$ si comporta come un unico corpo, la cui massa complessiva è
\[m = m_A + m_B + m_C = 30 \text{ kg} + 50 \text{ kg} + 20 \text{ kg} = 100 \text{ kg}\]
Pertanto si può realizzare un nuovo diagramam a corpo libero, mostrato di seguito:

\vspace{1em}
\begin{figure}[H]
  \centering
  \begin{tikzpicture}[scale=1.5]
    \draw [-stealth] (0,0) node[circ]{} -- ++(0,1) node [at end, right] {$\vec{F}_{N}$};
    \draw [-stealth] (0,0) -- ++(0,-1) node [midway, right] {$\vec{F}_{t}$};
    \draw [-stealth] (0,0) -- ++(-1,0) node [midway, above] {$\vec{F}_{k}$};
    \draw [-stealth] (0,0) -- ++(1,0) node [midway, above] {$\vec{F}$};
  \end{tikzpicture}
  \caption{Diagramma a corpo libero di un'unica massa}
  \label{fig:diagramma_corpo_libero_unica_massa}
\end{figure}

\vspace{1em}
\noindent
Pertanto si ottiene che, per la $2^a$ legge della dinamica:
\[\sum \vec{F} = m \vec{a}\]
Naturalmente le forze possono essere scomposte nei loro rispettivi componenti, per cui si ottiene
\[
  y : \left\{
  \rowcolors{1}{white}{white}
  \begin{array}{l}
    \vec{F}_N = - \vec{F}_t\\
    F_N = mg
  \end{array}
  \right.
\]
mentre si ottiene che
\[
  x : \left\{
  \rowcolors{1}{white}{white}
  \begin{array}{l}
    m a_x = F - F_k\\
    F_k = \mu_k \cdot F_N
  \end{array}
  \right.
\]
da cui
\[a_x = \frac{F}{m} - \mu_k \cdot g = 2.0 \text{ m/s}^2 - 0.98 \text{ m/s}^2 = 1.0 \text{ m/s}^2\]
Per la risoluzione del secondo quesito, si determini dapprima la tensione della corda $A-B$, ovvero $\vec{F}_{AB}$ che, per la $3^a$ legge di Newton è uguale, in modulo, alla forza $\vec{F}_{TA}$.\\
Dalla $2^a$ legge della dinamica applicata al corpo $A$ si ottiene
\[\vec{F}_{NA} + \vec{F}_{tA} + \vec{F}_{AB} + \vec{F}_{kA} = m_A \cdot \vec{a}_A\]
Naturalmente si ha che, scomponedo tale equazione nelle sue componenti $x$ e $y$ si ottiene
\[
  \left\{
  \rowcolors{1}{white}{white}
  \begin{array}{l}
    F_{AB x} - \mu_k m_A g = m_A a_{A x}\\
    F_N + F_t = m_A a_{A y} = 0 \longrightarrow F_N = - F_t
  \end{array}
  \right.
\]
Si può procedere, ora, al calcolo di $F_{BA x}$ come segue (ricordando che l'accelerazione del sistema è la stessa di ciascuna massa, ovviamente):
\[F_{BA x} = m_A \cdot (a_{Ax} + \mu_k g) = 30 \text{ kg} \cdot \left[1 \text{ m/s}^2 + 0.1 \cdot 9.8 \text{ m/s}^2\right] = 60 \text{ N}\]
Da notare che tale formula poteva anche essere riscritta come segue
\[F_{BA} = m_A \cdot \left(\frac{F}{m} - \mu_k \cdot g + \mu_k \cdot g\right) = \frac{m_A}{m_A + m_B + m_C} \cdot F = 60 \text{ N}\]
Per la determinazione della tensione $B-C$, è sufficiente procedere come già fatto, impiegando la seconda legge della dinamica sul corpo $B$ (o anche sul corpo $C$, visto che sono noti tutti i dati del problema). Si applichi, allora, la $2^a$ legge della dinamica sul corpo $C$, ottenendo
\[\sum \vec{F} = m \cdot \vec{a} \longrightarrow \vec{F} + \vec{F}_{kc} + \vec{F}_{NC} + \vec{F}_{tC} + \vec{F}_{BC} = m_C \cdot \vec{a}_C\]
Naturalmente è possibile scomporre tale equazione nelle sue componenti $x$ e $y$, ottenendo
\[y : F_{NC} - F_{tC} = 0 \longrightarrow F_{NC} = F_{tC} = m_C g\]
\[x : F - F_{kC} - F_{BC} = m_C a_{xC} \longrightarrow F - \mu_k m_C g - F_{BC} = m_C \cdot a_{xC}\]
Da cui si evince che
\[F_{BC} = F - \mu_k g m_C - m_C \cdot a_{xC} = 200 \text{ N} - 20 \text{ N} - 20 \text{ N} = 160 \text{ N}\]

\newpage
\noindent
\begin{center}
  15 Marzo 2022
\end{center}
Quando bisogna risolvere un problema di meccanica, è essenziale visualizzare il problema tramite una rappresentazione grafica; dopodiché è fondamentale procedere alla realizzazione del diagramma a corpo libero del sistema.\\
Solamente a questo punto è possibile procedere alla stesura delle equazioni dinamiche che consentono di isolare le richieste del problema.

\vspace{1em}
\subsection{Attrito dovuto a un fluido (resistenza)}
Un'altra importante forza di attrito è la resistenza di un fluido: l'aria, per esempio, rallenta il moto di un corpo, come accade quando si lascia cadere dalla stessa altezza un martello e una piuma (sulla terra, dove c'è aria, il martello cade prima della piuma, mentre sulla Luna, in assenza di atmosfera, i due corpi cadono alla medesima velocià, impiegando lo stesso tempo).

% Tabella per le definizione di concetti, etc...
\vspace{1em}
\rowcolors{1}{black!5}{black!5}
\setlength{\tabcolsep}{14pt}
\renewcommand{\arraystretch}{2}
\noindent
\begin{tabularx}{\textwidth}{@{}|P|@{}}
    \hline
    {\textbf{ATTRITO DOVUTO A UN FLUIDO (RESISTENZA)}}\\
    \parbox{\linewidth}{La resistenza di un fluido è la forza di attrito che il fluido produce sul corpo in movimento, rallentandone la corsa. La causa di tale resistenza è dovuta alla \textbf{viscosità del fluido} che, naturalmente, esercita una \textbf{forza opposta} $\vec{F}_v$ al moto del fluido (naturalmente le proprietà del fluido sono determinanti).\\
    Il problema della forza esercitata da un fluido viscoso su un corpo in esso immerso è complesso; tuttavia, tale fenomeno vine modellizzato, in maniera approssimativa, attraverso due diversi schemi teorici:
    \begin{enumerate}
      \item George Stokes, nel $1845$, prese in considerazione il problema solo per un caso particolare, quello di un \textbf{oggetto di forma sferica}, \textbf{completamente immerso} in un fluido in \textbf{moto laminare}, di \textbf{densità costante} ed \textbf{incomprimibile}.\\
      Perrtanto, a velocità bassa con densità bassa, si ha un \textbf{flusso laminare} (ovvero si è in \textbf{assenza di turbolenza}) e per determinare la forza resistiva si impiega, appunto, la \textbf{legge di Stokes}
      \[\boxed{\vec{F} = -b \cdot \vec{v}}\]
      in cui $b$ prende il nome di \textbf{coefficiente di viscosità}, che dipende dalle proprietà del fluido ed è un coefficiente di proporzionalità diretta: più aumenta la velocità, maggiore sarà l'intensità della forza d'attrito. Tale forza è molto diversa rispetto alle forze precedentemente analizzate, in quanto dipende strettamente dalla tipologia di moto del corpo considerato. Naturalmente, in questo caso, l'equazione è molto semplice, ma difficilmente applicabile, in quanto nell'equazione di Newton si ha un'accelerazione, mentre la legge di Stokes fornisce una velocità: si tratta, quindi, di risolvere un'\textbf{equazione differenziale}.

      \item Se il \textbf{moto} invece è \textbf{turbolento} le forze inerziali dominano su quelle viscose e la forza di resistenza dipende da diversi fattori:
      \begin{itemize}
        \item \textbf{densità} $\rho$ del fluido;
        \item l'\textbf{area di proiezione} del fluido $A$, ovvero una stima di quando fluido viene spostato nel moto: un corpo lungo e fino non subirà molta resistenza, mentre un corpo grande e largo avrà una resistenza maggiore;
        \item il \textbf{coefficiente di caratterizzazione della forma del corpo} $C_d$ (in cui $D$ sta per \quotes{Drag}), determinato sperimentalmente e, per ovvie ragioni, \textbf{adimensionale}.
      \end{itemize}
      Considerando tali fattori, la forza di resistenza è
      \[\boxed{F_v = \frac{1}{2} \rho A C_d v^2}\]
    \end{enumerate}
    \vspace{1mm}}\\
    \hline
\end{tabularx}

\vspace{1em}
\noindent
\textbf{Esempio}: Si consideri un quadricottero che, inclinandosi orizzontalmente di un angolo $\theta$, produce uno spostamento di velocità $v$ che, naturalmente, viene rallentato dall'aria che oppone una forza di resistenza.\\
Si supponga, per ipotesi, che tale drone esegua un moto che è possibile modellizzare tramite la legge di Stokes, ovvero
\[\vec{F}_v = -b \cdot \vec{v}\]
Si determini, allora, l'angolo $\theta$ tale per cui il drone si muove ad una \textbf{velocità costante} $\vec{v}$. Si realizzi, allora, il diagramma a corpo libero seguente:

\vspace{1em}
\begin{figure}[H]
  \centering
  \begin{tikzpicture}[scale=1.5]
    \draw [-stealth] (0,0) node[circ]{} -- ++(0.5,1) node [at end, right] {$\vec{F}_s$};
    \draw [dotted] (0,0) -- ++(0,1);
    \coordinate (f) at (0,0.8);
    \coordinate (ctr) at (0,0);
    \coordinate (i) at (0.5,1);
    \pic [draw=red, text=red, <->, "$\theta$", angle radius=1cm, angle eccentricity=1.4] {angle = i--ctr--f};
    \draw [-stealth] (0,0) -- ++(0,-1) node [midway, right] {$\vec{F}_{t}$};
    \draw [-stealth] (0,0) -- ++(-1,0) node [midway, above] {$\vec{F}_{v}$};
  \end{tikzpicture}
  \caption{Diagramma a corpo libero di un drone inclinato}
  \label{fig:diagramma_corpo_libero_drone_inclina}
\end{figure}

\vspace{1em}
\noindent
Naturalmente, in questo caso, è possibile applicare la $2^a$ legge della dinamica, in quanto è noto che il drone si muove a velocità costante, quindi l'accelerazione è nulla e quindi la somma delle forze risultanti è nulla.\\
Pertanto si ha che
\[\vec{F}_t + \vec{F}_v + \vec{F}_s = m \cdot \vec{a} = 0\]
Naturalmente è possibile scomporre tale equazione vettoriale nelle sue componenti $x$ e $y$, ottenendo
\[x: F_s \cdot \sin(\theta) - b v = 0\]
\[y: F_s \cdot \cos(\theta) - m g = 0\]
Dalla seconda equazione si ottiene che
\[F_s = \frac{m g}{\cos(\theta)}\]
mentre dalla prima equazione si ha che
\[v = \frac{mg}{b} \cdot \tan(\theta)\]

\vspace{1em}
\noindent
\textbf{Esempio}: Si calcoli la velocità limite di un paracadutista che scente verticalmente, di cui si propone di seguito il diagramma a corpo libero:

\vspace{1em}
\begin{figure}[H]
  \centering
  \begin{tikzpicture}[scale=1]
    \draw [-stealth] (0,0) node[circ]{} -- ++(0,1) node [midway, right] {$\vec{F}_v$};
    \draw [-stealth] (0,0) -- ++(0,-1) node [midway, right] {$\vec{F}_{t}$};
  \end{tikzpicture}
  \caption{Diagramma a corpo libero di un drone inclinato}
  \label{fig:diagramma_corpo_libero_drone_inclinato}
\end{figure}

\vspace{1em}
\noindent
Infatti, ad un certo punto, la forza peso del corpo in caduta libera verrà eguagliata dalla forza di resistenza dell'aria e il paracadutista non aumenterà più la sua velocità, ma la manterrà costante.\\
Naturalmente, in questo caso, è necessario applicare la seconda formula della resistenza, semplicemente osservando che
\[mg = F_v = \frac{1}{2} \cdot \rho A C_d v^2\]
Per isolare la velocità limite, semplicemente si può scrivere
\[v = \sqrt{\frac{2 m g}{\rho A C_d}}\]
Naturalmente, se si suppone che
\begin{itemize}
  \item $m = 70$ kg
  \item $\rho = 1.2$ kg/m$^3$
  \item $C_d = 0.8$
  \item $A = 0.5$ m$^2$
\end{itemize}
Considerando tali dati per il problema si ottiene che
\[v = 55 \text{ m/s}\]
e tale risultato è totalmente ininflente dall'altezza dalla quale ci si paracaduta, in quanto la velocità limite che è possibile raggiungere è $v = 55 \text{ m/s}$.

\vspace{1em}
\subsection{Dinamica del moto circolare uniforme}
Naturalmente, è noto che nel moto circolare uniforme la velocità $v$ è costante, per cui l'accelerazione è soltanto centripeta e diretta verso il centro della circonferenza, ortogonalmente al vettore velocità.\\
È noto che il modulo dell'accelerazione nel moto circolare uniforme è pari a
\[a = \frac{v^2}{R} = \omega^2 R\]
essendo
\[\omega = \frac{v}{R}\]

% Tabella per le definizione di concetti, etc...
\vspace{1em}
\rowcolors{1}{black!5}{black!5}
\setlength{\tabcolsep}{14pt}
\renewcommand{\arraystretch}{2}
\noindent
\begin{tabularx}{\textwidth}{@{}|P|@{}}
    \hline
    {\textbf{FORZA CENTRIPETA}}\\
    \parbox{\linewidth}{La forza necessaria per mantenere il moto circolare è, per la seconda legge della dimanica:
    \[\vec{F} = m \vec{a}\]
    e sostitutendo ad $a$ la formula precedentemente ottenuta si ha
    \[\boxed{F_c = \frac{m v^2}{R}}\]
    orientata verso il centro della circonferenza. Naturalmente quest'ultima è la \textbf{forza centripeta}, così chiamata per descriverne la direzione e il verso, ma non la natura della forza (la forza centripeta non è un tipo di forza, in quanto anche la forza di attrazione gravitazionale è una forza centripeta quando si parla di attrazione tra pianeti, anche la forza d'attrito è una forza centripeta, come quando si gira con la macchina).\vspace{3mm}}\\
    \hline
\end{tabularx}

\vspace{2em}
\noindent
\textbf{Esempio}: Nelle curve delle strade, la carreggiata è inclinata, in quanto la forza normale si scompone in due componenti: una centrifuga e una centripeta; ciò, naturalmente, aiuta la forza di attrito ad evitare che la vettura esca fuori strada.\\
Si determini, allora, l'angolo $\theta$ di inclinazione della strada affinché la componente orizzontale della forza normale fornisca la forza centripeta necessaria per girare (supponendo in assenza di attrito):

\vspace{1em}
\noindent
\begin{figure}[H]
  \centering
  \newcommand{\ang}{30}

  \begin{tikzpicture}[scale=1.5]

  % triangle:
  \draw [draw = orange, fill = orange!15] (0,0) coordinate (O) -- (\ang:6)
  	coordinate [pos=.45] (M) |- coordinate (B) (O);

  % angles:
  \draw [draw = orange] (O) ++(.8,0) arc (0:\ang:0.8)
  	node [pos=.4, left] {$\theta$};
  \draw [draw = orange] (B) rectangle ++(-0.3,0.3);

  \begin{scope} [rotate=\ang]
    \draw (M) -- ++(0,0.25) -- ++(-0.25,0) -- ++(0,0.25) -- ++(0.25,0) -- ++(0,0.5) -- ++(1,0) -- ++(0,-0.5) -- ++(0.25,0) -- ++(0,-0.25) -- ++(-0.25,0) -- ++(0,-0.25) ++(-0.25,0) -- ++(0,0.25) -- ++(-0.5,0) -- ++(0,-0.25);
  \end{scope}

  \begin{scope} [-latex,rotate=\ang]
  % Weight Force and its projections
  \coordinate (MM) at ([xshift=0.5cm,yshift=0.6cm]M);

  \draw (MM) node[circ]{} -- ++ (-\ang-90:1.5)
  	node [very near end, right] {$\vec{F}_t$};

  % Normal Force
  \draw (MM) ++(0,-0.6) -- ++ (0,1.8)
    node [very near end, right] {$\vec{F}_N$};
  \end{scope}
  \end{tikzpicture}
  \caption{Auto in curva su un piano inclinato}
  \label{fig:auto_curva_piano_inclinato}
\end{figure}

\vspace{1em}
\noindent
Si realizzi il diagramma a corpo libero della vettura:

\vspace{1em}
\begin{figure}[H]
  \centering
  \begin{tikzpicture}[scale=1]
    \draw [-stealth] (0,0) node[circ]{} -- ++(-0.8,0.8) node [near end, above right] {$\vec{F}_N$};
    \draw [-stealth] (0,0) -- ++(0,-1) node [midway, right] {$\vec{F}_{t}$};
  \end{tikzpicture}
  \caption{Diagramma a corpo libero di una macchina in curva}
  \label{fig:diagramma_corpo_libero_macchina_curva}
\end{figure}

\vspace{1em}
\noindent
Naturalmente, scomponendo la forza normale nelle sue due componenti (orizzontale e verticale) si ottiene che
\[F_N \cdot \sin(\theta) = F_c = \frac{m v^2}{R}\]
\[F_N \cdot \cos(\theta) - mg = 0 \longrightarrow F_N \cdot \cos(\theta) = mg\]
Allora si ha che
\[\tan(\theta) = \frac{v^2}{R g}\]

\vspace{1em}
\noindent
\textbf{Osservazione}: Anche quando si considera il moto di una corda che viene fatta ruotare nell'aria, la tensione della corda è esattamente la forza centripeta necessaria a mantere il moto circolare uniforme (conoscendo il raggio di rotazione, ossia la lunghezza della corda, nonché la velocità di rotazione).\\
Per verificare che il coefficiente di attrito statico non può essere mai maggiore di $1$ è sufficiente porre una massa su una superficie inclinata e aumentare l'angolo di inclinazione.\\

\newpage
\noindent
\begin{center}
  16 Marzo 2022
\end{center}
\subsection{Sistemi non-inerziali}
Il concetto di sistema non-inerziale è essenziale per lo studio della dinamica dei corpi. Infatti, un sistema non-inerziale è un sistema di riferimento in cui l'accelerazione non é nulla, ovvero $a \neq 0$. Ciò impedisce di applicare la $2^a$ legge della dinamica, in quanto tale legge si applica solo a sistemi di riferimento inerziali.\\
Si consideri il caso di un treno in accelerazione e di una massa sospesa al soffitto; naturalmente, quando il treno accelera (e quindi la sua velocità non è costante), la massa si inclina nella direzione opposta al moto:

\begin{figure}[H]
  \centering
  \begin{tikzpicture}[scale=4]
    \draw [dashed] (0,2.5) node[circ]{} -- ++(0,-1);
    \draw [dashed] (0,2.5) node[circ]{} -- (-0.4,1.5);
    \draw (0,2.5) node[circ]{} -- (-0.2,2) node[circ]{};
    \coordinate (i) at (-0.2,2);
    \coordinate (ctr) at (0,2.5);
    \coordinate (f) at (0,2);
    \pic [draw=red, text=red, <->, "$\theta$", angle radius=1cm, angle eccentricity=1.4] {angle = i--ctr--f};
    \draw [-stealth] (0.2,2) -- ++(0.4,0) node[midway, above]{$\vec a$};
    \draw [-stealth] (-0.2,2) -- ++(0,-0.4) node[at end, right]{$\vec F_t$};
    \draw [-stealth,red] (-0.2,2) -- ++(0.1,0.25) node[left]{$\vec F_T$};
    \draw [-stealth,blue] (-0.2,2) -- ++(-0.2,0) node[above]{$\vec F_a$};
  \end{tikzpicture}
  \caption{Forza apparente}
  \label{fig:forza_apparente}
\end{figure}

\noindent
In cui, ovviamente, applicando la $2^a$ legge della dinamica si ottiene che
\[\vec{F}_T + \vec{F}_t = m \cdot \vec{a}\]
da cui
\[\vec{F}_T + m \vec{g} = m \vec{a} \longrightarrow \vec{F}_T = m \cdot \left(\vec{a} - \vec{g}\right)\]

\vspace{1em}
\noindent
\textbf{Osservazione}: Si osservi che la prima equazione può essere riscritta come segue:
\[\vec{F}_T + \vec{F}_t = m \cdot \vec{a} \longrightarrow \vec{F}_T + \vec{F}_t - m \cdot \vec{a} = 0\]
Ecco che allora la forza $\vec{F}_a = m \vec{a}$ è una nuova forza, la quale prende il nome di \textbf{forza apparente}, o anche \textbf{pseudo-forza apparente} o \textbf{forza inerziale}, che agisce ancora una volta sulla massa $m$, in direzione opposta alla direzione del moto.\\
Naturalmente, in questo caso, l'unico termine da considerare è $m \vec{a}$, in quanto l'accelerazione è costante, mentre se l'accelerazione non lo fosse, all'interno della formula dovrebbero figurare altri termini (si pensi, per esempio, ad una \textbf{pseudo-forza apparente centrifuga}, la quale può essere costante in modulo, ma cambia direzione e verso).

\vspace{1em}
\noindent
\textbf{Esempio}: Si consideri il caso in cui una massa $m$ è posta su un cuneo e l'intero sistema accelera con $\vec{a}$ costante:

\vspace{1em}
\noindent
\begin{figure}[H]
  \centering
  \newcommand{\ang}{30}

  \begin{tikzpicture} [xshift=-12em,font = \small, scale=1.5]
  \reflectbox{\rotatebox[origin=c]{360}{
    % triangle:
    \draw [draw = orange, fill = orange!15] (0,0) coordinate (O) -- (\ang:6)
    	coordinate [pos=.45] (M) |- coordinate (B) (O);

    % angles:
    \draw [draw = orange] (O) ++(.8,0) arc (0:\ang:0.8)
    	node [pos=.4, left] {\reflectbox{\rotatebox[origin=c]{360}{$\theta$}}};
    \draw [draw = orange] (B) rectangle ++(-0.3,0.3);
    \draw [-stealth] (5.5,1.5) ++(1,0) -- ++(-0.75,0) node[midway,above]{\reflectbox{\rotatebox[origin=c]{360}{$a$}}};

    \begin{scope} [-latex,rotate=\ang]
    % Object (rectangle)
    \draw [fill = purple!30,
    	draw = purple!50] (M) rectangle ++ (1,.6);

    % Weight Force and its projections
    \draw [dashed] (M) ++ (.5,.3) coordinate (MM) -- ++ (0,-1.29)
    	node [very near end, right] {\reflectbox{\rotatebox[origin=c]{360}{$\vec{F}_t \cdot \cos{\theta}$}}};

    \draw [dashed] (MM) -- ++ (-0.75,0)
    	node [very near end, left] {\reflectbox{\rotatebox[origin=c]{360}{$\vec{F}_t \cdot \sin{\theta}$}}};

    \draw (MM) -- ++ (-\ang-90:1.5)
    	node [very near end,below left ] {\reflectbox{\rotatebox[origin=c]{360}{$\vec{F}_t$}}};

    % Normal Force
    \draw (MM) -- ++ (0,1.29)
    node [very near end, right] {\reflectbox{\rotatebox[origin=c]{360}{$\vec{F}_N$}}};
    \end{scope}
  }}
  \end{tikzpicture}
  \caption{Piano inclinato in accelerazione}
  \label{fig:piano_inclinato_in_accelerazione}
\end{figure}

\noindent
Si determini, allora, quale deve essere l'accelerazione $\vec{a}$ affinché il blocco rimanga immobile sul cuneo.\\
Naturalmente, osservando il sistema da fuori, le uniche forze che agiscono su tale corpo sono la forza peso e la forza normale. Se, invece, tale corpo viene osservato dall'interno, impiegando un sistema di riferimento non-inserziale, allora a tali forze se ne deve aggiungere una terza, una \textbf{pseudo-forza apparente} che agisce in verso opposto a quello dell'accelerazione $\vec{a}$, ovvero la forza $\vec{F}_a = - m \vec{a}$, come mostrato di seguito:

\vspace{1em}
\begin{figure}[H]
  \centering
  \begin{tikzpicture}[scale=1.5]
    \draw [-stealth] (0,0) -- ++(0,-1) node [midway, right] {$\vec{F}_{t}$};
    \draw [-stealth] (0,0) -- ++(0.5,1) node [midway, above left] {$\vec{F}_{T}$};
    \draw [-stealth] (0,0) -- ++(-1,0) node [midway, above] {$\vec{F}_{a}$};
  \end{tikzpicture}
  \caption{Diagramma a corpo libero di una massa su un piano inclinato in accelerazione}
  \label{fig:diagramma_corpo_libero_massa_piano_inclinato_in_accelerazione}
\end{figure}

\vspace{1em}
\noindent
Da cui si evince che
\[\vec{F}_N + \vec{F}_t + \vec{F}_a = 0\]
e scomponento l'equazione nelle sue componenti orizzontali e verticali si perviene al risultato seguente:
\begin{align*}
  F_N \cdot \cos(\theta) - mg = 0\\
  -F_a + F_N \cdot \sin(\theta) = 0
\end{align*}
per cui si ottiene che
\[a = g \cdot \tan(\theta)\]

\vspace{1em}
\subsection{Prodotto vettoriale}
Si considerino due vettori $\vec{A}$ e $\vec{B}$, allora si ha che
\[\left \vert \vec{A} \times \vec{B} \right \vert = \left \vert A \right \vert \cdot \left \vert B \right \vert \cdot \sin(\theta)\]
ove $\theta$ è l'angolo compreso tra i vettori $\vec{A}$ e $\vec{B}$.

\vspace{1em}
\noindent
\textbf{Osservazione}: Si osservi che il prodotto vettoriale $\vec{A} \times \vec{B}$ è un vettore ortogonale ad $\vec{A}$ e $\vec{B}$:

\begin{figure}[H]
  \centering
  \begin{tikzpicture}
    \draw[-,fill=white!95!red](0,0)--(3,0)--(4,1)--(1,1)--cycle;
    \node at (2,0.5) {$|\textcolor{blue}{a}\times \textcolor{red}{b}|$};
    \draw[ultra thick,-latex,blue](0,0)--(3,0)node[midway,below]{$a$};
    \draw[ultra thick,-latex,red](0,0)--(1,1)node[midway,above]{$b$};
    \draw[ultra thick,-latex,blue!50!red](0,0)--(0,3)node[pos=0.7,right]{$a\times b$};
    \draw (0.6,0) arc [start angle=0,end angle=45,radius=0.6]
    node[pos=0.7,right]{$\theta$};
  \end{tikzpicture}
  \caption{Prodotto vettoriale}
  \label{fig:prodotto_vettoriale}
\end{figure}

\noindent
Inoltre si ha che
\begin{itemize}
  \item se $\vec{A} \perp \vec{B}$, allora si ha che
  \[\left \vert \vec{A} \times \vec{B} \right \vert = \left \vert A \right \vert \cdot \left \vert B \right \vert\]

  \item se $\vec{A} \parallel \vec{B}$, allora si ha che
  \[\left \vert \vec{A} \times \vec{B} \right \vert = 0\]

  \item negli altri casi bisogna impiegare la regola della mano destra. Se si considera il prodotto vettoriale seguente (che non è mai commutativo)
  \[\vec F = q \vec v \times \vec B\]
  la regola della mano destra si applica come segue:

  \begin{figure}[H]
    \centering
    \begin{tikzpicture}[scale=0.5]
      \coordinate (O) at (1.0,0.7);    % ORIGIN
      \coordinate (WT) at ( 2.9,-1.1); % WRIST TOP
      \coordinate (T1) at ( 2.3, 0.7); % THUMB
      \coordinate (T2) at ( 1.75, 2.3);
      \coordinate (T3) at ( 2.0, 3.1);
      \coordinate (T4) at (1.38, 3.15);
      \coordinate (T5) at ( 0.9, 2.3);
      \coordinate (T6) at ( 0.85, 1.2);
      \coordinate (T7) at ( 0.85, 0.2);
      \coordinate (I1) at (-1.1, 2.45); % INDEX
      \coordinate (I2) at (-2.9, 3.45);
      \coordinate (I3) at (-3.3, 2.9);
      \coordinate (I4) at (-1.5, 1.8);
      \coordinate (I5) at (-0.9, 1.1);
      \coordinate (I6) at (-0.9, 0.3);
      \coordinate (M1) at (-2.1, 0.9); % MIDDLE
      \coordinate (M2) at (-3.95,0.55);
      \coordinate (M3) at (-4.0,-0.15);
      \coordinate (M4) at (-2.3, 0.05);
      \coordinate (M5) at (-1.1, 0.20);
      \coordinate (R1) at (-1.9,-0.1); % RING
      \coordinate (R2) at (-1.8,-0.7);
      \coordinate (R3) at (-0.3,-1.5);
      \coordinate (R4) at ( 0.1,-1.7);
      \coordinate (R5) at ( 0.1,-1.0);
      \coordinate (R6) at (-0.5,-0.7);
      \coordinate (R7) at (-1.2,-0.3);
      \coordinate (P1) at (-1.9,-1.3); % PINKY
      \coordinate (P2) at (-0.8,-1.9);
      \coordinate (P3) at (-0.2,-2.1);
      \coordinate (P4) at (-0.05,-1.65);
      \coordinate (W1) at ( 0.4,-2.9); % WRIST BOTTOM
      \coordinate (W2) at ( 1.6,-3.5);

      % HAND
      \fill[pinkskin]
        (WT) -- (T6) -- (I5) -- (M5) -- (R2) -- (P2) -- (W2) to[out=25,in=-90] cycle;
      \draw[fill=pinkskin]
        (WT) to[out=120,in=-60] % THUMB
        (T1) to[out=120,in=-90]
        (T2) to[out=80,in=-110]
        (T3) to[out=80,in=50,looseness=1.5] % tip
        (T4) to[out=-130,in=80]
        (T5) to[out=-100,in=70]
        (T6) to[out=-100,in=100]
        (T7)
        (T6) to[out=150,in=-30] % INDEX
        (I1) to[out=150,in=-30]
        (I2) to[out=150,in=145,looseness=1.7] % tip
        (I3) to[out=-30,in=150]
        (I4) to[out=-30,in=105]
        (I5) to[out=-75,in=90]
        (I6)
        (I5) to[out=-170,in=10] % MIDDLE
        (M1) to[out=-170,in=10]
        (M2) to[out=-170,in=-175,looseness=1.8] % tip
        (M3) to[out=5,in=-170]
        (M4) to[out=10,in=-170] % bottom knuckle
        (M5)
        (M5) to[out=-160,in=50] % RING
        (R1) to[out=-130,in=140,looseness=1.2]
        (R2) to[out=-30,in=160]
        (R3) --
        (R4) to[out=-20,in=-20,looseness=1.5] % tip
        (R5) --
        (R6) to[out=140,in=8,looseness=0.9]
        (R7)
        (R2) to[out=-160,in=155] % PINKY
        (P1) to[out=-35,in=150]
        (P2) to[out=-30,in=160]
        (P3) to[out=-20,in=-30,looseness=1.5] % tip
        %(P4) --
        (R4)
        (P2) to[out=-50,in=140] % WRIST
        (W1) to[out=-40,in=160]
        (W2);

      % FOLDS
      \draw[very thin] (T5)++(-80:0.3) to[out=40,in=180]++ (25:0.45); % THUMB
      \draw[very thin] (I1)++(180:0.2) to[out=-160,in=90]++ (-130:0.6); % INDEX
      \draw[very thin] (I1)++(155:1.3) to[out=-150,in=80]++ (-130:0.55);
      \draw[very thin] (M4)++(30:0.2) to[out=80,in=-65]++ (95:0.5); % MIDDLE FINGER
      \draw[very thin] (M3)++(10:0.8) to[out=80,in=-75]++ (90:0.45);
      \draw[very thin] (M5)++(-140:0.1) to[out=-20,in=90]++ (-54:0.8); % RING
      \draw[very thin] (R6) to[out=160,in=10]++ (180:0.2);
      \draw[very thin] (R3)++(155:0.5) to[out=120,in=-100]++ (100:0.2);
      \draw[very thin] (P2)++(140:0.1) to[out=95,in=-110]++ (80:0.4); % PINKY
      %\draw[very thin] (P1)++( 10:0.04) to[out=95,in=-130]++ (70:0.4);
      \draw[very thin] (I5)++(-40:0.45) to[out=-70,in=90]++ (-70:1.7);    % PALM
      \draw[very thin] (P3)++(-155:0.05) to[out=-120,in=40]++ (-130:0.2); % PALM
      \draw[very thin] (W2)++(70:1.4) to[out=-175,in=-40]++ (160:1.4); % PALM

      % VECTORS
      \def\R{0.32}
      \draw[force]
        (O) --++ (82:3.2)
        node[above,scale=1.5] {$\vb{F} \color{black} = q {\color{veccol}\vb{v}} \times {\color{Bcol}\vb{B}}$};
      \draw[velocity]
        (O) --++ (148:3.3) coordinate (V)
        node[above=2,left=0,scale=1.5] {$\vb{v}$};
      \draw[charge+] (O) circle (\R) node[scale=1.4] {$+$};
      \draw[BField]
        (O)++(-172:0.7*\R) --++ (-172:3.25) coordinate (B)
        node[above=4,left=0,scale=1.5] {$\vb{B}$};
      \draw pic[->,"\huge$\theta$",draw=black,thick,angle radius=28,angle eccentricity=1.26] {angle = V--O--B};
    \end{tikzpicture}
    \caption{Regola della mano destra}
    \label{fig:regola_mano_destra}
  \end{figure}

  \item $\hat{i} \times \hat{j} = \hat{k}$
  \item $\hat{j} \times \hat{k} = \hat{i}$
  \item $\hat{k} \times \hat{i} = \hat{j}$
  \item $\hat{j} \times \hat{i} = -\hat{k}$
  \item $\hat{k} \times \hat{j} = -\hat{i}$
  \item $\hat{i} \times \hat{k} = -\hat{j}$
  \item $\hat{i} \times \hat{i} = \hat{j} \times \hat{j} = \hat{k} \times \hat{k} = 0$
\end{itemize}
Alternativamente, è noto che il prodotto vettoriale $\vec{A} \times \vec{B}$ si calcola come
\[\vec{A} \times \vec{B} = \det \left(
  \rowcolors{1}{white}{white}
  \begin{array}{ccc}
    \hat{i} & \hat{j} & \hat{k}\\
    A_x & A_y & A_z\\
    B_x & B_y & B_z
  \end{array}
\right)\]

\vspace{1em}
\noindent
\textbf{Osservazione}: Mentre il prodotto vettoriale è il determinante della matrice di cui sopra, il prodotto scalare, $\vec{B} \cdot \vec{A}$ si interpretava come $\vert A \vert$ per la componente di $B$ su $A$.\\
Inoltre si ha che il prodotto scalare tra $\vec{A}$ e $\vec{B}$ non viene alterato se ai due vettori vengono aggiunte ulteriori componenti parallele ad $\vec{A}$ e $\vec{B}$.

\vspace{1em}
\noindent
\textbf{Esempio}: Si consideri la forza di Lorentz, la cui formula viene di seguito esposta
\[\boxed{\vec{F} = g \vec{E} + g \vec{v} \times \vec{B}}\]
in cui, senza $\vec{E}$, sarebbe stato
\[\boxed{\vec{F} = g \vec{v} \times \vec{B}}\]
Allora, in questo caso, se il campo magnetico $\vec{B}$ va dentro la pagina e la velocità $\vec{v}$ è parallela alla pagina, la forza $\vec{F}$ è diretta verso il centro della spirale, ovvero $\vec{F}$ è una \textbf{forza centripeta}, calcolata come segue
\[\vec{F} = -e \cdot (\vec{v} \times \vec{B})\]
Ma dalla cinetica è anche noto che la forza centripeta si calcola come
\[F_c = \frac{m v^2}{R}\]
per cui operando una eguaglianza si ottiene
\[\frac{m v^2}{R} = e v B \longrightarrow \frac{m v}{R} = e B \longrightarrow m \omega = e B \longrightarrow \omega = \frac{eB}{m}\]
in cui $\omega$ prende il nome di \textbf{frequenza ciclotronica}, un valore utilizzabile per determinare la traiettoria e la natura del campo magnetico $B$, grazie all'emissione di fotoni da parte delle particelle.\\
Tale fenomeno non si verifica solo sperimentalmente grazie ad un acceleratore di particelle, ma anche nello spazio, dove sono presenti campi magnetici e particelle cariche, le quali ruotano con frequenza angolare corrispondente alla frequenza ciclotronica;

\vspace{1em}
\subsection{Pseudo-forza di Coriolis}
Si consideri il seguente diagramma di un moto circolare

\begin{figure}[H]
  \centering
  \begin{tikzpicture}[>=Triangle]
    \shade [top color=white, bottom color=gray!50, middle color=white]
      (120:8/3) arc (120:190:8/3) node [black, near end, left] {$\omega$}
      -- (190:25/9) -- (200:15/6) -- (190:20/9) -- (190:7/3)
      arc (190:120:7/3) -- cycle;

    \foreach \i in {90, 210, 330}{
      \draw [->, thick, blue!50!cyan] (\i-65:2) arc (\i-65:\i+60:2);
      \tikzset{shift={(\i:2)}, rotate=\i+180}
      \draw [->, very thick, orange] (0,0) -- (1,0)
        node [black, near end, anchor=\i+90] {$\vec a$};
      \draw [->, very thick, green!50!black] (0,0) -- (0,-2)
        node [black, near end, anchor=\i+180] {$\vec v$};
      \fill circle [radius=1/10];
  }
  \end{tikzpicture}
  % Set the Angles of the Axis
	\tdplotsetmaincoords{57}{120}
  \begin{tikzpicture}[scale=2,tdplot_main_coords]
		% Axis
	    \draw[->] (2,0,0) -- (-2,0,0) node[above right]{$y$};
	    \draw[->] (0,-1.3,0) -- (0,1.5,0)coordinate(C) node[right]{$x$};
	    \draw[->] (0,0,0)coordinate(B) -- (0,0,1.5) node[above]{$z$};

  	    % Circural Loop
  		\draw (0,0,0) circle [radius=1];

  		% Node
  		\draw (0,0,0) -- +(0.8,0,0) node [pos=0.6, above left] {\small$R$} ;

  		% Current Direction
  		\draw [blue, ->] (0.3,-1.2,0) arc (280:340:1.1) node [black, pos=0.4, left] {$\omega$};

  		% Vectors
  		\draw[->,very thick] (0,0,0) -- (0,0,1) node [pos=0.5, left] {$\vec \omega$};
      \draw[very thick, ->] (0,0,0) -- (-0.8,0.60,0) coordinate(A) node[above, pos=0.5] {$\vec r$};
      \draw[very thick, ->] (A) -- ++(-0.51,-0.51,0) node[above right, pos=0.5] {$\vec v$};

  		% Angle
  		\pic[draw, angle radius=7mm,"$\phi$", angle eccentricity=1.7, thick] {angle=C--B--A};
	\end{tikzpicture}
  \caption{Moto circolare uniforme e pseudo-forza di Coriolis}
  \label{fig:moto_circolare_uniforme_pseudo_forza_coriolis}
\end{figure}

\vspace{1em}
\noindent
Allora si può scrivere che, naturalmente, la velocità è il prodotto vettoriale tra il vettore velocità angolare e il vettore posizione, ovvero
\[\boxed{\vec v = \vec w \times \vec R}\]
E procedendo a derivare tale espressione rispetto al tempo si ottiene che
\[\boxed{\vec{a} = \vec{\omega} \times \vec{v} = \vec \omega \times \left(\vec \omega \times \vec R \right)}\]
in cui le parentesi sono importanti, essendo il prodotto vettoriale non associativo (infatti $\vec \omega \times \vec \omega = 0$, essendo paralleli).\\
La pseudo-forza centrifuga, pertanto,  si calcola come segue
\[\boxed{\vec F = -m \vec a = -m \vec \omega \times \left(\vec \omega \times \vec R \right)}\]
da cui si evince che se $\vec R = 0$, ovvero il punto si trova sull'asse di rotazione, allora non subisce alcuna forza, ma tanto più grande sarà il raggio, maggiore sarà anche la pseudo-forza che lo spinge in direzione opposta al moto.

\vspace{1em}
\noindent
\textbf{Osservazione}: Si consideri un punto materiale che si muove in linea retta e si supponga che poggi su un piano rotante con velocità angolare $\omega$; allora si osserva che, naturalmente, la pseudo-forza centrifuga è una componente che contribuisce a fare sì che il punto vada verso l'esterno, ma oltre a ciò, è da considerare anche un'altra forza che spinge il punto materiale in direzione opposta al verso di rotazione, ovvero la pseudo-forza di Coriolis, la quale si ha quando il sistema di riferimento non inerziale dal quale si osserva il fenomeno presenta una propria velocità, ossia non si muove con traiettoria circolare.

% Tabella per le definizione di concetti, etc...
\vspace{1em}
\rowcolors{1}{black!5}{black!5}
\setlength{\tabcolsep}{14pt}
\renewcommand{\arraystretch}{2}
\noindent
\begin{tabularx}{\textwidth}{@{}|P|@{}}
    \hline
    {\textbf{PSEUDO-FORZA DI CORIOLIS}}\\
    \parbox{\linewidth}{Tale forza prende il nome di \textbf{pseudo-forza di Coriolis}, la quale è perpendicolare a $\vec \omega$ e a $\vec v$ e si calcola come segue
    \[\boxed{\vec F = -2 m \vec \omega \times \vec v}\]
    \vspace{-1mm}}\\
    \hline
\end{tabularx}

\vspace{1em}
\noindent
\textbf{Esempio}: Si osservi che, in un vortice di bassa pressione, considerando la velocità angolare terrestre con verso uscente dal foglio e la velocità delle particelle di aria dirette verso il centro del vortice, si ottiene che la pseudo-forza di Coriolis, per la regola della mano destra, è diretta ortogonalmente al vettore velocità e giacente sul foglio stesso:

\newcommand\bonusspiral{} % just for safety
\def\bonusspiral[#1](#2)(#3:#4)(#5:#6)[#7]{% \bonusspiral[draw options](placement)(start angle:end angle)(start radius:final radius)[revolutions]
\pgfmathsetmacro{\domain}{#4+#7*360}
\pgfmathsetmacro{\growth}{180*(#6-#5)/(pi*(\domain-#3))}
\draw [#1,
       shift={(#2)},
       domain=#3*pi/180:\domain*pi/180,
       variable=\t,
       smooth,
       samples=int(\domain/5)] plot ({\t r}: {#5+\growth*\t-\growth*#3*pi/180})
}

\begin{figure}[H]
  \centering
  \rowcolors{1}{white}{white}
  \renewcommand{\arraystretch}{4}
  \begin{tabularx}{\textwidth}{|P|}
    \hline
    {
      \vspace{1.5em}
      \begin{tikzpicture}[scale=0.7]
      \node[circle,draw,minimum width=2cm] at (6.5,3.5){};
      \draw (6.5,3.5) node[circ]{} (6.5,4) node[]{$\vec \omega$} (0,0) node[]{\huge$L$};
      \bonusspiral[red](0,0)(60:270)(-1:-5)[2];
      \draw[very thick, -stealth] (2.8,-1) -- ++(-1.5,1) node[at end, above]{$\vec v$};
      \draw[very thick, -stealth] (2.8,-1) -- ++(0.5,1.5) node[at end, right]{$\vec F_c$};
      \end{tikzpicture}
    }\\
    \hline
  \end{tabularx}
  \caption{Vortice di bassa pressione e pseudo-forza di Coriolis}
  \label{fig:vortice_bassa_pressione_pseudo_forza_coriolis}
\end{figure}

\vspace{1em}
\noindent
\textbf{Osservazione}: Si osservi che il calcolo della velocità limite segue la seguente formula
\[v = \sqrt{\frac{2mg}{\rho A C_d}}\]
e sapendo che
\[m = V \cdot \rho\]
si evince che
\[v = \sqrt{\frac{2 V \rho g}{\rho A C_d}} = \sqrt{\frac{2 V g}{A C_d}}\]
pertanto, a parità di accelerazione gravitazione $g$ e coefficiente di resistenza $C_d$, ciò che determina la velocità dell'impatto con il suolo è il rapporto
\[\frac{V}{A}\]
pertanto più un corpo è voluminoso, maggiore sarà la velocità con cui impatta al suolo.

\newpage
\section{Gravità}
\noindent
\begin{center}
  17 Marzo 2022
\end{center}
La \textbf{legge di gravitazione universale} venne formulata da Isaac Newton nell'opera Philosophiae Naturalis Principia Mathematica (\quotes{Principia}) e pubblicata il 5 luglio 1687. Newton, infatti, ha affermato di voler formulare una teoria unica (appunto, universale) che descriva sia la caduta dei corpi sulla terra, sia il movimento degli astri.\\
Al fine di determinare tale formula, Newton si basò sull'osservazione della Luna e del suo moto; l'\textbf{accelerazione centripeta della Luna}, naturalmente, si calcola come segue
\[a_c = \frac{v^2}{R} = \omega^2 R\]
in cui, ovviamente, $\omega$ è la velocità angolare, mentre $R$ è il raggio dell'orbita della Luna, ovvero la distanza Terra-Luna (già nota dai tempi dei Greci); in particolare è noto che $R = 60 \cdot R_T$, in cui $R_T = 6371$ km, mentre per determinare la velocità angolare della Luna, è noto che
\[\omega = \frac{2\pi}{T}\]
ovvero il rapporto tra l'angolo descritto dal moto e il tempo impiegato: anche in questo caso è noto che $T \cong 27.3$ giorni. Ovviamente, ora, con questi dati è possibile calcolare l'accelerazione centripeta della Luna, ossia:
\[a_c = 2.7 \times 10^{-3} \text{ m/s}^2\]
inoltre è noto che l'accelerazione gravitazionale terrestre è $g=9.8$ m/s$^2$ per cui eseguendo il rapporto si ottiene
\[\frac{g}{a_c} \cong 60^2\]
che è un risultato importante, in quanto fa capire come il modulo dell'accelerazione centripeta sia \textbf{inversamente proporzionale al quadrato della distanza} tra i due corpi che si attraggono:
\[F \propto \frac{1}{d^2}\]
Tale proprietà viene espressa tramite la seguente legge di gravitazione universale
\[\boxed{F = G \cdot \frac{m_1 \cdot m_2}{r^2}}\]
in cui $G$ prende il nome di \textbf{costante di gravitazione universale}, la quale venne determinata significativamente dopo Newton.

\vspace{2em}
\noindent
\textbf{Esempio}: Si considerino due punti materiali $m_1$ e $m_2$ ad una certa posizione:

\vspace{1em}
\begin{figure}[H]
  \centering
  \begin{tikzpicture}[scale=1]
    \draw (0,0) node[circ](start){} (-1,2) node[circ, above](m_1){} ++(0,0.4) node[]{$m_1$} (2,3) node[circ](m_2){} ++(0,0.4) node[]{$m_2$};
    \draw [-stealth] (start) -- (m_1) node[midway, below left]{$\vec r_1$};
    \draw [-stealth] (start) -- (m_2) node[midway, below right]{$\vec r_2$};
    \draw [dashed] (m_1) -- coordinate[midway](mid) (m_2);
    \draw [-stealth] (m_1) -- (mid) node[midway, above]{$\vec r_{1,2}$};
    \draw [-stealth] (m_2) ++(-0.5,0.5) -- ++(-0.75,-0.25) node[midway, above]{$\vec F_{1,2}$};
  \end{tikzpicture}
  \caption{Forza di attrazione tra due masse}
  \label{fig:forza_attrazione_gravitazionale}
\end{figure}

\vspace{1em}
\noindent
Allora si ha che, normalmente, il modulo della forza di attrazione gravitazinale dal corpo $1$ al corpo $2$ è
\[\vec F_{1,2} = G \cdot \frac{m_1 \cdot m_2}{\left \vert \vec r_1 - \vec r_2 \right \vert}\]
e volendo determinare il vettore forza di attrazione, semplicemente si può scrivere
\[\boxed{\vec F_{1,2} = -G \cdot \frac{m_1 \cdot m_2}{\left \vert \vec r_1 - \vec r_2 \right \vert} \cdot \hat{r}_{1,2}}\]
in cui cambiando il versore $\hat{r}_{1,2}$ si osserva come tale formula si adatti perfettamente alla terza legge di Newton: infatti il versore cambia segno, ma il modulo della forza rimane lo stesso.

\vspace{1em}
\noindent
\textbf{Osservazione}: Si osservi che, naturalmente, sulla Terra, fissando come origine il centro della Terra:
\[\vec{F}_t = -G \cdot \frac{m_t \cdot m}{r^2} \cdot \hat{r} = - \left(G \cdot \frac{m_2}{R_t^2}\right) \cdot m \cdot \hat{r}\]
in cui, ovviamente
\[g = \left(G \cdot \frac{m_t}{R_t^2}\right)\]
in cui si è fissato come raggio il raggio terrestre, dal momento che la Terra è un corpo sferico e, quindi, tutta la sua massa può essere considerata concentrata nel suo nucleo (un'assunzione che corrisponde alla somma del contributo attrattivo di tutte le infinitesime masse che costiuiscono la Terra).

\vspace{1em}
\subsection{L'esperimento di Cavendish}
Per misurare la forza di attrazione gravitazionale, Cavendish ha impiegato un \textbf{pendolo a torsione} su cui venivano fissate due masse, in corrispondenza delle loro estremità:

\begin{figure}[H]
  \centering
  \begin{tikzpicture}
    \draw (0,0) node[circ]{} -- ++(0,3) -- ++(3,0);
    \draw (3,2) -- (3,4);
    \foreach \i in {22,23,...,39} {
      \draw (3,\i / 10) -- (3.3,\i / 10-0.3);
    }
    \draw (-2,5.5) -- (2,5.5);
    \draw (0,5) -- ++(0,0.5);
    \foreach \i in {-19,-18,...,19} {
      \draw (\i / 10,5.5) -- (\i / 10+0.3,5.8);
    }
    \draw (0,5) node[circ]{} -- ++(1,-1) -- ++(0,-4.7) node[circle,fill=black,minimum width=10pt]{};
    \draw (0,5) node[circ]{} -- ++(-1,-1) -- ++(0,-4.7) node[circle,fill=black,minimum width=10pt]{};
    \draw (-1.5,-0.5) node[circ]{} -- (1.5,0.5) node[circ]{};
    \draw (-1.5,-0.5) ++(0,-0.4) node[]{$m_1$};
    \draw (1.5,0.5) ++(0,0.4) node[]{$m_2$};
  \end{tikzpicture}
  \caption{Esperimento di Cavendish}
  \label{fig:esperimento_cavendish}
\end{figure}

\noindent
Tali piccole masse oscillavano leggermente, a causa dell'attrazione con masse più considenti, in un moto continuativo, ma molto lento (in questo caso il periodo si aggirava intorno a $20$ minuti); l'esperimento doveva svolgersi esattamente sul piano, in modo tale da essere ortogonale alla forza di gravità della terra e ciò richiedeva molta precisione e condizioni sperimentali costanti (quali temperatura, pressioni, vento, etc.).\\
Grazie a ciò Cavendish è riuscito a determinare la \textbf{densità della terra}, che era ciò che gli interessava maggiormente: sapendo, infatti, la densità e il volume della Terra, si sarebbe stati in grado di conoscerne la massa e, sapendo il valore di $g$, si sarebbe, in seguito, stati in grado di isolare la costante gravitazionale $G$, che presenta il valore seguente:
\[\boxed{G = 6.67 \times 10^{-11} \frac{\text{m}^3}{\text{s}^2 \text{ kg}}}\]
A partire da tale risultato è possibile analizzare il moto e le caratteristiche di diversi pianeti.

\vspace{1em}
\subsection{Campo gravitazionale}
A partire dall'equazione seguente
\[\vec{F}_t = -G \cdot \frac{m_t}{R_t^2} \cdot \hat{r} \cdot m\]
si osserva come la forza sia proporzionale alla massa: conoscendo la massa del corpo è possibile conoscere la forza su tale corpo.\\
Per campo è da intendersi una valutazione della \textbf{forza per unità di massa}, come mostrato di seguito:
\[\boxed{\frac{\vec{F}_t}{m} = -G \cdot \frac{m_t}{R_t^2} \cdot \hat{r} = \vec g}\]

\colorlet{myred}{red!65!black}
\colorlet{mygreen}{green!60!black}
\colorlet{mydarkred}{red!50!black}
\colorlet{mydarkblue}{blue!40!black}
\tikzstyle{measure}=[{Latex[length=4,width=3]}-{Latex[length=4,width=3]},line width=0.4,mydarkblue]
\tikzstyle{ground}=[preaction={fill,top color=black!10,bottom color=black!5,shading angle=20},
                    fill,pattern=north east lines,draw=none,minimum width=0.3,minimum height=0.6]
\tikzstyle{mass}=[line width=0.6,red!30!black,fill=red!40!black!10,rounded corners=1,
                  top color=red!40!black!20,bottom color=red!40!black!10,shading angle=20]
\tikzstyle{rope}=[brown!70!black,line width=1.2,line cap=round] %very thick
\tikzstyle{force}=[->,myred,thick,line cap=round]
\tikzstyle{unit}=[->,mygreen,thick,line cap=round]
\tikzset{fieldlines/.style={mydarkred,decoration={markings,mark=at position #1 with {\arrow{latex}}},
                            postaction={decorate},line width=0.7},
         fieldlines/.default=0.55}
\newcommand{\vbF}{\vb{F}}
\begin{figure}[H]
  \centering
  \begin{tikzpicture}[scale=2]
    \def\W{2.4}   % ground width
    \def\D{0.2}   % ground depth
    \def\H{1.5}   % height
    \def\E{0.35}  % Earth's radius
    \def\R{1.3}   % field line max. radius
    \def\N{10}    % number of field lines
    \def\rang{45} % angle or r axis

    % FIELD LINES
    \foreach \i [evaluate={\ang=\i*360/\N;}] in {1,...,\N}{
      \draw[fieldlines=0.6] (\ang:\R) -- (\ang:\E); % field lines
    }
    \node[mydarkred,right] at (-0.6,-1.6) {$\vec g = -G \cdot \dfrac{m_t}{R_t^2} \cdot \hat{r}$};
    \draw[->] (\rang:0.44*\R) --++ (\rang:0.4*\R) node[anchor=\rang+170,inner sep=1] {$r$};

    % EARTH
    \fill[blue!70!black!70] (0,0) circle(\E);
    \begin{scope}[rotate=-11]
      \clip (0,0) circle(\E);
      \fill[white] (0,\E) ellipse({0.6*\E} and {0.15*\E});
      \fill[white] (0,-\E) ellipse({0.8*\E} and {0.08*\E});
      \fill[green!70!black!60,rotate=-30] (160:1.1*\E) ellipse({0.2*\E} and {0.8*\E});
      \fill[green!70!black!60,rotate=40] (-10:1.14*\E) ellipse({0.2*\E} and {0.9*\E});
      \fill[green!60!black!60,very thick,rotate=-20] % Australia
        (230:0.86*\E) ellipse({0.25*\E} and {0.18*\E});
    \end{scope}
    \draw[line width=0.3] (0,0) circle(\E);
  \end{tikzpicture}
  \caption{Campo gravitazionale terrestre}
  \label{fig:campo_gravitazionale_terrestre}
\end{figure}

\noindent
che è esattamente lo stesso concetto di di campo elettrico, ovvero la forza per unità di carica (nel caso di un campo gravitazionale, la carica è proprio la massa): sapendo la carica, e qui la massa, è possibile conoscere la forza associata, in quanto è noto il campo, ossia la forza per unità di carica/massa.

\vspace{1em}
\noindent
\textbf{Osservazione}: Si osservi che il \textbf{campo gravitazionale è additivo}: infatti, date due masse identiche $m_1$ e $m_2$, poste vicine l'una all'altra, le linee di campo dell'una si fondono con quelle dell'altra, dando vita a delle deformazioni delle linee di campo.\\
Si osservi che, ovviamente, le linee di campo sono una convenzione grafica atta a rappresentare qualitativamente il fenomeno attrattivo: in ogni punto dello spazio vi sono dei vettori che dovrebbero essere rappresentati integralmente, ma per ovvie ragioni di comprensione, si preferisce la più elegante soluzione delle linee di campo.\\
Di fatto si ha che il campo gravitazionale risultante è
\[\vec{g} = \vec{g}_1 + \vec{g}_2\]

\vspace{1em}
\subsection{Massa gravitazionale e inerziale}
Dalla $2^a$ legge della dinamica
\[\vec{F} = m_I \cdot \vec{a}\]
si evince come la massa $m_I$ sia da considerarsi com l'\textbf{inerzia}, ovvero la \textbf{resistenza alla variazione di moto} (e quindi di velocità).\\
Mentre dalla legge di gravitazione universale
\[\vec{F} = - G \cdot \frac{m_1 \cdot m_G}{r^2} \cdot \vec{r}\]
si ha che $m_G$ è una sorta di \quotes{carica gravitazionale}; tuttavia $m_I$ e $m_G$ non sono da reputarsi il medesimo concetto dal punto di vista teorico: si consideri, per esempio, il caso di due particelle, una poco massiva, ma molto carica e l'altra più massiva, che presenta una carica elettrica pressoché nulla, tale che la forza che subiscono sia completamente slegata dalla loro massa; in generale, infatti, non vi è correlazione fra le due masse, per cui, nel caso di una caduta libera, si dovrebbe avere che
\[m_I \cdot \vec{a} = - G \cdot \frac{m_1 \cdot m_G}{r^2} \cdot \vec{r}\]
in cui, isolando l'accelerazione si ottiene
\[a_y = -G \cdot \frac{m_1}{r^2} \cdot \left(\frac{m_G}{m_I}\right)\]
in cui non è possibile, in linea teoria, confondere $m_I$ con $m_G$. Tuttavia, dal punto di vista sperimentale, sono state eseguite delle misurazioni con un altissimo livello di precisione che hanno confermato come
\[\frac{m_G}{m_I} = 1\]
per cui le due masse sono da reputarsi identiche. Anche nella teoria della relatività generale tale risultato trova significato: in essa, infatti, non si considera un'attrazione gravitazionale, ma solamente una deformazione spazio-temporale, per cui tutto il sistema è non-inerziale.

\vspace{1em}
\subsection{Corpi estesi}
Il campo gravitazionale è una quantità additiva, per cui per conoscere la forza di attrazione di un corpo non puntiforme, è sufficiente scomporlo in punti e sommare il contributo di ciascuno per pervenire al risultato cercato.\\
In altri termini si ha che
\[\boxed{\vec{g} = \sum_i \vec{g}_i = \sum_i - G \cdot \frac{m_i}{r_i^2} \cdot \hat{r} = -G \cdot \int \frac{dm}{r^2} \cdot \hat{r}}\]
in cui si deve integrare all'infinito il contributo di ciascuna punto.

\vspace{1em}
\noindent
\textbf{Esempio}: Se si considera una \textbf{distribuzione con simmetria sferica} (ovvero una sfera in cui la densità non deve necessariamente essere uniforme), allora tutta la massa è da reputarsi concentrata al centro della sfera, in quanto tale interpretazione è perfettamente equivalente alla somma infinita di tutti i contributi.

\vspace{1em}
\noindent
\textbf{Osservazione}: Si osservi che il numero di giri compiuti nell'unità di tempo è, ovviamente, data da
\[\frac{1}{T}\]
Inoltre, si ha che la forza centripeta descrive la natura della forza, non il tipo di forza, per cui la forza centripeta può essere causata da qualsiasi tipo di forza.\\
Si osservi che un punto materiale si muove con velocità di modulo costante lungo un'ellisse, allora, l'accelerazione punta sempre verso il centro dell'elisse, in quanto il moto considerato è lo stesso di un moto circolare uniforme, ma visto da un sistema di riferimento a velocità costante (per cui l'accelerazione è nulla).\\
Naturalmente, i corpi liberi su una vettura in decelerazione cadono in avanti perché mantengono la stessa velocità che avevano prima della decelerazione, mentre la pseudo-forza di Coriolis ha effetto solamente su grandi distanza, come si può evincere dalla seguente formula:
\[\vec F_c = -2 m  \vec \omega \times \vec v\]
ricordando che
\[\vec \omega = \frac{\vec v}{R}\]

\newpage
\noindent
\begin{center}
  21 Marzo 2022
\end{center}
La forza di attrazione tra due corpi è direttamente proporzionale alla massa dei due corpi e inversamente proporzionale alla distanza fra i due (e tale concetto si trasla anche alla teoria della carica, per la \textbf{Forza di Coulomb}, in cui al posto della massa si deve considerare la carica di una particella e in questo caso la carica può essere positiva o negativa, ottenendo forze attrattive o repulsive). La formula per il calcolo della forza di attrazione gravitazionale presenta un $-$ in quanto si tratta di una forza attrattiva, ovviamente.\\
Il \textbf{campo gravitazionale} è definita come la forza per unità di massa: è, quindi, una vera e propria forza che si applica indistintamente ad ogni corpo, indipendentemente dalla massa di ogni corpo.\\
La differenza tra massa gravitazionale e massa inserziale è puramente concettuale, anche se nella pratica tali masse coincidono; infine, per quanto concerne i corpi estesi, al fine di considerare la forza attrattiva totale, è sufficiente sommare i contributi di tutte le masse; nel caso di un corpo con simmetria sferica, invece, è sufficiente considerare che la massa sia idealmente concentrata nel centro della sfera stessa, un'agevolazione che permette anche di semplificare i calcoli da svolgere.

\vspace{1em}
\subsection{Leggi di Keplero}
Kelpero, impiegando dei dati molto precisi, è riuscito a formulare delle valutazioni estremamente importanti sull'orbita dei pianeti: egli, infatti, ha scoperto che l'orbita dei pianeti è ellettica, in cui il sole si trova in uno dei due fuochi.\\
Non solo, Keplero, osservando il moto degli astri, è riuscito anche a formulare due leggi che spiegano la cinetica del moto dei pianeti, attraverso calcoli molto complessi e basandosi su risultati sperimentali. Di seguito si espongono le tre fondamentali leggi di Keplero:

% Tabella per le definizione di concetti, etc...
\vspace{1em}
\rowcolors{1}{black!5}{black!5}
\setlength{\tabcolsep}{14pt}
\renewcommand{\arraystretch}{2}
\noindent
\begin{tabularx}{\textwidth}{@{}|P|@{}}
    \hline
    {\textbf{LEGGI DI KEPLERO}}\\
    \parbox{\linewidth}{Le leggi di Keplero sono:
    \begin{enumerate}
      \item \emph{Tutti i pianeti si muovono lungo orbite ellittiche di cui il sole occupa uno dei fuochi};
      \item \emph{La congiungente di un pianeta con il Sole spazza aree uguali in intervalli di tempo uguali} (ciò significa che i pianeti, quando sono più lontani dal sole, ruotano più lentamente rispetto a quando si trovano nelle sue prossimità);
      \item \emph{Il quadrato del periodo di un qualunque pianeta è proporzionale al cubo della distanza media del pianeta dal Sole}.
    \end{enumerate}
    \vspace{1mm}}\\
    \hline
\end{tabularx}

\vspace{2em}
\noindent
La terza legge di Keplero può essere facilmente giustificata alla luce della legge di gravitazione universale di Newton.\\
Se, infatti, si considera un'orbita circolare (anziché ellittica) è possibile confondere la forza di attrazione gravitazionale $\vec F_a$ con la forza centripeta $\vec F_c$, per cui si ottiene
\[\vec F_c = \vec F_a \longrightarrow m_P \cdot \frac{v^2}{d} = G \cdot \frac{m_P \cdot m_S}{d^2}\]
o anche
\[\vec F_c = \vec F_a \longrightarrow m_P \cdot \omega^2 \cdot d = G \cdot \frac{m_P \cdot m_S}{d^2}\]
in cui, ovviamente, $m_P$ è la massa del pianeta, mentre $m_S$ è la massa del sole e $d$ è la distanza tra il pianeta e il sole. Da questo si ottiene che:
\[\frac{1}{\omega^2} = \frac{1}{G \cdot m_S} \cdot d^3\]
Ma siccome $\omega$ è la velocità angolare, ovvero il rapporto fra un giro completo e il periodo di rotazione, è facile capire che:
\[\frac{T^2}{(2\pi)^2} = \frac{1}{G \cdot m_S} \cdot d^3\]
Allora isolando il quadrato del periodo si ottiene:
\[\boxed{T^2 = \frac{(2 \pi)^2}{G \cdot m_S} \cdot d^3}\]

\vspace{1em}
\noindent
\textbf{Esercizio}: Un'orbita geostazionaria è l'orbita di un satellite attorno alla Terra che presenta un periodo di rotazione identico a quello della terra, per cui si mantiene sempre sopra lo stesso punto sopra la terra: secondo la terza di legge di Keplero, esiste una sola altezza tale per cui l'orbita di un satellite possa essere geostazionaria.\\
Si calcoli, allora, la velocità alla quale deve ruotare un satellite (idealmente sulla superficie della Terra) affinché la sua orbita sia geostazionaria. È facile capire che tale velocità si calcola come segue
\[v = \frac{\Delta s}{\Delta t} = \frac{2 \pi r}{\sqrt{\dfrac{(2 \pi)^2}{g \cdot r^2} \cdot r^3}} = r \cdot \sqrt{\frac{g}{r}} = \sqrt{g \cdot r}\]
ricordando che
\[g=\frac{G \cdot m_t}{R_t^2} \longrightarrow G \cdot m_t = g \cdot R_t^2\]
Ecco che allora la velocità cercata è
\[\boxed{v = \sqrt{g \cdot r}}\]
Si possono, naturalmente, anche considerare le due uguaglianze seguenti:
\[\boxed{\frac{v^2}{R_t} = g} \hspace{1em} \text{e} \hspace{1em} \boxed{\omega^2 \cdot R_t = g}\]
Ovviamente, per determinare il periodo di rotazione di un satellite attorno alla terra è sufficiente impiegare la terza legge di Kelpero, per cui
\[\boxed{T=\sqrt{\frac{(2\pi)^2}{g} \cdot d}}\]

\vspace{2em}
\noindent
\textbf{Osservazione}: Si osservi che il periodo dell'orbita per diversi pianeti risula essere molto simile; in particolare è noto che, per un qualsiasi pianeta, dalla terza legge di Keplero si ottiene che
\[d^3 = \frac{G \cdot m_S}{(2\pi)^2} \cdot T^2\]
ma più in generale, per un qualsiasi pianeta si ottiene che
\[T = 2\pi \cdot \sqrt{\frac{R_P^3}{G \cdot m_P}}\]
in cui, sotto la radice, compare il rapporto
\[\frac{R_P^3}{m_P}\]
che è esattamente il reciproco della densità del pianeta, essendo il volume di una sfera
\[V_S = \frac{4}{3} \cdot \pi \cdot R^3\]
da cui si evince che
\[T = 2\pi \cdot \sqrt{\frac{3}{4\pi} \cdot \frac{1}{\rho}}\]
in cui, naturalmente
\[\rho = \frac{m}{V}\]
Pertanto si ottiene che tutti i pianeti che sono stati considerati presentano lo stesso periodo di orbita in quanto hanno tutti la \textbf{stessa densità} (in quanto formati da ghiaccio, rocce, acqua, etc.).

\vspace{1em}
\noindent
\textbf{Osservazione}: Se fosse possibile tagliare la terra in due parti uguali e si lasciasse cadere una massa in tale foro, allora la massa accelerebbe fino a raggiungere il centro della terra, dove la forza di attrazione si annullerebbe; pertanto, la massa continuerebbe a scendere venenedo progressivamente decelerata fino a raggiungere velocità nulla al culmine della sua corsa dall'altra parte del pianeta, venendo nuovamente attratta al centro e così via, fino a quando non si viene a creare un moto oscillatorio di periodo esattamente pari al periodo di orbita attorno alla Terra.

\vspace{1em}
\noindent
\textbf{Esercizio}: Si consideri una macchina che cerca di accelerare su diverse superfici e avente diverse configurazioni di trazione.\\
Allora se il coefficiente di attrito statico è $\mu_s=1.0$ e la macchina presenta solo due ruote trainanti, l'accelerazione della vettura deve essere
\[a \leq \frac{\mu_s}{2} \cdot g = 4.9 \text{ m/s}^2\]
e così via, variando il coefficiente di attrito statico e la distribuzione del peso.

\vspace{1em}
\noindent
\textbf{Esercizio}: Si consideri un'ascensore sospeso da un cavo, avente massa $1.7 \times 10^3$ kg.\\
Allora, richiamando la seconda legge della dinamica, si può scrivere:
\[\vec F_T + F_t = m \cdot \vec a\]
e sostiutendo a tali vettori i moduli, si ottiene
\[F_T = m \cdot (a + g)\]
Taluna è la formula che si può adattare a qualsiasi scenario al fine di calcolare la forza di tensione sul cavo: se essso si muove di moto rettilineo uniforme, se accelera o decelera.

\newpage
\noindent
\begin{center}
  22 Marzo 2022
\end{center}
\section{Energia}
L'energia è la grandezza fisica che misura la capacità di un corpo o di un sistema fisico di compiere lavoro, a prescindere dal fatto che tale lavoro sia o possa essere effettivamente svolto.\\
Di seguito si espongono diverse forme di energia e ne viene analizzata nel dettaglio il significato fisico.

\vspace{1em}
\subsection{Energia cinetica}
L'\textbf{energia cinetica} (originariamente chiamata \emph{vis viva}) venne teorizzata da Leibniz e da Decartes: il primo asseriva che l'energia cinetica era direttamente proporzionale alla massa e al quadrato della velocità (ovvero $m v^2$), mentre il secondo considerava soltanto la massa e la velocità e non il suo quadrato (ovvero $m v$, che prende il nome di \emph{quantità di moto}).\\
Grazie a Thomas Young, il concetto fisico teorizzato da Leibniz prese il nome di \textbf{energia} ed infine Gustave Gasparre Coriolis gli attribuì il nome di \textbf{energia cinetica}, definita come segue:

% Tabella per le definizione di concetti, etc...
\vspace{1em}
\rowcolors{1}{black!5}{black!5}
\setlength{\tabcolsep}{14pt}
\renewcommand{\arraystretch}{2}
\noindent
\begin{tabularx}{\textwidth}{@{}|P|@{}}
    \hline
    {\textbf{ENERGIA CINETICA}}\\
    \parbox{\linewidth}{L'\textbf{energia cinetica} viene calcolata come segue:
    \[\boxed{K=\frac{1}{2}mv^2}\]
    per quanto riguarda un solo punto materiale di massa $m$. Se, invece, si considera un agglomerato di punti materiali, allora, essendo l'\textbf{energia additiva}, si ottiene
    \[\boxed{K=\frac{1}{2} \sum_{i} m_i \cdot v_i^2}\]
    Naturalmente, l'energia cinetica è uno scalare (non può avere una direzione), in cui $v^2=\vert \vec v \vert^2 = \vec v \cdot \vec v $.\vspace{3mm}}\\
    \hline
\end{tabularx}

\vspace{2em}
\noindent
\textbf{Esempio}: Si consideri una macchina di massa $m=1000$ kg e avente una velocità di $v_1=50$ km/h. Allora l'energia cinetica si calcola come segue:
\[K_1=\frac{1}{2}m v_1^2 = \frac{1}{2} \cdot 1000 \text{ kg} \cdot \left( \frac{50}{3.6} \text{ m/s} \right)^2 = 96.5 \text{ kJ}\]
in cui
\[\boxed{1 \text{ J} = \frac{\text{kg m}^2}{\text{s}^2}}\]
Mentre se la velocità della vettura è $v_2=60$ km/h allora l'energia cinetica della macchina diventa:
\[K_2=\frac{1}{2}m v_2^2 = \frac{1}{2} \cdot 1000 \text{ kg} \cdot \left( \frac{60}{3.6} \text{ m/s} \right)^2 = 139 \text{ kJ}\]
che significa che l'aumento di velocità di soli $10$ km/h comporta un aumento del $50\%$ dell'energia e, quindi, della dissipazione di energia in calore in caso di frenata (aumento dello spazio di frenata).

\newpage
\noindent
\subsection{Lavoro}
Di seguito si espone la definizione di \textbf{lavoro}, basandosi sulla Figura \ref{fig:lavoro_compiuto_molla_non_parallela_spostamento}:

\begin{figure}[H]
  \colorlet{myred}{red!65!black}
  \colorlet{mydarkblue}{blue!30!black}
  \colorlet{xcol}{blue!70!black}
  \colorlet{vcol}{green!70!black}
  \colorlet{acol}{red!50!blue!80!black!80}
  \tikzstyle{ground}=[preaction={fill,top color=black!10,bottom color=black!5,shading angle=20},
                      fill,pattern=north east lines,draw=none,minimum width=0.3,minimum height=0.6]
  \tikzstyle{mass}=[line width=0.6,red!30!black,fill=red!40!black!10,rounded corners=1,
                    top color=red!40!black!20,bottom color=red!40!black!10,shading angle=20]
  \tikzstyle{vector}=[->,very thick,xcol,line cap=round]
  \tikzstyle{force}=[->,myred,thick,line cap=round]
  \tikzstyle{Fproj}=[force,myred!40]
  \tikzstyle{mydashed}=[dash pattern=on 2pt off 2pt]
  \tikzstyle{smallarrow}=[{Latex[length=2,width=2]}-{Latex[length=2,width=2]}]
  \def\tick#1#2{\draw[thick] (#1) ++ (#2:0.1) --++ (#2-180:0.2)}

  \centering
  \begin{tikzpicture}[scale=2]
    \def\W{2.7}  % ground width
    \def\D{0.2}  % ground depth
    \def\h{0.8}  % mass height
    \def\w{1.0}  % mass width
    \def\F{1.1}  % force magnitude
    \def\ang{30} % angle force
    \coordinate (F0) at (0.4*\w,0.85*\h);
    \coordinate (Fx) at ($(F0)+({\F*cos(\ang)},0)$);
    \coordinate (F)  at ($(F0)+(\ang:\F)$);
    \draw[ground] (-0.3*\W,0) rectangle++ (\W,-\D);
    \draw (-0.3*\W,0) --++ (\W,0);
    \draw[mass] (-\w/2,0) rectangle++ (\w,\h) node[midway] {$m$};
    \draw[force,xcol] (\w/2,0.15*\h) --++ (0.4*\W,0) node[midway,above=0] {$\vb*{\Delta r}$};
    \draw[dashed,myred!80!black!60] (Fx) -- (F);
    \draw[Fproj] (F0) -- (Fx) node[above=1,right=0] {$F\cos\theta$}; %\vu{x}
    \draw[force] (F0) -- (F)  node[above=1,right=-1] {$\vbF$};
    \draw pic["$\theta$",draw=black,angle radius=14,angle eccentricity=1.4] {angle=Fx--F0--F};
  \end{tikzpicture}
  \caption{Lavoro compiuto da una forza non parallela allo spostamento}
  \label{fig:lavoro_compiuto_molla_non_parallela_spostamento}
\end{figure}

% Tabella per le definizione di concetti, etc...
\vspace{1em}
\rowcolors{1}{black!5}{black!5}
\setlength{\tabcolsep}{14pt}
\renewcommand{\arraystretch}{2}
\noindent
\begin{tabularx}{\textwidth}{@{}|P|@{}}
    \hline
    {\textbf{LAVORO}}\\
    \parbox{\linewidth}{Quando una forza $\vec F$ viene applicata su un corpo e produce uno spostamento $\Delta \vec r$, allora si ha che la forza ha compiuto un lavoro, calcolato come
    \[\boxed{W = \vec F \cdot \Delta \vec r \cdot \cos(\theta)}\]
    che significa che
    \begin{itemize}
      \item Se $\vec F$ è parallelo a $\Delta \vec r$, allora il lavoro è massimo.
      \item Se $\vec F$ è perpendicolare a $\Delta \vec r$, allora il lavoro è nullo (come nel moto circolare uniforme).
      \item Se $\vec F$ presenta verso opposto rispetto a $\Delta \vec r$, allora il lavoro è negativo (si ha sottrazione di energia al sistema, come per l'attrito).
    \end{itemize}
    Naturalmente, nel caso in cui vi siano più forze agenti sul sistema, ciascuna delle quali produce uno spostamento, si ottiene che
    \[W_{\text{tot}} = \vec F_1 \cdot \Delta \vec r_1 + \vec F_2 \cdot \Delta \vec r_2 + ... = \sum_{i} F_i \cdot \Delta \vec r_i\]
    Infine, se il corpo non ruota e non si deforma, tutti gli spostamenti sono uguali, ovvero $\Delta \vec r_1 = \Delta \vec r_2 = ... + \Delta \vec r_n$, e quindi il lavoro totale sarà dato solo dal prodotto scalare tra la somma delle forze e il singolo spostamento:
    \[W_{\text{tot}} = \sum_{i} F_i \cdot \Delta \vec r\]
    \vspace{-1mm}}\\
    \hline
\end{tabularx}

\vspace{2em}
\noindent
\textbf{Osservazione}: Si osservi che un astronauta in orbita attorno alla terra \textbf{ha peso}, in quanto è grazie a tale forza che risce a mantenere la propria orbita, altrimenti continuerebbe il proprio moto in linea retta.\\
È facile, inoltre, determinare il raggio di orbita geostazionaria (partendo dal centro della Terra), calcolabile come:
\[d=\frac{T^2 \cdot g \cdot R_t^2}{(2 \pi)^2} = 42 221 \text{ km}\]

\vspace{1em}
\subsection{Teorema lavoro-energia cinetica}
Si osservi che dalla cinetica, per quanto riguarda il moto uniformemente accelerato, è nota la formula seguente:
\[v^2 - v_0^2 = 2a \cdot (x-x_0)\]
Ma se ambo i membri si moltiplicano per $\frac{1}{2}m$ si ottiene
\[\frac{1}{2} m v^2 - \frac{1}{2} m v_0^2 = ma \cdot (x-x_0)\]
che corrisponde a
\[K - K_0 = F \cdot (x-x_0) \longrightarrow \Delta K = F \cdot \Delta x \longrightarrow \Delta K = W\]
ovvero si è ottenuto il \textbf{lavoro compiuto sul sistema da una forza esterna (trasferimento di energia verso (o da) il sistema)}. In altre parole, quando una forza viene applicata su un sistema, si assiste ad un trasferimento di energia (come quando una forza applicata ad un corpo ne aumenta la velocità, la forza ha compiuto un lavoro sul sistema che si è tramutato in aumento di energia cinetica).\\
Più in generale, è possibile affermare dalla seconda legge della dinamica applicata ad un punto materiale di massa $m$ che
\[\sum \vec F = m \cdot \vec a\]
considerando, ora, uno \textbf{spostamento infinitesimale} $d \vec r$ (che è da interpretarsi come $\Delta \vec r \rightarrow 0$), si può calcolare il \textbf{lavoro infinitesimale} compiuto dalla forza sul punto materiale per determinare tale spostamento:
\[dW = \left(\sum \vec F\right) \cdot d \vec r = m \vec a \cdot d \vec r\]
ma dalla cinetica è anche noto che
\[\vec v = \frac{d \vec r}{d t} \longrightarrow d \vec r = \vec v \cdot dt\]
che significa che in intervallo di tempo infinitesimale $dt$, un punto materiale alla velocità $\vec v$ compie uno spostamento infinitesimale $d \vec r$. Alla luce di tale evidenza, la formula ottenuta precedentemente diviene:
\[m \vec a \cdot d \vec r = m \vec a \cdot \vec v dt\]
Ma calcolando la derivata della velocità al quadrato si ottiene
\[\frac{d}{dt} \vec v^2 = \frac{d}{dt} \left(\vec v \cdot \vec v\right) = \vec a \cdot \vec v + \vec v \cdot \vec a = 2 \cdot \vec a \cdot \vec v\]
Alla luce di tale risultato si può quindi scrivere
\[m \vec a \cdot \vec v dt = \frac{1}{2}m \cdot \frac{d}{dt} (\vec v)^2\]
ovvero
\[\boxed{dW = \frac{1}{2}m \cdot \frac{d}{dt} (\vec v)^2}\]
Avendo ottenuto la formula per il calcolo del lavoro infinitesimale $dW$, è possibile eseguire una somma infinita di tali contributi per determinare il lavoro compiuto per effettuare uno spostamento maggiore, come mostrato di seguito:
\[W_{\text{tot}} = \int_i^f dW = \int_{t_i}^{t_f} \left(\sum \vec F\right) \cdot d \vec{r} = \int_{t_i}^{t_f} \frac{1}{2}m \cdot \frac{d}{dt} (\vec v)^2 \cdot dt = \frac{1}{2}m \cdot \left(v_f^2 - v_i^2\right)\]
per cui la formula finale ottenuta è proprio quanto ci si aspettava:
\[\boxed{W_{\text{tot}} = \frac{1}{2}m \cdot \left(v_f^2 - v_i^2\right)}\]

\vspace{1em}
\noindent
\textbf{Osservazione}: Il \textbf{teorema lavoro-energia cinetica} è una \textbf{conseguenza diretta} della seconda legge di Newton
\[F = m \vec a\]
e si applica ai punti materiali (ma non ad un sistema più complesso).\\
Si osservi che tale risultato non è equivalente al teorema di \textbf{conservazione dell'energia}, anche se fornisce risultati uguali in importanti casi.\\
È significativo osservare che il lavoro totale $W_{\text{tot}}$ così calcolato non corrisponde al \quotes{lavoro} in senso termodinamico. In questo caso si ha che
\[W_{\text{tot}} = \Delta K\]
da cui si può capire come il lavoro di una \textbf{forza costante} si possa calcolare come segue:
\[\boxed{W = \int_i^f \vec F \cdot d \vec r = \vec F \cdot (\vec r_f - \vec r_i) = \vec F \cdot \vec l}\]
in cui $\vec l$ indica lo \textbf{spostamento totale}, il quale non dipende assolutamente dal percorso compiuto.

\vspace{1em}
\noindent
\textbf{Esercizio}: Si consideri una vettura di massa $m=1000$ kg che viagga ad una velocità $v_0=50$ km/h. Si calcoli la distanza di arresto, considerando che la frenatura è resa possibile dalla forza di attrito cinetico $\vec F_k$.\\
Naturalmente è ovvio che $F_N=F_t=mg$, mentre $F_k=\mu_k F_N=\mu_k mg$. Allora il lavoro compiuto dalla forza di attrito è
\[W = \int_i^f \vec F_k \cdot d \vec r\]
ed essendo la forza di attrito di verso opposto al moto, il lavoro $W$ avrà segno negativo.\\
Pertanto si ottiene
\[W = \int_i^f \vec F_k \cdot d \vec r = \int_0^d -F_k \cdot dx = -F_k \cdot d = -\mu m g d\]
E richiamando il teorema lavoro-energia cinetica, si ottiene che
\[W=\Delta K \longrightarrow -\mu_k m g d = K_f-K_i=-\frac{1}{2}mv_i^2\longrightarrow d=\frac{v_i^2}{2\mu_k g}\]
e com'era da aspettarsi, lo spazio di frenata aumenta con il quadrato della velocità.

\vspace{1em}
\noindent
\textbf{Esercizio}: Si consideri un proiettile che viene sparato in aria, in direzione verticale, con velocità $\vec v_i$ sufficiente per raggiungere l'altezza di $h$ in cui l'accelerazione gravitazionale non può essere considerata costante.\\
Naturalmente, dalla legge di gravitazione universale si ottiene che
\[\vec F = - G \cdot \frac{m_t \cdot m}{(R_t + h)^2} \cdot \hat{j}\]
Allora per il calcolo del lavoro si può procedere come segue:
\[W=\int \vec F \cdot d \vec r = \int -G \cdot \frac{m_t \cdot m}{(R_t + h)^2} \cdot dh\]
Naturalmente, in questo caso, la forza non è costante, per cui bisogna calcolare l'integrale, posto $h'=R_t+h$
\[W=\int \vec F \cdot d \vec r = \int -G \cdot \frac{m_t \cdot m}{(R_t + h)^2} \cdot dh = \int_{R_t}^{R_t+h} - \frac{G m_t m}{{h'}^2} \cdot dh' = G m_t m \cdot \left(\frac{1}{R_t+h}-\frac{1}{R_t}\right)\]
in cui si ottiene che
\[\Delta K = W_{\text{tot}} \longrightarrow 0 - \frac{1}{2}m \cdot v_i^2 = G m_t m \cdot \left(\frac{1}{R_t+h}-\frac{1}{R_t}\right)\]
in cui, isolando $h$ si ottiene
\[h = \frac{1}{2} \cdot \frac{R_t^2 \cdot v_i^2}{G m_t} \cdot \left(1 - \frac{1}{2} \cdot \frac{R_t \cdot v_i^2}{G m_t} \right)^{-1} \longrightarrow h = \frac{1}{2} \cdot \frac{v_i^2}{g} \cdot \left(1 - \frac{1}{2} \cdot \frac{v_i^2}{g R_t} \right)^{-1}\]
in cui appare evidente come l'altezza dipenda dal quadrato della velocità; inoltre se il secondo termine si annulla, con una velocità sufficiente, essendo valutato il regiproco, l'altezza diviene idealmente infinita e quindi il proiettile esce dall'orbita terrestre.

\newpage
\noindent
\begin{center}
  23 Marzo 2022
\end{center}
\subsection{Lavoro compiuto da una forza variabile}
Il lavoro compiuto da una forza variabile si ha quando la forza non è costante, ma dipende dalla distanza: in altre parole, sussiste una \textbf{relazione lineare tra forza e distanza}:


\vspace{2em}
\noindent
\rowcolors{1}{white}{white}
\begin{tabularx}{\textwidth}{P}
  {
      \centering
      \begin{tikzpicture}
        \begin{axis}[
          grid=both,
          axis lines = middle,
          xlabel = \(x\),
          ylabel = {\(F_x\)},
          legend pos=outer north east,
          ymajorgrids=true,
          xmajorgrids=true,
          grid style=dashed,
        ]
      \addplot[
        domain=-5:5,
        samples=100,
        color=orange,
      ]
      {-x};
      \end{axis}
      \end{tikzpicture}
    }
\end{tabularx}

\noindent
Per cui si perviene al risultato seguente:
\[\boxed{F_x=-kx}\]
la quale prende il nome di \textbf{legge di Hooke}, in cui $k$ è la costante di proporzionalità. Tipicamente, tale legge viene impiegata per descrivere la \textbf{forza di una molla}: quando lo spostamento è positivo, la forza è negativa (si chiama, infatti, \textbf{forza di richiamo}), e viceversa.

\begin{figure}[H]
  \centering
  \colorlet{xcol}{blue!70!black}
  \colorlet{darkblue}{blue!40!black}
  \colorlet{myred}{red!65!black}
  \tikzstyle{mydashed}=[xcol,dashed,line width=0.25,dash pattern=on 2.2pt off 2.2pt]
  \tikzstyle{axis}=[->,thick] %line width=0.6
  \tikzstyle{ell}=[{Latex[length=3.3,width=2.2]}-{Latex[length=3.3,width=2.2]},line width=0.3]
  \tikzstyle{dx}=[-{Latex[length=3.3,width=2.2]},darkblue,line width=0.3]
  \tikzstyle{ground}=[preaction={fill,top color=black!10,bottom color=black!5,shading angle=20},
                      fill,pattern=north east lines,draw=none,minimum width=0.3,minimum height=0.6]
  \tikzstyle{mass}=[line width=0.6,red!30!black,fill=red!40!black!10,rounded corners=1,
                    top color=red!40!black!20,bottom color=red!40!black!10,shading angle=20]
  \tikzstyle{spring}=[line width=0.8,blue!7!black!80,snake=coil,segment amplitude=5,segment length=5,line cap=round]
  \tikzset{>=latex} % for LaTeX arrow head
  \tikzstyle{force}=[->,myred,very thick,line cap=round]
  \def\tick#1#2{\draw[thick] (#1)++(#2:0.1) --++ (#2-180:0.2)}

  \begin{tikzpicture}[scale=1.5]
    \def\H{1.1} % wall height
    \def\T{0.3} % wall thickness
    \def\W{3.9} % ground length
    \def\D{0.2} % ground depth
    \def\h{0.7} % mass height
    \def\w{0.8} % mass width
    \def\x{2.0} % mass x position
    \def\dx{0.9} % extension
    \def\y{1.22*\H} % x axis y position
    \def\F{0.8} % force

    % AXIS
    \draw[mydashed] (\x,0) --++ (0,\y) (\x-\dx,0) --++ (0,1.1*\y);
    \draw[axis] (\x-0.4*\W,\y) -- (\x+0.4*\W,\y) node[right] {$x$};
    \tick{\x,\y}{-90} node[scale=0.8,above=0] {$0$};
    \draw[ell] (0,1.3*\h) --++ (\x,0) node[pos=0.4,fill=white,inner sep=0] {$\ell_0$};
    \draw[dx] (\x,1.6*\h) --++ (-\dx,0)
      node[pos=0.45,fill=white,inner sep=0,scale=0.9] {$x$};

    % SPRING & MASS
    \draw[spring,segment length=2.9] (0,\h/2) --++ (\x-\dx,0);
    \draw[ground] (0,0) |-++ (-\T,\H) |-++ (\T+\W,-\H-\D) -- (\W,0) -- cycle;
    \draw (0,\H) -- (0,0) -- (\W,0);
    \draw[mass] (\x-\dx,0) rectangle++ (\w,\h) node[midway] {$m$};
    \draw[force] (\x-\dx+0.8*\w,0.8*\h) --++ (\F,0) node[below=0,right=0] {$\vb{F}$};

  \end{tikzpicture}
  \caption{Fisica di una molla}
  \label{fig:fisica_molla}
\end{figure}

\noindent
Volendo calcolare il lavoro compiuto da tale forza, è sufficiente considerare l'integrale della forza risultante moltiplicata per lo spostamento, come mostrato di seguito:
\[\boxed{W=\int_{x_i}^{x_f}-k x \cdot dx = - \frac{1}{2}k \cdot (x_f^2-x_i^2)}\]
che costiuisce un lavoro che è \textbf{indipendente dal percorso compiuto}, ma dipende solamente dalla posizione iniziale e finale.

\newpage
\noindent
\textbf{Esempio $\boldsymbol{1}$}: Si consideri un blocco sospeso da una molla, in cui la costante di elasticità della molla è $k$, mentre il blocco è immobile:

\begin{figure}[H]
  \centering
  \colorlet{xcol}{blue!70!black}
  \colorlet{darkblue}{blue!40!black}
  \colorlet{myred}{red!65!black}
  \tikzstyle{mydashed}=[xcol,dashed,line width=0.25,dash pattern=on 2.2pt off 2.2pt]
  \tikzstyle{axis}=[->,thick] %line width=0.6
  \tikzstyle{ell}=[{Latex[length=3.3,width=2.2]}-{Latex[length=3.3,width=2.2]},line width=0.3]
  \tikzstyle{dx}=[-{Latex[length=3.3,width=2.2]},darkblue,line width=0.3]
  \tikzstyle{ground}=[preaction={fill,top color=black!10,bottom color=black!5,shading angle=20},
                      fill,pattern=north east lines,draw=none,minimum width=0.3,minimum height=0.6]
  \tikzstyle{mass}=[line width=0.6,red!30!black,fill=red!40!black!10,rounded corners=1,
                    top color=red!40!black!20,bottom color=red!40!black!10,shading angle=20]
  \tikzstyle{spring}=[line width=0.8,blue!7!black!80,snake=coil,segment amplitude=5,segment length=5,line cap=round]
  \tikzset{>=latex} % for LaTeX arrow head
  \tikzstyle{force}=[->,myred,very thick,line cap=round]
  \def\tick#1#2{\draw[thick] (#1)++(#2:0.1) --++ (#2-180:0.2)}

  \begin{tikzpicture}[scale=2]
    \def\H{0.25}     % ceiling height
    \def\W{2.6}      % ceiling width
    \def\h{0.7}      % mass height
    \def\w{0.6}      % mass width
    \def\l{0.5*\y}   % rest length without weight
    \def\dl{0.7*\y}  % rest length with weight
    \def\y{2.4}      % mass y position
    \def\xy{0.38*\W} % mass y position
    \def\F{0.8}      % force magnitude
    \draw[spring,segment length=7.2] (0,0) -- (0,-\y);
    \draw[ground] (-\W/2,0) rectangle++ (\W,\H);
    \draw (-\W/2,0) --++ (\W,0);
    \draw[axis] (-\xy,0) --++ (0,-\y-0.7*\h) node[left] {$y$};
    \draw[axis] ( \xy,0) --++ (0,-\y-0.7*\h) node[right] {$y'$};
    \draw[mydashed] (-\xy,-\l) --++ (2.3*\xy,0);
    \draw[mydashed] (-\xy,-\dl) --++ (2*\xy,0);
    \draw[mydashed] (-0.46*\W,-\y) --++ (0.92*\W,0);
    \tick{-\xy,-\l}{0} node[left] {$0$};
    \tick{-\xy,-\dl}{0} node[left] {$y_0$};
    \tick{ \xy,-\dl}{180} node[right] {$0$};
    \draw[mass] (-\w/2,-\y) rectangle++ (\w,-\h) node[midway] {$m$};
    \draw[force] (0.4*\w,-\y-0.3*\h) --++ (0,1.6*\F) node[pos=0.9,right=0] {$\vb{F}$};
    \draw[force] (0.3*\w,-\y-0.7*\h) --++ (0,-\F) node[above right=0] {$m\vb{g}$};
    \draw[ell] (0.45*\W,0) --++ (0,-\l) node[midway,right=-2] {$\ell_0$};
    %\draw[ell] (-0.4*\W,-0.75*\y) --++ (0,-0.25*\y) node[midway,left=1] {$y_0$};
  \end{tikzpicture}
  \caption{Fisica di una molla verticale}
  \label{fig:fisica_molla_verticale}
\end{figure}

\noindent
Naturalmente, le due forze che agiscono su tale massa sono la forza peso $\vec F_t$ e la forza di richiamo della molla $\vec F_m$. Inoltre, essendo il blocco immobile ($\vec a = 0$), deve essere che
\[\sum \vec F = m \vec a = 0\]
Applicando la seconda legge della dinamica, quindi, si ottiene che
\[k \cdot \Delta y = mg \longrightarrow \Delta y = \frac{mg}{k}\]

\vspace{1em}
\noindent
\textbf{Esempio $\boldsymbol{2}$}: Si consideri un carrello che si muove di velocità $\vec v_i$ e che deve arrestare la sua corsa per mezzo di una molla posta orizzontalmente al moto:

\begin{figure}[H]
  \centering
  \colorlet{xcol}{blue!70!black}
  \colorlet{darkblue}{blue!40!black}
  \colorlet{myred}{red!65!black}
  \tikzstyle{mydashed}=[xcol,dashed,line width=0.25,dash pattern=on 2.2pt off 2.2pt]
  \tikzstyle{axis}=[->,thick] %line width=0.6
  \tikzstyle{ell}=[{Latex[length=3.3,width=2.2]}-{Latex[length=3.3,width=2.2]},line width=0.3]
  \tikzstyle{dx}=[-{Latex[length=3.3,width=2.2]},darkblue,line width=0.3]
  \tikzstyle{ground}=[preaction={fill,top color=black!10,bottom color=black!5,shading angle=20},
                      fill,pattern=north east lines,draw=none,minimum width=0.3,minimum height=0.6]
  \tikzstyle{mass}=[line width=0.6,red!30!black,fill=red!40!black!10,rounded corners=1,
                    top color=red!40!black!20,bottom color=red!40!black!10,shading angle=20]
  \tikzstyle{spring}=[line width=0.8,blue!7!black!80,snake=coil,segment amplitude=5,segment length=5,line cap=round]
  \tikzset{>=latex} % for LaTeX arrow head
  \tikzstyle{force}=[->,myred,very thick,line cap=round]
  \def\tick#1#2{\draw[thick] (#1)++(#2:0.1) --++ (#2-180:0.2)}

  \begin{tikzpicture}[scale=2]
    \def\H{1}    % wall height
    \def\T{0.3}  % wall thickness
    \def\W{2.6}  % ground length
    \def\D{0.25} % ground depth
    \def\h{0.6}  % mass height
    \def\w{0.7}  % mass width
    \def\x{1.6}  % mass x position
    \draw[spring] (0,\h/2) --++ (\x/2,0);
    \draw[ground] (0,0) |-++ (-\T,\H) |-++ (\T+\W,-\H-\D) -- (\W,0) -- cycle;
    \draw (0,\H) -- (0,0) -- (\W,0);
    \draw[mass] (\x/2+0.5,0) rectangle++ (\w,\h) node[midway] {$m$};
    \draw[mass] (\x/2,0) rectangle++ (0.05,\h);
    \draw[-stealth] (\x/2+0.5,\h+0.2) -- (\x/2,\h+0.2) node[midway,above] {$\vec v_i$};
  \end{tikzpicture}
  \caption{Arresto di un carrello tramite freno a molla}
  \label{fig:arresto_carrello_freno_molla}
\end{figure}

\noindent
Si determini, allora, la costante di elasticità della molla $k$ affinché il modulo dell'accelerazione sia al massimo $a_{max}=5g$.\\
Naturalmente, se tale costante è troppo elevata, la molla si comporterà come un muro, mentre se è troppo bassa non sarà sufficiente a rallentare in tempo la corsa del mezzo.\\
Sa la molla è troppo dura, la decelerazione sarà molto forte e rapida, mentre se è troppo morbida, la sua decelerazione non sarà sufficiente ad evitare l'impatto; è chiaro che l'accelerazione sarà massima nel punto di massima compressione, mentre sarà nulla prima del punto di contatto e crescerà in modo lineare secondo la legge di Hooke:

\vspace{2em}
\noindent
\rowcolors{1}{white}{white}
\begin{tabularx}{\textwidth}{P}
  {
      \centering
      \begin{tikzpicture}
        \begin{axis}[
          grid=both,
          axis lines = middle,
          xlabel = \(x\),
          ylabel = {\(a\)},
          legend pos=outer north east,
          ymajorgrids=true,
          xmajorgrids=true,
          grid style=dashed,
          xmin=-6,
          xmax=6,
          ymin=-2,
          ymax=6,
          xtick={0,5},
          xticklabels={O,$x_{max}$},
          ytick={5},
          yticklabels={$a_{max}$},
        ]
      \addplot[
        domain=0:5,
        samples=100,
        color=blue,
        thick,
      ]
      {x};
      \draw[thick,blue] (axis cs:-5,0) -- (axis cs:0,0);
      \end{axis}
      \end{tikzpicture}
    }
\end{tabularx}

\vspace{1em}
\noindent
Dalla seconda legge della dinamica si ottiene che
\[\sum \vec F = m \vec a\]
per cui considerando solamente la componente orizzontale, si ottiene
\[F_x = m a_x \longrightarrow -k x = m a_x \longrightarrow a_x = - \frac{k}{m} x\]
Applicando, ora, il teorema lavoro-energia cinetica si ottiene che
\[W = \Delta K\]
in cui è noto che $\vec v_i = \vec v_i$ mentre $x_i = 0$, e $\vec v_f = 0$, mentre $x_f = x_{max}$ da cui:
\[W = - \frac{1}{2}k \cdot \left(x_f^2 - x_i^2\right) = - \frac{1}{2}k \cdot x_{max}^2\]
invece si ottiene che
\[\Delta K = K_f - K_i = 0 - \frac{1}{2}m v_i^2\]
Applicando infine il teorema lavoro-energia cinetica si ottiene che
\[- \frac{1}{2}k \cdot x_{max}^2 = - \frac{1}{2}m v_i^2 \longrightarrow x_{max} = \sqrt{\frac{m}{k}} \cdot v_i\]
Ecco che avendo isolato $x_{max}$ si può scrivere che
\[a_{max} = \left \vert \frac{k}{m} x_{max} \right \vert \longrightarrow a_{max} = \sqrt{\frac{k}{m}} \cdot v_i\]
Ma sapendo che $a_{max}=5g$, si può facilmente isolare $k$, ottenendo:
\[k \leq \frac{a_{max}^2}{v_{i}^2} \cdot m = \frac{25 g^2}{v_{i}^2} \cdot m\]

\vspace{1em}
\subsection{Lavoro compiuto dalla forza di gravità}
Naturalmente la forza di gravità è una forza costante, per cui il lavoro che essa compie dipende unicamente dalla posizione iniziale e finale scelta.\\
Allora, applicando la formula per il calcolo del lavoro compiuto da una forza costante si ottiene
\[W=\int_i^f \vec F \cdot d \vec r = -mg \cdot (y_f - y_i)\]
pertanto il lavoro compiuto dalla forza di gravità è
\[\boxed{W=-mg \cdot (y_f - y_i)}\]

\vspace{1em}
\noindent
\textbf{Esempio}: Si consideri una massa $m$ che viene lasciata scivolare (senza attrito) su una discesa, partendo con velocità iniziale $\vec v_i = 0$ e giungendo alla base di tale discesa con velocità finale $\vec v_f$. Si calcoli, allora, la velocità finale della massa.\\
Naturalmente, applicando il teorema lavoro-energia cinetica si ottiene che
\[W=\Delta K \longrightarrow -mg \cdot (y_f - y_i) = \frac{1}{2}m \cdot v_f^2 \longrightarrow mgh = \frac{1}{2} m v_f^2\]
Ecco allora che la velocità finale di un corpo che viene lasciato cadere da una altezza $h$, in assenza di attriti è
\[\boxed{v_f=\sqrt{2gh}}\]

\vspace{1em}
\noindent
\textbf{Osservazione}: Si osservi che il lavoro compiuto da una forza può essere sia positivo o negativo (se la forza è diretta nello stesso verso dello spostamento, oppure no).\\
Una forza che non compie mai lavoro è il campo magnetico (in quanto nell'equazione di Lorents, la forza è ortogonale alla velocità, e quindi al vettore spostamento): ciò significa che il campo magnetico non può mai accelerare delle particelle; per farlo è necessario impiegare un campo elettrico.\\
Ovviamente il lavoro è l'integrale della forza per lo spostamento, per cui rappresenta l'area della parte di piano sottesa al grafico della forza in funzione dello spostamento.

\vspace{1em}
\subsection{Forze conservative}
Di seguito si espone la definzione di \textbf{forza conservativa}:

% Tabella per le definizione di concetti, etc...
\vspace{1em}
\rowcolors{1}{black!5}{black!5}
\setlength{\tabcolsep}{14pt}
\renewcommand{\arraystretch}{2}
\noindent
\begin{tabularx}{\textwidth}{@{}|P|@{}}
    \hline
    {\textbf{FORZA CONSERVATIVA}}\\
    \parbox{\linewidth}{Una forza $\vec F$ si definisce \textbf{conservativa} se
    \[\boxed{\int_i^f \vec F \cdot d \vec r}\]
    è \textbf{indipendente dal percorso}. Analogamente si ha che il lavoro compiuto da una forza $\vec F$ conservativa su un \textbf{percorso chiuso è nullo}
    \[\boxed{\oint \vec F \cdot d \vec r = 0}\]
    che corrisponde a dire che
    \[\boxed{\int_i^f \vec F \cdot d \vec r + \int_f^i \vec F \cdot d \vec r = 0}\]
    In altri termini, si potrebbe dire che una forza è definita conservativa se esiste una funzione $\mathcal{U}$ tale che
    \[\boxed{\vec F = \nabla \mathcal{U}}\]
    Ove per $\nabla \mathcal{U}$ è da intendersi il \textbf{gradiente} di $\mathcal{U}$, ossia la derivata coalcolata in tutte le direzioni:
    \[\nabla \mathcal{U} = \frac{d \mathcal{U}}{d x} \cdot \hat{i} + \frac{d \mathcal{U}}{d y} \cdot \hat{j} + \frac{d \mathcal{U}}{d z} \cdot \hat{k}\]
    \vspace{-1mm}}\\
    \hline
\end{tabularx}

\vspace{1em}
\noindent
\textbf{Osservazione}: Si osservi che la funzione $\mathcal{U}$ impiegata nella definizione prende il nome di \textbf{energia potenziale}.

\vspace{1em}
\noindent
\textbf{Esempio}: Si osservi che un esempio molto semplice di \textbf{forza non-conservativa} è l'attrito, in quanto il lavoro compiuto da tale forza dipende necessariamente dal percorso, essendo la forza d'attrito sempre opposta allo spostamento. Pertanto si ottiene che:
\[W_k = \int_i^f F_k \cdot d \vec r = \int_i^f -\mu_k m g \cdot d \vec r = -\mu_k m g \cdot s\]
in cui $s$ è la lunghezza del percorso.

\vspace{1em}
\noindent
\textbf{Osservazione}: Si osservi che Richard Feynman, parlando di fisica fondamentale delle particelle, ha affermato che:\\\\
\quotes{\emph{Abbiamo speso un tempo considerevole per discutere le forze conservative; che cosa diremo delle forze non conservative? Approfondiremo l'argomento più di quanto non si faccia solitamente, e stabiliremo che non esistono forze non conservative! In realtà, tutte le forze fondamentali nella natura appaiono conservative. Questa non è una conseguenza delle leggi di Newton. Infatti, per quanto ne sapeva Newton, le forze avrebbero potuto essere non conservative, come apparentemente è l'attrito. Quando diciamo che l'attrito apparentemente lo è, usiamo un punto di vista moderno, essendo stato scoperto che tutte le forze elementari, le forze fra le particelle a livello fondamentale, sono conservative.}}\\\\
\hspace{10em} Richard Feynman\\\\
Per cui le forze microscopiche sono conservative, ma quando si considera un assieme, la seconda legge della termodinamica stabilirà che deve esserci calore dissipato: pertanto, anche se tutti i meccanismi fondamentali sono conservativi, vi deve essere una aumento di \textbf{entropia}. Si conclude, quindi, che il problema delle forze dissipative (e quindi non conservative), come l'attrito, è un problema prettamente termodinamico.

\newpage
\noindent
\begin{center}
  24 Marzo 2022
\end{center}
Il lavoro viene definito come l'integrale del prodotto tra la forza applicata e lo spostamento ottenuto: naturalmente il lavoro può essere positivo o negativo e può essere eseguito da qualsiasi forza.\\
Le forze conservative sono forze per le quali il lavoro non dipende dal percorso e, quindi, è possibile applicare il principio di conservazione dell'energia (in altre parole, le forze conservative sono forze capaci di immagazzinare dell'energia potenziale, potenziale per svolgere un lavoro). Esempi di forze conservative sono:
\begin{itemize}
  \item Gravità;
  \item Forza elastica (non sempre);
  \item Forza elettrica;
  \item etc.
\end{itemize}
Mentre forze non conservative (e quindi dissipative) sono:
\begin{itemize}
  \item Attrito;
  \item Resistenza dell'aria;
  \item Forza compiuta da una persona;
  \item etc.
\end{itemize}

\vspace{1em}
\subsection{Energia potenziale}
Di seguito si espone la definizione di \textbf{energia potenziale}:

% Tabella per le definizione di concetti, etc...
\vspace{1em}
\rowcolors{1}{black!5}{black!5}
\setlength{\tabcolsep}{14pt}
\renewcommand{\arraystretch}{2}
\noindent
\begin{tabularx}{\textwidth}{@{}|P|@{}}
    \hline
    {\textbf{ENERGIA POTENZIALE}}\\
    \parbox{\linewidth}{Data una forza conservativa $\vec F$, è noto che il lavoro compiuto da tale forza è
    \[W_{i,f}=\int_{i}^f \vec F \cdot d \vec r\]
    il quale, essendo $\vec F$ conservativa, dipende solamente dai vettori posizione iniziale $\vec r_i$ e finale $\vec r_f$. Naturalmente, però, essendo $\vec F$ conservativa, essa può essere definita come il gradiente di una funzione $\mathcal{U}$, ovvero $\vec F = \nabla \mathcal{U}$, per cui eseguendo l'integrale indefinito di $\vec F$, si ottiene solamente l'energia potenziale $\mathcal{U}$, che deve essere valutata da una posizione finale ad una iniziale, come mostrato di seguito:
    \[W_{i,f} = \left(-\mathcal{U}(\vec r_f)\right) - \left(-\mathcal{U}(\vec r_i)\right)\]
    In altre parole si ha che
    \[\boxed{\mathcal{U}(\vec r) = - \int \vec F \cdot d \vec r + c}\]
    in cui $c=$ costante. Da ciò segue che
    \[\boxed{W_{i,f}=-\Delta \mathcal{U}_{i,f}}\]
    per cui il lavoro compiuto da $i$ a $f$ è uguale all'opposto della differenza di energia potenziale.\vspace{3mm}}\\
    \hline
\end{tabularx}

\vspace{2em}
\noindent
\textbf{Osservazione}: Si osservi che ogni qualvolta vi è una forza conservativa vi è energia potenziale.\\
L'energia potenziale, più in generale, può essere interpretata come un modo di immagazzinare energia per essere successivamente impiegata per compiere un lavoro (si parla di \quotes{potenziale di compiere un lavoro}).

\vspace{1em}
\subsubsection{Energia potenziale gravitazionale}
Per quanto concerne la forza di gravità, è noto che $\vec F_t = - mg \cdot \hat{j}$ e $d \vec r = dy \cdot \hat{j}$, per cui l'energia potenziale gravitazionale diviene
\[\mathcal{U}(y)=-\int(-mg \cdot dy) = mg \cdot y + c\]
Per cui l'energia potenziale gravitazionale si calcola come segue:
\[\boxed{\mathcal{U}(h)=mgh}\]

\vspace{1em}
\subsubsection{Energia potenziale elastica}
Per quanto concerne la forza elastica, è noto che $\vec F \cdot d \vec r = -k x \cdot dx$, da cui:
\[\mathcal{U}(x) = -\int - k x \cdot dx = \frac{1}{2}k \cdot x^2 + c\]
Per cui l'energia potenziale elastica si calcola come segue:
\[\boxed{\mathcal{U}(x)=\frac{1}{2}k x^2}\]

\vspace{1em}
\noindent
\textbf{Esempio}: Si consideri il pompaggio idroelettrico che prevede un bacino idrico in rilievo con $h=1050 \text{ m}$, avente una capacità di $V=9000000 \text{ m}^3$. Allora per conoscere l'energia potenziale è sufficiente impiegare la formula precedente:
\[\mathcal{U}=mgh=V \rho g h = 90 TJ\]

\vspace{1em}
\subsection{Potenza}
Di seguito si espone la definizione di \textbf{potenza}:

% Tabella per le definizione di concetti, etc...
\vspace{1em}
\rowcolors{1}{black!5}{black!5}
\setlength{\tabcolsep}{14pt}
\renewcommand{\arraystretch}{2}
\noindent
\begin{tabularx}{\textwidth}{@{}|P|@{}}
    \hline
    {\textbf{POTENZA}}\\
    \parbox{\linewidth}{La \textbf{potenza} viene definita come
    \[\boxed{P=\frac{dW}{dt}}\]
    ovverosia la quantità di energia trasferita per unità di tempo. L'unità di misura è, naturalmente, il \textbf{Watt}, per cui
    \[\boxed{1 \text{ W} = 1 \frac{\text{J}}{\text{s}}}\]
    mentre per l'unità di energia $1 \text{ kWh}$ è da considerarsi
    \[\boxed{1 \text{ kWh} = 1000 \text{ W} \cdot 3600 \text{ s} = 3.6 \text{ MJ}}\]
    \vspace{-3mm}}\\
    \hline
\end{tabularx}

\vspace{1em}
\noindent
\textbf{Esempio}: Se si considera un lavoro di $90 \text{ TJ}$, allora tale lavoro corrisponde ad una potenza di $25 \text{ GWh}$. Se si considera il flusso massimo di una turbina pari a $130 \text{ m}^3/\text{s}$, allora la potenza massima diviene:
\[P=\frac{dW}{dt}=\frac{d\mathcal{U}}{dt}=\frac{d}{dt}(mgh)=\frac{dm}{dt} gh = \rho \cdot \frac{dV}{dt} gh = 1.4 \text{ MW}\]

\vspace{1em}
\noindent
\textbf{Esercizio}: Si consideri una macchina di $10^3 \text{ kg}$ che necessita una potenza di $16 \text{ hp}$ per andare ad una velocità costante di $80 \text{ km/h}$. Si determini, allora, la potenza necessaria per salire una pendenza di $10^\circ$.

\vspace{1em}
\begin{figure}[H]
  \centering
  \begin{tikzpicture}[scale=1]
    \draw (-1.8,0) -- ++(6.2,0);
    \foreach \i in {-20,-18,...,40} {
      \draw (\i / 10,-0.3) -- (\i / 10 + 0.3,0);
    }
    \node[minimum size=0.5cm,circle,draw] (circle) at (0.5,0.25){};
    \node[minimum size=0.5cm,circle,draw] (circle) at (1.5,0.25){};
    \draw (0.25,0.25) -- ++(-0.3,0) -- ++(0,0.5) -- ++(0.5,0) -- ++(0.3,0.3) -- ++(0.8,0) -- ++(0.3,-0.3) -- ++(0.5,0) -- ++(0,-0.5) -- ++(-0.6,0);

    \draw [-stealth] (2,2) -- node[midway, above]{$\vec{v}$} (3,2) ;
    \draw [-stealth] (0.85,0) -- node[midway, left]{$\vec{F}_N$} (0.85,3) ;
    \draw [-stealth] (1.15,0.65) node[circ]{} -- ++(0,-3) node [midway, below right] {$\vec{F}_{t}$};
    \draw [-stealth, red] (2,0.5) node[circ]{} -- ++(2,0) node [midway, above right] {$\vec{F}$};
    \draw [-stealth, red] (0.5,0) node[circ]{} -- ++(-2,0) node [midway, above left] {$\vec{F}_{r}$};
  \end{tikzpicture}
  \caption{Vettura a velocità costante su una strada orizzontale}
  \label{fig:vettura_velocita_costante_strada_orizzontale}
\end{figure}

\noindent
Naturalmente le forze in gioco sono la forza di gravità, la forza normale, la forza di attrito e la forza di trazione:

\vspace{1em}
\begin{figure}[H]
  \centering
  \begin{tikzpicture}[scale=1]
    \draw [-stealth] (0,0) node[circ]{} -- ++(0,1) node [at end, right] {$\vec{F}_N$};
    \draw [-stealth] (0,0) -- ++(0,-1) node [midway, right] {$\vec{F}_{t}$};
    \draw [-stealth] (0,0) -- ++(-1,0) node [midway, above] {$\vec{F}_{r}$};
    \draw [-stealth] (0,0) -- ++(1,0) node [midway, above] {$\vec{F}$};
  \end{tikzpicture}
  \caption{Diagramma a corpo libero di una vettura a velocità costante su una strada orizzontale}
  \label{fig:diagramma_corpo_libero_vettura_velocita_costante_strada_orizzontale}
\end{figure}

\noindent
Giacché la velocità è costante, l'accelerazione della vettura è nulla, per cui
\[\sum \vec F = 0\]
Da ciò si evince la forza di trazione $\vec F$ e la forza di resistenza $\vec F_r$ sono uguali ed opposte, per cui:
\[\vec F = \vec F_r\]
Applicando il teorema lavoro-energia cinetica, si ha che
\[W_{\text{tot}}=\Delta K=0\]
in cui è da considerare $W_{\text{tot}}=W_m+W_r$, i quali sono uguali e oppposti. Ovviamente si ha che
\[W_m = \int \vec F \cdot d \vec r \hspace{1em} \text{e} \hspace{1em} W_r = \int \vec F_r \cdot d \vec r\]
e sono entrambi uguali a $W=F \cdot d$. Naturalmente, in questo caso, si ottiene che la potenza cercata è proprio:
\[P=\frac{d}{dt}W_m=\frac{d}{dt}F \cdot d = F \cdot v\]
che è un risultato fondamentale:
\[\boxed{P=\vec F \cdot \vec v}\]
Dai dati del problema si ha che tale potenza, ovvero $P=F \cdot v = 16 \text{ hp}$.\\
Se la macchina deve affrontare una salita di $10^\circ$, allora la forza peso deve essere scomposta nelle sue due componenti, e non viene cancellata totalmente dalla forza normale; tuttavia, si ha sempre che:
\[\sum \vec F = 0 = \vec F_r + \vec F_N + \vec F_t + \vec F\]
Ruotando il sistema di $10^\circ$ si ottiene che, scomponendo l'equazione nelle sue due componenti:
\begin{align*}
    \hat{j}:F_N - F_t \cdot \cos(\theta) = 0\\
    \hat{i}:F - F_r  - F_t \cdot \sin(\theta) = 0
\end{align*}
Volendo cercare la potenza della mattina necessaria per affrontare la salita, si deve procedere al calcolo seguente:
\[P=F \cdot v=(F_r + F_t \cdot \sin(\theta)) \cdot v\]
Ma volendo calcolare $F_r \cdot v$, si può impiegare il risultato precedente, sapendo che $F_r \cdot v = 16 \text{ hp}$. Da ciò si evince che:
\[P=F_r \cdot v + mg \cdot \sin(\theta) \cdot v = 16 \text{ hp} + \frac{37.8 \text{ kW}}{743 \text{ W/hp}} = 67 \text{ hp}\]
In cui, ovviamente, $mg \cdot \sin(\theta) \cdot v$ è il tasso di cambio della propria velocità in energia potenziale.

\vspace{1em}
\noindent
\textbf{Esercizio}: Si calcoli la velocità di un motorino elettrico avente una potenza di $3000 \text{ W}$ in salita avente una pendenza del $10^\circ$, in cui la massa complessiva del sistema è di $200 \text{ kg}$ (trascurando la resistenza dell'aria).\\
Naturalmente, giacché $v$ è costante, significa che non vi è accelerazione ($\vec a=0$). Da ciò si ha che la forza di trazione del motorino è $F=mg \cdot \sin(\theta)$, mentre la potenza cercata è
\[P = mg \cdot \sin(\theta) \cdot v = F \cdot v = \frac{d}{dt} \left(\mathcal{U}(y)\right) = mg \cdot \frac{dy}{dt}\]
Da cui si evince che la velocità del sistema cercata è proprio data dal rapporto seguente:
\[v = \frac{P}{mg \cdot \sin(\theta)} = 55 \text{ km/h}\]

\newpage
\noindent
\begin{center}
  28 Marzo 2022
\end{center}
\textbf{Esercizio}: Si consideri una massa $m=500$ g che viene lasciato cadere da un'altezza $h=60$ cm su un ripiano collegato ad una molla di costante elastica $k=120$ N/m.

\begin{figure}[H]
  \centering
  \colorlet{xcol}{blue!70!black}
  \colorlet{darkblue}{blue!40!black}
  \colorlet{myred}{red!65!black}
  \tikzstyle{mydashed}=[xcol,dashed,line width=0.25,dash pattern=on 2.2pt off 2.2pt]
  \tikzstyle{axis}=[->,thick] %line width=0.6
  \tikzstyle{ell}=[{Latex[length=3.3,width=2.2]}-{Latex[length=3.3,width=2.2]},line width=0.3]
  \tikzstyle{dx}=[-{Latex[length=3.3,width=2.2]},darkblue,line width=0.3]
  \tikzstyle{ground}=[preaction={fill,top color=black!10,bottom color=black!5,shading angle=20},
                      fill,pattern=north east lines,draw=none,minimum width=0.3,minimum height=0.6]
  \tikzstyle{mass}=[line width=0.6,red!30!black,fill=red!40!black!10,rounded corners=1,
                    top color=red!40!black!20,bottom color=red!40!black!10,shading angle=20]
  \tikzstyle{spring}=[line width=0.8,blue!7!black!80,snake=coil,segment amplitude=5,segment length=5,line cap=round]
  \tikzset{>=latex} % for LaTeX arrow head
  \tikzstyle{force}=[->,myred,very thick,line cap=round]
  \def\tick#1#2{\draw[thick] (#1)++(#2:0.1) --++ (#2-180:0.2)}

  \begin{tikzpicture}[scale=2]
    \def\H{1}    % wall height
    \def\T{0.3}  % wall thickness
    \def\W{2.6}  % ground length
    \def\D{0.25} % ground depth
    \def\h{0.6}  % mass height
    \def\w{0.7}  % mass width
    \def\x{1.6}  % mass x position
    \draw[spring] (\x/2,0) --++ (0,\h);
    \draw[ground] (0,0) |-++ (-\T,\H+1.5) |-++ (\T+\W,-\H-\D-1.5) -- (\W,0) -- cycle;
    \draw (0,\H+1.5) -- (0,0) -- (\W,0);

    \draw[mass] (\x/4,\h+1) rectangle++ (\w,\h) node[midway] {$m$};
    \draw[mass] (\x/4,\h) rectangle++ (\x/2,0.05);

    \draw[stealth-stealth] (\x,\h/2+0.5) -- ++(0,1) node[midway, right]{$60$ cm};
    \draw[stealth-stealth] (\x+0.75,\h/2) -- ++(0,1.5) node[midway, right]{$l$};
  \end{tikzpicture}
  \caption{Massa lasciata cadere su una molla}
  \label{fig:massa_lasciata_cadere_molla}
\end{figure}

\noindent
Si determini, allora, la lunghezza $l$ di compressione massima della molla fino a quando l'assieme blocco-piattaforma si ferma.\\
Al fine di risolvere tale problema è utile richiamare il teorema lavoro-energia cinetica: naturalmente si ha che il lavoro totale compiuto sul sistema è pari alla variazione di energia cinetica:
\[W_{\text{tot}} = \Delta K = \frac{1}{2}m \cdot \left(v_f^2 - v_i^2\right)\]
ma essendo la velocità iniziale e finale del blocco nulle, si deduce che il lavoro totale compiuto sul sistema è nullo. Le uniche forze che compiono lavoro, in questo caso ipotetico, sono la forza di gravità e la forza elastica di richiamo della molla; ciò significa che il lavoro compiuto da ambedue le forze è uguale, ma opposto, da cui:
\[w_{\text{tot}}=0=W_g+W_m=mgl-\frac{1}{2}k \cdot (l-d)^2\]
Da ciò si evince che:
\[-\frac{1}{2}kl^2 + (kd+mg) \cdot l - \frac{1}{2}kd^2=0\]
e volendo isolare $l^2$ si ottiene:
\[l^2 - 2 \cdot \left(\frac{mg}{k} + d\right) \cdot l + d^2=0\]
Naturalmente tale equazione presenterà due soluzioni, ma in questo caso si deve chiedere che $l>d$, per cui:
\[l_{1,2}=\frac{\dfrac{mg}{k}+l \pm \sqrt{4 \cdot \left(\dfrac{mg}{k} + d\right)^2 - 4d^2}}{2}\]
ma l'unica soluzione che ha significato è
\[l = \frac{mg}{k} \cdot \left(1+\sqrt{1+\frac{2kd}{mg}}\right)+d = 86.6 \text{ cm}\]

\vspace{1em}
\noindent
\textbf{Osservazione}: Si osservi che l'altra soluzione dell'equazione corrisponde non al caso in cui la molla è compressa, ma quando la molla è stesa.

\newpage
\noindent
\subsection{Conservazione dell'energia}
L'energia può essere interpretata come un fluido, la quale si traforma in una forma o in un'altra, ma conservando sempre la propria quantità. Tale concetto viene condensato dal \textbf{teorema di conservazione dell'energia} esposto di seguito:

% Tabella per le definizione di concetti, etc...
\vspace{1em}
\rowcolors{1}{black!5}{black!5}
\setlength{\tabcolsep}{14pt}
\renewcommand{\arraystretch}{2}
\noindent
\begin{tabularx}{\textwidth}{@{}|P|@{}}
    \hline
    {\textbf{CONSERVAZIONE DELL'ENERGIA}}\\
    \parbox{\linewidth}{Quando si impiega il teorema di conservazione dell'energia, è sempre fondamentale specificare il \textbf{sistema} su cui si sta lavorando, distinguendolo con l'\textbf{ambiente esterno}: tra il sistema e l'ambiente può esserci scambio (o trasferimento) di energia (in ambo le direzioni) oppure no; nel secondo caso si può applicare il \textbf{teorema di conservazione dell'energia} sul sistema considerato:
    \[\boxed{E_{\text{sistema}}=K+\mathcal{U}+U}\]
    in cui $K$ è l'\textbf{energia cinetica}, $\mathcal{U}$ è l'\textbf{energia potenziale}, mentre $U$ prende il nome di \textbf{energia interna}. Tuttavia, in \textbf{meccanica}, si ignora totalmente l'energia del sistema, e si preferisce considerare solamente l'\textbf{energia meccanica} $E_m=K+\mathcal{U}$.\\
    Se si ha conservazione dell'energia meccanica, si ottiene che
    \[\boxed{\Delta E_{\text{sistema}}=0=\sum\text{trasferimenti}=\text{lavoro}+\text{calore}}\]
    per cui su un sistema chiuso è possibile modificare l'energia del sistema compiendo lavoro su di esso, oppure tramite scambi di calore. Esempi di sistema possono essere costituiti da
    \begin{itemize}
      \item un corpo solo;
      \item due (o più corpi) che interagiscono;
      \item un corpo deformabile;
      \item una regione dello spazio (in questo caso si può avere non solo scambi d energia, ma anche di materia);
      \item etc.
    \end{itemize}
    in cui tali sistemi possono essere \textbf{aperti} (per cui si ha scambio sia di energia che di materia), \textbf{chiusi} (in cui si ha scambio solamente di energia) o \textbf{isolati} (in cui non si ha nè scambio di energia nè di materia).\\
    Più limitatamente all'energia meccanica, il teorema di conservazione dell'energia meccanica afferma che
    \[\boxed{\Delta E_{\text{sistema}}=\Delta K + \Delta \mathcal{U}=W_{\text{sistema}}}\]
    ove $W_{\text{sistema}}$ è da intendersi il lavoro totale compiuto sul sistema.\vspace{3mm}}\\
    \hline
\end{tabularx}

\vspace{1em}
\noindent
\textbf{Esempio}: Nel caso precedente, in cui si considerava una massa che cadeva sulla molla, si può considerare come sistema l'assieme blocco-molla-terra, in cui si assiste alla conservazione dell'energia meccanica, in quanto non c'è lavoro compiuto sul sistema dall'esterno: pertanto si assiste solamente alla trasformazione di energia potenziale gravitazionale in energia cinetica e, successivamente, in energia potenziale elastica.

\vspace{1em}
\noindent
\textbf{Osservazione}: Si osservi che l'energia cinetica non può mai essere negativa, mentre l'energia potenziale è totalmente arbitraria: ciò che è fondamentale da valutare è la variazione di energia potenziale.\\
Inoltre, l'energia totale di un sistema può cambiare se una forza esterna compie lavoro sul sistema: il lavoro totale compiuto sul sistema è uguale alla variazione in energia totale del sistema.

\vspace{1em}
\noindent
\textbf{Esercizio}: Se si considera una biglia poggiata su una molla in compressione: nello stadio iniziale vi sarà solamente energia potenziale elastica, che poi si tramuterà in energia cinetica ed energia potenziale gravitazionale, fino a quando la biglia raggiungerà il suo punto di massima altezza, in cui la totalità dell'energia del sistema è data soltanto dalla componente potenziale gravitazionale.

\vspace{1em}
\noindent
\textbf{Osservazione}: Si osservi che l'energia potenziale di una forza conservativa $\vec F$ era stata definita come
\[\mathcal{U} = - \int \vec F \cdot d \vec r\]
per cui si può scrivere che
\[\boxed{F_x = - \frac{d \mathcal{U}}{dx}}\]
ovvero la forza $F$ corrisponde all'opposto della pendenza del grafico energia potenziale-spostamento:

\vspace{2em}
\noindent
\rowcolors{1}{white}{white}
\begin{tabularx}{\textwidth}{P}
  {
      \centering
      \begin{tikzpicture}
        \begin{axis}[
          grid=both,
          axis lines = middle,
          xlabel = \(x\),
          ylabel = {\(\mathcal{U}\)},
          legend pos=outer north east,
          ymajorgrids=true,
          ymin=-5,
          xmajorgrids=true,
          grid style=dashed,
        ]

        \addplot [
          domain=-2:12,
          samples=100,
          color=red,
        ]
        {(x-5)^2+20};
        \draw (axis cs:5,20) node[circ]{} node[above=0.5cm]{$F=0$};
        \draw (axis cs:8,29) node[circ]{} node[below=0.5cm]{$\vec F$};
        \draw (axis cs:2,29) node[circ]{} node[below=0.5cm]{$\vec F$};
        \draw[-stealth, red, very thick] (axis cs:1,15) -- (axis cs:3,15);
        \draw[-stealth, red, very thick] (axis cs:9,15) -- (axis cs:7,15);
        \end{axis}
    \end{tikzpicture}
  }
\end{tabularx}

\noindent
Questo è un sistema che viene definito \textbf{stabile}, in quanto ad ogni spostamento positivo (o negativo) vi è una forza di richiamo negativa (o positiva, rispettivamente) che permette di bilanciare il sistema, portandolo in una condizione di equilibrio.\\
Analogamente, un sistema è \textbf{instabile} quando ad ogni spostamento viene associata una forza che sbilancia ancora di più il sistema nella direzione dello spostamento:

\vspace{2em}
\noindent
\rowcolors{1}{white}{white}
\begin{tabularx}{\textwidth}{P}
  {
      \centering
      \begin{tikzpicture}
        \begin{axis}[
          grid=both,
          axis lines = middle,
          xlabel = \(x\),
          ylabel = {\(\mathcal{U}\)},
          legend pos=outer north east,
          ymin=-5,
          ymax=18,
          xmin=-2,
          xmax=12,
          ymajorgrids=true,
          xmajorgrids=true,
          grid style=dashed,
        ]

        \addplot [
          domain=-2:12,
          samples=100,
          color=red,
        ]
        {-(x-5)^2+12};
        \draw (axis cs:5,12) node[circ]{} node[above=0.5cm]{$F=0$};
        \draw (axis cs:8,3) node[circ]{} node[above=1.5cm]{$\vec F$};
        \draw (axis cs:2,3) node[circ]{} node[above=1.5cm]{$\vec F$};
        \draw[-stealth, red, very thick] (axis cs:3,12) -- (axis cs:1,12);
        \draw[-stealth, red, very thick] (axis cs:7,12) -- (axis cs:9,12);
        \end{axis}
    \end{tikzpicture}
  }
\end{tabularx}

\noindent
Non da ultimo devono essere citati sistemi molto più complessi, in cui vi può essere la combinazione di più configurazioni stabili e instabili, valutabili tramite la derivata seconda, ovvero:
\[\frac{d^2 \mathcal{U}}{dx^2} > 0 \longrightarrow \text{sistema stabile} \hspace{1em} \text{e} \hspace{1em} \frac{d^2 \mathcal{U}}{dx^2} < 0 \longrightarrow \text{sistema instabile}\]

\vspace{1em}
\noindent
\textbf{Osservazione}: Si osservi che l'energia potenziale gravitazione può essere interpretata come
\[\boxed{\mathcal{U}(r) = - G \cdot \frac{m_1 \cdot m_2}{r}}\]
che graficamente corrisponde ad un'iperbole equilatera sul quarto quadrante.

\newpage
\noindent
\begin{center}
  29 Marzo 2022
\end{center}
A proposito dell'energia potenziale gravitazione, è nota la formula seguente, in quanto deriva dall'integrale della forza di attrazione gravitazionale per la distanza:
\[\boxed{\mathcal{U}(r) = - G \cdot \frac{m_1 \cdot m_2}{r}}\]
Che presenta un grafico sull'asse della distanza che rappresenta un'iperbole equilatera: infatti, due masse inizialmente a riposo presentano una certa energia potenziale, la quale, a mano a mano che le due masse si avvicinano per attrazione gravitazionale, diminuisce fino ad annullarsi, trasformandosi progressivamente in energia cinetica. Nella realtà, le masse sono sempre in movimento, con una certa velocità, e progressivamente le due masse (che sono fra di loro simili) entrano in orbita una attorno all'altra rispetto al loro centro di massa.\\
Di seguito si espone il grafico iperbolico dell'energia potenziale gravitazionale:

\vspace{2em}
\noindent
\rowcolors{1}{white}{white}
\begin{tabularx}{\textwidth}{P}
  {
      \centering
      \begin{tikzpicture}
        \begin{axis}[
          grid=both,
          axis lines = middle,
          xlabel = \(r\),
          ylabel = {\(\mathcal{U}\)},
          legend pos=outer north east,
          ymax = 4,
          ymajorgrids=true,
          xmajorgrids=true,
          grid style=dashed,
        ]

        \addplot [
          domain=0:3,
          samples=100,
          color=red,
        ]
        {- 1/x};
        \end{axis}
    \end{tikzpicture}
  }
\end{tabularx}

\vspace{1em}
\noindent
\textbf{Osservazione}: Si osservi che in base alla formula di cui sopra, l'energia potenziale gravitazionale è sempre negativa (anche se sarebbe possibile aggiungere a tale calcolo una costante $c$ di integrazione).

\vspace{1em}
\noindent
\textbf{Esempio}: L'energia meccanica totale di un satellite in orbita attorno alla terra è data dalla somma di due componenti: l'energia cinetica e l'energia potenziale
\[E_{\text{tot}} = K+\mathcal{U}=\frac{1}{2}mv^2 - G \cdot \frac{m_t \cdot m}{r}\]
in cui $m$ è la massa del satellite. Inoltre, in tali sistemi nello spazio, si assiste ad un minimo scambio di energia (a causa delle radiazioni), per cui con sufficiente precisione è possibile affermare che si ha conservazione di energia, ovvero
\[\Delta E = 0 \longrightarrow K+\mathcal{U}=\text{costante}\]
Per comprendere la condizione di orbita di un satellite, è opportuno studiare i possibili valori di $E$
\begin{itemize}
  \item se $E$ è \textbf{positivo}, allontanando progressivamente il satellite dalla terra, l'energia potenziale $\mathcal{U}$ diminuisce progressivamente, tendendo a $0$, mentre l'energia cinetica aumenta (dovendo essere $E>0$), e quindi il satellite si allontanerebbe eccessivamente dalla terra, incrementando la propria energia cinetica e riducento la propria energia potenziale, non potendo più trovarsi in orbita attorno alla terna.\\
  Pertanto $E$ non può essere positivo, in quanto ciò significherebbe che il satellite presenta troppa energia e non può stare in orbita attorno alla terra.
  \item se $E$ è \textbf{negativo}, invece, allora il satellite, ad una certa distanza dalla terra, diminuirebbe la propria energia cinetica, senza mai annullarla totalmente, però, in quanto essa è sempre tangenziale all'orbita e consente al satellite di compiere tale moto di rivoluzione.
\end{itemize}
Pertanto, è possibile definire il concetto di \emph{$E:=$energia di legame}, per cui se $E < 0$ si ha la condizione per avere un'orbita attorno alla terra. Ovviamente sussiste anche il caso limite in cui $E=0$, per cui l'energia cinetica del satellite eguaglia quella potenziale gravitazinale: ciò significa che il satellite presenta la velocità sufficiente (appunto, la \textbf{velocità di fuga}) da poter uscire dal campo gravitazionale terrestre.

\vspace{1em}
\noindent
\textbf{Osservazione}: Naturalmente, l'energia potenziale dipende solamente dal parametro $r$, per cui più $r$ aummenta, più l'energia potenziale diminuisce, mentre se $r$ tende a $0$, allora l'energia potenziale assume il suo massimo valore.\\
Disegnando l'energia potenziale in tre dimensioni, è possibile visualizzare le \textbf{linee equipotenziali} che permettono di descrivere le possibili orbite di due pianeti attorno al proprio centro di massa: ovviamente tale linee sono delle circonferenze che permettono di visualizzare tutti punti alla medesima distanza.

\vspace{2em}
\noindent
\rowcolors{1}{white}{white}
\begin{tabularx}{\textwidth}{P}
  {
      \centering
      \begin{tikzpicture}[scale=1.5]
        \begin{axis}
        \addplot3[
          color=red,
          very thick,
          surf
        ]
        {-(1 / x^2 + 1 / y^2)};
        \end{axis}
    \end{tikzpicture}
  }
\end{tabularx}

\vspace{1em}
\noindent
Ovviamente se un massa ruota seguendo una di tali linee quipotenziali, in piena conservazione dell'energia, rimarrà a compiere tale moto perpetuo, in maniera identica.\\
Tuttavia, a causa delle cosidette \textbf{forze di marea}, ciò non si verifica, in quando deformazioni del corpo in rotazione comportano dispersione di energia cinetica e quindi un allontanamento dei corpi.\\
Il punto in cui si è più distanti dal centro è il punto in cui si ha il massimo dell'energia potenziale (ossia il più grande valore negativo), e quindi l'energia cinetica dovrà avere il suo valore minimo: infatti il satellite si muove molto lentamente tanto più è lontano dal centro; Viceversa si verificherà quando la massa si trova molto vicina al centro di rotazione.

\vspace{1em}
\subsection{Velocità di fuga}
È noto che la velocità di fuga si ha quando l'energia di legame è pari a $0$, ossia $E=0$: taluna è la condizione che permette al satellite di fuggire dalla sua orbita, ossia di uscire dal campo gravitazionale, ovvero:
\[E=0=K+\mathcal{U} \longrightarrow K=-\mathcal{U} \longrightarrow \frac{1}{2}m v^2 = G \cdot \frac{m_t \cdot m}{r}\]
in cui $v$ è proprio la velocità di fuga, per cui
\[\boxed{v_{\text{fuga}} = \sqrt{\frac{2G \cdot m_t}{r}}}\]
e dipende naturalmente da $r$: se la distanza è molto elevata, è chiaro che l'energia potenziale gravitazionale sarà inferiore e quindi la velocità di fuga sarà inferiore.\\
Nel caso della terra, allora, volendo calcolare la velocità di fuga sulla superficie della terra si ottiene:
\[v_{\text{fuga}} = \sqrt{\frac{2G \cdot m_t}{R_t}} = \sqrt{2} \cdot \sqrt{g \cdot R_t} \cong 11 000 \text{ m/s} \cong 40 000 \text{ km/h}\]
ossia pari a $\sqrt{2}$ volte la velocità necessaria per stabilire un'orbita attorno alla terra con raggio esattamente pari a quello della terra: un proiettile sulla superficie della terra avente tale velocità può effettivamente uscire dal campo gravitazionale terrestre senza mai farvi più ritorno.

\vspace{1em}
\noindent
\textbf{Osservazione}: Quando si considera un sistema stabile, partendo con una certa energia cinetica, tale sistema converte molteplici volte tale energia in energia potenziale e viceversa; se vi è dissipazione, ovviamente, parte dell'energia viene dispersa (viene dissipata in calore) ad ogni oscillazione fino a quando l'energia totale del sistema descresce a tal punto che che si raggiunge la configurazione stabile di equilibrio, ovvero l'energia potenziale minima.

\vspace{1em}
\noindent
\textbf{Esempio}: Si consideri una molla verticale sulla quale è attaccata una massa:

\begin{figure}[H]
  \centering
  \colorlet{xcol}{blue!70!black}
  \colorlet{darkblue}{blue!40!black}
  \colorlet{myred}{red!65!black}
  \tikzstyle{mydashed}=[xcol,dashed,line width=0.25,dash pattern=on 2.2pt off 2.2pt]
  \tikzstyle{axis}=[->,thick] %line width=0.6
  \tikzstyle{ell}=[{Latex[length=3.3,width=2.2]}-{Latex[length=3.3,width=2.2]},line width=0.3]
  \tikzstyle{dx}=[-{Latex[length=3.3,width=2.2]},darkblue,line width=0.3]
  \tikzstyle{ground}=[preaction={fill,top color=black!10,bottom color=black!5,shading angle=20},
                      fill,pattern=north east lines,draw=none,minimum width=0.3,minimum height=0.6]
  \tikzstyle{mass}=[line width=0.6,red!30!black,fill=red!40!black!10,rounded corners=1,
                    top color=red!40!black!20,bottom color=red!40!black!10,shading angle=20]
  \tikzstyle{spring}=[line width=0.8,blue!7!black!80,snake=coil,segment amplitude=5,segment length=5,line cap=round]
  \tikzset{>=latex} % for LaTeX arrow head
  \tikzstyle{force}=[->,myred,very thick,line cap=round]
  \def\tick#1#2{\draw[thick] (#1)++(#2:0.1) --++ (#2-180:0.2)}

  \begin{tikzpicture}[scale=2]
    \def\H{0.25}     % ceiling height
    \def\W{2.6}      % ceiling width
    \def\h{0.7}      % mass height
    \def\w{0.6}      % mass width
    \def\l{0.5*\y}   % rest length without weight
    \def\dl{0.7*\y}  % rest length with weight
    \def\y{2.4}      % mass y position
    \def\xy{0.38*\W} % mass y position
    \def\F{0.8}      % force magnitude
    \draw[spring,segment length=7.2] (0,0) -- (0,-\y);
    \draw[ground] (-\W/2,0) rectangle++ (\W,\H);
    \draw (-\W/2,0) --++ (\W,0);
    \draw[axis] (-\xy,0) --++ (0,-\y-0.7*\h) node[left] {$y$};
    \draw[axis] ( \xy,0) --++ (0,-\y-0.7*\h) node[right] {$y'$};
    \draw[mydashed] (-\xy,-\l) --++ (2.3*\xy,0);
    \draw[mydashed] (-\xy,-\dl) --++ (2*\xy,0);
    \draw[mydashed] (-0.46*\W,-\y) --++ (0.92*\W,0);
    \tick{-\xy,-\l}{0} node[left] {$0$};
    \tick{-\xy,-\dl}{0} node[left] {$y_0$};
    \tick{ \xy,-\dl}{180} node[right] {$0$};
    \draw[mass] (-\w/2,-\y) rectangle++ (\w,-\h) node[midway] {$m$};
    \draw[force] (0.4*\w,-\y-0.3*\h) --++ (0,1.6*\F) node[pos=0.9,right=0] {$\vb{F}$};
    \draw[force] (0.3*\w,-\y-0.7*\h) --++ (0,-\F) node[above right=0] {$m\vb{g}$};
    \draw[ell] (0.45*\W,0) --++ (0,-\l) node[midway,right=-2] {$\ell_0$};
    %\draw[ell] (-0.4*\W,-0.75*\y) --++ (0,-0.25*\y) node[midway,left=1] {$y_0$};
  \end{tikzpicture}
  \caption{Fisica di una molla verticale}
  \label{fig:fisica_molla_verticale}
\end{figure}

\noindent
Naturalmente, l'energia potenziale del sistema in funzione di $y$ è data da
\[\mathcal{U}(y) = mg y + \frac{1}{2} k y^2\]
e per conoscere il minimo valore di energia potenziale (corrispondente proprio al punto di equilibrio nel grafico di una parabola con concavità rivolta verso l'alto), è sufficiente derivare tale espressione e porre la condizione che la derivata sia nulla (corrispondente a tangente orizzontale):
\[\frac{d}{dy} \mathcal{U}(y) = mg + ky \longrightarrow y = -\frac{mg}{k}\]
che è il medesimo risultato che sarebbe stato ottenuto ragionando con la $2^a$ legge della dinamica, imponendo che la forza peso e la forza elastica siano uguali, ma opposte (giacché la massa è immobile, quindi con $\vec a = 0$):
\[mg=-ky \longrightarrow y = -\frac{mg}{k}\]

\vspace{1em}
\noindent
\textbf{Esercizio $\boldsymbol{1}$}: Il raggio dell'orbita di Marte è $1.52$ volte quello dell'orbita terrestre. Utilizzando la terza legge di Keplero si determini il periodo della rivoluzione di Marte in anni.

\vspace{1em}
\noindent
La terza legge di Keplero afferma che il quadrato del periodo di rivoluzione di un pianeta attorno al sole è proporizionale al cubo della distanza tra i due:
\[T^2 \propto R^3\]
Ma naturalmente si ha che la costante di proporizionalità sarà
\[\frac{T^2}{R^3} = \text{costante}\]
e tale costante dipende solamente dalla massa del sole, per cui è la stessa per tutti i pianeti, pertanto si ha che
\[\frac{T_t^2}{R_t^3} = \frac{T_m^2}{R_m^3}\]
ma sapendo che $R_m=1.52 R_t$ è facile isolare $T_m$, ottenendo
\[T_m = T_t \cdot \left(\frac{R_m}{R_t}\right)^{\frac{3}{2}} = T_t \cdot (1.52)^{\frac{3}{2}}=1.87 \text{ anni}\]

\vspace{1em}
\noindent
\textbf{Esercizio $\boldsymbol{2}$}: Si supponga che il disco da hockey abbia una velocità di modulo $v_0$ nella posizione più bassa.\\
Si determini il valore minimo $v_0$ che consente al disco di completare il suo percorso circolare.

\vspace{1em}
\noindent
Naturalmente la tensione del filo nel punto più alto è nulla. Per cui, affinché si abbia un moto circolare, è necessario che la componente diretta nel piano della forza di gravità sia la forza centripeta necessaria affinché la massa compia un moto circolare uniforme.\\
Pertanto appare evindente come si debba imporre che
\[F_{tx} = mg \cdot \sin(\theta) = m \cdot \frac{v^2}{r}\]
da cui si evince come la velocità nel punto più in alto sia
\[v = \sqrt{rg \cdot \sin(\theta)}\]
E per determinare $v_0$ è sufficiente applicare il teorema di conservazione dell'energia, da cui
\[K_i + \mathcal{U}_i = K_f + \mathcal{U}_f\]
e volendo determinare $K_f$, ossia l'energia cinetica al termine della rotazione della massa nel punto più basso si ottiene:
\[K_f = K_i + \mathcal{U}_i - \mathcal{U}_f \longrightarrow \frac{1}{2} m v_0^2 = \frac{1}{2} m v^2 + mg \cdot (2L \cdot \sin(\theta))\]
per cui, volendo isolare $v_0$ si ottiene che
\[v_0 = \sqrt{5L g \cdot \sin(\theta)}\]
Volendo determninare la tensione del filo nella posizione più bassa, sarà sufficiente imporre che la somma della forza di tensione e della componente nel piano della forza peso devono essere pari alla forca centripeta nel punto più bassso, da cui:
\[F_{tx} + F_T = F_c \longrightarrow F_T + F_{tx} = -m \cdot \frac{v_0^2}{L}\]
essendo diretta verso il centro della circonferenza. Da ciò è immediato capire che
\[F_T = -m \cdot \frac{v_0^2}{L} - mg \cdot \sin(\theta) = -6 mg \cdot \sin(\theta)\]

\newpage
\noindent
\begin{center}
  30 Marzo 2022
\end{center}
\textbf{Esercizio $\boldsymbol{1}$}: Naturalmente, se si devono percorrere $200$ km ad una velocità di $120$ km/h oppure ad una velocità di $135$ km/h, si ha una variazione di tempo impiegato pari a:
\[\Delta t = \frac{200 \text{ km}}{120 \text{ km/h}} - \frac{200 \text{ km}}{135 \text{ km/h}} = 11 \text{ minuti}\]
L'aumento percentuale dal lavoro compiuto dal motore per viaggiare a due diverse velocità, considerando solamente la resistenza dell'aria, è dato semplicemente dal rapporto del quadrato delle due velocità.\\
Infatti è noto che la forza di resistenza dell'aria è proporzionale al quadrato della velocità, da cui:
\[F_r \propto v^2 \longrightarrow F_r = \beta \cdot v^2 \hspace{1em} \text{ in cui } \hspace{1em} \beta=\frac{1}{2}AC_d\rho\]
Pertanto, il lavoro compiuto dalla forza di resistenza è dato da:
\[W = \int \vec F \cdot d \vec r = - F_r \cdot d\]
e per calcolare l'aumento relativo percentuale è da considerare il rapporto seguente:
\[\frac{\vert W_{135} \vert - \vert W_{120} \vert}{\vert W_{120} \vert} \cdot 100 = \frac{(135 \text{ km/h})^2 - (120 \text{ km/h})^2}{(120 \text{ km/h})^2} \cdot 100 = 26.7 \%\]

\vspace{1em}
\noindent
\textbf{Esercizio $\boldsymbol{2}$}: La forza compiuta da una corda bungee segue una funzione non-lineare con la sua estensione:
\[F(x) = k_1 \cdot x + k_2 \cdot x^3\]
dove $x$ è la lunghezza di estensione della corda, mentre $k_1=-204$ N/m e $k_3 = 0.233 \text{ N/m}^3$.\\
Si determini, allora, quanto lavoro è necessario (cioé quanto lavoro deve compiere una forza esterna) per estendere la corda di $15$ m, come segue:
\[W_{c} = \int_0^{15} \vec F \cdot d x =  \int_0^{15} \left(k_1 \cdot x + k_2 \cdot x^3 \right) \cdot d x = \left[\frac{1}{2}k_1x^2 + \frac{1}{4}k_2 x^4 \right]_0^15 \cong 20 \text{ kJ}\]
Inoltre, se $15$ m è l'estensione massima per un'operazione sicura del bungee, si determini la massa massima che può avere una persona che salta legata a questa corda, trascurando la resistenza dell'aria.\\
Naturalmente, essendo in assenza di attriti, è possibile applicare il \textbf{teorema di conservazione dell'energia meccanica}, ovvero:
\[E_i = E_f \longrightarrow K_i + \mathcal{U}_{gi} + \mathcal{U}_{e\i} = K_f + \mathcal{U}_{gf} + \mathcal{U}_{ef}\]
Ma naturalmente l'energia iniziale del sistema è pari a $0$, mentre l'energia potenziale gravitazionale finale è negativa e l'energia potenziale elastica finale è naturalmente data dall'opposto del lavoro calcolato in precedenza, da cui:
\[0 = -mgh + W_c \longrightarrow m = \frac{W_c}{gh} = 136 \text{ kg}\]

\vspace{1em}
\noindent
\textbf{Esercizio $\boldsymbol{3}$}: Una cassetta portautensili con massa $15$ kg è ferma sul ripiano orizzontale liscio di un autocarro. Quand l'autocarrio si mette in modo da fermo con un'accelerazione di $2.5 \text{ m/s}^2$, la cassetta scivola con un coefficiente di attrito cinetico pari a $0.20$.\\
Essa parte dalla quiete nella parte anteriore del pianale e scivola finché non urta contro la sponta posteriore del pianale e scivola finché non urta contro la sponda posteriore del pianale lungo $2.0$ m. Si calcoli, allora, il lavoro compiuto dalla forza di attrito sulla cassetta durante la fase di moto.\\
Naturalmente è ovvio calcolare la forza di attrito come segue:
\[F_k = F_N \cdot \mu_k = mg \cdot \mu_k\]
Il lavoro della forza di attrito, in tale sistema di riferimento, è positivo, in quanto bisogna considerare lo spostamento della cassetta sull'autocarro, per cui bisogna scomporre il moto in quello dell'autocarro e in quello della cassetta, da cui:
\begin{flalign*}
  x_a(t) & = \frac{1}{2}at^2\\
  x_c(t) & = \frac{1}{2} \frac{F_k}{m} \cdot t^2 + l = \frac{1}{2} \mu_k \cdot g \cdot t^2 + l\\
\end{flalign*}
La condizione di risoluzione prevede di determinare l'istante di tempo nel quale le due posizioni saranno identiche, prendendo come punto di riferimento la sponda dell'autocarro. Allora si ha che
\[x_a(t_1) = x_c(t1) \longrightarrow \frac{1}{2} a t_1^2 = l + \frac{1}{2} \mu_k g t_1^2 \longrightarrow t_1 = \sqrt{\frac{2l}{1 - \mu_k \cdot g}}\]
per cui, ora, ovviamente, è facile ottenere la distanza $d$ compiuta dalla cassetta nel moto considerato.\\
Tale distanza è semplicemente data da
\[d=x_c(t_1)-l=\frac{1}{2}a \left(\frac{2l}{a - \mu_k \cdot g}\right) - l=\frac{l \mu_k g}{a - \mu_k \cdot g}\]
Per cui ora, il lavoro compiuto dalla forza di attrito sulla distanza $d$ è:

\vspace{1em}
\noindent
\textbf{Esercizio $\boldsymbol{4}$}: Un bambino sta seduto in cima a un serbatoio cilindrico di raggio $R$. La superficie è molto liscia e il bambino compincia a scivolare con attrito tracurabile. Si determini il valoer dell'angolo in corrispondenza del quale il bambino si stacca dalla superficie cilindrica.\\
Bisogna, quindi, determinare l'angolo $\theta$ tale $F_N=0$. Naturalmente, è possibile applicare a tale problema la seconda legge della dinamica, per cui
\[m \cdot \vec a = \vec F_N + \vec F_t\]
È utile, in tal senso, impiegare due nuovi versori: $\hat r$ in direzione radiale rispetto alla circonferenza, e $\hat \theta$, in direzione angolare rispetto al sistema. Da ciò si evince che:
\[\vec a = a_r \cdot \hat{r} + a_\theta \cdot \hat{\theta}\]
Pertanto, la componente radiale dell'accelerazione si calcola come segue:
\[m \cdot a_r = m \cdot \vec a \cdot \hat{r} = \vec F_N \cdot \hat{r} + \vec{F_t} \cdot \hat{r} = F_n - mg \cdot \cos(\theta)\]
Ma è noto che nel moto circolare uniforme, l'accelerazione entripeta, e quindi in questo caso radiale, è data
\[a_r = - \frac{v^2}{R}\]
Pertanto è possibile ottenere l'equazione seguente:
\[-m \cdot \frac{v^2}{R} = F_n - mg \cdot \cos(\theta)\]
Al fine di isolare $v$ è sufficiente applicare il teorema di conservazione dell'energia, per cui si ottiene che:
\[E_i = E_f \longrightarrow K_i + \mathcal{U}_{i} = K_f + \mathcal{U}_f \longrightarrow K_f = \mathcal{U}_i - \mathcal{U}_f \longrightarrow \frac{1}{2}mv_f^2 = mgh\]
in cui è facile ottenere $h=R-R \cdot \cos(\theta)$. Da ciò si evince come
\[v^2 = 2g R \cdot (1-\cos(\theta))\]
Per cui ora è facile isolare l'angolo $\theta$ cercato, operando come segue:
\[2g R \cdot (1-\cos(\theta))=gR \cdot \cos(\theta) - \frac{F_N \cdot R}{m}\]
Per cui si ottiene come
\[\cos(\theta) = \frac{2}{3} \longrightarrow \theta=48^\circ\]

\newpage
\noindent
\begin{center}
  4 Aprile 2022
\end{center}
\textbf{Osservazione}: Si consideri un blocco di massa $m=500$ g, appoggiato su un piano rimovibile. Il blocco è sospeso con una molla di costante $k=50$ N/m inizialmente a riposo, mostrata di seguito:

\begin{figure}[H]
  \centering
  \colorlet{xcol}{blue!70!black}
  \colorlet{darkblue}{blue!40!black}
  \colorlet{myred}{red!65!black}
  \tikzstyle{mydashed}=[xcol,dashed,line width=0.25,dash pattern=on 2.2pt off 2.2pt]
  \tikzstyle{axis}=[->,thick] %line width=0.6
  \tikzstyle{ell}=[{Latex[length=3.3,width=2.2]}-{Latex[length=3.3,width=2.2]},line width=0.3]
  \tikzstyle{dx}=[-{Latex[length=3.3,width=2.2]},darkblue,line width=0.3]
  \tikzstyle{ground}=[preaction={fill,top color=black!10,bottom color=black!5,shading angle=20},
                      fill,pattern=north east lines,draw=none,minimum width=0.3,minimum height=0.6]
  \tikzstyle{mass}=[line width=0.6,red!30!black,fill=red!40!black!10,rounded corners=1,
                    top color=red!40!black!20,bottom color=red!40!black!10,shading angle=20]
  \tikzstyle{spring}=[line width=0.8,blue!7!black!80,snake=coil,segment amplitude=5,segment length=5,line cap=round]
  \tikzset{>=latex} % for LaTeX arrow head
  \tikzstyle{force}=[->,myred,very thick,line cap=round]
  \def\tick#1#2{\draw[thick] (#1)++(#2:0.1) --++ (#2-180:0.2)}

  \begin{tikzpicture}[scale=2]
    \def\H{0.25}     % ceiling height
    \def\W{2.6}      % ceiling width
    \def\h{0.7}      % mass height
    \def\w{0.6}      % mass width
    \def\l{0.5*\y}   % rest length without weight
    \def\dl{0.7*\y}  % rest length with weight
    \def\y{2.4}      % mass y position
    \def\xy{0.38*\W} % mass y position
    \def\F{0.8}      % force magnitude
    \draw[spring,segment length=7.2] (0,0) -- (0,-\y);
    \draw[ground] (-\W/2,0) rectangle++ (\W,\H);
    \draw (-\W/2,0) --++ (\W,0);
    \draw[axis] (-\xy,0) --++ (0,-\y-0.7*\h) node[left] {$y$};
    \draw[axis] ( \xy,0) --++ (0,-\y-0.7*\h) node[right] {$y'$};
    \draw[mydashed] (-\xy,-\l) --++ (2.3*\xy,0);
    \draw[mydashed] (-\xy,-\dl) --++ (2*\xy,0);
    \draw[mydashed] (-0.46*\W,-\y) --++ (0.92*\W,0);
    \tick{-\xy,-\l}{0} node[left] {$0$};
    \tick{-\xy,-\dl}{0} node[left] {$y_0$};
    \tick{ \xy,-\dl}{180} node[right] {$0$};
    \draw[mass] (-\w/2,-\y) rectangle++ (\w,-\h) node[midway] {$m$};
    \draw[mass] (-\w,-\y-\h-0.05) rectangle++ (2*\w,0.05);
    \draw[force] (0.4*\w,-\y-0.3*\h) --++ (0,1.6*\F) node[pos=0.9,right=0] {$\vb{F}$};
    \draw[force] (0.3*\w,-\y-0.7*\h) --++ (0,-\F) node[above right=0] {$m\vb{g}$};
    \draw[ell] (0.45*\W,0) --++ (0,-\l) node[midway,right=-2] {$\ell_0$};
    %\draw[ell] (-0.4*\W,-0.75*\y) --++ (0,-0.25*\y) node[midway,left=1] {$y_0$};
  \end{tikzpicture}
  \caption{Fisica di una molla verticale}
  \label{fig:fisica_molla_verticale}
\end{figure}

 Il piano viene spostato seguendo due scenari diversi:
\begin{enumerate}
  \item il piano è spostato lentamente verso il basso finché il blocco si stacca dal piano e rimane immobile sospeso alla molla molla.
  \item il piano è rimosso velocemente, in modo da non impedire il moto del blocco.
\end{enumerate}
Solamente nel secondo caso si ha conservazione dell'energia, giacché non si ha nessun intervento di una forza esterna.\\
Per determinare l'estensione della molla nel primo scenario, è sufficiente considerare il diagramma a corpo libero delle forze nella condizione di equilibrio, per cui la forza elastica eguaglia quella gravitazionale, ottenendo
\[-k \Delta y = - mg \longrightarrow \Delta y = \frac{mg}{k} = \frac{0.50 \text{ kg} \cdot 9.8 \text{ m/s}^2}{50 \text{ N/m}} = 9.8 \text{ cm}\]
Per determinare l'estensione massima della molla nel secondo caso, siccome si ha conservazione dell'energia meccanica, ossia $\Delta E=0$, per cui
\[E_i = E_f \longrightarrow K_i + \mathcal{U}_i = K_f + \mathcal{U}_f\]
Ma naturalmente l'energia cinetica iniziale e finale è nulla, così come l'energia potenziale iniziale (data dalla somma di energia potenziale elastica e gravitazionale, fissando lo zero di riferimento in corrispondenza della posizione di equilibrio iniziale della molla).\\
Mentre l'energia potenziale finale è data dalla somma di energia potenziale gravitazionale ed energia potenziale elastica, da cui:
\[\mathcal{U}_{m,f}=-\mathcal{U}_{g,l} \longrightarrow \frac{1}{2}k h^2 = m g h \longrightarrow h = 2 \cdot \frac{mg}{k} \cong 20 \text{ cm}\]
che è esattamente il doppio del risultato ottenuto in precedenza; ciò non sorprende in quanto lasciare cadere una massa attaccata ad una molla, senza intervento di forze esterne, comporta l'innesco di un moto oscillatorio tale che i due estremi di oscillazione siano proprio a distanza doppia dal punto di equilibrio.\\
Se, invece, si considerano, al posto di una molla, due molle, è molto utile considerare il punto di collegamento tra la prima e la seconda molla come un punto materiale di massa nulla, sul quale agiscono due forze uguali ed opposte; infatti, se si considera il punto di connessione tra le due molle, su di esso agisce la forza della prima molla verso l'alto e la forza della seconda molla verso il basso, ottenendo:
\[\vec F_{m_1} + \vec F_{m_2} = m \vec a\]
ma essendo la massa nulla, indipendentemente dall'istante di tempo, le due forze devono essere sempre uguali ed opposte:
\[\vec F_{m_1} + \vec F_{m_2} = 0 \longrightarrow k \cdot l_1 - k \cdot l_2 = 0 \longrightarrow l_1 = l_2\]
Applicando la $3^a$ legge di Newton, poi, si capisce immediatamente come la forza $\vec F_{m_2}$ che agisce sul punto materiale sia esattamente uguale e contraria alla forza che si contrappone alla forza peso della massa, per cui:
\[\vec F_t + \vec F_{m_2} = m \vec a \longrightarrow mg = k \cdot l_2 \longrightarrow l_2 = \frac{mg}{k}\]
per cui l'estensione totale finale è
\[l_1 + l_2 = 2 \cdot \dfrac{mg}{k} = \dfrac{mg}{\dfrac{k}{2}}\]
ossia equivalente ad un'unica molla avente costante di elasticità dimezzata.\\
Nel caso in cui le due molle da considerare siano in parallelo, allora è come se si considerasse una molla con costante di elasticità doppia; a partire dall'equazione seguente
\[\vec F_t + \vec F_{m_1} + \vec F_{m_2} = m \vec a\]
si evince come l'estensione sarà data da:
\[l=\frac{mg}{2k}\]

\vspace{1em}
\noindent
\textbf{Osservazione}: Si osservi che se si considerano delle molle in serie, allora si può sostituire a tali molle un'unica molla di costante elastica
\[\boxed{k_{\text{serie}}=\dfrac{1}{\dfrac{1}{k_1}+\dfrac{1}{k_2}+\dfrac{1}{k_3}+...+\dfrac{1}{k_n}}}\]
mentre se le molle sono poste in parallelo, allora la costante elastica della molla equivalente è
\[\boxed{k_{\text{parallelo}}=k_1+k_2+k_3+...+k_n}\]

\newpage
\section{Moto dei sistemi}
Fino a questo momento, ogni corpo è sempre stato approssimato ad un punto materiale; tuttavia, i \textbf{corpi reali} presentano
\begin{itemize}
  \item una \textbf{distribuzione di massa};
  \item più \textbf{punti di applicazione delle forze};
  \item delle \textbf{forze interne} (tra le parti del medesimo corpo).
\end{itemize}
Ciò che, naturalmente, presenta un importante significato fisico è considerare più masse insieme e le forze che si applicano su ogni elemento di massa, andando così a studiare un \textbf{sistema di masse}.

\vspace{1em}
\subsection{Centro di massa}
Di seguito si espone il significato fisico di \textbf{centro di massa}:

% Tabella per le definizione di concetti, etc...
\vspace{1em}
\rowcolors{1}{black!5}{black!5}
\setlength{\tabcolsep}{14pt}
\renewcommand{\arraystretch}{2}
\noindent
\begin{tabularx}{\textwidth}{@{}|P|@{}}
    \hline
    {\textbf{CENTRO DI MASSA}}\\
    \parbox{\linewidth}{Il \textbf{centro di massa} è un vettore posizione che viene definito come una \textbf{media pesata dei vettori posizione di ogni massa} (tale per cui si può considereare come se tutta la massa sia concetrata in tale punto):
    \[\boxed{\vec r_{\text{CM}} = \frac{\displaystyle{\sum_i m_i \cdot \vec r_i}}{\displaystyle{\sum_i m_i}}}\]
    quindi, essendo $M=\displaystyle{\sum_i m_i}$, il vettore posizione del centro di massa è il rapporto tra il vettore posizione di ogni massa e la massa totale del sistema, ovvero
    \[\boxed{\vec r_{\text{CM}} = \sum_i \frac{m_i}{M} \cdot \vec r_i}\]
    \vspace{-1mm}}\\
    \hline
\end{tabularx}

\vspace{1em}
\noindent
\textbf{Esempio}: Si considerino le due masse seguenti:

\vspace{1em}
\begin{figure}[H]
  \centering
  \begin{tikzpicture}[scale=1]
    \draw (0,0) node[circ](start){} (-1,2) node[circ, above](m_1){} ++(0,0.4) node[]{$m_1$} (2,3) node[circ](m_2){} ++(0,0.4) node[]{$m_2$};
    \draw [-stealth] (start) -- (m_1) node[midway, below left]{$\vec r_1$};
    \draw [-stealth] (start) -- (m_2) node[midway, below right]{$\vec r_2$};
    \draw [dashed] (m_1) -- coordinate[midway](mid) (m_2);
    \draw [-stealth, thick] (0,0) -- (mid) node[midway, above left]{$\vec r_{\text{CM}}$};
  \end{tikzpicture}
  \caption{Centro di massa tra due masse}
  \label{fig:centro_di_massa}
\end{figure}

\vspace{1em}
\noindent
In cui, ovviamente, si ha che
\[\vec r_{\text{CM}} = \frac{m_1 \cdot \vec r_1 + m_2 \cdot \vec r_2}{m_1 + m_2}\]
in cui è facile capire come, a seconda dell'entità delle due masse $m_1$ e $m_2$, si possono formulare le seguenti osservazioni:
\begin{itemize}
  \item se $m_1 >> m_2$, allora $\vec r_{\text{CM}} \to \vec r_1$
  \item se $m_1 << m_2$, allora $\vec r_{\text{CM}} \to \vec r_2$
\end{itemize}

\vspace{1em}
\noindent
Ovviamente, gli oggetti reali, ossia i corpi estesi, sono costituiti da un sistema continuo di punti materiali, per cui è necessario fornire una definizione di centro di massa anche per i corpi estesi, da cui
\[\boxed{\vec r_{\text{CM}} = \sum_{i} \frac{\Delta m_i}{M} \cdot \vec r_i = \frac{1}{M} \cdot \int \vec r \cdot dm}\]
in cui si sommano i contributi di ogni porzione di massa $\Delta m$ tendente all'infinitesimale $\Delta m \to dm$. Inoltre, impiegando il concetto di \textbf{densità}, ossia la massa per unità di volume
\[\rho = \frac{m}{V} \hspace{1em} \text{ossia} \hspace{1em} \rho = \frac{dm}{dV}\]
essa può essere anche definita in funzione dello spazio, ottenendo $\rho(\vec r)$, ossia una \textbf{densità variabile} nello spazio. Da ciò si evince che
\[\rho = \frac{dm}{dV} \longrightarrow dm = \rho \cdot dV\]
Avendo a disposizione tale concetto si ottiene che il calcolo del centro di massa diviene
\[\boxed{\vec r_{\text{CM}} = \frac{1}{M} \cdot \int \vec r \cdot \rho(\vec r) \cdot dV}\]
da cui si può evincere come
\[M=\int \rho(\vec r) \cdot dV\]

\vspace{1em}
\noindent
\textbf{Esempio}: Si consideri cilindro di lunghezza $l$ e di massa $M$, avente una densità uniforme. Allora è evidente come il centro massa sia proprio al centro della bacchetta, in corrispondenza di $\frac{l}{2}$:

\begin{figure}[H]
  \centering
  \colorlet{xcol}{blue!70!black}
  \colorlet{vcol}{green!60!black}
  \colorlet{myred}{red!65!black}
  \colorlet{acol}{red!50!blue!80!black!80}
  \tikzstyle{mass}=[line width=0.6,red!30!black,fill=red!40!black!10,rounded corners=1,
                    top color=red!40!black!20,bottom color=red!40!black!10,shading angle=20]
  \tikzstyle{rvec}=[->,xcol,very thick,line cap=round]
  \tikzstyle{pvec}=[->,myred,very thick,line cap=round]
  \tikzstyle{velocity}=[->,vcol,very thick,line cap=round]
  % CENTER OF MASS 1D
  \begin{tikzpicture}
    \def\xmax{2.5} % max x axis
    \coordinate (O) at (0,0);
    \coordinate (M1) at (-0.75*\xmax,0);
    \coordinate (CM) at (0,0);
    \draw[->,thick] (-\xmax,0) -- (\xmax,0) coordinate (X) node[below] {$x$};
    \draw[thick] (0,0.1) -- (0,-0.1) node[below=0.8,scale=0.9] {$\dfrac{l}{2}$};
    \fill[myred!80!black] (CM) circle(0.08) node[above=0.3] {CM};
  \end{tikzpicture}
  \caption{Centro di massa di una bacchetta}
  \label{fig:centro_di_massa_bacchetta}
\end{figure}

\vspace{1em}
\noindent
Infatti si ha che
\[\vec r_{\text{CM}} = \frac{1}{M} \cdot \int \vec r \cdot \rho(\vec r) \cdot dV\]
Essendo questa una equazione vettoriale, è possibile considerarne la sola componente orizzontale (supponendo che la bacchetta sia così fine che il suo spessore sia trascurabile), ottenendo
\[x_{\text{CM}}=\frac{1}{M} \cdot \int x \cdot \rho(\vec r) \cdot dV\]
Definendo con $A$ l'area della sezione trasversale della bacchetta e considerando una porzione $\Delta x$ di bacchetta, è ovvio che il volume di tale porzione sarà dato da $\Delta V = \Delta x \cdot A$ (in quanto il volume di tutta la bacchetta è $V=l \cdot A$). Inoltre, \textbf{essendo la densità uniforme}, si ha che $\rho = \frac{M}{V}$, ma per quanto appena osservato si evince che:
\[\rho \cdot \Delta V = \frac{M}{lA} \cdot \Delta x \cdot A = \frac{M}{l} \cdot \Delta x\]
in cui $\frac{M}{l}$ prende il nome di \textbf{densità lineare}. Considerando una porzione infinitesimale di bacchetta, si ottiene che
\[\rho(\vec r) \cdot dV = \frac{M}{l} dx\]
per cui l'integrale visto in partenza diviene
\[x_{\text{CM}} = \frac{1}{M} \cdot \int x \cdot \rho(\vec r) \cdot dV = \frac{1}{M} \cdot \int x \cdot \frac{M}{l} dx = \frac{1}{l} \cdot \int_0^l x \cdot dx = \frac{1}{l} \cdot \left(\frac{1}{2}l^2\right) = \frac{l}{2}\]
come ci si aspettava; si osservi, inoltre, come tale risultato sia totalmente indipendente dal sistema di coordinate assunto.

\vspace{1em}
\noindent
\textbf{Osservazione}: Si consideri una distribuzione di masse (approssimate a dei punti materiali) non necesariamente uniforme (per esempio, potrebbe accadere che vi sia uno squilibrio di distribuzione di masse tra due parti distinte di uno stesso corpo); allora, è possibile calcolare il centro di massa di differenti parti dello stesso corpo e poi sommarle fra loro, ovvero
\[\vec r_{\text{CM}} = \sum_i \frac{m_i}{M} \cdot \vec r_i = \sum_{i \in A} \frac{m_i}{M} \cdot \vec r_i + \sum_{i \in B} \frac{m_i}{M} \cdot \vec r_i\]
ma essendo, ovviamente
\[M_{A} = \sum_{i \in A} m_i \hspace{1em} \text{e} \hspace{1em} M_{B} = \sum_{i \in B} m_i\]
si può scrivere che
\[\vec r_{\text{CM}} = \sum_{i \in A} \frac{m_i}{M} \cdot \vec r_i + \sum_{i \in B} \frac{m_i}{M} \cdot \vec r_i = \frac{1}{M} \cdot M_A \cdot \underbrace{\sum_i \frac{m_i}{M_A} \cdot \vec r_i}_{\vec r_{\text{CM}_A}} + \frac{1}{M} \cdot M_B \cdot \underbrace{\sum_i \frac{m_i}{M_B} \cdot \vec r_i}_{\vec r_{\text{CM}_B}}\]
pertanto si ottiene che, essendo $M=M_A+M_B$:
\[\boxed{\vec r_{\text{CM}} = \frac{M_A \cdot \vec r_{\text{CM}_A} + M_B \cdot \vec r_{\text{CM}_A}}{M}}\]
ciò singifica che se si riesce a calcolare il centro di massa di singoli elementi individuali di un sistema, allora il centro di massa del sistema non è altro che la media pesata, rispetto alle masse di ciascuno, di tali centri di massa.

\vspace{1em}
\noindent
\textbf{Esempio}: Si consideri un disco di densità uniforme con un buco di forma circolare adiacente alla circonferenza esterna, mostrato di seguito:

\begin{figure}[H]
  \centering
  \begin{tikzpicture}
    \node[circle,draw,fill=blue!25,minimum width=4cm](c1) at (0,0){};
    \node[circle,draw,fill=white,minimum width=2cm]  (c2) at (3,0){};
  \end{tikzpicture}
  \caption{Disco con foro circolare}
  \label{fig:disco_foro_circolare}
\end{figure}

\noindent
Al fine di determinare il centro di massa di tale oggetto, si dovrebbe procedere a calcolare l'integrale della figura, andando ad escludere, ovviamente, i punti del foro.\\
Tuttavia, con il risultato precedentemente ottenuto diviene molto più semplice; è ovvio che il centro di massa di una circonferenza è collocato nel centro della stessa, per semplice simmetria, si può considerare il concetto di massa negativa, per cui il centro di massa del disco forato si può ottenere come differenza pesata del centro di massa del disco pieno meno il centro di massa del foro, avente, appunto, massa negativa.\\
Ovviamente, il centro di massa del disco forato sarà collocato sull'asse passante per i due centri dei due cerchi raffigurati in Figura \ref{fig:disco_foro_circolare}, per cui è sufficiente considerare solamente la componente $x$ del centro massa
\[x_{\text{CM}} = \frac{M_D \cdot x_{\text{CM}_D} + M_F \cdot x_{\text{CM}_F}}{M_D+M_F}\]
in cui, per quanto detto, $M_F<0$. Pertanto, se $x_{\text{CM}_D}=0$ e $x_{\text{CM}_F}=\frac{R}{2}$ (ponendo il raggio del disco più grande pari a $R$) e la densità $\rho$ è uniforme, si ha che
\[M_D=\rho \cdot V = \rho \cdot \pi R^2 \cdot h \hspace{1em} \text{e} \hspace{1em} M_F=-\rho \cdot V = -\frac{1}{4} \cdot \rho \cdot \pi R^2 \cdot h\]
ciò significa che
\[x_{\text{CM}} = \frac{0 \cdot M_D + \dfrac{R}{2} \cdot M_F}{M_D+M_F} = \frac{R}{2} \cdot \frac{M_D}{M_D+M_F} = -\frac{R}{2} \cdot \frac{\displaystyle{\frac{1}{4} \rho \cdot \pi R^2 \cdot h}}{\displaystyle{\frac{3}{4} \rho \cdot \pi R^2 \cdot h}} = -\frac{R}{6}\]

\newpage
\noindent
\begin{center}
  5 Aprile 2022
\end{center}
\textbf{Esempio}: Si consideri la seguente figura:

\begin{figure}[H]
  \centering
  \begin{tabular}{|c|c|c|c|c|c|c|}
    \hline
    $ $ & $ $ & $ $ & $ $ & $ $ & $ $ & $ $\\
    \hline
    $ $ & \cellcolor{blue!25}$ $ & \cellcolor{blue!25}$ $ & \cellcolor{blue!25}$ $ & \cellcolor{blue!25}$ $ & \cellcolor{blue!25}$ $ & $ $\\
    \hline
    $ $ & \cellcolor{blue!25}$ $ & $ $ & $ $ & $ $ & $ $ & $ $\\
    \hline
    $ $ & \cellcolor{blue!25}$ $ & $ $ & $ $ & $ $ & $ $ & $ $\\
    \hline
    $ $ & \cellcolor{blue!25}$ $ & \cellcolor{blue!25}$ $ & \cellcolor{blue!25}$ $ & $ $ & $ $ & $ $\\
    \hline
    $ $ & \cellcolor{blue!25}$ $ & $ $ & $ $ & $ $ & $ $ & $ $\\
    \hline
    $ $ & \cellcolor{blue!25}$ $ & $ $ & $ $ & $ $ & $ $ & $ $\\
    \hline
    $ $ & \cellcolor{blue!25}$ $ & $ $ & $ $ & $ $ & $ $ & $ $\\
    \hline
    $ $ & \cellcolor{blue!25}$ $ & $ $ & $ $ & $ $ & $ $ & $ $\\
    \hline
    $ $ & $ $ & $ $ & $ $ & $ $ & $ $ & $ $\\
    \hline
  \end{tabular}
  \caption{Oggetto di cui calcolare il centro di massa}
  \label{fig:calcolo_centro_massa_figura}
\end{figure}

\noindent
Al fine di calcolarne il centro di massa, è utilie scomporre la stessa in tre oggetti di cui è facile determinare il centro di massa, come mostrato di seguito:

\begin{figure}[H]
  \centering
  \begin{tabular}{|c|c|c|c|c|c|c|}
    \hline
    $ $ & $ $ & $ $ & $ $ & $ $ & $ $ & $ $\\
    \hline
    $ $ & \cellcolor{red!25}$ $ & \cellcolor{blue!25}$ $ & \cellcolor{blue!25}$ $ & \cellcolor{blue!25}$ $ & \cellcolor{blue!25}$ $ & $ $\\
    \hline
    $ $ & \cellcolor{red!25}$ $ & $ $ & $ $ & $ $ & $ $ & $ $\\
    \hline
    $ $ & \cellcolor{red!25}$ $ & $ $ & $ $ & $ $ & $ $ & $ $\\
    \hline
    $ $ & \cellcolor{red!25}$ $ & \cellcolor{orange!25}$ $ & \cellcolor{orange!25}$ $ & $ $ & $ $ & $ $\\
    \hline
    $ $ & \cellcolor{red!25}$ $ & $ $ & $ $ & $ $ & $ $ & $ $\\
    \hline
    $ $ & \cellcolor{red!25}$ $ & $ $ & $ $ & $ $ & $ $ & $ $\\
    \hline
    $ $ & \cellcolor{red!25}$ $ & $ $ & $ $ & $ $ & $ $ & $ $\\
    \hline
    $ $ & \cellcolor{red!25}$ $ & $ $ & $ $ & $ $ & $ $ & $ $\\
    \hline
    $ $ & $ $ & $ $ & $ $ & $ $ & $ $ & $ $\\
    \hline
  \end{tabular}
  \caption{Oggetto di cui calcolare il centro di massa diviso in parti}
  \label{fig:calcolo_centro_massa_figura_divisa_parti}
\end{figure}

\vspace{1em}
\noindent
Ecco che in questo caso è facile capire il centro di masssa di ciascuna delle figure:
\begin{itemize}
  \item Centro di massa \textcolor{blue!25}{blu}: $r_b=(3,7.5)$;
  \item Centro di massa \textcolor{red!25}{rosso}: $r_r=(0.5,4)$;
  \item Centro di massa \textcolor{orange!25}{arancione}: $r_a=(2,4.5)$;
\end{itemize}
Inoltre, sapendo che $m_b=4$, $m_r=8$ e $m_a=2$, si possono facilmente determinare le coordinate del centro di massa:
\[x_{\text{CM}}=\frac{x_1 \cdot m_1 + x_2 \cdot m_2 + x_3 \cdot m_3}{m_1+m_2+m_3} = \frac{4 \cdot 3 + 8 \cdot 0.5 + 2 \cdot 2}{4+8+2}=\frac{20}{14}\cong 1.4\]
e
\[y_{\text{CM}}=\frac{y_1 \cdot m_1 + y_2 \cdot m_2 + y_3 \cdot m_3}{m_1+m_2+m_3} = \frac{4 \cdot 7.5 + 8 \cdot 4 + 2 \cdot 4.5}{4+8+2}=\frac{71}{14}\cong 5\]

\vspace{1em}
\subsection{Moto del centro di massa}
Volendo descrivere il moto del centro di massa, si considera la definizione dello stesso, ovvero
\[\vec{r_{\text{CM}}} = \frac{\displaystyle{\sum_{i} \vec r_i \cdot m_i}}{M}\]
e volendo conoscere la velocità del centro di massa, è sufficiente considerare la derivata rispetto al tempo di tale vettore, ottenendo:
\[\vec v_{\text{CM}} = \frac{1}{M} \cdot \frac{d}{dt} \sum_i \vec r_i \cdot m_i = \frac{1}{M} \cdot \sum_i m_i \cdot \vec v_i\]
ovvero la media pesata delle velocità di ciascuna massa. Analogamente per l'accelerazione si ottiene
\[\vec a_{\text{CM}} = \frac{1}{M} \cdot \frac{d}{dt} \sum_i \vec r_i \cdot m_i = \frac{1}{M} \cdot \sum_i m_i \cdot \vec a_i\]
ovvero la media pesata delle accelerazioni di ciascuna massa.

\vspace{1em}
\subsection{Dinamica del centro di massa}
Si consideri un punto materiale $i$ all'interno di un assieme; allora, per la seconda legge di Newton, la forza che agisce su tale punto è la risultante delle forze applicate sul punto stesso, ovvero
\[m_i \cdot \vec a_i = \sum_i \vec F_i = \vec F_{a,i} + \vec F_{b,i} + \vec F_{c,i} + ...\]
A tal proposito, diviene utile effettuare una significativa distinzione tra \textbf{forze interne al sistema} e \textbf{forze esterne al sistema}:
\begin{itemize}
  \item Le forze esterne al sistema si denotano con
  \[\vec F_{\text{ext},i}\]
  ovvero la risultante di tutte le forze esterne che si applicano sulla massa $m_i$ (approssimata ad un punto materiale);

  \item Le forze interne al sistema, invece, dal momento che originano da altri punti del sistema, si denotano con
  \[\vec F_{j,i}\]
  ovvero la forza causata dal punto $j$ sul punto $i$, ossia la massa $m_i$ (approssimata ad un punto materiale);
\end{itemize}
Ciò permette di considerare che:
\[\boxed{m_i \cdot \vec a_i = \vec F_{\text{ext},i} + \sum_j \vec F_{j,i}}\]

\vspace{1em}
\noindent
\newcommand{\tikzmark}[1]{\tikz[overlay,remember picture] \node (#1) {};}
\textbf{Osservazione}: Si osservi, allora, che volendo determinare l'accelerazione del centro di massa si ottiene
\[\vec a_{\text{CM}} = \frac{1}{M} \cdot \sum_i m_i \cdot \vec a_i = \frac{1}{M} \cdot \sum_i \left( \vec F_{\text{ext},i} + \sum_j \vec F_{j,i} \right) = \frac{1}{M} \cdot \left(\sum_i \vec F_{\text{ext},i} + \sum_i \sum_j \vec F_{i,j}\right)\]
Ma applicando la $3^a$ legge della dinamica, è immmediato osservare come:
\begin{equation*}
  \sum_i \sum_j \vec F_{i,j} = \tikzmark{a} \vec F_{2,1} + \vec F_{3,1} + \vec F_{4,1} + ... + \vec F_{1,2} \tikzmark{b} + \vec F_{3,2} + \vec F_{4,2} + ... = 0
  \begin{tikzpicture}[overlay,remember picture,out=315,in=225,distance=0.7cm]
      \draw[<->,blue,shorten >=3pt,shorten <=3pt] (a.east) to node[midway,below](n){opposti} (b.west);
  \end{tikzpicture}
\end{equation*}
Pertanto, essendo tale somma nulla, in quanto $F_{i,j}=-F_{j,i}, \forall i \neq j$, si ottiene semplicemente che
\[\boxed{M \cdot \vec a_{\text{CM}} = \sum_i \vec F_{\text{ext},i}}\]
ovvero la forza risultante applicata al centro di massa di un corpo dipende unicamente dalle forze esterne e non da quelle interne: ciò giustifica perfettamente l'approssimazione di un corpo ad un punto materiale (\textbf{se questo è il centro di massa} del corpo stesso) in quanto il centro di massa segue la medesima fisica, ovvero segue la $2^a$ legge della dinamica come se fosse un punto materiale; in altre parole, \textbf{il moto del centro di massa è determinato solamente dalle forze esterne ed è indipendente dal moto relativo delle singole masse}.

\vspace{1em}
\noindent
\textbf{Osservazione}: Nel caso di un corpo che viene lanciato in aria e lasciato cadere, trascurando l'attrito dell'aria, si ha che:
\[\vec F_{\text{ext}} = M \cdot \vec a_{\text{CM}} = M \cdot \vec g = \sum_{i} m_i \cdot \vec g\]
Inoltre, se la somma delle forze esterne è nulla, ossia $\vec F_{\text{ext}}=0$, allora ciò significa che l'accelerazione del centro di massa è nulla ($\vec a_{\text{CM}} = 0$) e quindi la velocità $\vec v_{\text{CM}} = $ costante (per cui se si considera un sistema inizialmente fermo e poi avente una certa accelerazione che non dipende da forze esterne, come nel caso di una molla, il centro di massa continua a rimanere lo stesso, nella medesima posizione).

\vspace{1em}
\subsection{Conservazione della quantità di moto}
Di seguito si espone il \textbf{principio di consdervazione della quantità di moto}:

% Tabella per le definizione di concetti, etc...
\vspace{1em}
\rowcolors{1}{black!5}{black!5}
\setlength{\tabcolsep}{14pt}
\renewcommand{\arraystretch}{2}
\noindent
\begin{tabularx}{\textwidth}{@{}|P|@{}}
    \hline
    {\textbf{CONSERVAZIONE DELLA QUANTITÀ DI MOTO}}\\
    \parbox{\linewidth}{Dal momento che
    \[\vec F_{\text{ext}} = 0 = \sum_i m_i \cdot \vec a_i = \sum_i m_i \cdot \frac{d}{dt} \cdot \vec v_i = \frac{d}{dt} \cdot \sum_i m_i \cdot \vec v_i\]
    Pertanto, siccome la derivata nel tempo della quantità considerata è nulla, allora tale quantità viene conservata, ovvero
    \[\frac{d}{dt} \sum_i m_i \cdot \vec v_i = 0 \longrightarrow \sum_i m_i \cdot v_i = \text{ costante}\]
    Ciò porta a definire il concetto di \textbf{quantità di moto} (dall'inglese \emph{momentum}):
    \[\boxed{\vec p = m \cdot \vec v}\]
    per cui se la \textbf{risultante delle forze su un sistema è nulla}, allora la \textbf{quantità di moto del sistema si conserva}.\vspace{3mm}}\\
    \hline
\end{tabularx}

\newpage
\noindent
\textbf{Osservazione}: La $2^a$ legge di Newton, originariamente come il fisico l'aveva definita, recitava:\\\\
\quotes{\emph{Il cambiamento di moto è proporzionale alla forza mmotrice impressa, ed avviene lungo la linea retta secondo la quale la forza è stata impressa.}}\\\\
in cui è evidente come Newton non parlasse espressamene di accelerazione proporzionale alla forza motrice impressa, ma proprio di \emph{cambiamento di moto}, intendendo, più propriamente, la derivata nel tempo della quantità di moto. Ecco che allora la $2^a$ legge di Newton può essere, reinterpretata alla luce della definizione di quantità di moto come segue
\[\boxed{\Delta \vec p \propto \vec F}\]
che ha perfettamente senso in quanto è noto che
\[\vec F = m \vec a = \frac{d}{dt} \vec p\]
per cui la $2^a$ legge della dinamica diviene
\[\boxed{F=\frac{d\vec p}{dt}}\]
che può riguardare tanto la \textbf{derivata della velocità} nel tempo quanto la \textbf{derivata della massa} nel tempo (come nel caso del moto dei razzi): ciò garantisce una maggiore generalizzazione del secondo principio della dinamica.

\newpage
\noindent
\begin{center}
  7 Aprile 2022
\end{center}
Com'è noto, il centro di massa è così definito:
\[\vec r_{\text{CM}} = \frac{1}{M} \cdot \sum_i \vec r \cdot m_i = \int \vec r \cdot g(\vec r) \cdot dV\]
Naturalmente, derivando nel tempo tale quantità si ottiene
\[\vec v_{\text{CM}} = \frac{1}{M} \cdot \sum_i \vec v_i \cdot m_i\]
e derivando nuovamente
\[\vec a_{\text{CM}} = \frac{1}{M} \cdot \sum_i \vec a_i \cdot m_i\]
Inoltre, è noto che la dinamica del centro di massa è totalmente indipendente dalle forze interne, ovvero:
\[M \cdot \vec a_{\text{CM}} = \sum F_{\text{ext}}\]
Inoltre, interpretando anche la seconda equazione come segue:
\[M \cdot \vec v_{\text{CM}} = \sum_i m_i \cdot \vec v_i\]
in cui, essendo $\vec p = m \cdot \vec v$ la \textbf{quantità di moto}, si ha che:
\[\boxed{\vec p_{\text{CM}} = \sum_i \vec p_i}\]
Pertanto, se
\[\sum \vec F_{\text{ext}} = 0\]
allora si ha che la quantità di moto del centro di massa si conserva, ovvero $\vec p_{\text{CM}}$ è costante: tale risultato prende il nome di \textbf{conservazione della quantità di moto}.

\vspace{1em}
\noindent
\textbf{Osservazione}: Si osservi che considerando solamente la Luna e il suo moto di rivoluzione attorno alla Terra, giacché sulla Luna agisce la forza di attrazione gravitazionale, essa non conserva la sua quantità di moto; tuttavia, se si considera il sistema Terra-Luna (ingnorando anche il Sole), allora la forza di gravità è una forza interna, per cui il centro di massa conserva la propria quantità di moto e quindi si muove a velocità costante.

\vspace{1em}
\subsection{Seconda legge di Newton reinterpretata}
Si considerino due masse $m_A=m$ e $m_B=3m$ collegate fra di loro da una molla:

\begin{figure}[H]
  \centering
  \colorlet{xcol}{blue!70!black}
  \colorlet{darkblue}{blue!40!black}
  \colorlet{myred}{red!65!black}
  \tikzstyle{mydashed}=[xcol,dashed,line width=0.25,dash pattern=on 2.2pt off 2.2pt]
  \tikzstyle{axis}=[->,thick] %line width=0.6
  \tikzstyle{ell}=[{Latex[length=3.3,width=2.2]}-{Latex[length=3.3,width=2.2]},line width=0.3]
  \tikzstyle{dx}=[-{Latex[length=3.3,width=2.2]},darkblue,line width=0.3]
  \tikzstyle{ground}=[preaction={fill,top color=black!10,bottom color=black!5,shading angle=20},
                      fill,pattern=north east lines,draw=none,minimum width=0.3,minimum height=0.6]
  \tikzstyle{mass}=[line width=0.6,red!30!black,fill=red!40!black!10,rounded corners=1,
                    top color=red!40!black!20,bottom color=red!40!black!10,shading angle=20]
  \tikzstyle{spring}=[line width=0.8,blue!7!black!80,snake=coil,segment amplitude=5,segment length=5,line cap=round]
  \tikzset{>=latex} % for LaTeX arrow head
  \tikzstyle{force}=[->,myred,very thick,line cap=round]
  \def\tick#1#2{\draw[thick] (#1)++(#2:0.1) --++ (#2-180:0.2)}

  \begin{tikzpicture}[scale=2]
    \def\H{1}    % wall height
    \def\T{0.3}  % wall thickness
    \def\W{2.6}  % ground length
    \def\D{0.25} % ground depth
    \def\h{0.6}  % mass height
    \def\w{0.7}  % mass width
    \def\x{1.6}  % mass x position
    \draw[spring] (\x/2+0.5+\w,\h/2) --++ (\x/2,0);
    %\draw[ground] (0,0) |-++ (-\T,\H) |-++ (\T+\W,-\H-\D) -- (\W,0) -- cycle;
    %\draw (0,\H) -- (0,0) -- (\W,0);
    \draw[mass] (\x/2+0.5,0)  rectangle++ (\w,\h) node[midway] {$m$};
    \draw[mass] (\x+\w+0.5,0) rectangle++ (\w,\h) node[midway] {$3m$};
    \draw[-stealth] (\x/2+0.5,\h/2) -- ++(-0.5,0);
    \draw[-stealth] (\x+2*\w+0.5,\h/2) -- ++(0.5,0);
    \draw[-stealth] (\w,-\h/2) -- ++(5*\w,0);
    \draw (\x/2+\w/2+0.5,-\h/4) -- ++(0,-\h/2) node[below]{$x_A$};
    \draw (\x+3*\w/2+0.5,-\h/4) -- ++(0,-\h/2) node[below]{$x_B$};
  \end{tikzpicture}
  \caption{Masse collegate da una molla}
  \label{fig:masse_collegate_molla}
\end{figure}

\noindent
Inizialmente, la molla compressa è a riposo, per cui la quantità di moto iniziale è nulla
\[\vec p_{i}=0\]
Successivamente, la molla, estendendosi, innesca il moto delle due masse, ma la quantità di moto rimane la stessa, ovvero
\[\vec p_f=\vec p_i=0\]
dal momento che su tale sistema non agiscono forze esterne (o meglio, la risultante delle forze esterne impresse, forza di gravità e forza normale, è nulla)
\[\sum \vec F_{\text{ext}} = 0\]
Pertanto, a qualsiasi istante di tempo $t$ si ha che
\[\vec p = 0 = \vec p_A + \vec p_B = m_A \cdot \vec v_A + m_B \cdot \vec v_B = m \cdot \vec v_A + 3m \cdot \vec v_B \longrightarrow \vec v_A =-3 \vec v_B\]

\vspace{1em}
\noindent
\textbf{Esercizio}: Un nucleo di radio $(^{226}\text{Ra})$ inizialmente a riposo si decompone in un nucleo di radon $(^{222}\text{Rn})$ e una particella alfa (nucleo di $^{4}\text{He}$). Se l'energia cinetica della particella alfa è di $6.72 \times 10^{-13}$ J, si determini quali sono il modulo della velocità di rinculo dell'atomo di radon e la sua energia cinetica.\\
Essendo il sistema inizialmente a riposo è evidente come la quantità di moto iniziale sia nulla, ovvero $\vec p_i = 0$; applicando, poi, il principio di conservazione della quantità di moto, si ha che $\vec p_i = \vec p_f=0$, dal momento che
\[\sum F_{\text{ext}}=0\]
Ma dai dati del problema è evidente come
\[\vec p_f=\vec p_{\text{Rn}} + \vec p_{\alpha}=0 \longrightarrow \vec p_{\text{Rn}} = - \vec p_\alpha\]
Inoltre, è noto che
\[K_\alpha = \frac{1}{2}m_\alpha v_\alpha^2 = \frac{1}{2}\frac{1}{m_\alpha} p_\alpha^2\]
Dovendo determinare la quantità di moto $p_\alpha$ si ottiene che
\[\left \vert p_\alpha \right \vert = \sqrt{2 m_\alpha K_\alpha} = \left \vert p_{\text{Rm}} \right \vert\]
per cui
\[v_{\text{Rn}}=\frac{p_{\text{Rn}}}{m_{\text{Rn}}} = \frac{1}{m_{\text{Rn}}} \cdot \sqrt{2 m_\alpha K_\alpha}\]
e per determinare l'energia cinetica si esegue il calcolo seguente
\[K_{\text{Rn}} = \frac{1}{2}m_{\text{Rn}} \cdot \left( \frac{1}{m_{\text{Rn}}} \cdot \sqrt{2 m_\alpha K_\alpha} \right)^2 = \frac{m_\alpha}{m_{\text{Rn}}} \cdot K_\alpha\]
che è un risultato fondamentale da considerare, in quanto è stato ottenuto senza sapere nel dettaglio ciò che è successo tra l'istante iniziale e l'istante finale presi in esame.

\vspace{1em}
\noindent
\textbf{Esercizio}: Una ragazza di massa $45$ kg si tuffa da una barca di massa $1000$ kg, allontanandosi da essa con una velocità orizzontale di $5.2$ m/s. Ammettendo che la barca fosse inizialmente in quiete e libera di muoversi nell'acqua, si determini con quale velocità essa si mette in movimento.\\
Dal momento che la quantità di moto è un vettore, si può considerare solamente la conservazione in una direzione (come nel caso della caduta libera, in cui si ha conservazione nel movimento orizzontale, ma non verticale, a causa della forza di gravità), a differenza dell'energia, che o si conserva o non si conserva. Ciò che è utile in questo caso è considerare la consdervazione della quantità di moto orizzontale, per cui:
\[\vec p_i=0 \hspace{1em} \text{e} \hspace{1em} \vec p_f=\vec p_r + \vec p_b=0\]
ovvero la quantità di moto finale è data dalla somma della quantità di moto della ragazza e della barca. Dal momento che $p_{i_x}=p_{f_x}=0$ si può facilmente capire come
\[p_{r_x}=-p_{b_x} \longrightarrow m_r v_{r_x} = -m_{b} v_{b_x}\]
In questo caso, però, viene fornita una \textbf{velocità relativa}, ossia la velocità misurata da un sistema di riferimento in cui uno dei due corpi è a riposo, ovvero, in generale
\[\boxed{\vec v_{\text{rel}} = \vec v_B - \vec v_A}\]
si può considerare $\vec v_{\text{rel}} = \vec v_r - \vec v_b$, da cui si evince come
\[
  \left\{
  \rowcolors{1}{white}{white}
  \begin{array}{l}
    m_r v_{r_x} = -m_b v_{b_x}\\
    \vec v_{\text{rel}} = \vec v_r - \vec v_b
  \end{array}
  \right.
\]
Da cui si evince come $v_{r_x}=-\dfrac{m_b}{m_r} \cdot v_{b_x}$ e impiegando tale risultato si ottiene che la velocità relativa diviene
\[v_{\text{rel}_x} = -\frac{m_b}{m_r} \cdot v_{b_x} - v_{b_x} = -v_{b_x} \cdot \left(1+\frac{m_b}{m_r}\right)\]
Pertanto, ora, si possono determinare sia la velocità della barca che quella della ragazza:
\[v_{b_x}=-\frac{v_{\text{rel}_x} \cdot m_r}{m_r+m_b} \hspace{1em} \text{e} \hspace{1em} v_{r_x}=\frac{v_{\text{rel}_x} \cdot m_b}{m_r+m_b}\]

\vspace{1em}
\subsection{Impulso}
Si consideri un uovo lasciato cadere prima su un materasso e poi su un piano rigido: nel primo caso l'uovo non si rompre, mentre nel secondo si rompe, nonostante la quantità di moto iniziale e finale (prima e dopo l'impatto) nei due casi sia la stessa; la ragione fisica di tale fenomeno sta nel concetto di \textbf{impulso}: nel primo caso la forza che rallenta la caduta del corpo agisce in un tempo superiore rispetto al secondo caso.\\
Infatti, dalla seconda legge di Newton generalizzata si ha che:
\[\vec F = \frac{d \vec p}{dt}\]
Procedendo all'integrazione di ambo le parti si ottiene che:
\[\int \vec F \cdot dt = \int \frac{d \vec p}{dt} \cdot dt = \vec p_f - \vec p_i = \Delta \vec p\]
Ciò permette di capire come la variazione della quantità di moto sia proprio l'area sottesa al grafico forza-tempo: nonostante nei due casi l'area sia la stessa, il picco della forza è necessariamente maggiore nel secondo caso che nel primo:

\vspace{2em}
\noindent
\rowcolors{1}{white}{white}
\begin{tabularx}{\textwidth}{P}
  {
      \centering
      \begin{tikzpicture}
        \begin{axis}[
          grid=both,
          axis lines = middle,
          xlabel = \(t\),
          ylabel = {\(F\)},
          legend pos=outer north east,
          ymajorgrids=true,
          xmajorgrids=true,
          grid style=dashed,
          ymin=0,
        ]
      \addplot[
        domain=0:15,
        samples=100,
        color=orange,
      ]
      {-0.2*(x-5)^2+2};
      \addlegendentry{Uovo su materasso}
      \addplot[
        domain=0:15,
        samples=100,
        color=violet,
      ]
      {-2*(x-3)^2+5};
      \addlegendentry{Uovo su pavimento}
      \end{axis}
      \end{tikzpicture}
    }
\end{tabularx}

\vspace{1em}
\noindent
È stato, così facendo, definito implicitamente il \textbf{concetto di impulso}:
\[\boxed{\vec j = \Delta \vec p = \int \vec F \cdot dt}\]

\vspace{1em}
\subsection{Urti}
Il concetto di urto risulta fondamentale nello studio e nella comprensione della fisica moderna: tutta la fisica moderna può essere interpretata come un'infinità di interazioni, come una sequenza di urti, sia a livello microscopico che macroscopico:

\begin{figure}[H]
  \colorlet{xcol}{blue!70!black}
  \colorlet{vcol}{green!60!black}
  \colorlet{myred}{red!65!black}
  \colorlet{acol}{red!50!blue!80!black!80}
  \tikzstyle{mass}=[line width=0.6,red!30!black,fill=red!40!black!10,rounded corners=1,
                    top color=red!40!black!20,bottom color=red!40!black!10,shading angle=20]
  \tikzstyle{ground}=[preaction={fill,top color=blue!50!black!10,bottom color=blue!50!black!5,shading angle=20},
                      fill,pattern color=blue!20!black,pattern=north east lines,draw=none,minimum width=0.3,minimum height=0.6]
  \tikzstyle{velocity}=[->,vcol,very thick,line cap=round]

  \tikzset{
    pics/collision/.style={
      code={
        \draw[line width=0.5*#1,orange,fill=yellow]
          (0:0.20*#1) -- (30:0.06*#1) -- (50:0.25*#1) -- (80:0.10*#1) -- (105:0.32*#1) --
          (140:0.08*#1) -- (170:0.25*#1) -- (190:0.08*#1) -- (220:0.25*#1) --
          (250:0.08*#1) -- (270:0.24*#1) -- (300:0.08*#1) -- (320:0.25*#1) -- (340:0.09*#1) -- cycle;
    }},
    pics/collision/.default=1,
  }
\centering
% COLLISION 2D together
\begin{tikzpicture}[scale=1.2]
    % COLLISION 2D before
    \def\w{0.8}    % mass width
    \def\h{0.6}    % mass height
    \def\d{1.5}    % distance
    \def\xmax{2.6} % max x axis
    \def\ymax{2.2} % max y axis
    \def\d{1.9}    % distance
    \def\v{0.7}    % mass velocity
    \def\ang{40}   % angle after
    \coordinate (O) at (0,0);
    \coordinate (M1) at (-0.7*\xmax,0);
    \coordinate (M2) at (0,-0.7*\ymax);
    \coordinate (M')  at (\ang:\d); %(\d-0.07*\w,0.7*\d+0.23*\w);
    \coordinate (M1') at ($(\ang:\d)+(-0.4*\w,0)$) ; %(\d-0.6*\w,0.7*\d);
    \coordinate (M2') at ($(\ang:\d)+(0.4*\w,-0.3*\w)$); %(\d+0.2*\w,0.7*\d-0.3*\w);
    \draw[dashed] (0,0) -- (M');
    \draw[->,thick] (0,-\ymax) -- (0,\ymax) node[left] {$y$};
    \draw[->,thick] (-\xmax,0) -- (\xmax,0) coordinate (X) node[below] {$x$};
    \draw[velocity] (M')++(\ang:0.3*\w) --++ (\ang:1.3*\v) node[above right=0] {$\vb{v}'$};
    \draw[mass]
      (M1')++(\w/2,\h/2) -|++ (-\w,-\h) --++ (\w,0) coordinate (I0)
      to[out=100,in=-100]++(-0.05*\w,0.3*\h) coordinate (I1)
      to[out=100,in=-100]++( 0.08*\w,0.3*\h) coordinate (I2)
      to[out=100,in= -80]++(-0.03*\w,0.2*\h) coordinate (I3) --  cycle;
      %node[midway] {$m_2$};
    \draw[mass]
      (M2')++(0,\w/2) -|++ (\h/2,-\w) -|++ (-\h,0.4*\w) -- ([xshift=0.4]I0)
      to[out=100,in=-100] ([xshift=0.4]I1)
      to[out=100,in=-100] ([xshift=0.4]I2)
      to[out=100,in= -80] ([xshift=0.4]I3) -- cycle;
    \node[red!30!black] at (M1') {$m_1$};
    \node[red!30!black] at (M2') {\,$m_2$};
    \draw pic["$\theta$",xcol,draw=xcol,angle radius=14,angle eccentricity=1.4] {angle=X--O--M'}; %pic text options={shift={(0.05,-0.05)}}
    \draw[velocity] (M1)++(\w/2,0) --++ (1.1*\v,0) node[above=0] {$\vb{v}_1$};
    \draw[velocity] (M2)++(0,\h/2) --++ (0,\v) node[right=0] {$\vb{v}_2$};
    \draw[mass] (M1)++(-\w/2,-\h/2) rectangle++ (\w,\h) node[midway] {$m_1$};
    \draw[mass] (M2)++(-\h/2,-\w/2) rectangle++ (\h,\w) node[midway] {$m_2$};
    \pic[scale=1,rotate=110] at (O) {collision={1.1}};
  \end{tikzpicture}
  \caption{Esempio di urto}
  \label{fig:esempio_di_urto}
\end{figure}

\vspace{1em}
\noindent
\textbf{Osservazione}: Si osservi che, dal momento che gli urti avvengono in un intervallo di tempo pressoché infinitesimo, si può considerare che sul sistema non agiscano forze esterne.\\
Quindi, dal momento che il \textbf{tempo di interazione} è estremamente corto rispetto all'effetto delle forze esterne, negli urti la \textbf{quantità di moto è sempre conservata}.\\
Alla luce di ciò, si distinguono tre categorie di urto:
\begin{enumerate}
  \item \textbf{Urto elastico}: in cui si ha \textbf{conservazione dell'energia cinetica};
  \item \textbf{Urto anelastico}: i corpi interagenti rimangono attaccati dopo la collisione, per cui presenteranno la \textbf{stessa velocità finale} (si potrebbe dire che tutta l'energia viene dispersa in questo caso);
  \item \textbf{Urto parzialmente anelastico}: in tale urto (una via di mezzo tra i precedenti) parte dell'energia è dispersa;
\end{enumerate}

\vspace{1em}
\noindent
\textbf{Osservazione $\boldsymbol{1}$}: Per lo studio degli urti, è fondamentale collocarsi nel centro di massa del sistema: infatti, il risultato precedentemente trovato afferma che se la risultante delle forze esterne è nulla, allora l'accelerazione del centro di massa è nulla, e quindi esso si muove a velocità costante: adottando tale sistema di riferimento, siccome il centro di massa non si muove né prima né dopo l'urto, la quantità di moto (essendo conservata per ipotesi) del sistema prima e dopo l'urto è sempre nulla.

\vspace{1em}
\noindent
\textbf{Osservazione $\boldsymbol{2}$}: Naturalmente, si può cambiare la quantità di moto di una particcella mentre la sua energia cinetica rimane la stessa: infatti, la quantità di moto è un vettore, per cui anche cambiando la direzione della velocità, pur mantenendo inalterato il suo modulo, si ha cambiamento di quantità di moto, ma non di energia cinetica; tuttavia, non può essere il contrario.\\
Ovviamente, affinché un urto sia definibile elastico, giacché la quantità di moto, negli urti, si conserva sempre, bisogna dimostrare che l'energia cinetica si conserva, ossia essa permane la stessa prima e dopo l'urto.

\newpage
\noindent
\textbf{Esercizio}: Si consideri l'urto seguente
\begin{figure}[H]
  \centering
  \begin{tikzpicture}
    \node[circle,draw,minimum width=1cm](a1) at (0,0){};
    \node[circle,draw,minimum width=1cm](b1) at (2.5,0){};
    \draw (0,0) node[]{$m$} (2.5,0) node[]{$2m$};
    \draw[-stealth] (0.5,0) -- ++(1,0) node[midway,above]{$3v$};
    \draw (1.25,-1) node[]{PRIMA};
    \draw ([xshift=5mm,yshift=5mm]current bounding box.north east) rectangle ([xshift=-5mm,yshift=-5mm]current bounding box.south west);
  \end{tikzpicture}
  \hspace{5em}
  \begin{tikzpicture}
    \node[circle,draw,minimum width=1cm](a1) at (0,0){};
    \node[circle,draw,minimum width=1cm](b1) at (2.5,0){};
    \draw (0,0) node[]{$m$} (2.5,0) node[]{$2m$};
    \draw[-stealth] (-0.5,0) -- ++(-1,0) node[midway,above]{$v$};
    \draw[-stealth] (3,0) -- ++(1,0) node[midway,above]{$?$};
    \draw (1.25,-1) node[]{DOPO};
    \draw ([xshift=5mm,yshift=5mm]current bounding box.north east) rectangle ([xshift=-5mm,yshift=-5mm]current bounding box.south west);
  \end{tikzpicture}
  \caption{Esempio di urto}
  \label{esempio_di_urto_1}
\end{figure}

\noindent
Al fine di verificare la natura di tale urto, è sufficiente determinare la velocità della della massa $2m$ dopo l'urto. Applicando la conservazione della quantità di moto, è facile capire come:
\[p_{i_x} = p_{f_x} \longrightarrow 3mv = -mv + 2m \cdot x \longrightarrow x = \frac{4mv}{2m} = 2v\]
Alla luce di ciò, appare evidente come
\[K_i=K_f \longrightarrow \frac{9}{2}mv^2 = \frac{1}{2}m v^2 + 4 mv^2\]
per cui l'urto è elastico.

\newpage
\noindent
\begin{center}
  11 Aprile 2022
\end{center}
\subsubsection{Urti tra due corpi}
Si considerino due corpi che collidono, come mostrato di seguito:

\begin{figure}[H]
  \colorlet{xcol}{blue!70!black}
  \colorlet{vcol}{green!60!black}
  \colorlet{myred}{red!65!black}
  \colorlet{acol}{red!50!blue!80!black!80}
  \tikzstyle{mass}=[line width=0.6,red!30!black,fill=red!40!black!10,rounded corners=1,top color=red!40!black!20,bottom color=red!40!black!10,shading angle=20]
  \tikzstyle{ground}=[preaction={fill,top color=blue!50!black!10,bottom color=blue!50!black!5,shading angle=20},
                      fill,pattern color=blue!20!black,pattern=north east lines,draw=none,minimum width=0.3,minimum height=0.6]
  \tikzstyle{velocity}=[->,vcol,very thick,line cap=round]

  \tikzset{
    pics/collision/.style={
      code={
        \draw[line width=0.5*#1,orange,fill=yellow]
          (0:0.20*#1) -- (30:0.06*#1) -- (50:0.25*#1) -- (80:0.10*#1) -- (105:0.32*#1) --
          (140:0.08*#1) -- (170:0.25*#1) -- (190:0.08*#1) -- (220:0.25*#1) --
          (250:0.08*#1) -- (270:0.24*#1) -- (300:0.08*#1) -- (320:0.25*#1) -- (340:0.09*#1) -- cycle;
    }},
    pics/collision/.default=1,
  }
\centering
% COLLISION 2D together
\begin{tikzpicture}[scale=1.2]
    % COLLISION 2D before
    \def\w{0.8}    % mass width
    \def\h{0.6}    % mass height
    \def\d{1.5}    % distance
    \def\xmax{2.6} % max x axis
    \def\ymax{2.2} % max y axis
    \def\d{1.9}    % distance
    \def\v{0.7}    % mass velocity
    \def\ang{40}   % angle after
    \coordinate (O) at (0,0);
    \coordinate (M1) at (-0.7*\xmax,0);
    \coordinate (M2) at (0,-0.7*\ymax);
    \coordinate (M')  at (\ang:\d); %(\d-0.07*\w,0.7*\d+0.23*\w);
    \coordinate (M1') at ($(\ang:\d)+(-0.4*\w,0)$) ; %(\d-0.6*\w,0.7*\d);
    \coordinate (M2') at ($(\ang:\d)+(0.4*\w,-0.3*\w)$); %(\d+0.2*\w,0.7*\d-0.3*\w);
    \draw[dashed] (0,0) -- (M');
    \draw[->,thick] (0,-\ymax) -- (0,\ymax) node[left] {$y$};
    \draw[->,thick] (-\xmax,0) -- (\xmax,0) coordinate (X) node[below] {$x$};
    \draw[velocity] (M')++(\ang:0.3*\w) --++ (\ang:1.3*\v) node[above right=0] {$\vb{v}'$};
    \draw[mass]
      (M1')++(\w/2,\h/2) -|++ (-\w,-\h) --++ (\w,0) coordinate (I0)
      to[out=100,in=-100]++(-0.05*\w,0.3*\h) coordinate (I1)
      to[out=100,in=-100]++( 0.08*\w,0.3*\h) coordinate (I2)
      to[out=100,in= -80]++(-0.03*\w,0.2*\h) coordinate (I3) --  cycle;
      %node[midway] {$m_2$};
    \draw[mass]
      (M2')++(0,\w/2) -|++ (\h/2,-\w) -|++ (-\h,0.4*\w) -- ([xshift=0.4]I0)
      to[out=100,in=-100] ([xshift=0.4]I1)
      to[out=100,in=-100] ([xshift=0.4]I2)
      to[out=100,in= -80] ([xshift=0.4]I3) -- cycle;
    \node[red!30!black] at (M1') {$m_1$};
    \node[red!30!black] at (M2') {\,$m_2$};
    \draw pic["$\theta$",xcol,draw=xcol,angle radius=14,angle eccentricity=1.4] {angle=X--O--M'}; %pic text options={shift={(0.05,-0.05)}}
    \draw[velocity] (M1)++(\w/2,0) --++ (1.1*\v,0) node[above=0] {$\vb{v}_1$};
    \draw[velocity] (M2)++(0,\h/2) --++ (0,\v) node[right=0] {$\vb{v}_2$};
    \draw[mass] (M1)++(-\w/2,-\h/2) rectangle++ (\w,\h) node[midway] {$m_1$};
    \draw[mass] (M2)++(-\h/2,-\w/2) rectangle++ (\h,\w) node[midway] {$m_2$};
    \pic[scale=1,rotate=110] at (O) {collision={1.1}};
  \end{tikzpicture}
  \caption{Esempio di urto}
  \label{fig:esempio_di_urto}
\end{figure}

\noindent
Allora, in questo caso, la massa $m_1$ presenta velocità iniziale $\vec V_1$, mentre la massa $m_2$ si muove a velocità $\vec V_2$: taluno è un problema generale che viene studiato tramite un sistema di riferimento esterno, di un osservatore.\\
Questo sistema di riferimento, tuttavia, non risulta essere pratico nello studio, per cui si può adottare un sistema di riferimento inerziale che prende in considerazione il centro di massa del sistema, la cui velocità, ovviamente, è data da
\[\vec v_{\text{CM}} = \frac{1}{M} \cdot \sum_i m_i \cdot \vec v_i = \frac{m_1 \cdot \vec V_1 + m_2 \cdot \vec V_2}{m_1+m_2} = \frac{\vec P_1 + \vec P_2}{m_1+m_2}=\frac{\vec P_{\text{tot}}}{m_{\text{tot}}}\]
Pertanto, se ora si considera il sistema di riferimento dato dal centro di massa, il quale si sposta con velocità $\vec v_{\text{CM}}$, le velocità delle due masse vengono così ricalcolate
\[\boxed{\vec v_1 = \vec V_1 - \vec v_{\text{CM}}} \hspace{1em} \text{e} \hspace{1em} \boxed{\vec v_2 = \vec V_2 - \vec v_{\text{CM}}}\]
cosicchè si ha che $\vec p_{\text{tot}}=0$, infatti:
\[\vec p_{\text{tot}}=m_1 \vec v_1 + m_2 \vec v_2 = m_1 \vec V_1 - m_1 \vec v_{\text{CM}} + m_2 \vec V_2 - m_2 \vec v_{\text{CM}}\]
È sufficiente, ora, calcolare il minimo comune denominatore e si ottiene
\[\vec p_{\text{tot}} = \frac{m_1^2 \vec V_1 + m_1m_2 \vec V_2 + m_1m_2 \vec V_1 + m_2^2 \vec V_2}{m_1+m_2} - \frac{m_1^2 \vec V_1 + m_1m_2 \vec V_2 + m_1m_2 \vec V_1 + m_2^2 \vec V_2}{m_1+m_2} = 0\]
Tale risultato è fondamentale: infatti, siccome la durata dell'urto è pressoché infinitesima, la quantità di moto totale si conserva, per cui $\vec p_{\text{tot},i} = \vec p_{\text{tot},f} = 0$.\\
Ciò significa che nel sistema di riferimento centro massa, per qualsiasi tipologia di urto
\[\boxed{\vec p_{\text{tot}} = 0 \longrightarrow \vec p_1 = - \vec p_2}\]
ovvero le quantità di moto delle due masse che si urtano sono uguali ed opposte; tuttavia ciò non equivale a dire che le velocità delle due masse siano uguali e opposte: infatti, le due masse potrebbero essere diverse, per cui anche le loro velocità sarebbero proporzionate.\\
Applicando la conservazione della quantità di moto, si evince come
\[\boxed{\vec p_{\text{tot},i} = \vec p_{1,i} + \vec p_{2,i} = 0 = \vec p_{\text{tot},f} = \vec p_{1,f} + \vec p_{2,f}}\]
che è una proprietà, per quanto già esposto, valida per qualsiasi urto. Tuttavia, se ora si considera l'energia cinetica, si ottiene che
\[K_i = \frac{1}{2}m_1v_{1,i}^2 + \frac{1}{2}m_2v_{2,i}^2 = \frac{1}{2m_1} p_{1,i}^2 + \frac{1}{2m_2} p_{2,i}^2\]
in quanto è sempre verificata l'identità seguente
\[\boxed{K=\frac{1}{2}mv^2=\frac{p^2}{2m}}\]
Ma essendo le quantità di moto delle due particelle \textbf{uguali in modulo} (ovvero $\vert p_{1,i} \vert = \vert p_{2,i} \vert$), allora anche i loro quadrati saranno uguali, per cui
\[\frac{1}{2m_1} p_{1,i}^2 + \frac{1}{2m_2} p_{2,i}^2 = p_{1,i}^2 \cdot \left(\frac{1}{2m_1} + \frac{1}{2m_2}\right)\]
Analogamente, per quanto concerne l'energia cinetica finale si ottiene che
\[K_f = \frac{1}{2m_1} p_{1,f}^2 + \frac{1}{2m_2} p_{2,f}^2 = p_{1,f}^2 \cdot \left(\frac{1}{2m_1} + \frac{1}{2m_2}\right)\]
Ciò, tuttavia, non significa che l'energia cinetica si conserva sempre (è possibile che vi sia dispersione di energia, ma anche un'aggiunta di energia, come nel caso di un'esplosione, che è comunque una forza interna che non produce un'accelerazione del centro di massa che, quindi, permane suo moto a velocità costante $\vec v_{\text{CM}}$ e garantisce sempre la conservazione della quantità di moto).

\vspace{1em}
\subsubsection{Urti elastici}
Nel caso di urti elastici, si ha \textbf{conservazione di energia cinetica} oltre che di quantità di moto, per cui
\[K_i = p_{1,i}^2 \cdot \left(\frac{1}{2m_1} + \frac{1}{2m_2}\right) = K_f = p_{1,f}^2 \cdot \left(\frac{1}{2m_1} + \frac{1}{2m_2}\right) \longrightarrow \vert p_i \vert = \vert p_f \vert\]
Ciò significa che
\[\boxed{\vert v_{1,i} \vert = \frac{\vert p_i \vert}{m_1} = \frac{\vert p_f \vert}{m_1} = \vert v_{1,f} \vert} \hspace{1em} \text{e} \hspace{1em} \boxed{\vert v_{2,i} \vert = \frac{\vert p_i \vert}{m_2} = \frac{\vert p_f \vert}{m_2} = \vert v_{2,f} \vert}\]

\vspace{1em}
\noindent
\textbf{Osservazione}: Pertanto, nel caso di urti elastici, le due masse che si scontrano conservano la propria velocità prima e dopo l'urto (in quanto quantità di moto ed energia cinetica sono conservati), per cui:

\begin{table}[H]
  \begin{tabularx}{\textwidth}{|P|P|}
    \hline
    \cellcolor{blue!25}\textbf{Conservazione dell'energia cinetica} & \cellcolor{green!25}\textbf{Conservazione della quantità di moto}\\
    \hline
    \cellcolor{blue!25}$\vert v_{1,i} \vert = \vert v_{1,f} \vert$ & \cellcolor{green!25}$\vert p_{1,i} \vert = \vert p_{1,f} \vert$\\
    \hline
    \cellcolor{blue!25}$\vert v_{2,i} \vert = \vert v_{2,f} \vert$ & \cellcolor{green!25}$\vert p_{2,i} \vert = \vert p_{2,f} \vert$\\
    \hline
  \end{tabularx}
  \caption{Conclusioni sugli urti elastici}
  \label{tab:conclusioni_urti_elastici}
\end{table}

\noindent
Ovviamente, considerando come sistema di riferimento il centro massa, per tutti gli urti è possibile affermare che
\[\vec p_{1,i} = - \vec p_{2,i} \hspace{1em} \text{e} \hspace{1em} \vec p_{1,f} = - \vec p_{2,f}\]
in forza della conservazione della quantità di moto.\\
Tuttavia, solamente nel caso degli urti elastici si ha che
\[\vert p_{1,i} \vert = \vert p_{1,f} \vert \hspace{1em} \text{e} \hspace{1em} \vert p_{2,i} \vert = \vert p_{2,f} \vert\]
in forza della conservazione dell'energia cinetica.\\
Non solo, ma siccome le velocità inziali e finali delle due particelle presentano la stessa direzione, ma verso opposto (in quanto è noto che, nel centro di riferimento centro massa, $\vec p_{1,i} = - \vec p_{2,i}$ e $\vec p_{1,f} = - \vec p_{2,f}$), si può affermare, in generale che
\[\vert \vec v_{\text{rel}} \vert = \vert \vec v_{1,i} - \vec v_{2,i} \vert = \vert v_{1,i} \vert + \vert v_{2,i} \vert\]
ma sapendo che anche l'energia cinetica è conservata, si può scrivere che
\[\vert \vec v_{\text{rel}} \vert =  \vert v_{1,f} \vert + \vert v_{2,f} \vert = \vert \vec v_{1,f} - \vec v_{2,f} \vert\]
e tale risultato diviene generale, valido per gli urti elastici indipendentemente dal sistema di riferimento adottato (ricordando come si ottengono $\vec v_1$ e $\vec v_2$ da $\vec V_1$ e da $\vec V_2$): il modulo della velocità relativa di avvicinamento è uguale al modulo della velocità relativa di separazione.

\vspace{1em}
\subsection{Urto in 1D - Elastico}
Nel caso di urto in 1D, si può presentare solamente la situazione seguente:

\begin{figure}[H]
  \centering
  \begin{tikzpicture}
    \draw[thick,-stealth,blue] (0,0) node[circ]{} -- ++(2,0) node[midway,below]{$\vec p_{1,i}$};
    \draw[thick,-stealth,blue] (5,0) node[circ]{} -- ++(-2,0) node[midway,below]{$\vec p_{2,i}$};
  \end{tikzpicture}
  \hspace{3em}
  \begin{tikzpicture}
    \draw[thick,-stealth,red] (2,0) node[circ]{} -- ++(-2,0) node[midway,below]{$\vec p_{1,f}$};
    \draw[thick,-stealth,red] (3,0) node[circ]{} -- ++(2,0) node[midway,below]{$\vec p_{2,f}$};
  \end{tikzpicture}
  \caption{Urto elastico in 1D}
  \label{fig:urto_elastico_1D}
\end{figure}

\noindent
In cui, ovviamente, i vettori quantità di moto iniziali (così come quelli finali) delle due masse sono uguali ed opposti, proprietà sempre vera per qualsiasi urto osservato dal sistema di riferimento centro massa. Ma nel caso di urto elastico si ha anche che per ogni massa la quantità di moto iniziale e finale è uguale in modulo; ciò significa che, essendo in 1D, deve essere necessariamente che
\[\vec p_{1,i} = - \vec p_{1,f} \hspace{1em} \text{e} \hspace{1em} \vec p_{2,i} = - \vec p_{2,f} \hspace{1em} \longrightarrow \hspace{1em} \vec v_{1,i} = - \vec v_{1,f} \hspace{1em} \text{e} \hspace{1em} \vec v_{2,i} = - \vec v_{2,f}\]
Passando, però, ad un nuovo sistema di riferimento e non più a quello del centro di massa, si ottiene che
\[V_{1,f} = v_{1,f} + v_{\text{CM}} = -v_{1,i} + v_{\text{CM}} = - \left(V_{1,i} - v_{\text{CM}}\right) + v_{\text{CM}} = -V_{1,i} + 2 v_{\text{CM}}\]
e sapendo come determinare la velocità del centro di massa si ottiene che:
\[\boxed{V_{1,f} = \frac{2 m_2}{m_1 + m_2} V_{2,i} + \frac{m_1-m_2}{m_1+m_2} V_{1,i}}\]
Da cui si può evincere come, nel caso in cui $m_1=m_2$, allora $V_{1,f}=V_{2,i}$ (ovvero le due masse si scambiano la velocità con l'urto).\\
Invece, nel caso in cui $m_1 \to +\infty$, allora si ha che la sua velocità finale è identica a quella iniziale:
\[V_{1,f} = \frac{2 m_2}{m_1 + m_2} V_{2,i} + \frac{m_1-m_2}{m_1+m_2} V_{1,i} = V_{1,i}\]
mentre la massa $m_2$ assume due volte la velocità della prima massa, sottratta alla propria velocità iniziale:
\[V_{2,f} = \frac{2 m_1}{m_1 + m_2} V_{1,i} + \frac{m_2-m_1}{m_1+m_2} V_{2,i} = 2 V_{1,i} - V_{2,i}\]

\newpage
\subsection{Urto in 1D - Anelastico}
Nel caso di un urto parzialmente anelastico non si ha conservazione dell'energia cinetica, mentre se si ha un urto completamente anelastico le due masse interagenti rimangono attaccate l'una all'altra: ciò significa che se si ossserva tale urto attraverso il sistema di riferimento centro di massa, le due masse iniziale presentano una quantità di moto non nulla e dopo l'urto, rimanendo attaccate l'una all'altra, presenteranno quantità di moto pari a $0$.\\
Ovviamente, anche in questi due casi, si ha che $\vec p_{1,i} = -\vec p_{2,i}$, così come $\vec p_{1,f} = - \vec p_{2,f}$. Parlando di urto completamente anelastico, siccome le due masse rimangono attaccate l'una all'altra, considerando il sistema di riferimento centro massa è ovvio che $\vec p_{1,f} = \vec p_{2,f} = 0$, mentre se si considera un sistema di riferimento differente dal centro di massa, si ottiene
\[\boxed{V_{1,f} = V_{2,f} = v_{\text{CM}} = \frac{m_1 \cdot V_{1,i} + m_2 \cdot V_{2,i}}{m_1 + m_2}}\]

\vspace{1em}
\subsection{Urti - Riepilogo}
\begin{itemize}
  \item Nello studio degli urti, \textbf{la conservazione della quantità di moto si applica sempre}, per cui, in un qualsiasi sistema di riferimento:
  \[\boxed{\vec P_{\text{tot}, i} = \vec P_{\text{tot}, f}}\]
  mentre nel sistema di riferimento del centro di massa si ha
  \[\boxed{\vec p_{1,i} = - \vec p_{2,i}} \hspace{1em} \text{e} \hspace{1em} \boxed{\vec p_{1,f} = - \vec p_{2,f}}\]
  in quanto la quantità di moto iniziale e finale è sempre nulla.

  \item Nel caso di un urto elastico, si ha conservazione dell'energia cinetica
  \[\boxed{K_i=K_f}\]
  per cui il modulo della velocità relativa all'avvicinamento è uguale al modulo della velocità relativa dopo l'urto.\\
  Mentre nel sistema di riferimento del centro di massa si ha che
  \[\boxed{\vert p_{1,i} \vert = \vert p_{1,f} \vert = \vert p_{2,i} \vert = \vert p_{2,f} \vert}\]

  \item Nel caso di un urto completamente anelastico, la velocità finale è quella del centro di massa, ovvero
  \[\boxed{V_{1,f} = V_{2,f} = v_{\text{CM}} = \frac{m_1 \cdot V_{1,i} + m_2 \cdot V_{2,i}}{m_1 + m_2}}\]
\end{itemize}

\newpage
\noindent
\begin{center}
  12 Aprile 2022
\end{center}
\textbf{Esercizio}: Si consideri un carrello posizionato sulla sommità di una rampa avente altezza $h$. Quando viene lasciato cadere, da fermo, acquista velocità così come la rampa. Si determini la velocità della rampa.\\
Naturalmente a tale problema si applica la conservazione della quantità di moto, per cui
\[\vec p_i = \vec p_f\]
ed esplicitando la quantità di moto iniziale e finale si ottiene:
\begin{flalign*}
  \vec p_i &=\vec p_{\text{cart},i} + \vec p_{\text{ramp},i} = 0 + 0 = \vec 0\\
  \vec p_f &= \vec p_{\text{cart},f} + \vec p_{\text{ramp},f} = \vec v_c \cdot m + \vec v_r \cdot M = 0
\end{flalign*}
ma siccome $\vec p_{\text{cart},f} = - \vec p_{\text{ramp},f}$ si ottiene che
\[\vec v_{cart} \cdot m = - \vec v_{\text{ramp}} \cdot M \longrightarrow \vec v_{cart} = - M \cdot \frac{\vec v_{\text{ramp}}}{m}\]
Applicando il principio di conservazione dell'energia meccanica, si ottiene anche che
\[E_i=E_f \longrightarrow E_i=K_{\text{cart},i}+K_{\text{ramp},i} + \mathcal{U}_i = mgh\]
mentre
\[E_f = K_{\text{cart},f} + K_{\text{ramp},f} + \mathcal{U}_f = \frac{1}{2}m v_{cart}^2 + \frac{1}{2}M v_{\text{ramp}}^2\]
volendo determinare $v_{\text{ramp}}$ si ottiene, applicando la conservazione dell'energia meccanica e il fatto che $v_{\text{cart}} = M \cdot \frac{v_{\text{ramp}}}{m}$
\[mgh=\frac{1}{2}m v_{\text{cart}}^2 + \frac{1}{2}M v_{\text{ramp}}^2 \longrightarrow mgh=\frac{1}{2}m \left(M \cdot \frac{v_{\text{ramp}}}{m}\right)^2 + \frac{1}{2}M v_{\text{ramp}}^2\]
Ciò permette di pervenire facilmente al risultato seguente:
\[v_{\text{ramp}} = \sqrt{\frac{2mgh}{\displaystyle{M \cdot \left(1 + \frac{M}{m}\right)}}}\]

\vspace{1em}
\noindent
\textbf{Esercizio}: Si consideri l'urto seguente
\begin{figure}[H]
  \centering
  \begin{tikzpicture}
    \node[circle,draw,minimum width=1cm](a1) at (0,0){};
    \node[circle,draw,minimum width=1cm](b1) at (2.5,0){};
    \node[circle,draw,minimum width=1cm](c1) at (2.5,1.5){};
    \draw (b1.120) -- (c1.240) (b1.60) -- (c1.300);
    \draw (0,0) node[]{$m$} (2.5,0) node[]{$m$} (2.5,1.5) node[]{$m$};
    \draw[-stealth] (0.5,0) -- ++(1,0) node[midway,above]{$2v$};
    \draw (1.25,-1) node[]{PRIMA};
    \draw ([xshift=5mm,yshift=5mm]current bounding box.north east) rectangle ([xshift=-5mm,yshift=-5mm]current bounding box.south west);
  \end{tikzpicture}
  \hspace{5em}
  \begin{tikzpicture}
    \node[circle,draw,minimum width=1cm](a2) at (0.5,0){};
    \node[circle,draw,minimum width=1cm](b2) at (3,0){};
    \node[circle,draw,minimum width=1cm](c2) at (2.5,1.5){};
    \draw (b2.60) -- (c2.320) (b2.130) -- (c2.260);
    \draw (0.5,0) node[]{$m$} (3,0) node[]{$m$} (2.5,1.5) node[]{$m$};
    \draw[-stealth] (2.75,0.75) -- ++(1,0) node[midway,above]{$v$};
    \draw [-stealth] (c2) [out=210, in=170] to (b2);
    \draw (1.7,0.5) node[]{$\omega$};
    \draw (1.75,-1) node[]{DOPO};
    \draw ([xshift=5mm,yshift=5mm]current bounding box.north east) rectangle ([xshift=-5mm,yshift=-5mm]current bounding box.south west);
  \end{tikzpicture}
  \caption{Esempio di urto}
  \label{esempio_di_urto_2}
\end{figure}

\noindent
Essendo un urto, si può applicare la conservazione della quantità di moto, per cui
\[\vec p_i = \vec p_f \longrightarrow 2 m v_i = \frac{m v_{1,f} + m v_{2,f}}{2m}\]
Per verificare se si conserva l'energia cinetica si consideri:
\[K_i=\frac{1}{2}m \cdot (2v)^2\]
mentre l'energia cinetica finale è data solamente dal contributo delle due masse collegate fra loro e può essere calcolata ricavando le velocità delle singole masse proprio dall'informazione fornita dal problema che $\vec v_{\text{CM}}=v$, per cui
\[\vec v_{\text{CM}} = \frac{m v_{1,f} + m v_{2,f}}{2m}\]
ma siccome delle due masse fra di loro collegate, solamente quella più in basso si muove (considerando l'istante appena dopo l'urto), è facile capire come
\[\vec v_{\text{CM}} = \vec v_{\text{CM}} = v \longrightarrow \frac{v_{1,f}}{2} = v \longrightarrow \vec v_{1,f} = 2v\]
per cui l'energia cinetica finale è data unicamente dal contributo della prima massa, da cui:
\[K_f=\frac{1}{2}m (2v)^2\]
per cui l'urto considerato è elastico.

\vspace{1em}
\subsection{Moto dei razzi}
Il moto dei razzi ha radici storiche molto profonde; un esempio di fallimento nella progettazione di un razzo riguarda l'esperimento di Gottard:\\\\
\quotes{Come un metodo per inviare un proiettile verso la parte pia alta, persino la parte le pia alta, dell'atmosfera terrestre, il razzo a carica multipla del Prof. Goddard è un dispositivo praticabile e quindi promettente.\\
(...) dopo che it razzo ha lasciato la nostra aria ed ha iniziato davvero il suo viaggio più lungo, il suo vuolo non sarebbe né accelerato né mantenuto dall'esplosione delle cariche avrebbe potuto ancora avere. Affermare che lo sarebbe è negare una legge fondamentale della dinamica, e solo il Dott. Einstein e la sua dozzina prescelta, pochi e capaci, sono autorizzati a farlo.\\
Che it Prof. Goddard (...) non conosca la relazione tra azione e reazione, e della necessità di avere qualcosa di meglio di un vuoto contro cui reagire — dire questo sarebbe assurdo.}

\vspace{1em}
\noindent
\textbf{Osservazione}: Al fine di analizzare il moto dei razzi, è necessario considerarne il modello seguente:

\begin{figure}[H]
  \centering
  \begin{tikzpicture}
    \draw[fill=cyan!30] (0,0) arc [start angle=200, delta angle=-110, radius=1cm] -- ++(0,-1) coordinate(a) -- (0,0);
    \draw[fill=yellow!30] (a) -- ++(2,0) -- ++(0,4) -- ([xshift=1cm,yshift=6cm]a) -- ([yshift=4cm]a)-- ++(0,-4);
    \draw[fill=cyan!30] ([xshift=2cm]a) ++(0,1) arc [start angle=90, delta angle=-110, radius=1cm] -- ([xshift=2cm]a) -- ([xshift=2cm]a);
    \node[minimum size=0.8cm,circle,draw,red] (circle) at ([xshift=1cm,yshift=3cm]a){};
    \node[minimum size=0.6cm,circle,draw,fill=blue!25] (oblo) at ([xshift=1cm,yshift=3cm]a){};
    \draw[-stealth] ([xshift=1cm,yshift=6cm]a) -- ++(0,1) node[midway,right]{$\vec v_r$};
    \draw[-stealth] ([xshift=1cm]a) -- ++(0,-1) node[midway,right]{$\vec v_g$};
  \end{tikzpicture}
  \caption{Modello di un razzo}
  \label{fig:modello_razzo}
\end{figure}

\noindent
Naturalmente il razzo si muove ad una velocità $\vec v_r$ e, come risultato della combusitone del carburante, viene emessa posteriormente anche una massa $\Delta m$ di gas che, a causa dell'elevata temperatura, presenta delle molecole che si muovono ad un'elevata velocità $\vec v_g$; ovviamente è fondamentale osservare come la velocità $\vec v_g$ delle molecole di gas sia relativa al razzo e non all'osservatore.\\
A tale contesto si applica la seconda legge di Newton generalizzata, per cui si ha che
\[\vec F_{\text{ext}} = \frac{d\vec p}{dt}\]
che, naturalmente, può anche essere interpretata come:
\[\vec F_{\text{ext}} = \frac{\vec p_f - \vec p_i}{\Delta t}\]
La quantità di moto iniziale del razzo è
\[\vec p_i = m \cdot \vec v_r\]
in cui $m$ è la massa totale iniziale del razzo e $\vec v_r$ è la velocità iniziale dello stesso. Invece, la quantità di moto finale, dopo un intervallo di tempo $\Delta t$ è
\[\vec p_f = \left(m - \Delta m\right) \cdot \left(\vec v_r + \Delta \vec v_r\right) + \Delta m \cdot \left(\vec v_g + \vec v_r\right)\]
in quanto il razzo ha perso una quantità $\Delta m$ di gas e, quindi, ha diminuito la propria massa di $\Delta m$ e aumentato la propria velocità di $\Delta v$; inoltre è necessario anche considerare la quantità di moto della quantità $\Delta m$ di gas che viene espulso dal razzo stesso, la cui velocità, relativa al razzo, è proprio $\vec v_g + \vec v_r$ (sempre interpretando l'intero sistema dal punto di vista di un osservatore esterno). La semplificazione porta a
\[\vec p_f = \left(m - \Delta m\right) \cdot \left(\vec v_r + \Delta \vec v_r\right) + \Delta m \cdot \left(\vec v_g + \vec v_r\right) = m\vec v_r - \Delta m \vec v_r + m \Delta \vec v_r - \Delta m \Delta \vec v_r + \Delta m \vec v_g + \Delta m \vec v_r\]
in cui, ovviamente, $\Delta m \Delta \vec v_r \to 0$, in quanto due quantità infinitesime moltiplicate fra loro. In altre parole si ottiene
\[\vec p_f = m \vec v_r + m \Delta v_r + \Delta m \vec v_g\]
per cui, ritornando alla seconda legge della dinamica generalizzata, si ottiene
\[\vec F_{\text{ext}} = \frac{\vec p_f - \vec p_i}{\Delta t}=\frac{1}{\Delta t} \cdot \left(m \Delta \vec v_r + \Delta m \vec v_g - \Delta m \Delta \vec v_r\right)\]
Ma siccome, per quanto detto, le variazioni considerate sono infinitesimali, allora $\Delta m \Delta \vec v_r \to 0$, da cui
\[\vec F_{\text{ext}} = m \cdot \frac{\Delta \vec v_r}{\Delta t} + \vec v_g \cdot \frac{\Delta m}{\Delta t}\]
Considerando un tempo infinitesimale, ossia $\displaystyle{\lim \Delta t \to 0}$, si ottiene che
\[\frac{\Delta \vec v_r}{\Delta t} \longrightarrow \frac{d \vec v_r}{dt} = \vec a_r \hspace{1em} \text{e} \hspace{1em} \frac{\Delta m}{\Delta t} \longrightarrow \frac{dm}{dt}\]
Ecco, quindi, che si è ottenuto:
\[\boxed{\vec F_{\text{ext}} = m \vec a_r + \vec v_g \cdot \frac{dm}{dt}}\]
in cui $\vec v_g \cdot \dfrac{dm}{dt}$ è proprio la \textbf{spinta del razzo}: questo è proprio il meccanismo di funzionamento di un razzo, in quanto se il gas viene espulso e aquista una certa velocità, allora anche il razzo deve accelerare per garantire la conservaizone della quantità di moto.

\vspace{1em}
\noindent
\textbf{Esempio}: Naturalmente, nello spazio, si ha che $\vec F_{\text{ext}}=0$, per cui l'equazione di cui sopra diviene:
\[\boxed{m \vec a_r = - \vec v_g \cdot \frac{dm}{dt}}\]
che è un'equazione che ha senso, in quanto al decremento della massa del razzo corrisponde un aumento della velocità dello stesso. Per quanto riguarda la componente orizzontale, riscrivendo l'accelerazione secondo la definizione si ottiene:
\[m \cdot \frac{dv}{dt} =- v_g \cdot \frac{dm}{dt} \longrightarrow \frac{dv}{dt} = - \frac{v_g}{m} \cdot \frac{dm}{dt}\]
Naturalmente taluna è una equazione differenizale; ma sapendo che
\[\frac{d}{dt} \log(m(t)) = \frac{1}{m} \frac{dm}{dt}\]
si può facilmente evincere come
\[\frac{dv}{dt}=-v_g \cdot \frac{d}{dt} \log(m(t))\]
e integrando, ora, rispetto al tempo, si ottiene che
\[v_f-v_i =-v_g \cdot \left[\log(m_f) - \log(m_i)\right]\]
che permette di ottenere
\[\boxed{v_f-v_i=\Delta v=-v_g \cdot \log \left(\frac{m_f}{m_i}\right)}\]
che prende il nome di \textbf{equazione di Ciolkovsky}. È facile capire come, ovviamente, essendo $m_f < m_i$ (a causa della perdita di carburante), si ottiene
\[\frac{m_f}{m_i} < 1 \longrightarrow \log \left(\frac{m_f}{m_i}\right) < 0\]
ma essendo precedeuto da un segno $-$, si ottiene un aumento di velocità, ovviamente.

\vspace{1em}
\noindent
\textbf{Osservazione}: È anche possibile considerare la velocità in funzione del tempo, ovvero
\[v(t)=v_i-v_g \log \left(\frac{m(t)}{m_i}\right)\]
che, graficamente, corrisponde a:

\vspace{2em}
\noindent
\rowcolors{1}{white}{white}
\begin{tabularx}{\textwidth}{P}
  {
      \centering
      \begin{tikzpicture}
        \begin{axis}[
          grid=both,
          axis lines = middle,
          xlabel = \(t\),
          ylabel = {\(v(t)\)},
          legend pos=outer north east,
          ymajorgrids=true,
          xmajorgrids=true,
          grid style=dashed,
        ]

        \addplot [
          domain=-2:100,
          samples=100,
          color=red,
        ]
        {50-2*ln((4-0.1*x)/4)};
        \end{axis}
    \end{tikzpicture}
  }
\end{tabularx}

\newpage
\noindent
\begin{center}
  13 Aprile 2022
\end{center}
Per la risoluzione di un problema fisico, è fondamentale compiere i passi seguenti:
\begin{enumerate}
  \item \textbf{Visualizzazione}
  \begin{enumerate}
    \item Creare un'immagine mentale del problema;
    \item Identificare i concetti che possono essere utili;
    \item Riformulare la domanda a parole più chiare, ed in termini che possono essere calcolati.
  \end{enumerate}
  \item \textbf{Descrizione}
  \begin{enumerate}
    \item Realizzare diagrammi, inclusi diagrammi di corpi liberi;
    \item Scegliere un sistema di coordinate appropriato;
    \item Definire le quantità che sono importanti nel problema;
    \item Identificare la quantità da determinare per fornire la soluzione al problema.
  \end{enumerate}
  \item \textbf{Pianificazione}
  \begin{enumerate}
    \item Trasferire i concetti in forma matematica;
    \item È possibile partire dalla soluzione e lavorare all'indietro, oppure costruire un percorso risolutivo da ciò che è noto a ciò che si vuole determinare.
  \end{enumerate}
  \item \textbf{Esecuzione}
  \begin{enumerate}
    \item Risolvere i sistemi di equazioni;
    \item Verificare le unità;
    \item Sostituire i valori numerici alle variabili in gioco.
  \end{enumerate}
  \item \textbf{Valutazione della soluzione}
  \begin{enumerate}
    \item Verificare se la soluzione risponde alla domanda del problema;
    \item Validare il risultato testando i casi limite;
    \item Valutare la ragionevolezza del risultato;
    \item Valutare la completezza del risultato.
  \end{enumerate}
  \item \textbf{Modificazione del piano risolutore}
  \begin{enumerate}
    \item Per correggere il piano risolutore si può adottare una strategia diferente;
    \item È possibile cambiare il sistema di coordinate o il sistema di riferimento.
  \end{enumerate}
\end{enumerate}

\vspace{1em}
\section{Moto e dinamica dei corpi rigidi}
Di seguito si fornisce la definizione di \textbf{corpo rigido}:

% Tabella per le definizione di concetti, etc...
\vspace{1em}
\rowcolors{1}{black!5}{black!5}
\setlength{\tabcolsep}{14pt}
\renewcommand{\arraystretch}{2}
\noindent
\begin{tabularx}{\textwidth}{@{}|P|@{}}
    \hline
    {\textbf{CORPO RIGIDO}}\\
    \parbox{\linewidth}{Un corpo rigido è un corpo la cui \textbf{forma} e \textbf{dimensione} permangono fissi. In altre parole, le \textbf{posizioni relative} tra le parti del corpo sono \textbf{fisse}.\vspace{3mm}}\\
    \hline
\end{tabularx}

\vspace{1em}
\noindent
\textbf{Osservazione}: Ovviamente, la definizione di corpo rigido è relativa e fortemente limitata a dei problemi specifici, in quanto la deformazione di un corpo può avvenire sempre, ma con gradi diversi.

\vspace{1em}
\subsection{Corpi rigidi - Cinematica}
La cinematica di un corpo rigido è lo studio e la descrizione del movimento di un corpo, il quale può avvenire con modalità differenti.

\vspace{1em}
\subsubsection{Traslazione}
Naturalmente, in una traslazione di un corpo rigido, tutte le parti del corpo subiscono lo \textbf{stesso spostamento} $\Delta \vec r$ (altrimenti si avrebbe deformazione) e, quindi, presentano la \textbf{medesima velocità} $\vec v$: ciò significa che sono sufficienti solamente $3$ parametri per descrivere una traslazione, ossia $(x,y,z)$.

\vspace{1em}
\subsubsection{Rotazione}
Per quanto concerne la rotazione di un corpo rigido, è necessario nuovamente specificare $3$ parametri, come
\begin{itemize}
  \item 3 angoli, uno per ciascun asse ($x$,$y$,$z$);
  \item 2 parametri per l'asse di rotazione (in quanto l'asse è un versore di lunghezza unitaria) più un angolo di rotazione attorno all'asse.
\end{itemize}
Approssimando un corpo ad un punto materiale, si osservi che se esso compie uno spostamento angolare di raggio $R$, descrivendo un angolo $\theta$ e un arco di lunghezza $s$, come mostrato di seguito

\begin{figure}[H]
  \centering
  \begin{tikzpicture}[scale=1.5]
    \draw (0,0) node[circ]{};
    \draw[-stealth] (0,0) -- ++(2.5,0) node[at end, below right]{$x$};
    \draw[-stealth] (0,0) -- ++(0,2.5) node[at end, above left]{$y$};
    \draw (0,0) -- ++(2,0.75) node[midway,above left]{$R$} node[circ](a){};
    \draw[-stealth] (a) arc [start angle=30,end angle=75,radius=2.2] node[midway,above right]{s} coordinate(b);
    \draw (b) -- (0,0);
    \draw [draw = orange] (0,0) ++(0.5,0.18) arc (30:60:0.8) node [pos=.4, left] {\textcolor{orange}{$\theta$}};
  \end{tikzpicture}
\end{figure}

\noindent
allora l'angolo $\theta$ descritto si calcola come
\[\boxed{\theta=\frac{s}{R}}\]
che viene naturalmente espresso in \textbf{radianti}. Inoltre $\theta$ non è solo un angolo, ma è anche una \quotes{\textbf{coordinata angolare}} che permette di esprimere
\begin{itemize}
  \item la \textbf{velocità angolare}, ossia la derivata nel tempo dell'angolo $\theta$ descritto
  \[\boxed{\dfrac{d\theta}{dt}=\omega}\]
  \item l'\textbf{accelerazione angolare}, ossia la derivata nel tempo della velocità angolare $\omega$, o anche la derivata seconda dell'angolo $\theta$ descritto
  \[\boxed{\dfrac{d\omega}{dt}=\dfrac{d^2\theta}{dt^2}=\alpha}\]
\end{itemize}

\vspace{1em}
\noindent
\textbf{Esempio}: Nel caso caso di \emph{accelerazione angolare costante}, si possono impiegare le quantità note per fornire la legge oraria seguente
\[\boxed{\theta(t)=\theta_0+\omega t + \frac{1}{2}\alpha t^2}\]
che ricorda esattamente la cinematica di un moto uniformemente accelerato (infatti, in questo caso, si una un \textbf{momento di forza costante}).\\

\vspace{1em}
\noindent
\textbf{Osservazione}: Nel caso di un moto rotatorio, è fondamentale considerare l'attrito (che è un \textbf{attrito statico}, non essendoci moto relativo tra le superifici di contatto):

\begin{figure}[H]
  \centering
  \colorlet{xcol}{blue!70!black}
  \colorlet{vcol}{green!60!black}
  \colorlet{myred}{red!65!black}
  \colorlet{mypurple}{blue!60!red!80}
  \colorlet{acol}{red!50!blue!80!black!80}
  \tikzstyle{rvec}=[->,xcol,very thick,line cap=round]
  \tikzstyle{vvec}=[->,vcol,very thick,line cap=round]
  \tikzstyle{myarr}=[{Latex[length=3,width=3]}-,xcol]
  \tikzstyle{force}=[->,myred,very thick,line cap=round]
  \tikzstyle{mass}=[line width=0.6,draw=red!30!black, %rounded corners=1,
                    top color=red!40!black!30,bottom color=red!40!black!10,shading angle=30]
  \tikzstyle{ground}=[preaction={fill,top color=black!10,bottom color=black!5,shading angle=20},
                      fill,pattern=north east lines,draw=none,minimum width=0.3,minimum height=0.6]
  \tikzstyle{metal}=[fill,top color=black!40,bottom color=black!20,shading angle=10]
  \begin{tikzpicture}[scale=1.5]
    \def\W{4.8} % ground width
    \def\H{3.2} % ground height
    \def\D{0.3} % ground depth
    \def\R{0.8} % disk radius
    \def\d{1.0} % disks' distances from edge
    \def\angb{{(2*\d-\W)*180/pi-90}} % angle point 2
    \coordinate (A)  at (\d,\R);     % disk 1 origin
    \coordinate (B)  at (\W-\d,\R);  % disk 2 origin
    \coordinate (RA) at ($(A)+(-90:\R)$);   % point 1
    \coordinate (RB) at ($(B)+(\angb:\R)$); % point 2
    \coordinate (BB) at ($(B)+(-90:\R)$);   % point 2 bottom

    % GROUND
    \draw[ground] %(0,0) rectangle++ (-\D,\H) (-\D,\H) rectangle++ (\W,\D);
      (0,0) rectangle++ (\W,-\D);
    \draw (0,0) --++ (\W,0);

    % DISK 1
    \draw[mass] (A) circle(\R);
    \draw[blue!50!black,fill=blue!90!black] (RA) circle(0.06*\R);
    \draw[->] (A) -- (RA) node[midway,above=0,left=0,scale=0.8] {$R$}; %node[right=3] {$M$}
    \draw[vvec] (A) --++ (0.7*\R,0) node[below] {$\vb{v}$};
    \draw[red!40!black,fill=red!80!black!80] (A) circle(0.08*\R) node[above=0.1,scale=0.8] {$M$}; %node[right=1,scale=0.8] {CM}
    %\draw[force] (TD)++(0.08,0) --++ (0,0.8*\RD) node[right] {$\vb{T}$};
    \draw[->] (A)++(30:1.2*\R) arc(30:-10:1.0*\R) node[midway,above right=0] {$\omega$};

    % DISK 2
    \draw[mass] (B) circle(\R);
    \draw[thick,blue!40!black] (BB) arc(-90:\angb:\R);
    \draw[blue!50!black,fill=blue!90!black] (RB) circle(0.06*\R);
    \draw[->] (B) -- (RB);
    \draw[dashed] (B) --++ (0,-\R) --++ (0,-0.1*\R);
    \draw[vvec] (B) --++ (0.7*\R,0) node[below] {$\vb{v}$};
    \draw[red!40!black,fill=red!80!black] (B) circle(0.08*\R);
    \draw pic[<-,scale=0.8,"$\theta$",draw,angle radius=8,angle eccentricity=1.6] {angle=RB--B--BB};
    \draw[->,blue!50!black] (A)++(0,1.3*\R) --++ (\W-2*\d,0) node[midway,fill=white,inner sep=1,scale=0.8] {$s=R\theta$};
    %\draw[->] (B)++(30:1.2*\R) arc(30:-10:1.0*\R) node[midway,above right=-1] {$\omega$};
  \end{tikzpicture}
  \caption{Corpo in rotolamento}
  \label{fig:corpo_rotolamento}
\end{figure}

\noindent
Inoltre, se un corpo rotola con velocità $\vec v$, allora ciò significa che si può definire una funzione spostamento nel tempo:
\begin{flalign*}
  x(t) & := \text{ posizione del centro del cerchio in funzione del tempo};\\
  \theta(t) & := \text{ rapporto fra lo spostamento traslatorio orizzontale e il raggio della circonferenza};
\end{flalign*}
Si osservi, infatti, che nel caso di piccoli spostamenti, l'arco descritto da una circonferenza $\Delta s$ può essere approssimato allo spostamento orizzontale $\Delta x$, per cui
\[\Delta \theta = \frac{\Delta s}{r} \cong \frac{\Delta x}{r}\]
Ciò, allora, porta alla seguente approssimazione:
\[\boxed{\theta(t) = \theta_0 + \frac{x(t)}{r}}\]
e da ciò segue anche la relazione seguente $v=\omega r$ (che non è sempre necessariamente vera, ma per il rotolamento sì).

\vspace{1em}
\subsubsection{Rototraslazione}
Talvolta si parla anche di \textbf{rototraslazione}, ovvero un moto che è sia rotatorio che traslatorio, per la cui descrizione sono fondamentali $6$ parametri.

\vspace{1em}
\subsection{Corpi rigidi - Dinamica}
Partendo dal presuppposto che un corpo rigido è approssimabile tramite il modello di un corpo esteso, ossia un \textbf{assieme di punti materiali con distanze relative fisse}.\\
L'applicazione del concetto di centro di massa ad un assieme di punti permette di scrivere la legge:
\[\boxed{\vec F_{\text{ext}} = M \cdot \vec a_{\text{CM}}}\]
che è una proprietà totalemente generale e, quindi, include anche il caso dei corpi rigidi. Ciò permette di affermare che la risultate delle forze esterne, anche se ciascuna presenta un punto di applicazione diverso, si comporta come se fosse un'unica forza che agisce sul centro di massa, come se l'intero corpo fosse un singolo punto materiale.

\begin{corollary}
  Naturalmente, se non vi sono forze esterne applicate al corpo esteso
  \[\boxed{\vec F_{\text{ext}} = 0 \longrightarrow \vec a_{\text{CM}} = 0}\]
  ossia l'accelerazione del centro di massa è nulla: ciò significa che il corpo si sposta a velocità costante. Tuttavia, ciò non significa che non vi sia accelerazione, in quanto un corpo può anche ruotare con un'\textbf{accelerazione angolare non nulla}.
\end{corollary}

\vspace{1em}
\noindent
\subsection{Corpi rigidi - Energia cinetica}
Si consideri un sistema di punti materiali (ovvero un insieme di punti le cui velocità e masse possono essere diverse fra di loro); naturalmente, l'energia cinetica totale del sistema è
\[K=\frac{1}{2}\sum_i m_i v_i^2\]
mentre la velocità del centro di massa si calcola come
\[\vec v_{\text{CM}}=\frac{1}{M} \sum_i m_i \vec v_i\]
e volendo spostarsi nel sistema di riferimento del centro di massa, si calcoli la velocità relativa al centro di massa come:
\[\vec v_i = \vec v_{\text{CM}} + \vec v_i'\]
in cui $\vec v_i'$ è proprio la velocità relativa tra il sistema di riferimento dell'osservatore e il sistema di riferimento del centro di massa, ossia la velocità relativa al centro di massa. Allora, ricalcolando nuovamente l'energia cinetica totale, si ottiene
\[K=\frac{1}{2}\sum_i m_i \cdot \left(\vec v_{\text{CM}} + \vec v_{i}'\right)^2=\frac{1}{2}\sum_i m_i \cdot \left(\vec v_{\text{CM}} + \vec v_{i}'\right) \cdot \left(\vec v_{\text{CM}} + \vec v_{i}'\right) = \frac{1}{2}\sum_i m_i \cdot \left(\vec v_{\text{CM}}^2 + \vec v_{i}'^2 + 2 \vec v_{\text{CM}} \cdot \vec v_i'\right)\]
che permette di ottenere
\[K=\frac{1}{2}M v_{\text{CM}}^2 + \frac{1}{2}\sum_i m_i v_i'^2 + \vec v_{\text{CM}} \cdot \hspace{-4em} \underbrace{\sum_i m_i \cdot \vec v_i'}_{0 \text{ perché } \vec p_{\text{tot}}=0 \text{ nel centro di massa}}\]
per cui si ottiene il seguente \textbf{risultato generale} (che non dipende dal fatto che si parli di corpi rigidi o meno)
\[\boxed{K_{\text{tot}}=K_{\text{CM}}+K_{\text{rel}}}\]
in cui $K_{\text{CM}}$ è l'energia cinetica del centro di massa, mentre $K_{\text{rel}}$ è l'energia cinetica del moto relativo al centro di massa.

\vspace{1em}
\noindent
\textbf{Osservazione}: Si osservi che, nel caso di un \textbf{corpo rigido}, il moto relativo al centro di massa è proprio la rotazione (in quanto si parla di un moto che conserva le distanze fisse rispetto al centro di massa), per cui
\[\boxed{K_{\text{tot}}=K_{\text{CM}}+K_{\text{rot}}}\]
in cui $K_{\text{rot}}$ prende il nome di \textbf{energia cinetica di rotazione}. Al fine di determinare tale energia, si consideri una massa $m$ che ruota con velocità angolare $\omega$ descrivendo una circonferenza di raggio $r$:

\begin{figure}[H]
  \centering
  \begin{tikzpicture}[scale=1.5,shorten >=1pt,auto,node distance=1cm,thick,main node/.style={circle,draw,font=\sffamily\Large\bfseries}]
    \node[main node](m) at (2,0.75) {$m$};
    \draw (0,0) node[circ]{} node[below]{asse};
    \draw (0,0) -- (m) node[midway,above left]{$R$};
    \draw[-stealth] ([xshift=-0.18,yshift=0.26cm]m) arc [start angle=30,end angle=75,radius=2.2] node[midway,above right]{$\omega$} coordinate(b);
  \end{tikzpicture}
\end{figure}

\noindent
Da ciò segue che l'energia cinetica del singolo punto materiale è, ovviamente:
\[K=\frac{1}{2}mv^2=\frac{1}{2}m(\omega r)^2=\frac{1}{2}m\omega^2r^2\]
in quanto è nota la relazione $v=\omega r$.\\
Se, invece, si considera un assieme di punti, il risultato generale diviene:
\[K=\frac{1}{2} \sum_i m_i v_i^2 = \frac{1}{2} \sum_i m_i (\omega r_i)^2 = \frac{1}{2} \omega^2 \underbrace{\sum_i m_i r_i^2}_{I}\]
in quanto tutti i punti devono necessariamente ruotare alla stessa velocità angolare $\omega$, la quale è costante. Tale risultato permette di introdurre la quantià seguente
\[\boxed{I=\sum_i m_i r_i^2}\]
che prende il nome di \textbf{momento di inerzia}, in cui $r_i$ è la distanza della massa $i$-esima dall'asse di rotazione. Ecco che sulla base di tale quantità è possibile definire l'energia cinetica totale di un corpo rigido come
\[\boxed{K_{\text{tot}} = K_{\text{CM}} + K_{\text{rot}} = \frac{1}{2}M v_{\text{CM}}^2 + \frac{1}{2}I \omega^2}\]
in cui $I$ è il momento di interzia attorno ad un asse che passa per il centro di massa del corpo rigido (sempre ricordando come $v=\omega r$). È interessante, però, osservare, che l'energia cinetica di un corpo in rotolamento, non è totalmente spesa per aumentare la velocità di traslazione del corpo stesso, ma una parte è anche spesa per il rotolamento steso, come viene esposto chiaramente nell'esempio che segue.

\vspace{1em}
\noindent
\textbf{Esempio}: Si considerano due piani inclinati identici, sul primo del quale vi è una massa quadrata $m$ e sul secondo una sfera, sempre di massa $m$:

\begin{figure}[H]
  \centering
  \colorlet{xcol}{blue!70!black}
  \colorlet{vcol}{green!60!black}
  \colorlet{myred}{red!65!black}
  \colorlet{mypurple}{blue!60!red!80}
  \colorlet{acol}{red!50!blue!80!black!80}
  \tikzstyle{rvec}=[->,xcol,very thick,line cap=round]
  \tikzstyle{vvec}=[->,vcol,very thick,line cap=round]
  \tikzstyle{myarr}=[{Latex[length=3,width=3]}-,xcol]
  \tikzstyle{force}=[->,myred,very thick,line cap=round]
  \tikzstyle{mass}=[line width=0.6,draw=red!30!black, %rounded corners=1,
                    top color=red!40!black!30,bottom color=red!40!black!10,shading angle=30]
  \tikzstyle{ground}=[preaction={fill,top color=black!10,bottom color=black!5,shading angle=20},
                      fill,pattern=north east lines,draw=none,minimum width=0.3,minimum height=0.6]
  \tikzstyle{metal}=[fill,top color=black!40,bottom color=black!20,shading angle=10]
  \begin{tikzpicture}[scale=1.5]
    \def\R{0.5} % disk radius
    \def\W{3.2}  % ground width
    \def\ang{30} % ground angle
    \def\mx{2.5} % mass x position
    \coordinate (X) at (\ang:\mx); % wheel position
    \coordinate (C) at ($(X)+(\ang+90:\R)$); % wheel center
    \draw[thick,top color=blue!20!black!30,bottom color=white,shading angle=\ang+10]
      (0,0) coordinate (O) -- (\ang:\W) coordinate (T) -- ({\W*cos(\ang)},0) coordinate (L) -- cycle;
      \begin{scope}
        \tikzset{shift={(30:2)}, rotate=30}
        \draw[mass] ([xshift=-0.3cm,yshift=-0.5cm]C) rectangle ++(3*\R/2,3*\R/2) ([yshift=-0.2cm]C) node[myred!70!black] {$m$};
      \end{scope}
    \draw pic["$\theta$",draw=black,angle radius=22,angle eccentricity=1.3] {angle=L--O--T};
    \draw[<->] (X) --++ (0,{-\mx*sin(\ang)}) node[midway,right] {$h$};
  \end{tikzpicture}
  \hspace{5em}
  \begin{tikzpicture}[scale=1.5]
    \def\R{0.5} % disk radius
    \def\W{3.2}  % ground width
    \def\ang{30} % ground angle
    \def\mx{2.5} % mass x position
    \coordinate (X) at (\ang:\mx); % wheel position
    \coordinate (C) at ($(X)+(\ang+90:\R)$); % wheel center
    \draw[thick,top color=blue!20!black!30,bottom color=white,shading angle=\ang+10]
      (0,0) coordinate (O) -- (\ang:\W) coordinate (T) -- ({\W*cos(\ang)},0) coordinate (L) -- cycle;
    \draw[mass] (C) circle(\R) node[myred!70!black] {$m$};
    \draw pic["$\theta$",draw=black,angle radius=22,angle eccentricity=1.3] {angle=L--O--T};
    \draw[<->] (X) --++ (0,{-\mx*sin(\ang)}) node[midway,right] {$h$};
  \end{tikzpicture}
  \caption{Confronto di corpi in rotolamento}
  \label{fig:confronto_corpi_rotolamento}
\end{figure}

\noindent
In questo caso, se entrambi partono da un'altezza $h$ e si spostano senza attrito, applicando la conservazione dell'energia meccanica, si ottiene:
\begin{enumerate}
  \item $\displaystyle{\frac{1}{2}m v_f^2 = mgh}$
  \item $\displaystyle{\frac{1}{2}m v_f^2 + \frac{1}{2} I \frac{v_f^2}{r^2} = mgh}$
\end{enumerate}
In cui è evidente come la velocità finale della prima massa sarà maggiore della seconda (e quindi arriva prima), in quanto nel secondo caso parte dell'energia potenziale viene \quotes{persa o spesa} per effettuare la rotazione.

\vspace{1em}
\subsection{Momento di inerzia}
La definizione generale di \textbf{momento di inerzia} è
\[I=\sum_i m_i r_i^2\]
ma se si considera un corpo continuo si ottiene
\[I=\int r^2 \cdot dm\]
e se la densità è data si ha
\[I=\int r^2 \cdot \rho(\vec r) \cdot dV\]
che, in ognuno di taluni tre casi, il concetto di momento di inerzia è interpretabile come \textbf{la resistenza di un corpo al cambiamento di velocità angolare} (che è molto simile alla definizione della massa inerziale): più il momento di inerzia è grande, maggiore sarà la resistenza che il corpo oppone alla variazione della propria velocità angolare e viceversa.\\
Tuttavia, la definizione del momento di inerzia dipende dalla \textbf{posizione} e dall'\textbf{orientazione} dell'asse di rotazione (su uno stesso corpo, ad assi di rotazione differenti corrispondono momenti di interzia differenti: più la massa è distante dall'asse di rotazione, maggiore sarà il suo momento di inerzia).

\vspace{1em}
\noindent
\textbf{Osservazione}: Naturalmente, nel caso di un punto materiale di massa $m$, a distanza $r$ dall'asse di rotazione, si ha che il suo momento di inerzia è calcolabile come:
\[I_{\text{punto}}= m r^2\]
se, invece, si considerano due punti di massa $\dfrac{m}{2}$ (affinché la massa totale del sistema sia sempre $m$) a distanza $r$ dall'asse di rotazione, si ha che
\[I_{2 \text{ punti}} = \frac{m}{2}r^2 + \frac{m}{2}r^2 = mr^2\]
Più in generale, considerando $n$ punti di massa $\dfrac{m}{n}$ a distanza $r$ dal centro si avrà che
\[I_{n \text{ punti}} = \sum_{i=1}^n \frac{m}{n} r^2 = m r^2\]
e si va al limite si ottiene sempre che
\[I_{\infty \text{ punti}} = \sum_{i=1}^\infty \frac{m}{n} r^2 =  \int r^2 \cdot dm = m r^2\]
ed è possibile calcolare il momento di inerzia anche per altri corpi, come mostrato nella tabella seguente:

\begin{table}[H]
  \centering
  \rowcolors{1}{white}{white}
  \setlength{\tabcolsep}{4pt}
  \renewcommand{\arraystretch}{2.3}
  \begin{tabularx}{0.5 \textwidth}{|P|P|}
    \hline
    \textbf{Oggetto} & \textbf{Momento di inerzia}\\
    \hline
    Anello & $\displaystyle{I=m R^2}$\\
    \hline
    Sfera piena & $\displaystyle{I=\frac{2}{5} m r^2}$\\
    \hline
    Sfera cava & $\displaystyle{I=\frac{2}{3} m r^2}$\\
    \hline
    Cilindro & $\displaystyle{I=\frac{1}{2} m r^2}$\\
    \hline
    Cilindro cavo & $\displaystyle{I=m R^2}$\\
    \hline
  \end{tabularx}
  \caption{Tabella dei momenti di inerzia}
  \label{tab:tabella_momenti_inerzia}
\end{table}

\newpage
\noindent
\begin{center}
  21 Aprile 2022
\end{center}
La formulazione per l'energia cinetica del moto di rotazione si calcola considerando il momento di inerzia, il quale dipede dall'asse di rotazione del corpo stesso (in quanto il momento di inerzia si calcola in base alla distanza dei contributi di massa dall'asse di rotazione). È stato osservato, inoltre, quale sia il momento di inerzia di un anello che, come si può facilmente intuire, è lo stesso di un anello cavo.\\
Al fine di determinare il periodo di rotazione sapendo la velocità angolare $\omega$, è possibile impiegare la definizione e isolare $T$ come segue
\[\omega=\frac{2\pi}{T} \longrightarrow T=\frac{2\pi}{\omega}\]
Se si considera un'accelerazione angolare costante $\alpha$ (e negativa) e una velocità angolare iniziale $\omega$, allora il tempo impiegato per arrestarsi è
\[t=\frac{\omega}{\alpha}\]
secondo la definizione per la quale $\omega= \alpha t$ (che rassomiglia alla legge oraria del moto uniformemente accelerato).\\
Se si considera la definizione di energia cinetica di rotazione e di velocità angolare, si ha che
\[K=\frac{1}{2}I \omega^2 \longrightarrow K=\frac{1}{2} I \frac{4\pi^2}{T^2} \longrightarrow T=\sqrt{\frac{2\pi^2 I}{K}}\]
e sapendo che $I=m r^2$, avendo due masse $m$ a distanza $d$ l'una dall'altra, si ottiene
\[I=2m \left(\frac{d}{2}\right)^2 \longrightarrow \frac{1}{2}md^2\]
per cui si ha:
\[T=\sqrt{\frac{2\pi^2 I}{K}}=\sqrt{\frac{\pi^2 m d^2}{K}} = \pi\]
nell'ipotesi in cui tutte le misure siano unitarie.

\vspace{1em}
\noindent
\textbf{Esercizio}: Un blocco di massa $m$ è attacato ad un filo di massa trascurabile, che è avvolto intorno a un ciclindro omogeneo di massa $M$ e raggio $R_0$, come mostrato nella figura seguente:

\begin{figure}[H]
  \centering
  \pgfmathsetmacro{\xdeg}{30}
  \pgfmathsetmacro{\xx}{cos(\xdeg)}
  \pgfmathsetmacro{\xy}{sin(\xdeg)}

  \pgfmathsetmacro{\ydeg}{150}
  \pgfmathsetmacro{\yx}{cos(\ydeg)}
  \pgfmathsetmacro{\yy}{sin(\ydeg)}

  \pgfmathsetmacro{\zdeg}{90}
  \pgfmathsetmacro{\zx}{cos(\zdeg)}
  \pgfmathsetmacro{\zy}{sin(\zdeg)}
  \newcommand{\tdcyl}[5]{% origin x, origin y, origin z, radius, height
        \path (1,0,0);
        \pgfgetlastxy{\cylxx}{\cylxy}
        \path (0,1,0);
        \pgfgetlastxy{\cylyx}{\cylyy}
        \path (0,0,1);
        \pgfgetlastxy{\cylzx}{\cylzy}
        \pgfmathsetmacro{\cylt}{(\cylzy * \cylyx - \cylzx * \cylyy)/ (\cylzy * \cylxx - \cylzx * \cylxy)}
        \pgfmathsetmacro{\ang}{atan(\cylt)}
        \pgfmathsetmacro{\ct}{1/sqrt(1 + (\cylt)^2)}
        \pgfmathsetmacro{\st}{\cylt * \ct}
        \filldraw[fill=white] (#4*\ct+#1,#4*\st+#2,#3) -- ++(0,0,#5) arc[start angle=\ang,delta angle=-180,radius=#4] -- ++(0,0,-#5) arc[start angle=\ang+180,delta angle=180,radius=#4];
        \filldraw[fill=white] (#1,#2,#3+#5) circle[radius=#4];
    }

    \newcommand{\tdcylxy}[5]{% origin x, origin y, origin z, radius, height
        \path (1,0,0);
        \pgfgetlastxy{\cylxx}{\cylxy}
        \path (0,1,0);
        \pgfgetlastxy{\cylyx}{\cylyy}
        \path (0,0,1);
        \pgfgetlastxy{\cylzx}{\cylzy}
        \pgfmathsetmacro{\cylt}{(\cylzy * \cylyx - \cylzx * \cylyy)/ (\cylzy * \cylxx - \cylzx * \cylxy)}
        \pgfmathsetmacro{\ang}{atan(\cylt)}
        \pgfmathsetmacro{\ct}{1/sqrt(1 + (\cylt)^2)}
        \pgfmathsetmacro{\st}{\cylt * \ct}
        \filldraw[fill=white] (#4*\ct+#1,#4*\st+#2,#3) -- ++(0,0,#5) arc[start angle=\ang,delta angle=180,radius=#4] -- ++(0,0,-#5) arc[start angle=\ang+180,delta angle=180,radius=#4];
        \filldraw[fill=white] (#1,#2,#3) circle[radius=#4];
    }
  \begin{tikzpicture}[x={(\xx*1cm,\xy*1cm)},y={(\yx*1cm,\yy*1cm)},z={(\zx*1cm,\zy*1cm)}]
      % \tdcyl{0}{0}{3}{1}{3} % x y z   r h
      \begin{scope}[z={(\xx*1cm,\xy*1cm)},x={(\yx*1cm,\yy*1cm)},y={(\zx*1cm,\zy*1cm)}]
          % This is a x-growing cylinder
          \tdcylxy{0}{0}{0}{1.5}{3} % y z x  r h
          \tdcylxy{0}{0}{0}{1.5}{2.4} % y z x  r h
          \tdcylxy{0}{0}{0}{1.5}{1.8} % y z x  r h
          \tdcylxy{0}{0}{0}{1.5}{1.2} % y z x  r h
      \end{scope}
      \draw (0,0,0) node[circ]{} -- ++(-3,0,0);
      \draw (4.05,0,0) -- ++(2,0,0);
      \draw (0,0,0) -- (0,-1.5,0) node[midway,below]{$M$};
      \draw (0,0,0) -- (0,0,1.5) node[midway,left]{$R_0$};
      \draw (2.4,-1.5,0) -- ++(0,0,-3) node[circ]{} node[below]{$m$} ++(-1.5,1,0) -- ++(3,0,0) -- ++(0,-2,0) -- ++(-3,0,0) -- ++(0,2,0) -- ++(0,0,-1) -- ++(0,-2,0) -- ++(0,0,1) ++(0,0,-1) -- ++(3,0,0) -- ++(0,0,1);
  \end{tikzpicture}
  \caption{Massa collegata ad un cilindro rotante}
  \label{fig:massa_cilindro_rotante}
\end{figure}

\noindent
Il cilindro è libero di girare, con attrito trascurabile, intorno a un asse orizzontale fisso passante per il suo centro. Dopo che il blocco è sceso di un tratto verticale $h$ a partire dalla quiete, si determini la velocità lineare del centro del blocco e la velocità angolare del cilindro rispetto al suo asse di rotazione.\\
Immediatamente è possibile applicare la conservazione dell'energia meccanica del sistema cilindro-blocco: le energie coinvolte sono l'energia cinetica (anche di rotazione) e l'energia potenziale, per cui
\[E_i=E_f \longrightarrow K_i+\mathcal{U}_i=K_f+\mathcal{U}_f\]
ma siccome il sistema, inizialmente, parte dalla quiete, l'energia cinetica iniziale è nulla. Per cui si ha
\[K_f=\mathcal{U}_i-\mathcal{U}_f\]
Ovviamente, la variazione di energia cinetica è $\Delta \mathcal{U}=mgh$, mentre l'energia cinetica finale è data dalla somma dell'energia cinetica del blocco di massa $m$ e la velocità angolare del cilindro, per cui:
\[\frac{1}{2}mv^2 + \frac{1}{2}I\omega^2 = mgh\]
ma siccome $v$ e $\omega$ sono legate fra di loro dalla relazione
\[\omega=\frac{v}{R_0}\]
in quanto un punto a velocità $v$ sulla circonferenza del cilindro presenta la velocità della massa $m$, essendo connessi tramite il filo. Pertanto si ottiene che
\[v^2 \cdot \left(1 + \frac{I}{m \cdot R_0^2}\right) = 2gh\]
e sapendo che il cilindro è omogeneo, allora
\[I=\frac{1}{2} M R_0^2\]
per cui la velocità lineare del centro del blocco si ottiene:
\[\boxed{v = \sqrt{\frac{2gh}{1 + \dfrac{M}{2m}}}}\]
e volendo conoscere la velocità angolare del cilindro, è sufficiente dividere $v$ per $R_0$. Dalla formula ottenuta si capisce bene che se la massa $M$ del cilindro è $0$, allora si ottiene una semplice caduta libera per la quale si può ignorare il cilindro stesso. Se la massa $M \to +\infty$, allora la velocità del blocco $v \to 0$, in quanto il momento d'inerzia del cilindro diviene molto grande e, quindi, la resistenza alla variazione di velocità angolare del cilindro stesso cresce.

\vspace{1em}
\noindent
\textbf{Esempio}: Si consideri un oggetto rotondo si massa $m$ e raggio $R$ che rotola giù da un piano inclinato, partendo da un'altezza $h$ nello stato di quiete:

\begin{figure}[H]
  \centering
  \colorlet{xcol}{blue!70!black}
  \colorlet{vcol}{green!60!black}
  \colorlet{myred}{red!65!black}
  \colorlet{mypurple}{blue!60!red!80}
  \colorlet{acol}{red!50!blue!80!black!80}
  \tikzstyle{rvec}=[->,xcol,very thick,line cap=round]
  \tikzstyle{vvec}=[->,vcol,very thick,line cap=round]
  \tikzstyle{myarr}=[{Latex[length=3,width=3]}-,xcol]
  \tikzstyle{force}=[->,myred,very thick,line cap=round]
  \tikzstyle{mass}=[line width=0.6,draw=red!30!black, %rounded corners=1,
                    top color=red!40!black!30,bottom color=red!40!black!10,shading angle=30]
  \tikzstyle{ground}=[preaction={fill,top color=black!10,bottom color=black!5,shading angle=20},
                      fill,pattern=north east lines,draw=none,minimum width=0.3,minimum height=0.6]
  \tikzstyle{metal}=[fill,top color=black!40,bottom color=black!20,shading angle=10]
  \begin{tikzpicture}[scale=1.5]
    \def\R{0.5} % disk radius
    \def\W{3.2}  % ground width
    \def\ang{30} % ground angle
    \def\mx{2.5} % mass x position
    \coordinate (X) at (\ang:\mx); % wheel position
    \coordinate (C) at ($(X)+(\ang+90:\R)$); % wheel center
    \draw[thick,top color=blue!20!black!30,bottom color=white,shading angle=\ang+10]
      (0,0) coordinate (O) -- (\ang:\W) coordinate (T) -- ({\W*cos(\ang)},0) coordinate (L) -- cycle;
    \draw[mass] (C) circle(\R) node[myred!70!black] {$m$};
    \draw pic["$\theta$",draw=black,angle radius=22,angle eccentricity=1.3] {angle=L--O--T};
    \draw[<->] (X) --++ (0,{-\mx*sin(\ang)}) node[midway,right] {$h$};
  \end{tikzpicture}
  \caption{Massa rotonda in rotolamento}
  \label{fig:massa_rotonda_rotolamento}
\end{figure}

\noindent
Allora, considerando l'attrito trascurabile, si può applicare la conservazione dell'energia meccanica, per cui
\[mgh=\frac{1}{2}mv^2 + \frac{1}{2}I\omega^2\]
ma siccome si parla sempre di \textbf{rotolamento} (e solo in questo caso), è nota la relazione
\[\omega=\frac{v}{R}\]
da cui si ha che
\[\boxed{v=\sqrt{\frac{2gh}{1 + \dfrac{I}{mR^2}}}}\]
Si osservi che il rapporto
\[\dfrac{I}{mR^2}\]
è necessariamente un numero puro, in quanto $I$ presenta, nella sua definizione, sempre il fattore $mR^2$, che quindi si semplifica con il denominatore. Infatti, sapendo come determinare il momento di inerzia $I$ di diversi oggetti, si ottengono i risultati seguenti
\begin{itemize}
  \item Sfera piena
  \[I=\frac{2}{5}mR^2 \longrightarrow v=\sqrt{2gh} \cdot \sqrt{\frac{5}{7}}\]
  \item Sfera cava
  \[I=\frac{2}{3}mR^2 \longrightarrow v=\sqrt{2gh} \cdot \sqrt{\frac{3}{5}}\]
\end{itemize}
Pertanto, \textbf{la velocità di movimento di un corpo in rotolamento} non dipende dalla sua massa, dalla sua dimensione, dal raggio, etc., ma \textbf{dipende solamente dalla sua distribuzione di massa} (ovvero dalla forma dell'oggetto, ossia se è un cilindro, una sfera, etc.)

\vspace{1em}
\noindent
\textbf{Osservazione}: Pertanto, due cilindri, di raggio diverso e massa diversa, ma con densità uniforme, si muoveranno alla stessa velocità.\\
Si considerino, poi, due biglie identiche che sono posizionate su due rotaie a distanze diverse:

\begin{figure}[H]
  \centering
  \begin{tikzpicture}
      \draw[line width=3pt,very thick] (-.75,.75) -- ++(2,2);
      \draw[fill=myred!80!black] (1,1) circle [radius=1];
      \draw[line width=3pt,very thick] (.75,-.75) -- ++(2,2);
  \end{tikzpicture}
  \caption{Biglie in rotazione su rotaie}
  \label{fig:biglie_rotazione_rotaie}
\end{figure}

\noindent
Allora, la biglia che si posiziona sulle rotaie più ravvicinate fra di loro si muove a velocità maggiore, in quanto il raggio di rotazione della sfera rispetto al punto di contatto con le rotaie è più grande rispetto ad una biglia poggiata su rotaie a distanza maggiore. Infatti, in questo caso non è vero che $v = \omega R$ (con $R$ raggio della biglia), ma si ha che $v=\omega r'$, in cui $r'$ è il raggio descritto dalla biglia in rotazione sulle rotaie.\\
Ecco che allora la velocità alla quale si muovono le due biglie si calcola come segue:
\[\boxed{v=\sqrt{\frac{2gh}{1 + \dfrac{I}{m r'^2}}}}\]
in cui più $r' \to R$ (ossia più le rotaie sono vicine tra loro), più la velocità aumenta, ma se $r' \to 0$ (ovvero più le rotaie si allontanano), allora la velocità di traslazione diminuisce a scapito di quella di rotazione.

\vspace{1em}
\subsection{Teorema degli assi paralleli}
Il teortema degli assi paralleli è estremamente utile per conoscere il momento di inerzia di un corpo attorno ad un certo asse sapendo già qual è il suo momento di inerzia attorno ad un altro asse.\\
Si consideri, allora, una massa uniforme, il cui centro di massa, a distanza $d$ dall'asse di rotazione, ruota a velocità angolare $\omega$. Allora, per quanto già visto, si ha
\[K_{\text{tot}} = K_{\text{rot}} + K_{\text{CM}} = \frac{1}{2}I_{\text{CM}} \omega^2 + \frac{1}{2}M v_{\text{CM}}^2\]
per cui si ottiene che
\[K_{\text{tot}}=\frac{1}{2} \underbrace{\left(I_{\text{CM}} + Md^2\right)}_{I_{\text{tot}}} \cdot \omega^2\]
in cui, essendo $d$ la distanza tra il centro di massa e l'asse di rotazione, allora la velocità del centro di massa è $v_{\text{CM}}=\omega d$. Il teorema degli assi paralleli, quindi, è il seguente

% Tabella per le definizione di concetti, etc...
\vspace{1em}
\rowcolors{1}{black!5}{black!5}
\setlength{\tabcolsep}{14pt}
\renewcommand{\arraystretch}{2}
\noindent
\begin{tabularx}{\textwidth}{@{}|P|@{}}
    \hline
    {\textbf{TEOREMA DEGLI ASSI PARALLELI}}\\
    \parbox{\linewidth}{Volendo calcolare il momento di inerzia di un corpo attorno ad un qualsiasi asse, conoscendo il momento di inerzia dello stesso attorno al centro di massa, si può impiegare il \textbf{teorema degli assi paralleli} che afferma che
    \[\boxed{I_p = I_{\text{CM} + Md^2}}\]
    in cui $p$ è il punto attorno a quale il sistema ruota e $d$ è la distanza $p-\text{CM}$, ovverosia la distanza tra il punto attorno al quale avviene la rotazione e il centro di massa.
    \vspace{3mm}}\\
    \hline
\end{tabularx}

\vspace{2em}
\noindent
\textbf{Esempio}: Si consideri una porta di forma rettangolare (di dimensioni $l,h,s$) avente una densità uniforme. Allora il momento di inerzia del suo centro di massa si calcola come
\[I_{\text{CM}}=\int r^2 \cdot dm = \int r^2 \rho dV = \int r^2 \frac{M}{l \cdot h \cdot s} \cdot dv\]
A questo punto, considerando la posizione del centro di massa pari a $\frac{l}{2}$ e siccome $dV = dx \cdot hs$ (ove $hs$ è la sezione della porta), è facile determinare
\[I_{\text{CM}}=\int_{-\frac{l}{2}}^{\frac{l}{2}} x^2 \frac{M}{l \cdot s \cdot h} \cdot h \cdot s \cdot dx = \frac{M}{l} \cdot \int_{-\frac{l}{2}}^{\frac{l}{2}} x^2 \cdot dx = \frac{M}{l} \cdot \left[\frac{x^3}{3}\right]_{-\frac{l}{2}}^{\frac{l}{2}}\]
che si traduce nel seguente risultato
\[\boxed{I_{\text{CM}} = \frac{1}{12}Ml^2}\]
che è un risultato molto più generale che per una porta, ma per un qualsiasi oggetto a parallelepipedo, come un'asta (ovviamente considerando sempre l'asse di rotazione passante per il centro di massa).\\
Tuttavia, una porta non ruota attorno al proprio centro di massa, ma attorno ad uno dei lati, per cui ora è possibile applicare il teorema degli assi paralleli e considerare la distanza del centro di massa da uno dei lati come $\frac{l}{2}$, ottenendo
\[\boxed{I=I_{\text{CM}}+M \cdot \left(\frac{l}{2}\right)^2=\frac{1}{12}Ml^2+\frac{1}{4}Ml^2=\frac{1}{3}Ml^2}\]
per cui è evidente come sia più facile far ruotare una porta attorno al proprio centro di massa invece che attorno ad uno dei lati.

\newpage
\noindent
\subsection{Prodotto vettoriale - Richiamo}
Dati due vettori $\vec{A}$ e $\vec{B}$ aventi un angolo $\theta$ compreso tra i due

\begin{figure}[H]
  \centering
  \begin{tikzpicture}
    \draw[-,fill=white!95!red](0,0)--(3,0)--(4,1)--(1,1)--cycle;
    \node at (2,0.5) {$|\textcolor{blue}{a}\times \textcolor{red}{b}|$};
    \draw[ultra thick,-latex,blue](0,0)--(3,0)node[midway,below]{$a$};
    \draw[ultra thick,-latex,red](0,0)--(1,1)node[midway,above]{$b$};
    \draw[ultra thick,-latex,blue!50!red](0,0)--(0,3)node[pos=0.7,right]{$a\times b$};
    \draw (0.6,0) arc [start angle=0,end angle=45,radius=0.6]
    node[pos=0.7,right]{$\theta$};
  \end{tikzpicture}
  \caption{Prodotto vettoriale}
  \label{fig:prodotto_vettoriale_1}
\end{figure}

\noindent
Allora il prodotto vettoriale tra $\vec A$ e $\vec B$ è
\[\vec C = \vec A \times \vec B\]
in cui
\[\vert \vec C \vert = \vert \vec A \vert \cdot \vert \vec B \vert \cdot \sin(\theta)\]
in cui $\vec C \perp \vec A$ e $\vec C \perp \vec B$ e il suo verso è dato dalla regola della mano destra.\\
Non solo, ma era anche noto come il vettore velocità potesse essere descritto anche come prodotto vettoriale tra il vettore $\vec \omega$ e $\vec r$, per cui $\vec v = \vec \omega \times \vec r$. È molto conveniente, infatti, descrivere il versore dell'asse di rotazione del vettore $\vec \omega$ come
\[\frac{\vec \omega}{\vert \vec \omega \vert}\]
in cui $\vert \omega \vert$ è semplicemente la velocità angolare, calcolabile come il rapporto tra $v$ e $R$.\\
Allo stesso modo, per l'accelerazione angolare $\vec \alpha$ si ha che il suo asse è
\[\frac{\vec \alpha}{\vert \vec \alpha \vert}\]
essendo $\vert \alpha \vert$ l'accelerazione angolare.

\vspace{1em}
\noindent
\textbf{Osservazione}: Si osservi che se l'asse è fisso si ha che $\vec a \parallel \vec \omega$ e se $\vec \alpha$ è costante, allora si ha che
\[\boxed{\vec \omega = \vec \omega_0 + \vec \alpha \cdot t}\]
per cui
\[\boxed{\vec \alpha=\frac{\vec \omega}{t}}\]

\newpage
\noindent
\begin{center}
  26 Aprile 2022
\end{center}
Il teorema degli assi paralleli è fondamentale al fine di determinare il momento di inerzia attorno ad un asse che non passa per il centro di massa: per farlo è sufficiente sommare a tale momento di inerzia il prodotto tra la massa e il quadrato della distanza tra la l'asse considerato e quello passante per il centro di massa.\\
A proposito dell'azione di aprire una porta, si può riflettere in merito a che cosa può influire sulla facilità o meno di compiere tale atto, ovviamente determinabile in funzione del momento di inerzia e della distribuzione di massa:
\begin{itemize}
  \item più grande è la massa, più difficile è aprire la porta;
  \item più grande è la distanza dell'asse di rotazione dall'asse passante per il centro di massa, più difficile è aprire la porta;
  \item se la distribuzione di massa è favorevole, ovvero se vi è più massa concentrata in prossimità dell'asse di rotazione, allora sarà più facile aprire la porta.
  \item se la forza applicata non è parallela allo spostamento, essa dovrà essere maggiore per aprire la porta.
\end{itemize}

\vspace{1em}
\subsection{Momento di forza}
Di seguito si espone la definizione di \textbf{momento di forza}:

% Tabella per le definizione di concetti, etc...
\vspace{1em}
\rowcolors{1}{black!5}{black!5}
\setlength{\tabcolsep}{14pt}
\renewcommand{\arraystretch}{2}
\noindent
\begin{tabularx}{\textwidth}{@{}|P|@{}}
    \hline
    {\textbf{MOMENTO DI FORZA}}\\
    \parbox{\linewidth}{Il \textbf{momento di forza} viene definito come il prodotto vettoriale tra il vettore punto di applicazione (che presenta esso stesso come punto di applicazione l'asse di rotazione) e il vettore forza, ovvero
    \[\boxed{\vec \tau = \vec r \times \vec F}\]
    e per determinare il modulo di tale momento è sufficiente calcolare
    \[\vert \vec \tau \vert = \vert \vec r \vert \cdot \vert \vec F \vert \cdot \sin(\theta) = r \cdot (F \cdot \sin(\theta)) = (r \cdot \sin(\theta)) \cdot F\]
    ove $\theta$ è l'angolo compreso tra il vettore $\vec r$ e il vettore $\vec F$ (ovviamente $\theta$ dovrà essere misurato facendo coincidere il punto applicazione dei due vettori).\vspace{3mm}}\\
    \hline
\end{tabularx}

\vspace{1em}
\noindent
\textbf{Osservazione}: In altre parole il modulo del momento di forza è dato dal prodotto tra la distanza dall'asse e la componente perpendicolare della forza (ovvero $r \cdot (F \cdot \sin(\theta))$):

\begin{figure}[H]
  \centering
  \begin{tikzpicture}
    \draw[thick, -stealth] (0,0) -- ++(2,0) node[midway,above]{$\vec r$};
    \draw[thick, -stealth] (2,0) -- ++(2,2) node[midway,below right]{$\vec F$};
    \draw[thick, -stealth] (2,0) -- ++(0,2) node[midway,left]{$\vert F_y \vert = \vert F \vert \sin(\theta)$};
    \draw[dashed] (2,2) -- (4,2) (2,0) -- ++(1,0);
    \draw [draw = orange] (2,0) ++(.8,0) arc (0:45:0.8) node [pos=.4, left] {\textcolor{orange}{$\theta$}};
  \end{tikzpicture}
\end{figure}

\noindent
O ancora, il modulo del momento di forza è dato dal prodotto tra la forza e il \textbf{braccio}, ossia la distanza del vettore forza dal punto di applicazione del vettore $\vec r$ (ovvero $(r \cdot \sin(\theta)) \cdot F$, in cui $(r \cdot \sin(\theta))$ prende il nome di braccio):

\begin{figure}[H]
  \centering
  \begin{tikzpicture}
    \draw[thick, -stealth] (0,0) -- ++(3,0) node[midway,above]{$\vec r$};
    \draw[thick, -stealth] (3,0) -- ++(1,1) node[midway,below right]{$\vec F$};
    \draw[dashed] (3,0) -- ++(-1,-1);
    \draw[thick, -stealth] (0,0) -- (1.5,-1.5) node[midway,below left]{$r \sin(\theta)$};
    \draw[thick, -stealth] (1.5,-1.5) -- ++(1,1) node[midway,below right]{$\vec F$};
  \end{tikzpicture}
\end{figure}

\vspace{1em}
\noindent
\textbf{Osservazione}: Si consideri la seguente rappresentazione tridimensionale del momento di forza, nel quale si pone l'asse di rotazione in corrispondenza dell'asse $z$:

\begin{figure}[H]
  \centering
  \begin{tikzpicture}[scale=1.5]
    \draw[-stealth] (0,0)  -- ++(0,2) node[at end, above]{$\hat{z}$};
    \draw[-stealth] (-1,1) -- ++(4,-4) node[at end, below]{$y$};
    \draw[-stealth] (1,1)  -- ++(-4,-4) node[at end, below]{$x$};
    \draw[very thick, -stealth] (0,0)  -- ++(.5,-2) node[midway,below left]{$\vec r$};
    \draw[very thick, -stealth] (.5,-2)  -- ++(2,0) node[midway,above]{$\vec F$};
    \draw[dashed] (2.5,-2) -- ++(-.45,.65) coordinate(a);
    \draw[dashed] (2.5,-2) -- ++(-1.8,-.75) coordinate(b);
    \draw[very thick, -stealth] (.5,-2)  -- (a) node[midway,above]{$\vec F_{\theta}$};
    \draw[very thick, -stealth] (.5,-2)  -- (b) node[midway,left]{$\vec F_{\vec r}$};
    \draw [draw = orange] (.5,-2) ++(.5,0) arc (0:-73:0.5) node [pos=.4, left] {\textcolor{orange}{$\theta$}};
  \end{tikzpicture}
  \caption{Calcolo del momento di forza in 3 dimensioni}
  \label{fig:momento_forza_tridimensionale}
\end{figure}

\noindent
Allora, giacché è noto che è possibile scomporre $\vec F$ nelle sue componenti \emph{radiale} $\vec F_{\vec r}$ e \emph{angolare} $\vec F_{\theta}$
\[\vec F = F_r \cdot \hat{r} + F_{\theta} \cdot \hat{\theta}\]
si può semplicemente calcolare il momento di forza come
\[\vec \tau = \vec r \times \vec F = r \cdot F \sin(\theta) \cdot \hat{k} = \tau_z \cdot \hat{k} = r \cdot F_\theta \cdot \hat{k}\]
Applicando, inoltre, la seconda legge della dinamica $\vec F = m \vec a$, si evince facilmente come la componente di $\vec F$ in direzione $\theta$ sia proprio data dal prodotto tra la massa $m$ e l'accelerazione in direzione $\theta$
\[F_\theta=m a_\theta=mr\alpha\]
l'ultima uguaglianza è giustificata dal fatto che l'accelerazione perpendicolare all'asse di rotazione si può calcolare come
\[a_\theta=r \cdot \alpha=r \cdot \frac{d \omega}{d t} = r \cdot \frac{1}{r} \cdot \frac{dv}{dt} = \frac{dv}{dt}\]
dal momento che
\[\boxed{\vec a_\theta = \vec r \cdot \alpha}\]
Pertanto si evince come
\[\boxed{\tau_z=r F_\theta=m r^2 \alpha}\]
ciò permette di capire come il momento di forza $\tau$ produca un'\textbf{accelerazione angolare}.

\vspace{1em}
\noindent
Se ora si consiera un corpo rigido, al fine di determinare il momento di forza totale è necessario sommare i contributi di tutte le forze applicate al corpo rigido, per cui
\[\tau_z=\sum_i \left(\vec r_i \times \vec F_i\right) = \sum_{i} r_i \cdot F_{i,\theta}\]
ma siccome in un corpo rigido tutti i punti presentano la stessa accelerazione angolare è facile capire come
\[\sum_{i} r_i \cdot F_{i,\theta} = \sum_{i} r_i \cdot m_i r_i \alpha = \sum_{i} r_i^2 m_i \alpha = I \alpha\]
per cui si ottiene che
\[\boxed{\vec \tau = I \vec \alpha}\]
che è un esatto analogo della seconda legge della dinamica $\vec F = m \vec a$.

\vspace{1em}
\noindent
\textbf{Attenzione}: Si osservi che la legge
\[\vec \tau = I \vec \alpha\]
è valida e applicabile solamente se
\[\left\{
    \rowcolors{1}{white}{white}
    \begin{array}{l}
      \textbf{asse è fisso}\\
      \text{oppure}\\
      \textbf{asse passa per il centro di massa}
    \end{array}
\right.\]

\vspace{1em}
\noindent
\textbf{Esempio}: Si consideri il cilindro seguente:

\begin{figure}[H]
  \centering
  \pgfmathsetmacro{\xdeg}{30}
  \pgfmathsetmacro{\xx}{cos(\xdeg)}
  \pgfmathsetmacro{\xy}{sin(\xdeg)}

  \pgfmathsetmacro{\ydeg}{150}
  \pgfmathsetmacro{\yx}{cos(\ydeg)}
  \pgfmathsetmacro{\yy}{sin(\ydeg)}

  \pgfmathsetmacro{\zdeg}{90}
  \pgfmathsetmacro{\zx}{cos(\zdeg)}
  \pgfmathsetmacro{\zy}{sin(\zdeg)}
  \newcommand{\tdcyl}[5]{% origin x, origin y, origin z, radius, height
        \path (1,0,0);
        \pgfgetlastxy{\cylxx}{\cylxy}
        \path (0,1,0);
        \pgfgetlastxy{\cylyx}{\cylyy}
        \path (0,0,1);
        \pgfgetlastxy{\cylzx}{\cylzy}
        \pgfmathsetmacro{\cylt}{(\cylzy * \cylyx - \cylzx * \cylyy)/ (\cylzy * \cylxx - \cylzx * \cylxy)}
        \pgfmathsetmacro{\ang}{atan(\cylt)}
        \pgfmathsetmacro{\ct}{1/sqrt(1 + (\cylt)^2)}
        \pgfmathsetmacro{\st}{\cylt * \ct}
        \filldraw[fill=white] (#4*\ct+#1,#4*\st+#2,#3) -- ++(0,0,#5) arc[start angle=\ang,delta angle=-180,radius=#4] -- ++(0,0,-#5) arc[start angle=\ang+180,delta angle=180,radius=#4];
        \filldraw[fill=white] (#1,#2,#3+#5) circle[radius=#4];
    }

    \newcommand{\tdcylxy}[5]{% origin x, origin y, origin z, radius, height
        \path (1,0,0);
        \pgfgetlastxy{\cylxx}{\cylxy}
        \path (0,1,0);
        \pgfgetlastxy{\cylyx}{\cylyy}
        \path (0,0,1);
        \pgfgetlastxy{\cylzx}{\cylzy}
        \pgfmathsetmacro{\cylt}{(\cylzy * \cylyx - \cylzx * \cylyy)/ (\cylzy * \cylxx - \cylzx * \cylxy)}
        \pgfmathsetmacro{\ang}{atan(\cylt)}
        \pgfmathsetmacro{\ct}{1/sqrt(1 + (\cylt)^2)}
        \pgfmathsetmacro{\st}{\cylt * \ct}
        \filldraw[fill=white] (#4*\ct+#1,#4*\st+#2,#3) -- ++(0,0,#5) arc[start angle=\ang,delta angle=180,radius=#4] -- ++(0,0,-#5) arc[start angle=\ang+180,delta angle=180,radius=#4];
        \filldraw[fill=white] (#1,#2,#3) circle[radius=#4];
    }
  \begin{tikzpicture}[x={(\xx*1cm,\xy*1cm)},y={(\yx*1cm,\yy*1cm)},z={(\zx*1cm,\zy*1cm)}]
      % \tdcyl{0}{0}{3}{1}{3} % x y z   r h
      \begin{scope}[z={(\xx*1cm,\xy*1cm)},x={(\yx*1cm,\yy*1cm)},y={(\zx*1cm,\zy*1cm)}]
          % This is a x-growing cylinder
          \tdcylxy{0}{0}{0}{1.5}{3} % y z x  r h
          \tdcylxy{0}{0}{0}{1.5}{2.4} % y z x  r h
          \tdcylxy{0}{0}{0}{1.5}{1.8} % y z x  r h
          \tdcylxy{0}{0}{0}{1.5}{1.2} % y z x  r h
      \end{scope}
      \draw (0,0,0) node[circ]{} -- ++(-3,0,0);
      \draw (4.05,0,0) -- ++(2,0,0);
      \draw (0,0,0) -- (0,-1.5,0) node[midway,below]{$M$};
      \draw (0,0,0) -- (0,0,1.5) node[midway,left]{$R_0$};
      \draw (2.4,-1.5,0) -- ++(0,0,-3) node[circ]{} node[below]{$m$} ++(-1.5,1,0) -- ++(3,0,0) -- ++(0,-2,0) -- ++(-3,0,0) -- ++(0,2,0) -- ++(0,0,-1) -- ++(0,-2,0) -- ++(0,0,1) ++(0,0,-1) -- ++(3,0,0) -- ++(0,0,1);
  \end{tikzpicture}
  \caption{Massa collegata ad un cilindro rotante}
  \label{fig:massa_cilindro_rotante_1}
\end{figure}

\noindent
Si determini l'accelerazione della massa $m$. Naturalmente, sulla massa $m$ vi sono due forze applicate, ossia la forza peso $\vec F_t$ e la forza di tensione $\vec F_T$, mentre sul cilindro vi è applicata una forza di tensione che produce una rotazione dello stesso, come mostrato di seguito:

\begin{figure}[H]
  \centering
  \begin{tikzpicture}
    \draw (0,0) node[circ]{};
    \draw[-stealth] (0,0) -- ++(0,2) node[midway,left]{$\vec F_T$};
    \draw[-stealth] (0,0) node[circ]{} (0,0) -- ++(0,-2) node[midway,left]{$\vec F_t$};
  \end{tikzpicture}
  \hspace{5em}
  \begin{tikzpicture}
    \draw (0,0) node[circ]{};
    \draw (0,0) circle[radius=1.5];
    \draw[dashed] (0,0) -- ++(1.5,0) node[midway,above]{$R_0$};
    \draw[-stealth] (1.5,0) -- ++(0,-2) node[midway,right]{$\vec F_T$};
  \end{tikzpicture}
  \caption{Diagramma a corpo libero della massa e del cilindro}
  \label{fig:diagramma_corpo_libero_massa_cilindro}
\end{figure}

\noindent
Naturalmente, dalla seconda legge di Newton applicata al blocco $m$, si evince come
\[ma = -mg + F_T\]
mentre per quanto riguarda il cilindro, essendo l'asse fisso e $\vec F_T \perp \vec R_0$ si ha che
\[\tau = I \alpha \longrightarrow F_T \cdot R_0 = \frac{1}{2}M R_0^2 \cdot \left(-\frac{a}{R_0}\right) \longrightarrow F_T = -\frac{1}{2}M a\]
in quanto la fune consente di vincolare l'accelerazione angolare del cilindro con quella tangenziale della massa. A questo punto, pertanto, è possibile ottenere la seguente equazione
\[ma = -mg + F_T \longrightarrow ma=-mg+\frac{1}{2}Ma \longrightarrow ma-\frac{1}{2}Ma=-mg \longrightarrow a \cdot \left(m-\frac{1}{2}M\right)=-mg\]
pertanto si ottiene come
\[a=-\frac{m}{\displaystyle{\left(m-\frac{1}{2}M\right)}} \cdot g\]

\vspace{1em}
\noindent
\textbf{Osservazione}: Com'è noto, per quanto riguarda una traiettoria angolare, per la quale un punto materiale percorre uno spostamento $\Delta s$ a distanza $R$ dal centro e descrivendo un angolo $\Delta \theta$, si ha che
\[\Delta \theta=\frac{\Delta s}{R} \longrightarrow \Delta s = R \cdot \Delta \theta \]
Dervivando l'ultima quantità rispetto al tempo si ottiene
\[\frac{\Delta s}{\Delta t} = R \cdot \frac{\Delta \theta}{\Delta t} \longrightarrow v = R \omega\]
e derivando nuovamente rispetto al tempo si ottiene
\[\frac{v}{\Delta t} = R \cdot \frac{\omega}{\Delta t} \longrightarrow a = R \alpha\]
per cui si ottiene
\[\boxed{\alpha=\frac{a}{R}}\]

\vspace{1em}
\subsubsection{Momento di forza dovuto alla gravità}
Si consideri un corpo il cui asse di rotazione non coincide con il centro di massa; allora, considerando una generica una massa $m_i$ che compone il corpo rigido, su di esso agisce una forza peso $\vec F_{t,i}$ che presenta un punto di applicazione posizionato in $\vec r_i$.\\
Allora è ovvio, per quanto detto, che
\[\vec \tau = \sum_i \vec r_i \times \vec F_{t,i} = \sum_i \vec r_i \times (-\hat{j} \cdot m_i g) = -g \cdot \sum_i m_i \vec r_i \times \hat{j}\]
Ma naturalmente si ha che
\[\vec r_i = x_i \cdot \hat{i} + y_i \cdot \hat{j} + z_i \cdot \hat{k}\]
per cui si perviene al risultato seguente
\[\vec r_i \times \hat{j} = x_i \cdot (\hat{i} \times \hat{j}) + y_i \cdot (\hat{j} \times \hat{j}) + z_i \cdot (\hat{k} \times \hat{j}) = x_i \cdot \hat{k} + 0 - z_i \cdot \hat{i}\]
pertanto si ottiene che
\[-g \cdot \sum_i m_i \vec r_i \times \hat{j} = -g \cdot \sum_i m_i \left(x_i \cdot \hat{k} - z_i \cdot \hat{i}\right) = -g \cdot \left(\sum_i m_i x_i\right) \cdot \hat{k} + g \cdot \left(\sum_i m_i z_i\right) \cdot \hat{j}\]
ma ricordando che
\[\frac{\displaystyle{\sum_i m_i x_i}}{\displaystyle{\sum_i m_i}} = x_{\text{CM}} \hspace{1em} \text{e} \hspace{1em} \frac{\displaystyle{\sum_i m_i z_i}}{\displaystyle{\sum_i m_i}} = z_{\text{CM}}\]
si ottiene
\[-g \cdot \left(\sum_i m_i x_i\right) \cdot \hat{k} + g \cdot \left(\sum_i m_i z_i\right) \cdot \hat{j} = -g M x_{\text{CM}} \cdot \hat{k} + g M z_{\text{CM}} \cdot \hat{i}\]
ovvero
\[\boxed{\tau=\vec r_{\text{CM}} \times \left(- \hat{j} M g\right) = \vec r_{\text{CM}} \times \vec F_{t}}\]
per cui il \textbf{momento di forza dovuto alla gravità è uguale al momento di forza applicato solo sul centro di massa}; indipendentemente dalla distribuzione di massa, il momento di forza dovuto alla gravità si calcola considerando tutta la massa concentrata nel centro di massa.\\
Ciò significa anche che se l'asse di rotazione è proprio il centro di massa, allora il momento di forza associato è nullo (essendo il braccio nullo): la forza di gravità non produce alcun momento di forza su un oggetto libero di muoversi.

\vspace{1em}
\subsection{Baricentro}
Di seguito si espone la definizione del concetto di \textbf{baricentro}:

% Tabella per le definizione di concetti, etc...
\vspace{1em}
\rowcolors{1}{black!5}{black!5}
\setlength{\tabcolsep}{14pt}
\renewcommand{\arraystretch}{2}
\noindent
\begin{tabularx}{\textwidth}{@{}|P|@{}}
    \hline
    {\textbf{CENTRO DI GRAVITÀ}}\\
    \parbox{\linewidth}{La definizione di \textbf{baricentro} coincide con quella di \emph{centro di massa}, anche definito \textbf{centro di gravità}, in quanto è il punto di applicazione della forza di gravità.
    \vspace{3mm}}\\
    \hline
\end{tabularx}

\vspace{1em}
\noindent
\textbf{Esempio $\boldsymbol{1}$}: Si consideri un oggetto sospeso ad una parete con l'asse di rotazione non allineato con il centro di massa; allora, in questo caso, si ha un momento di forza dovuto alla forza peso, determinabile come
\[\vec \tau = \vec r_{\text{CM}} \times \vec F_{t} \neq 0\]
Si ha equilibrio solamente quando il \textbf{centro di massa} è allineato alla verticale passante per il perno, ossia il punto attorno al quale il corpo ruota ed è, quindi, vincolato al soffitto. Per tale ragione è sufficiente considerare due misure per determinare il centro di massa: se il corpo è in equilibrio, allora è noto che il centro di massa si trova sulla verticale passante per il perno; ora è sufficiente appendere il medesimo copo ma su un altro punto per determinare una nuova retta nella quale è collocato il centro di massa: l'intersezione delle due rette permette di determinare il baricentro.

\vspace{1em}
\noindent
\textbf{Esempio $\boldsymbol{2}$}: Si consideri un'asta di lunghezza $L$ che, inizalmente verticale, viene inclinata con un angolo $\theta$. Naturalmente si ha che il momento di forza dovuto alla forza di gravità è
\[\vec \tau = \vec r \times \vec F = r F_t \sin(\theta)\]
ma è noto che la distanza del centro di massa è pari a $r=\frac{L}{2}$, per cui
\[\vec \tau = r F_t \sin(\theta) = \frac{L}{2} mg \sin(\theta)\]
inoltre, dal momento che si suppone l'asta di densità uniforme e l'asse in corrispondenza di un una sua estremità, si ottiene
\[\vec \tau=I \alpha = \frac{1}{3}mL^2 \cdot \alpha\]
per cui alla fine è immediato determinare l'accelerazione angolare $\alpha$ come segue
\[\frac{L}{2}mg \sin(\theta)=\frac{1}{3} mL^2 \alpha \longrightarrow \alpha = \frac{3}{2}\frac{g}{L} \sin(\theta)\]
Se ora si volesse calcolare l'accelerazione lineare $a$ nel punto più alto si ottiene:
\[a = \alpha L = \frac{3}{2}g \sin(\theta)\]
e ciò significa che ad un certo angolo si avrà che $a > g$, ossia il punto più alto dell'asta accelera più velocemente del centro di massa, per cui l'intera struttura in caduta potrebbe rompersi.

\vspace{1em}
\subsection{Ribaltamento}
Si consideri un oggetto posizionato su un piano inclinato di un angolo $\theta$ e avente un centro di massa CM posizionato in corrispondenza di $\vec r_{\text{CM}}$, come mostrato di seguito

\vspace{1em}
\noindent
\begin{figure}[H]
  \centering
  \newcommand{\ang}{30}

  \begin{tikzpicture} [font = \small, scale=1.5]

  % triangle:
  \draw [draw = orange, fill = orange!15] (0,0) coordinate (O) -- (\ang:6)
  	coordinate [pos=.45] (M) |- coordinate (B) (O);

  % angles:
  \draw [draw = orange] (O) ++(.8,0) arc (0:\ang:0.8)
  	node [pos=.4, left] {$\theta$};
  \draw [draw = orange] (B) rectangle ++(-0.3,0.3);

  \begin{scope} [-latex,rotate=\ang]

  % Object (rectangle)
  \draw [fill = purple!30,
  	draw = purple!50] (M) rectangle ++ (1,.6);

  % Weight Force and its projections
  \draw [dashed] (M) ++ (.5,.3) coordinate (MM) -- ++ (0,-1.29)
  	node [very near end, right] {$\vec{F}_t \cdot \cos{\theta}$};

  \draw [dashed] (MM) -- ++ (-0.75,0)
  	node [very near end, left] {$\vec{F}_t \cdot \sin{\theta}$};

  \draw (MM) -- ++ (-\ang-90:1.5)
  	node [very near end,below left ] {$\vec{F}_t$};

  % Normal Force
  \draw (MM) -- ++ (0,1.29)
  node [very near end, right] {$\vec{F}_N$};

  \draw (M) node[circ]{};
  \draw (MM) node[above right]{CM};
  \draw[-stealth] (M) -- (MM) node[midway,below left]{$\vec r_{\text{CM}}$};
  \end{scope}
  \end{tikzpicture}
  \caption{Piano inclinato}
  \label{fig:piano_inclinato}
\end{figure}

\noindent
Ovviamente, su tale oggetto agisce la forza peso, la forza normale e la forza di attrito statico; facilmente, essendo tutto statico, si può applicare la seconda legge della dinamica, ottenendo:
\[\vec F = m \vec a = 0 = \vec F_s + \vec F_N + \vec F_t\]
per cui si può trovare che
\[F_N=F_t \cdot \cos(\theta) \hspace{1em} \text{e} \hspace{1em} F_s=F_t \cdot \sin(\theta)\]

\vspace{1em}
\noindent
\textbf{Osservazione}: Ovviamente, quando si considerano i vettori posizione, bisogna considerare come loro punto di applicazione l'asse di rotazione al fine di calcolare, poi, il momento di forza.

\newpage
\noindent
\begin{center}
  27 Aprile 2022
\end{center}
Ovviamente quando si considera un oggetto posizionato su un piano inclinato, la condizione di ribaltamento prevede che il centro di massa dell'oggetto sia perfettamente allineata con il perno di rotazione.\\
Chiamata $b$ la distanza tra il perno e il piede della perpendicolare al piano inclinato passante per il centro di massa, la condizione di stabilità prevede semplicemente che
\[\boxed{\tan(\theta) < \frac{b}{y_{\text{CM}}}}\]
per cui per aumentare la stabilità è sufficiente abbassare il baricentro o aumentare la base di appoggio, in modo tale da rendere il rapporto più grande possibile.

\vspace{1em}
\noindent
\textbf{Esercizio}: Dal lato passeggiero, una Jeep si ribalta se è inclinata più di $41^\circ$. La velocità non deve superare per evitare di ribaltarsi in una curva di raggio $R=100$ m.\\
Al fine di risolvere tale problema, è utile passare ad un sistema di riferimento inerziale, in cui si può considerare la forza apparente centrifuga (se, invece, si considerasse un sistema di riferimento inerziale vi sarebbero troppi contributi da considerare e, siccome, si sta considerando un sistema che accelera, l'asse non è fisso, per cui non si può applicare la condizione $\vec \tau = I \vec \alpha$).\\
Se si passa ad un sistema di riferimento non inerziale, allora si ha solo la forza apparente $\vec F_a$ e la forza peso $\vec F_t$, per cui
\[\vec F_{\text{tot}} = \vec F_a + \vec F_t\]
Pertanto, la condizione di stabilità è che
\[\tan(\theta) < \frac{F_a}{F_t}\]
Naturalmente è noto che la forza apparente è una forza centrifuga, calcolabile come segue
\[f_a=\frac{mv^2}{R} \longrightarrow \tan(\theta)=\frac{v^2}{Rg}\]
Peranto, la velocità che non deve essere superata è
\[\boxed{v=\sqrt{Rg\tan(\theta)}}\]

\vspace{1em}
\subsection{Leve e ingranaggi}
Il concetto di momento di forza si applica perfettamente a quello di leva. Affinché l'intero sistema costituito da una leva sia in equilibrio, si deve verificare la condizione seguente
\[\sum \vec \tau = 0\]
ovvero si deve avere che
\[\vec \tau_1 + \vec \tau_2 = -F_1 L_1 + F_2 L_2 = 0 \longrightarrow F_1 L_1 = F_2 L_2\]
Ottenendo, semplicemente, la condizione seguente
\[\boxed{F_1=F_2 \cdot \frac{L_2}{L_1}}\]
Nel caso degli ingranaggi, è sufficiente considerare il punto di contatto nel quale $\vec F_1 = - \vec F_2$, per cui si ottiene che
\[\tau_1 = r_1 F_1 \hspace{1em} \text{e} \hspace{1em} \tau_2 = r_2 F_2\]
per cui si ottiene semplicemente che
\[\boxed{\frac{\tau_1}{\tau_2}=\frac{r_1}{r_2} \longrightarrow \tau_1=\tau_2 \cdot \frac{r_1}{r_2}}\]
ciò significa che applicando un piccolo momento di forza sulla ruota più grande, questo verrà applificato di un fattore $\frac{r_2}{r_1}$ sulla ruota più piccola (oppure verrà ridotto, nel caso contrario).
... continua ...

\vspace{1em}
\subsection{Potenza}
Com'è noto nel caso di forze e lavoro che la potenza era data dalla variazione del lavoro nel tempo, per cui
\[\boxed{P=\frac{dW}{dt}}\]
ma essendo $dW=\vec F \cdot d \vec r$ e $d \vec r = \vec v dt$ è facile capire come $dW=\vec F \cdot \vec v dt$, per cui, analogamente
\[\boxed{P=\vec F \vec v}\]
Pertanto, siccome è noto che
\[dr = R d\theta = R \omega dt\]
è facile evincere come
\[F dr = F R \omega dt = dW \longrightarrow P=FR \omega\]
ma essendo $FR=\tau$, si perviene facilmente al risultato seguente
\[\boxed{P=\vec \tau \cdot \vec \omega}\]


\vspace{1em}
\subsection{Momento angolare}
Di seguito si espone la definzione di \textbf{momento angolare}:

% Tabella per le definizione di concetti, etc...
\vspace{1em}
\rowcolors{1}{black!5}{black!5}
\setlength{\tabcolsep}{14pt}
\renewcommand{\arraystretch}{2}
\noindent
\begin{tabularx}{\textwidth}{@{}|P|@{}}
    \hline
    {\textbf{MOMENTO ANGOLARE}}\\
    \parbox{\linewidth}{Il \textbf{momento angolare} di un punto materiale viene definito come il prodotto vettoriale tra il vettore posizione e il vettore quantità di moto, ovvero
    \[\boxed{\vec l = \vec r \times \vec p}\]
    in cui è ovvio come $\vec r$ dipende dall'origine scelta. Il momento angolare può essere sempre calcolato sia che vi sia o meno rotazione e può essere calcolato prendendo come punto di riferimento qualsiasi punto.
    \vspace{3mm}}\\
    \hline
\end{tabularx}

\vspace{1em}
\noindent
\textbf{Osservazione}: Si osservi che nel caso di un moto circolare, la quantità di moto è $\vec p = m \vec v$, per cui il momento angolare è
\[\vec l = \vec r \times m \vec v = mR^2 \vec \omega\]
essendo $v=\omega R$ e $\vert \vec r \vert = R$.

\vspace{1em}
\noindent
\textbf{Esempio $boldsymbol{1}$}: Il momento angolare della terra che ruota attorno al Sole è, ovviamente, per quanto appena trovato
\[l=mR^2 \omega = m_t \cdot d_{t-s}^2 \cdot \frac{2\pi}{T} = 2.7 \times 10^{40} \frac{\text{kg m}^2}{\text{s}}\]

\vspace{1em}
\noindent
\textbf{Esempio $boldsymbol{2}$}: Il momento angolare di un elettrone in orbita intorno ad un protone (come l'idrogeno) è sempre dato da
\[l=m R^2 \omega=\frac{h}{2\pi}=\hbar= 1.05 \times 10^{-34} \frac{\text{kg m}^2}{\text{s}}\]
in cui ovviamente è possibile determinare
\[R=\sqrt{\frac{\hbar}{m \omega}}=\sqrt{\frac{\hbar T}{2\pi m}}=5.4 \times 10^{-14} \text{ m} = 0.5 \text{\AA}\]

\vspace{1em}
\subsection{Momento angolare e dinamica}
Per quanto concerne un punto materiale si ha che
\[\frac{d}{dt} \vec l = \frac{d}{dt}(\vec r \times \vec p) = \frac{d \vec r}{dt} \times \vec p + \vec r \times \frac{d \vec p}{dt} = \vec v \times m \vec v + \vec r \times \vec F\]
ma è ovvio come $\vec v \times m \vec v = 0$, mentre $\vec r \times \vec F = \tau$, per cui si ottiene come
\[\frac{d \vec l}{dt}=\vec tau\]

\vspace{1em}
\subsection{Equazioni cardianli della meccanica}
Dato un sistema di punti materiali, è possibile calcolare il momento angolare totale di tale sistema, ovver
\[\vec L = \sum_i \vec l_i\]
calcolando la derivata nel tempo di tale quantità, è facile capire come si ottenga la seguente quantità
\[\frac{d \vec L}{dt} = \sum_i \frac{\vec l_i}{dt} = \sum_i \vec \tau_i\]
ma applicando nuovamente la $3^\text{a}$ legge di Newton, si ottiene che la somma di tutti i momenti di forza interni si cancellano, per cui
\[\frac{d \vec L}{dt} = \sum \vec \tau_{\text{ext}}\]
Ecco che sono state ottenute le due equazioni cardinali della dinamica
\[\boxed{\frac{d \vec L}{dt} = \sum \vec \tau_{\text{ext}}} \hspace{1em} \text{e} \hspace{1em} \boxed{\frac{d \vec p}{dt} = \sum \vec F_{\text{ext}}}\]

\vspace{1em}
\noindent
\textbf{Osservazione}: È nota, ovviamente, a la relazione seguente
\[\vec r \overset{\dfrac{d}{dt}}{\longrightarrow} \vec v \overset{\dfrac{d}{dt}}{\longrightarrow} \vec a\]
e, inoltre, si ha
\begin{itemize}
  \item $m \vec v = \vec p$
  \item $m \vec a = \vec F$
  \item $\frac{d \vec p}{dt} = \vec F$
  \item $\vec F \cdot \vec v = P$
  \item ... continua ...
\end{itemize}


\vspace{1em}
\subsection{Equilibrio statico di un corpo rigido}
È noto, ovviamente, che quando si ha
\[\sum \vec F_{\text{ext}} = 0\]
si ha la conservazione della quantità di moto e, conseguentemente, il centro di massa si muove a velocità costante.\\
Ora, però, è anche noto che se
\[\sum \vec \tau_{\text{ext}} = 0\]
allora il momento angolare è conservato e, quindi, la velocità angolare è costante.\\
Ciò significa che un corpo a riposo, con queste due condizioni, permane nel suo stato di quiete. Pertanto, le condizioni di equilibrio per l'equilibrio statico sono
\[\boxed{\sum \vec F = 0} \hspace{1em} \text{e} \hspace{1em} \boxed{\sum \vec \tau = 0}\]
e tale condizione può essere valutata scegliendo arbitrariamente qualsiasi punto di riferimento.

\vspace{1em}
\noindent
\textbf{Esempio}: Si consideri un albero di natale sul quale agisce, naturalmente, la forza peso $\vec F_t$, e due forze di sostegno $\vec F_1$ e $\vec F_2$ aventi braccio $\vec r_1$ e $\vec r_2$ dal centro di massa.\\
Naturalmente, affinché si abbia equilibrio statico deve essere che
\begin{flalign*}
  \displaystyle{\sum \vec F = 0} & \longrightarrow \vec F_1 + \vec F_2 + \vec F_t = 0 \longrightarrow F_1 + F_2 = mg\\
  \displaystyle{\sum \vec \tau = 0} & \longrightarrow \vec r_1 \times \vec F_1 + \vec r_2 \times \vec F_2 = 0 \longrightarrow x_1 F_1 - x_2 F_2 = 0 \longrightarrow \frac{F_1}{F_2} = \frac{x_2}{x_1}
\end{flalign*}
Al fine di determinare $F_1$ e $F_2$ è sufficiente scrivere, impiegando la seconda equazione
\[F_1=F_2 \frac{x_2}{x_1}\]
mentre dalla prima equazione appare veidente come
\[F_2 \cdot \left(\frac{x_2}{x_1}+1\right) = mg\]
per cui, in conclusione, si ha che
\[F_2 = \frac{mg}{1 + \dfrac{x_2}{x_1}} \hspace{1em} \text{e} \hspace{1em} F_1 = \frac{mg}{1 + \dfrac{x_1}{x_2}}\]

\vspace{1em}
\subsection{Orbita e spin}
Dato un corpo sul quale vengono messi in evidenza il centro di massa e un punto diverso dal centro di massa, in posizione $\vec r_i$ e $\vec r_{\text{CM}}$, è ovviamente possibile determinare
\[\vec r_i = \vec r_{\text{CM}} + \vec r_{\text{rel}}\]
e se ora si calcola il momento angolare, si ottiene che
\[\vec l_i = \vec r_i \times \vec p_i = \vec r_{\text{CM}} \times \vec p_i + \vec r_{\text{rel}} \times \vec p_i\]
Se ora si prende in considerazione l'intera totalità di tutti i punti costituenti il corpo si ottiene
\[\vec L = \sum_i \vec l_i = \vec r_{\text{CM}} \times \underbrace{\sum_i \vec p_i}_{\vec p_{\text{CM}}} + \sum_i \vec r_{\text{rel}} \times \vec p_i\]
per cui è possibile dividere il momento angolare totale in \textbf{momento angolare orbitale} $\vec L_o$ e \textbf{spin} $\vec L_s$, come illustrato di seguito
\[\vec L = \underbrace{\vec r_{\text{CM}} \times \vec p_{\text{CM}}}_{\vec L_o = \text{orbitale}} + \underbrace{\sum_{i} \vec r_{\text{rel}} \times \vec p_i}_{\vec L_s = \text{spin}}\]
per cui si ottiene la relazione fondamentale
\[\boxed{\vec L = \vec L_o + \vec L_s}\]

\newpage
\noindent
\begin{center}
  28 Aprile 2022
\end{center}
È noto che il momento angolare totale cambia quando la risultante totale dei momenti di forza è nullo. Inoltre, affinché vi sia equilibrio statico, la risultante delle forze applicate e dei momenti angolari deve essere nulla.

\vspace{1em}
\subsection{Momento angolare - Corpo rigido}
Si consideri un corpo rigido che ruota attorno ad un asse fissato con velocità angolare $\omega$; si consideri, poi, un generico punto materiale $m_i$ a distanza $\vec r_i$ dall'asse avente velocità $\vec v_i$; allora si ha che
\[\vec L = \sum_i \vec l_i = \sum_i \vec r_i \times m_i \vec v_i = \sum_i m_i \vec r_i \times \vec v_i\]
ma essendo, com'è noto, verificata la relazione seguente
\[\vert \vec v_i \vert = \vert \vec r_i \vert \cdot \omega\]
si può esprimere $\vec v_i$ in funzioene di $\omega$, per cui
\[\vec L = \sum_i m_i r_i^2 \vec \omega = I \omega^2\]
ma siccome $\vec \omega$ è costante per tutti i corpi, si ottiene come il momento angolare per un corpo rigido sia
\[\boxed{\vec L = I \vec \omega}\]

\vspace{1em}
\noindent
\textbf{Esempio}: Il momento angolare di spin della terra è
\[L=I\omega \cong \frac{2}{5} M R^2 \omega = \frac{2}{5}M R^2 \cdot \frac{2\pi}{T} = 7 \times 10^{33} \frac{\text{kg m}^2}{\text{s}}\]

\vspace{1em}
\noindent
\subsection{Conservazione del momento angolare}
È già noto che se non vi sono momenti di forza esterni che agiscono sul sistema, allora è possibile applicare la conservazione del momento angolare, ovvero
\[\frac{d \vec l}{dt} = 0 \longrightarrow \vec L_i = \vec L_f\]

\vspace{1em}
\noindent
\textbf{Esempio $\boldsymbol{1}$}: Si consideri un bambino di massa $m$ che si muove ad una velocità $\vec v$ e che ad un certo punto salta su una giostra circolare di massa $M$ e di raggio $R$, proprio sull'estremità della circonferenza. Se la giostra è inizialmente a riposo, si determini la velocità angolare del sistema alla fine.\\
In questo caso, l'energia in generale non è conservata, in quanto si tratta di una sorta di urto anelastico, in cui i due corpi, dopo l'urto, rimangono collegati fra di loro. Tuttavia, non essendoci momenti di forza esterni che agiscono sul sistema, si può applicare la conservazione del momento angolare, ovvero
\[\vec L_i = \vec L_f\]
Il momento angolare iniziale è dato solamente dal contributo del bambino, il quale si muove ad una velocità $\vec v$ a distanza $R$ dal centro della giostra, per cui
\[\vec L_i = \vec r \times m \vec v = m R v \cdot \hat{k}\]
Per quanto concerne il momento angolare finale, siccome i due corpi rimangono attaccati l'uno all'altro, si può approssimare come un corpo rigido avente momento angolare $I \omega$, in cui $I=I_g+I_b$, per cui
\[\vec L_f = (I_g + I_b) \cdot \omega = \left(\frac{1}{2}MR^2 + mR^2\right) \cdot \vec \omega\]
per cui, siccome $L_i=L_f$, è facile capire come la velocità angolare cercata sia data da
\[\omega = \frac{mv}{\left(\dfrac{1}{2}M + m\right) \cdot R}\]

\vspace{1em}
\noindent
\textbf{Esempio $\boldsymbol{2}$}: Si consideri il sistema Terra-Luna e si determini la posizione relativa del centro di massa dalla Terra e dalla Luna, come mostrato di seguito:
\[x_{\text{CM}} = \frac{m_T \cdot 0 + m_L \cdot d}{m_T + m_L} = \frac{m_L}{m_T + m_L} \cdot d\]
supponendo la posizione $x=0$ in corrispondenza della Terra. Ecco, quindi, che la distanza Terra-centro di massa e Luna-centro di massa si calcola come:
\[d_L=d \frac{m_T}{m_T+m_L} \hspace{1em} \text{e} \hspace{1em} d_T=d \frac{m_L}{m_T+m_L}\]
in questo modo il momento orbitale del sistema Terra-Luna si calcola come segue
\[\vec L_o = (m_T d_T^2 + m_L d_L^2) \cdot \omega = [...] = \frac{m_T m_L}{m_T+m_L} d^2 \omega\]
in cui il prodotto $m_T m_L$ prende il nome di \quotes{\textbf{massa ridotta}}.\\
Avendo, ora, determinato, il momento angolare orbitale, è possibile procedere al calcolo del momento angolare di spin per la Terra (ossia $L_T = I_T \omega_T$) e per la Luna (ossia $L_L = I_L \omega_L$), sfruttando il fatto che la velocità angolare della Luna è la stessa di quella del sistema, visto che mostra alla terra sempre la stessa faccia. Avendo ciò, ora, è possibile calcolare il momento angolare totale, dato dalla somma del momento angolare orbitale e quello di spin.

\vspace{1em}
\noindent
\textbf{Osservazione}: Si osservi che sia nel caso della Terra e della Luna che nel caso del bambino sulla giostra, il momento d'inerzia è dato da $m R^2$

\vspace{1em}
\noindent
\textbf{Esercizio}: Si consideri un'asta \emph{non omogenea} di lunghezza $L=101$ cm e forza peso $F_t=2.0$ N.\\
Naturalmente il diagramma a corpi liberi dello scenario di cui sopra è

\begin{figure}[H]
  \centering
  \begin{tikzpicture}
    \draw (0,0) node[circ]{};
    \draw[-stealth] (0,0) -- ++(0,-1) node[midway,right]{$\vec F_t$};
    \draw[-stealth] (0,0) -- ++(-0.5,0.5) node[midway,left]{$\vec F_{T_1}$};
    \draw[-stealth] (0,0) -- ++(0.5,0.5) node[midway,right]{$\vec F_{T_2}$};
  \end{tikzpicture}
  \caption{}
  \label{}
\end{figure}

\noindent
Naturalmente le condizioni di equilibrio sono
\[\sum \vec F_{\text{ext}} = 0 \hspace{1em} \text{e} \hspace{1em} \sum \vec \tau_{\text{ext}} = 0\]
Tali condizioni si traducono nelle seguenti equazioni (considerando come perno il punto di applicazione della forza peso):
\[
  \left\{
  \rowcolors{1}{white}{white}
  \begin{array}{l}
    \vec F_{T_1} + \vec F_{T_1} + \vec F_t = 0\\
    -x \cdot F_{T_1} \cos(\theta) \cdot \hat{k} + (L-x) \cdot F_{T_2} \cos(\phi) \cdot \hat{k}\\
  \end{array}
  \right.
\]
in cui ovviamente il momento di forza della forza peso è nullo, essendo il braccio proprio $0$.

\vspace{1em}
\noindent
\textbf{Osservazione}: Si presti particolare attenzione al segno del prodotto vettoriale: impiegare sempre la regola della mano destra per determinare se il prodotto vettoriale \textbf{entra nella pagina} (ossia è \textbf{negativo}) oppure \textbf{esce dalla pagina} (ossia è \textbf{positivo}).

\newpage
\noindent
\begin{center}
  2 Maggio 2022
\end{center}
Considerando l'esempio precedente, è possibile, come già anticipato, considerare una differente origine attorno alla quale valutare i momenti di forza, come il punto di applicazione della prima forza di tensione $\vec F_{T_1}$, per cui si ottiene la seguente sommatoria:
\[\vec \tau_{T_1} + \vec \tau_{T_2} + \vec \tau_p = -x \cdot F_p + l \cdot F_{T_2} \cdot \cos(\phi)\]
in questo modo, di fatto, il momento di forza associato alla forza di tensione $\vec F_{T_1}$ è nullo e tutti gli altri momenti di forza si calcolano in modo più agevole. Ovviamente il risultato ottenuto è il medesimo a quello precedente.\\
Al fine di determinare il modulo delle forze, è utile richiamare l'equazione di equilibrio di forze e scomporre tale equazione nelle sue due componenti $x$ e $y$, per cui
\begin{flalign*}
  x &: - F_{T_1} \cdot \sin(\theta) + F_{T_2} \cdot \sin(\phi) = 0\\
  y &: F_{T_1} \cdot \cos(\theta) + F_{T_2} \cdot \cos(\phi) - F_t = 0
\end{flalign*}
a cui deve essere aggiunta anche l'equazione dei momenti di forza:
\[-x \cdot F_p + l \cdot F_{T_2} \cdot \cos(\phi)\]
È possibile, dall'ultima equazione, ricavare il vincolo
\[F_{T_2} \cos(\phi) = \frac{x}{L} F_t\]
per cui, dalla seconda equazione si evince come
\[F_{T_1} \cos(\theta) + \frac{x}{L} F_t - F_t \longrightarrow F_{T_1} \cos(\theta) = F_t \cdot \left(1 + \frac{x}{L}\right) \longrightarrow F_T = \frac{\displaystyle{\left(1 + \frac{x}{L}\right)}}{\cos(\theta)} \cdot F_t\]
Dovendo ora, determinare il valore di $x$, è utile impiegare la prima equazione, per cui
\[F_{T_1} \sin(\theta) = F_{T_2} \sin(\phi) \longrightarrow \left(1-\frac{x}{L}\right) \frac{\sin(\theta)}{\cos(\theta)} F_t = \frac{x}{L} \frac{\sin(\phi)}{\cos(\phi)} F_t\]
e moltiplicando ambo i membri per $L$, si ha
\[(L-x) \tan(\theta) = x \tan(\phi) \longrightarrow x \cdot \left(\tan(\phi) + \tan(\theta)\right) = L \tan(\theta)\]
ottenendo, quindi
\[x=L \frac{\tan(\theta)}{\tan(\phi) + \tan(\theta)}\]

\vspace{1em}
\noindent
\textbf{Esercizio}: Il dicso $B$ ruota liberamente con velocità angolare $\omega_B$ ed è solidale con un'asta cilindrica che ne sporge lungo l'asse di rotazione. Il disco $C$ ha un foro al centro che consente di infilarlo sull'asta; il disco $C$, inizialmente in quiete, viene lasciato cadere sul disco $B$. Si determini la velocità angolare del sistema dopo che l'attrito tra $B$ e $C$ li ha portati a una velocità angolare comune. Si determini la variazione dell'energia cinetica del sistema e si esprima il risultato in termini di $\omega_B$, $I_B$ e $I_C$.\\
In questo problema, giacché vi è attrito, non si ha conservazione dell'energia, ma si ha conservazione del momento angolare, per cui
\[\vec L_i = \vec L_f \longrightarrow I_B \vec \omega_B = (I_B+I_C) \vec \omega\]
Per cui si evince semplicemente come
\[\boxed{\vec \omega = \vec \omega_B \frac{I_B}{I_C + I_B}}\]
Al fine di determinare la variazione di energia, si ha
\[\Delta E=E_f-E_i=K_{\text{rot},f}-K_{\text{rot},i}=\left(\frac{1}{2}(I_B+I_C) \omega^2 \right) - \left( \frac{1}{2}I_B\omega_B^2\right) = \left(\frac{1}{2}(I_B+I_C) \cdot \omega_B^2 \frac{I_B^2}{(I_C + I_B)^2} \right) - \left( \frac{1}{2}I_B\omega_B^2\right)=\frac{1}{2} I_B \omega_B^2 \left(\frac{I_B}{I_B+I_C}-1\right)\]

\vspace{1em}
\subsection{Seconda legge di Keplero}
La seconda legge di Keplero afferma che\\
\vspace{1em}
\emph{La congiungente di un pianeta con il Sole spazza aree uguali in intervalli di tempo uguali} (ciò significa che i pianeti, quando sono più lontani dal sole, ruotano più lentamente rispetto a quando si trovano nelle sue prossimità), ovvero
\[A \propto \Delta t\]

\vspace{1em}
\noindent
Tale legge può essere dimostrata, ora, alla luce del teorema di conservazione del momento angolare: naturalmente la velocità $\vec v$ del pianeta attorno al sole è sempre tangenziale all'orbita, avente raggio $\vec r$, per cui
\[\vec l= \vec r \times \vec p = m \vec r \times \vec v\]
ma se si considera $\theta$ l'angolo compreso tra i vettori $r$ e $p$ si ottiene
\[\vec l = m \vec r \vec v = mrv \sin(\theta)\]
Dovendo determinare la relazione con l'area descritta dal pianeta, si osserva semplicemente come
\[dA=\frac{1}{2}ds \cdot r \sin(\theta) = \frac{1}{2}v dt r \sin(\theta)\]
Pertanto, si ottiene
\[dA = \underbrace{\left(\frac{1}{2} vr \sin(\theta) \right)}_{\frac{l}{2m}} dt\]
Pertanto, in conclusione si ottiene che
\[\vec l = \vec r \times \vec p = m \vec r \times \vec v = mr v \sin(\theta) \hat{k}\]
in cui $m r v \sin(\theta)$ è costante in quanto il momento angolare è conservato.

\vspace{1em}
\subsection{Fisica di un quadricottero}
I quadricotteri presentano quattro rotori, ciascuno dei quali garantisce una forza di propulsione $\vec F_s$ identica. Tale velivolo può quindi compiere $4$ movimenti, ciascuno per ogni asse e poi un movimento rotatorio attorno all'asse vertiale.\\
La forza di propulsione $\vec F_s$ può essere spiegata in modo analogo alla forza di propulsione di un razzo, per cui
\[F=\frac{dp}{dt}=v \frac{dm}{dt}\]
in cui, ovviamente, $dm$ è la massa del fluido che viene spostato. Considerando il flusso di aria di forma cilindrica, si ha che
\[dm=\rho dV = \rho dl A = \rho v dt A\]
per cui si ottiene che
\[\frac{dm}{dt}=\rho v A\]
che permette di concludere come la forza di propulsione è dato da
\[\boxed{F_s=v^2 \rho A} \longrightarrow F_s \cong 2A \rho v^2\]
anche se dovrebbe essere $F_s=v^2 \rho 2A$, in quanto è più aria ad essere spostata, pari a circa due volte l'area del cilindro considerato.\\
Avendo a disposizione i dati del drone, si può considerare la velocità per mantenere una posizione stazionaria è
\[v=...continua...\]
È importante, tuttavia, considerare il fatto che i droni possono ruotare su se stessi; non solo, se tutte le eliche ruotassero nello stesso verso, per la consdervazione del momento angolare, il drone stesso ruoterebbe in direzione opposta a quello delle eliche; in realtà, le eliche opposte ruotano nello stesso verso che è opposto rispetto a quello delle altre due; in questo modo, quindi, si ottiene che
\[\vec L_1 + \vec L_2 + \vec L_3 + \vec L_4 = 0\]
per cui il drone è stabile.\\
Al fine di muoversi, il drone deve modificare la forza di propulsione di opprtune eliche:
\begin{itemize}
  \item se il drone deve alzarsi, allora tutte le eliche devono fornire la stessa forza di propulsione;
  \item se il drone deve muoversi in orizzontale, è necessario aumentare la forza di propulsione delle eliche opposte al movimeno, affinché la forza risultante sia orizzontale.
  \item se il drone deve ruotare su se stesso, allora saranno le eliche che ruotano nello stesso verso a dover presentare la stessa forza di propulsione, ottenendo un momento angolare tale da garantire la rotazione cercata.
\end{itemize}




\newpage
\section{Oscillazioni}

\newpage
\section{Solidi e fluidi}

\newpage
\section{Temperatura e calore}

\newpage
\section{Il primo principio della termodinamica}

\newpage
\section{Il secondo principio della termodinamica}


\end{document}
