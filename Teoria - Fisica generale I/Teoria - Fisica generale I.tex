\documentclass[a4paper]{extarticle}
\usepackage[utf8]{inputenc}
\usepackage[italian]{babel}
\selectlanguage{italian}
\usepackage[table]{xcolor}
\usepackage{xcolor}
\usepackage{circuitikz}
\usetikzlibrary{positioning, circuits.logic.US}
\usetikzlibrary {shapes.gates.logic.US, shapes.gates.logic.IEC, calc}
\tikzset {branch/.style={fill, shape = circle, minimum size = 3pt, inner sep = 0pt}}
\usetikzlibrary{matrix,calc}
\usepackage{multirow}
\usepackage{float}
\usepackage{geometry}
\usepackage{pgfplots}
\usepackage{tabularx}
\usepackage{pgf-pie}
\usepackage{tikz}
\usepackage{amsmath}
\usepackage{amssymb}
\usepackage{color, soul}
\usepackage{fancyhdr}
\usepackage{graphicx}
\usepackage{subfig}
\graphicspath{ {./img/} }
\newtheorem{theorem}{Teorema}[section]
\newtheorem{corollary}{Corollario}[theorem]
\newtheorem{lemma}[theorem]{Lemma}

% Specifiche
\geometry{
 a4paper,
 top=20mm,
 left=30mm,
 right=30mm,
 bottom=30mm
}

\pagestyle{fancy}
\fancyhf{}
\fancyhead[LO]{\nouppercase{\leftmark}}
\fancyfoot[CE, CO]{\thepage}
\addtolength{\headheight}{1em}
\addtolength{\footskip}{-0.5em}

\newcommand{\quotes}[1]{``#1''}
\renewcommand\tabularxcolumn[1]{>{\vspace{\fill}}m{#1}<{\vspace{\fill}}}
\renewcommand\arraystretch{}
\newcolumntype{P}{>{\centering\arraybackslash}X}

\title{\textbf{Università di Trieste\\ \vspace{1em}
Laurea in ingegneria elettronica e informatica}}
\author{Enrico Piccin - Corso di Fisica generale I - Prof. Pierre Thibault}
\date{Anno Accademico 2021/2022 - 1 Marzo 2022}

\begin{document}

\vspace{-10mm}
\maketitle

\tableofcontents
\newpage

\noindent
\begin{center}
  1 Marzo 2022
\end{center}

\section{Introduzione}
La \textbf{Fisica} è lo studio della materia e delle sue interazioni. La \textbf{Fisica classica} è divisa in tre macroaree:
\begin{enumerate}
  \item Meccanica classica
  \item Termodinamica
  \item Elettromagnetismo
\end{enumerate}
La Fisica è organizzata in
\begin{itemize}
  \item \textbf{Leggi}: relazioni fra grandezze fisiche
  \item \textbf{Principi}: affermazioni generali da reputare vere
  \item \textbf{Modelli}: analogie o rappresentazioni pratiche su cui basare il proprio studio
  \item \textbf{Teoria}: insieme di leggi, principi e modelli
\end{itemize}

\vspace{1em}
\subsection{Metodo scientifico}
Il \textbf{metodo scientifico} si basa su \textbf{osservazioni} della realtà circostante, a cui seguono delle \textbf{ipotesi}, ossia delle possibili spiegazioni dei fonmeni osservati, basati sulle osservazioni precedentemente formulate.\\
Dopo aver esposto le proprie ipotesi, esse devono essere verificate, mediante degli \textbf{esperimenti}, a cui seguono delle \textbf{analisi} dei risultati sperimentali ottenuti. Il processo di analisi viene seguito da delle \textbf{conclusioni} che \quotes{concludono} il metodo scientifico.\\
Naturalmente tale circuito non è chiuso, in quanto ciascuna di queste fasi può essere ripetuta più e più volte. La parte più importante di tale \emph{metodo scientifico} è la dimostrazione, così come la verifica tramite \textbf{sperimentazioni} delle proprie ipotesi, in quanto le ipotesi devono essere \textbf{sempre verificate}. Tale processo permette di sviluppare leggi e teorie con un fondamento concreto e solido.

\vspace{1em}
\noindent
\textbf{Osservazione}: Si osservi che \textbf{verificare un'ipotesi} non significa dimostrare che un'ipotesi è vera, ma \textbf{verificare che un'ipotesi può essere contraddetta}, ovvero ci si deve assicurare che una \textbf{teoria deve essere \quotes{falsificabile}}, ossia che può essere dimostrato che essia sia falsa.

\newpage
\section{Unità e vettori}

\vspace{1em}
\subsection{Grandezza fisica}
Alla base della \textbf{Fisica} si pone il concetto di \textbf{grandezza fisica}. Non è facile, per esempio, definire che cosa sia il \emph{tempo}; tuttavia, la soluzione più immediata è quella che prevede di definire la misura del tempo come ciò che si riesce a misurare tramite, per esempio, un orologio.\\
Si parla, in tale caso, di \textbf{definizione operativa}:

% Tabella per le definizione di concetti, etc...
\vspace{1em}
\rowcolors{1}{black!5}{black!5}
\setlength{\tabcolsep}{14pt}
\renewcommand{\arraystretch}{2}
\noindent
\begin{tabularx}{\textwidth}{@{}|P|@{}}
    \hline
    {\textbf{DEFINIZIONE OPERATIVA}}\\
    \parbox{\linewidth}{Una grandezza fisica è definita solo dalle operazioni necessarie per misurarla.
    \vspace{3mm}}\\
    \hline
\end{tabularx}

\vspace{1em}
\noindent
Inoltre, le grandezze fisiche si esprimono in termini di un \textbf{campione}, il quale prende il nome di \textbf{unità}.\\
In Fisica, inoltre, si distinguono due diverse categorie di grandezze fisiche:
\begin{enumerate}
  \item Grandezze fisiche fondamentali
  \item Grandezze fisiche derivate
\end{enumerate}
Le \textbf{grandezze fisiche fondamentali} sono $3$:
\begin{enumerate}
  \item Tempo: il tempo presenta come unità il \textbf{secondo} (s) che, dal $1967$, è stato definito come
  \[9192631170 \text{ volte il periodo di oscillazione di una risonanza dell'atomo di Cesio } ^{133}C\]
  Prima di tale data, il secondo era definito come una suddivisione del giorno, ma tale definizione era imprecisa: la terra non ruota sempre con la stessa velocità.

  \item Lunghezza: la lunghezza presenta come unità il \textbf{metro} (m), il quale viene definito come
  \[\frac{1}{299 782 458} \text{ la distanza percorsa dalla luce in $1$ s}\]
  Prima di tale definizione, il metro era definito come \(\frac{1}{10000}\) la distanza tra equatore e polo.\\
  La nuova definizione, tuttavia, è più precisa, in quanto la velocità della luce è \textbf{costante}, fissata in quanto su tale costante si definisce il metro.

  \item Massa: la massa presenta come unità il \textbf{chilogrammo} (kg), il quale viene definito in funzione della \textbf{costante di Planck} ($h = 6.62607015 \times 10^{-34} \text{ kg} \text{ m}^{2} \text{ s}^{-1}$). Prima di tale definizione, il chilogrammo era definito con riferimento ad un campione presente a Parigi e su cui si faceva riferimento per ogni misura di massa.
\end{enumerate}

\noindent
Le grandezze fisiche fondamentali permettono, poi, di definire le grandezze fisiche derivate, quale il \textbf{Volume}, la \textbf{Forza}, etc.

\vspace{1em}
\subsection{Cifre significative e incertezza}
In Fisica, quando si effettuano delle misurazioni, deve essere sempre specificata la precisione e, dunque, l'incertezza. Infatti, \textbf{tutte le msiure hanno un livello di incertezza}.\\
Per esempio
\begin{flalign*}
  L & = 1.82 \pm 0.02 \text{ m}\\
  m & = 3.5 \pm 0.1 \text{ kg}
\end{flalign*}
Da notare che l'indicazione dell'incertezza è sempre (o quasi) data da una sola cifra: altrimenti si avrebbe incertezza nell'incertezza. L'indicazione dell'incertezza è la base della \textbf{fisica sperimentale}.\\
Nella pratica, tuttavia, l'indicazione dell'incertezza è ridondante e pesante. Per indicare il livello di precisione si ricorre alle cifre significative.\\
Per esempio
\begin{flalign*}
  L & = 1.82 \text{ m} = 1.82 \pm 0.01 \text{ m}\\
  m & = 3.5 \text{ kg} = 3.5 \pm 0.1 \text{ kg}
\end{flalign*}

\vspace{1em}
\subsubsection{Operazioni di base}
Per la gestione delle cifre significative nelle operazioni di calcolo è importante tenere a mente che
\begin{itemize}
  \item Moltiplicazione e Divisione: bisogna considerare come cifre significative del prodotto o del quoziente il più basso numero di cifre significative dei fattori o di dividendo e divisore.\\
  Per esempio
  \[1,1 \text{ m} \times 3.45 \text{ m} = 3.8 \text{ m}^2\]
  in quanto il più basso numero di cifre significative dei fattori è $1$.

  \item Addizione e Sottrazione: bisogna considerare come cifre significative della somma o differenza il più basso numero di decimali degli addendi o del minuendo e sottraendo.\\
  Per esempio
  \[1.1 \text{ m} - 12 \text{ cm} = 1.1 \text{ m} - 0.12 \text{ m} = 0.98 \text{ m} = 1.0 \text{ m}\]
  in quanto il più basso numero di decimali tra minuendo e sottraendo è $1$.
\end{itemize}

\vspace{1em}
\subsection{Ordini di grandezza}
Molto spesso, nelle stime è importante non tanto la precisione delle misure, ma l'ordine di grandezza delle stesse, in modo tale da effettuare un macroconfronto utile per delle valutazioni pratiche e veloci.\\
Lo scopo, quindi, dell'impiego degli ordini di grandezza è quello di effettuare dei calcoli veloci e, quindi, delle stime. Più precisamente:

% Tabella per le definizione di concetti, etc...
\vspace{1em}
\rowcolors{1}{black!5}{black!5}
\setlength{\tabcolsep}{14pt}
\renewcommand{\arraystretch}{2}
\noindent
\begin{tabularx}{\textwidth}{@{}|P|@{}}
    \hline
    {\textbf{ORDINE DI GRANDEZZA}}\\
    \parbox{\linewidth}{L'ordine di grandezza di una misura è la \textbf{potenza di $10$ più vicina}.
    \vspace{3mm}}\\
    \hline
\end{tabularx}

\vspace{1em}
\noindent
\textbf{Esempio}: Un ingegnere deve fabbricare un nuovo pacemaker. Si stimi quanti battiti di cuore deve fare senza malfunzionamento. Per effettuare tale stima è necessario conoscere la \textit{media dei battiti al secondo} e \textit{l'aspettativa di vita del soggetto}. Considerando, quindi, come media dei battiti $m_B = 1 \text{ battito}/\text{s}$ e come aspettativa di vita $a_V = 60 \text{ anni}$. La stima selectlanguage
\[m_B \times a_V \times \pi \times 10^7 \text{ s}/\text{anno} = 1 \text{ battito}/\text{s} \times 60 \text{ anni} \times \times 10^7 \text{ s}/\text{anno} = 2 \times 10^9 \text{ battiti}\]

\newpage
\noindent
\begin{center}
  2 Marzo 2022
\end{center}
Il metodo scientifico permette di \textbf{falsificare una teoria}, quindi non è vero che permette di validare una teoria senza ambiguità.

\vspace{1em}
\subsection{Analisi dimensionale}
Il concetto di \textbf{unità} è estremamente importante per parlare di \textbf{analisi dimensionale}. In particolare
\[A = B\]
non può essere valido e corretto formalmente se $A$ e $B$ hanno unità diverse. Questo è molto intuitivo per le grandezze fisiche fondamentali, ma quando si parla di grandezze derivate diventa un punto cruciale: tale concetto permette di validare anche delle possibili soluzioni di test.\\
Per esempio, l'unità di misura della costante di richiamo di una molla si può facilmente ricavare dalla formula della \emph{forza di richiamo}:
\[F = k \cdot x\]
Da cui è immediato capire che
\[\left[k\right] = \frac{\left[F\right]}{\left[x\right]} = \frac{\text{kg m s}^{-2}}{m} = \text{kg s}^{-2}\]

\vspace{1em}
\subsection{Scalari e vettori}
Di seguito si espone la definizione di \textbf{scalare}:

% Tabella per le definizione di concetti, etc...
\vspace{1em}
\rowcolors{1}{black!5}{black!5}
\setlength{\tabcolsep}{14pt}
\renewcommand{\arraystretch}{2}
\noindent
\begin{tabularx}{\textwidth}{@{}|P|@{}}
    \hline
    {\textbf{SCALARE}}\\
    \parbox{\linewidth}{Uno \textbf{scalare} è una grandezza specificata da un numero + unità.\\
    Per esempio la \emph{lunghezza}, la \emph{massa} o l'\emph{energia}.
    \vspace{3mm}}\\
    \hline
\end{tabularx}

\vspace{1em}
\noindent
Mentre un \textbf{vettore} è:

% Tabella per le definizione di concetti, etc...
\vspace{1em}
\rowcolors{1}{black!5}{black!5}
\setlength{\tabcolsep}{14pt}
\renewcommand{\arraystretch}{2}
\noindent
\begin{tabularx}{\textwidth}{@{}|P|@{}}
    \hline
    {\textbf{VETTORE}}\\
    \parbox{\linewidth}{Un \textbf{vettore} è  una quantità definita da un valore e una direzione (e un verso, che può essere implicito nella definizione di direzione).
    \vspace{3mm}}\\
    \hline
\end{tabularx}

\vspace{1em}
\noindent
Tale definizione, tuttavia, pur essendo molto intuitiva, non risulta particolarmente pratica. Si potrebbe anche considerare un vettore come una \textbf{quantità con più valori associati}, ovvero una \textbf{lista di numeri a cui conferiamo un significato}.\\
Per esempio, in algebra un vettore viene indicato come segue
\[\vec{v} = (1, 2, 3)\]
a cui la fisica attribuisce un significato preciso: $1$, $2$ e $3$ sono le componenti associate alle tre diverse dimensioni $x$, $y$ e $z$. Il vettore di cui sopra, allora, si può scrivere come
\[\vec{v} = (v_x, v_y, v_z)\]

\vspace{1em}
\noindent
\textbf{Osservazione}: Anche se tale definizione sembra identica alla definizione del \textbf{punto}, in realtà tale definizione è differente, in quanto
\begin{itemize}
  \item un punto non ha una lunghezza;
  \item non è possibile eseguire la somma di due punti, etc.
\end{itemize}

\vspace{1em}
\subsection{Prodotto con uno scalare}
Dato un vettore
\[\vec{v} = \left(v_x, v_y, v_z\right)\]
e si considera uno scalare $a \in \mathbb{R}$, allora
\[a \cdot \vec{v} = \left(a \cdot v_x, a \cdot v_y, a \cdot v_x\right)\]
in cui il vettore $a \cdot \vec{v}$ è un vettore che
\begin{itemize}
  \item presenta come lunghezza la lunghezza del vettore $\vec{v}$ moltiplicata per $\vert a \vert$;
  \item presenta come direzione la stessa direzione del vettore $\vec{v}$;
  \item presenta come verso lo stesso verso del vettore $\vec{v}$ se $a \geq 0$, mentre avrà verso opposto se $a \leq 0$.
\end{itemize}

\vspace{1em}
\subsection{Somma vettoriale}
Dati due vettori $\vec{u}$ e $\vec{v}$, la loro somma viene eseguita graficamente tramite la \textbf{regola del parallelogramma}, o il metodo \quotes{punta-coda}:

\begin{figure}[H]
  \centering
  \begin{tikzpicture}
    \draw [-stealth, thick, red]    (0,0) -- coordinate[midway](u1) (-3,2);
    \draw [-stealth, thick, blue]   (0,0) -- coordinate[midway](v)  (5,2);
    \draw [-stealth, thick, dashed] (5,2) -- coordinate[midway](u2) (2,4);
    \draw [-stealth, thick, orange] (0,0) -- coordinate[midway](r)  (2,4);
    \draw [thick, red]    (u1) node[above]{$\vec{u}$};
    \draw [thick, blue]   (v)  node[above]{$\vec{v}$};
    \draw [thick, dashed] (u2) node[above]{$\vec{u}$};
    \draw [thick, orange] (r)  node[left]{$\vec{r}$};
  \end{tikzpicture}
  \caption{Somma vettoriale con il metodo \quotes{punta-coda}}
  \label{fig:somma_vettoriale_metodo_punta_coda}
\end{figure}

\noindent
Se $\vec{u}$ e $\vec{v}$ sono espressi nello stesso sistema di riferimento, allora è chiaro che la loro somma sarà data \textbf{componente per componente}, ovvero
\[\vec{u} + \vec{v} = \left(u_x + v_x, u_y + v_y, u_z + v_z\right)\]

\vspace{1em}
\subsection{Versori}
Si definiscano tre versori come segue
\begin{flalign*}
  \hat{i} & = (1,0,0)\\
  \hat{j} & = (0,1,0)\\
  \hat{k} & = (0,0,1)
\end{flalign*}
Allora qualsiasi vettore può essere scritto come
\[\vec{v} = \left(v_x, v_y, v_z\right) = v_x \cdot \hat{i} + v_y \cdot \hat{j} + v_z \cdot \hat{k}\]
in cui, naturalmente, $v_x$, $v_y$ e $v_z$ sono le componenti di $\vec{v}$ in direzione $\hat{i}$, $\hat{j}$ e $\hat{k}$.\\
Naturalmente si scrive $\hat{i}$ e non $\vec{i}$ in quanto
\[\left \vert \hat{i} \right \vert = \left \vert \hat{j} \right \vert = \left \vert \hat{k} \right \vert = 1\]
essi, infatti, prendono il nome di \textbf{versori} o \textbf{vettori unità}.

\vspace{1em}
\subsection{Modulo e direzione}
Di seguito si espone la definizione di \textbf{modulo di un vettore}:

% Tabella per le definizione di concetti, etc...
\vspace{1em}
\rowcolors{1}{black!5}{black!5}
\setlength{\tabcolsep}{14pt}
\renewcommand{\arraystretch}{2}
\noindent
\begin{tabularx}{\textwidth}{@{}|P|@{}}
    \hline
    {\textbf{MODULO DI UN VETTORE}}\\
    \parbox{\linewidth}{Il modulo di un vettore è la sua \quotes{lunghezza geometrica} e si indica come segue
    \[v = \left \vert \vec{v} \right \vert\]
    È chiaro che il modulo può essere \textbf{positivo o nullo}, mai negativo. In termini di componenti il modulo si calcola come segue
    \[v = \sqrt{v_x^2 + v_y^2 + v_z^2}\]
    \vspace{3mm}}\\
    \hline
\end{tabularx}

\vspace{1em}
\noindent
Per esempio, si calcoli il modulo del vettore
\[\hat{n} = \frac{\vec{v}}{v}\]
ovviamente si procede come segue
\[\left \vert \frac{\vec{v}}{v} \right \vert = \frac{1}{\left \vert v \right \vert} \cdot \left \vert \vec{v} \right \vert = \frac{v}{v} = 1\]
Ecco che allora tale vettore è a tutti gli effetti un versore in direzione $\vec{v}$, in quanto di modulo $1$.\\
Questo fa capire come si possa definire un versore associato a qualunque vettore: basta dividere il vettore per il suo modulo.

\vspace{1em}
\noindent
Mentre di seguito si espone la definizione di \textbf{direzione di un vettore}:

% Tabella per le definizione di concetti, etc...
\vspace{1em}
\rowcolors{1}{black!5}{black!5}
\setlength{\tabcolsep}{14pt}
\renewcommand{\arraystretch}{2}
\noindent
\begin{tabularx}{\textwidth}{@{}|P|@{}}
    \hline
    {\textbf{DIREZIONE DI UN VETTORE}}\\
    \parbox{\linewidth}{La direzione di un vettore (e anche il suo verso) è definita, in due dimensioni, come l'angolo $\theta$ che il vettore descrive con il semiasse positivo delle ascisse.\\
    È immediato osservare che
    \begin{center}
      $\begin{array}{c}
        v_x = v \cdot \cos(\theta)\\
        v_y = v \cdot \sin(\theta)
      \end{array}$
    \end{center}
    e si può verificare che
    \[\left \vert \vec{v} \right \vert = \sqrt{v_x^2 + v_y^2} = \sqrt{v^2 \cdot \cos^2(\theta) + v^2 \cdot \sin^2(\theta)} = v \cdot \sqrt{\cos^2(\theta) + \sin^2(\theta)} = v\]
    \vspace{3mm}}\\
    \hline
\end{tabularx}

\newpage
\noindent
\subsection{Prodotto scalare}
Di seguito si espone la definizione di \textbf{prodotto scalare}:

% Tabella per le definizione di concetti, etc...
\vspace{1em}
\rowcolors{1}{black!5}{black!5}
\setlength{\tabcolsep}{14pt}
\renewcommand{\arraystretch}{2}
\noindent
\begin{tabularx}{\textwidth}{@{}|P|@{}}
    \hline
    {\textbf{PRODOTTO SCALARE}}\\
    \parbox{\linewidth}{Il prodotto scalare tra due vettori $\vec{v}$ e $\vec{u}$, in termini di componenti si definisce come segue:
    \[\vec{v} \cdot \vec{u} = v_x \cdot u_x + v_y \cdot u_y + v_z \cdot u_z\]
    che è, naturalmente, uno scalare.\\
    Analogamente si può interpretare il prodotto scalare tra due vettori $\vec{v}$ e $\vec{u}$ come il prodotto dei moduli per il \textbf{coseno} dell'angolo $\theta$ compreso tra i vettori stessi, ovvero
    \[\vec{v} \cdot \vec{u} = v \cdot u \cdot \cos(\theta)\]
    \vspace{-1mm}}\\
    \hline
\end{tabularx}

\vspace{1em}
\noindent
\textbf{Osservazione}: Naturalmente, da tale definizione seguono delle importanti osservazioni:
\begin{itemize}
  \item \(\vec{v} \cdot \vec{v} = v_x^2 + v_y^2 + v_z^2 = \left \vert \vec{v} \right \vert ^2\)
  \item \(\hat{i} \cdot \hat{i} = \hat{j} \cdot \hat{j} = \hat{k} \cdot \hat{k} = 1\)
  \item \(\hat{i} \cdot \hat{j} = \hat{i} \cdot \hat{k} = \hat{j} \cdot \hat{k} = 0\). Questo significa che i due vettori considerati sono ortogonali, ovvero i versori $\hat{i}$, $\hat{j}$ e $\hat{k}$ sono a due a due ortogonali.
\end{itemize}
Si consideri, invece, l'esempio seguente:
\[\vec{v} \cdot \hat{i} = \left(v_x \cdot \hat{i} + v_y \cdot \hat{j} + v_z \cdot \hat{k} \right) \cdot \hat{i} = v_x \cdot \hat{i} \cdot \hat{i} + v_y \cdot \hat{i} \cdot \hat{j} + v_z \cdot \hat{i} \cdot \hat{k} = v_x\]
e questo significa che $\vec{v} \cdot \hat{i}$ è la \textbf{proiezione} del vettore $\vec{v}$ in direzione $\hat{i}$. Tale metodo è molto efficace per effettuare un cambio di base: se al posto dei versori $\hat{i}$, $\hat{j}$ e $\hat{k}$, che presuppongono l'origine del sistema di riferimento in $O = (0,0,0)$ si scegliessere degli altri versori, moltiplicando il vettore $\vec{v}$ per taluni versori si otterrebbero le componenti del nuovo vettore in una nuova base.

\vspace{1em}
\noindent
\textbf{Osservazione}: Se si considerando due vettori $\vec{c}$ e $\vec{d}$, allora il loro prodotto scalare può essere interpretato come segue
\[\vec{c} \cdot \vec{d} \cdot \frac{d}{d} = d \cdot \left(\vec{c} \cdot \frac{\vec{d}}{d}\right)\]
per cui, ricordando che
\[\hat{n} = \frac{\vec{d}}{d}\]
è un versore in direzione del vettore $\vec{d}$, allora il prodotto scalare tra $\vec{c}$ e $\vec{d}$ è proprio la proiezione del vettore $\vec{c}$ sul vettore $\vec{d}$, per quanto appena detto a proposito delle \textbf{proiezioni}, moltiplicata per il modulo del vettore $\vec{d}$.

\newpage
\noindent
\begin{center}
  3 Marzo 2022
\end{center}

\section{Cinematica}
La descrizione del moto di un corpo prende il nome di cinematica.\\ Com'è noto, un vettore è una quantità con \textbf{modulo} e \textbf{direzione} (e \textbf{verso}). La descrizione di un vettore avviene tramite le sue componenti: in particolare, dato un versore $\hat{n}$, la componente di un vettore $\vec{v}$ in direzione del versore $\hat{n}$ è così definita
\[\vec{v} \cdot \hat{n}\]
Per esempio, la componente del vettore $\vec{v}$ lungo l'asse $x$ è
\[\vec{v} \cdot \hat{i} = v_x\]
Inoltre, il prodotto scalare tra due vettori viene definito come ... continua ....
Grazie a ciò è possibile definire il concetto di \textbf{Cinematica}:

% Tabella per le definizione di concetti, etc...
\vspace{1em}
\rowcolors{1}{black!5}{black!5}
\setlength{\tabcolsep}{14pt}
\renewcommand{\arraystretch}{2}
\noindent
\begin{tabularx}{\textwidth}{@{}|P|@{}}
    \hline
    {\textbf{CINEMATICA}}\\
    \parbox{\linewidth}{La cinematica è lo \textbf{studio del moto}, a differenza della \textbf{dinamica} che studia la \textbf{causa del moto} e della \textbf{statica} che studia l'\textbf{equilibrio meccanico}, ossia la \textbf{causa dell'immobilità}.
    \vspace{3mm}}\\
    \hline
\end{tabularx}

\vspace{1em}
\noindent
È chiaro che lo studio di un corpo complesso e non omogeneo è molto più elaborata dello studio di un solo \textbf{punto}. Pertanto, il primo passo per lo studio del moto è quello di studiare il comportamento di un modello standard a cui può essere ricondotto, tramite approssimazione, un altro corpo, a seconda della necessità.

\vspace{1em}
\subsection{Posizione e spostamento}
Dato un punto nello spazio, la sua posizione viene descritta tramite un \quotes{vettore} posizione $\vec{r}$, di cui è possibile calcolare la lunghezza $\left(\vec{r}\right)$, la quale, tuttavia, non ha molto significato dal momento che dipende dalla posizione dell'origine scelta: ovverosia dipende dalla posizione iniziale e, quindi, dal \textbf{sistema di riferimento adottato}. Conoscere il sistema di riferimento è fondamentale, in quanto in base a ciò possono essere effettuate diverse valutazioni in merito alle misurazioni effettuate in base a tale sistema.

\begin{figure}[H]
  \centering
  \begin{tikzpicture}

  \end{tikzpicture}
  \caption{}
  \label{}
\end{figure}

\vspace{1em}
\noindent
Lo \textbf{spostamento}, invece, è proprio un vettore e, com'é intuibile, taluno è definito come la differenza tra due posizioni, ovvero
\[\Delta \vec{r} = \vec{r_2} - \vec{r_1}\]
di cui è possibile calcolare il modulo come segue
\[\left \vert \Delta \vec{r} \right \vert = \left \vert \vec{r_2} - \vec{r_1} \right \vert = \sqrt{\left(x_2 - x_1\right)^2 + \left(y_2 - y_1\right)^2 + \left(z_2 - z_1\right)^2}\]
Per esempio, la lunghezza sull'asse $x$ è
\[\Delta \vec{r} = \Delta x \cdot \hat{i}\]
di cui
\[\left \vert \Delta \vec{r} \right \vert = \left \vert x_2 - x_1 \right \vert\]

\vspace{1em}
\subsection{Posizione in funzione del tempo}
Sia data una funzione spostamento, definita in funzione del tempo $t$, quale
\[\vec{r}(t) = x(t) \cdot \hat{i} + y(t) \cdot \hat{j}\]
in cui
\begin{flalign*}
  x(t) & = 2 \text{ m } + \left(2 \text{ m/s} \right) \cdot t\\
  y(t) & = 0 \text{ m } + \left(4 \text{ m/s} \right) \cdot t
\end{flalign*}
Naturalmente si ha

\vspace{2em}
\noindent
\rowcolors{1}{white}{white}
\begin{tabularx}{\textwidth}{P}
  {
      \centering
      \begin{tikzpicture}
        \begin{axis}[
          axis lines = left,
          xlabel = \(t\),
          ylabel = {\(x(t)\)},
          legend pos=outer north east,
          ymajorgrids=true,
          xmajorgrids=true,
          grid style=dashed,
        ]

        \addplot [
          domain=-2:10,
          samples=100,
          color=red,
        ]
        {2 + 2*x};
        \addlegendentry{\(x(t) = 2 \text{ m } + \left(2 \text{ m/s} \right) \cdot t\)}
        \end{axis}
    \end{tikzpicture}
  }
\end{tabularx}

\vspace{2em}
\noindent
\rowcolors{1}{white}{white}
\begin{tabularx}{\textwidth}{P}
  {
      \centering
      \begin{tikzpicture}
          \begin{axis}[
            axis lines = left,
            xlabel = \(t\),
            ylabel = {\(y(t)\)},
            legend pos=outer north east,
            ymajorgrids=true,
            xmajorgrids=true,
            grid style=dashed,
          ]

          \addplot [
            domain=-2:10,
            samples=100,
            color=blue,
          ]
          {4*x};
          \addlegendentry{\(y(t) = 0 \text{ m } + \left(4 \text{ m/s} \right) \cdot t\)}
          \end{axis}
      \end{tikzpicture}
  }
\end{tabularx}

\vspace{2em}
\noindent
\rowcolors{1}{white}{white}
\begin{tabularx}{\textwidth}{P}
  {
      \centering
      \begin{tikzpicture}
        \begin{axis}[
          axis lines = left,
          xlabel = \(t\),
          ylabel = {\(r(t)\)},
          legend pos=outer north east,
          ymajorgrids=true,
          xmajorgrids=true,
          grid style=dashed,
        ]
      \addplot[
        domain=-2:10,
        samples=100,
        color=orange,
      ]
      ({2 + 2*x},
      {4*x});
      \addlegendentry{\(\vec{r}(t) = x(t) \cdot \hat{i} + y(t) \cdot \hat{j}\)}
      \end{axis}
      \end{tikzpicture}
        }
\end{tabularx}

\vspace{1em}
\subsection{Velocità}
La \textbf{velocità} si pone alla base della cinematica. In fisica la velocità si distingue in due tipolgie
\begin{itemize}
  \item Velocità istantanea
  \item Velocità media
\end{itemize}
Intuitivamente si ha che la velocità media è proprio il rapporto tra uno spostamento e il tempo impiegato per effettuarlo, ovvero
\[\left<v\right> = \frac{\Delta s}{\Delta t} = \frac{x_2 - x_1}{t_2 - t_1}\]
che, graficamente, può essere interpretata come la pendenza (o coefficiente angolare, della congiungente i punti $(x_1,t_1)$ e $(x_2,t_2)$ nel gradico spazio/tempo.\\
Mentre la velocità istantanea è, naturalmente, la derivata nel tempo del vettore spostameto, ovvero
\[\vec{v}(t) = \frac{d\vec{t}}{dt} = \lim_{t \to 0} \frac{x - x_0}{t - t_0}\]
ovvero la retta tangente il grafico della funzione spostamento nel punto $(x_0,t_0)$. Naturalmente essendo un vettore la velocità istantanea, è possibile descriverlo tramite componenti come segue:
\[\vec{v} = \frac{d}{dt} \left(\vec{r}(t)\right) = \frac{dx}{dt} \cdot \hat{i} + \frac{dy}{dt} \cdot \hat{j} + \frac{dz}{dt} \cdot \hat{k}\]
in quanto $\hat{i}$, $\hat{j}$ e $\hat{k}$ non dipendono dal tempo (cosa che potrebbe accadre, comunque). Il modulo della velocità si calcola come segue
\[\left \vert \vec{v} \right \vert = v = \sqrt{v_x^2 + v_y^2 + v_z^2}\]

\vspace{1em}
\noindent
\textbf{Esempio}: Si consideri uno spostamento verso l'alto tale per cui $x_1 = 0 \text{ m}$ e $x_2 = 12000 \text{ m}$ e $t_1 = 2600 \text{ s}$ e $t_2 = 4000 \text{ s}$. Allora si ha che
\[\left<v_z\right> = \frac{x_2 - x_1}{t_2 - t_1} = \frac{12000}{4000 - 2600} = 8.6 \text{ m/s} = 308.6 \text{ km/h}\]
Che è una velocità irrisoria; tuttavia, ciò non sorprende, in quanto è opportuno conoscere anche le altre componenti della velocità, ossia $v_x$ e $v_y$.

\vspace{1em}
\subsection{Accelerazione}
Di seguito si espone la definizione di \textbf{accelerazione}:

% Tabella per le definizione di concetti, etc...
\vspace{1em}
\rowcolors{1}{black!5}{black!5}
\setlength{\tabcolsep}{14pt}
\renewcommand{\arraystretch}{2}
\noindent
\begin{tabularx}{\textwidth}{@{}|P|@{}}
    \hline
    {\textbf{ACCELERAZIONE}}\\
    \parbox{\linewidth}{L'accelerazione viene definita come la derivata prima della velocità nel tempo, o la derivata seconda dello spostamento nel tempo:
    \[\vec{a} = \frac{d \vec{v}}{dt} = \frac{d^2 \vec{r}}{dt}\]
    \vspace{3mm}}\\
    \hline
\end{tabularx}

\newpage
\noindent
\begin{center}
  7 Marzo 2022
\end{center}
Comè noto, l'accelerazione è la derivata prima della velocità in funzione del tempo, o la derivata seconda dello spostamento in funzione del tempo.\\
L'accelerazione è fondamentale in \textbf{meccanica}, in quanto grazie all'accelerazione è possibile definire il concetto di forza: l'accelerazione è la connessione tra cinematica e dimanica.

\vspace{2em}
\noindent
\rowcolors{1}{white}{white}
\begin{tabularx}{\textwidth}{P}
  {
      \centering
      \begin{tikzpicture}
        \begin{axis}[
          axis lines = left,
          xlabel = \(t\),
          ylabel = {\(x(t)\)},
          legend pos=outer north east,
          ymajorgrids=true,
          xmajorgrids=true,
          grid style=dashed,
        ]

        \addplot [
          domain=-2:10,
          samples=100,
          color=red,
        ]
        {(x-3)^3 + 5};
        \addlegendentry{\(x(t) = 2 \text{ m } + \left(2 \text{ m/s} \right) \cdot t\)}
        \end{axis}
    \end{tikzpicture}
  }
\end{tabularx}

\vspace{1em}
\noindent
Avendo a disposizione il grafico dello spostamento, è possibile ora studiarne il comportamento per poi descrivere il comportamento della velocità nel secondo grafico. È sufficiente vedere gli intervalli di crescenza e decrescenza e i punti in cui la derivata è nulla:

\vspace{2em}
\noindent
\rowcolors{1}{white}{white}
\begin{tabularx}{\textwidth}{P}
  {
      \centering
      \begin{tikzpicture}
        \begin{axis}[
          axis lines = left,
          xlabel = \(t\),
          ylabel = {\(v(t)\)},
          legend pos=outer north east,
          ymajorgrids=true,
          xmajorgrids=true,
          grid style=dashed,
        ]

        \addplot [
          domain=-2:10,
          samples=100,
          color=red,
        ]
        {(x-4)^2-2};
        \addlegendentry{\(x(t) = 2 \text{ m } + \left(2 \text{ m/s} \right) \cdot t\)}
        \end{axis}
    \end{tikzpicture}
  }
\end{tabularx}

\vspace{2em}
\noindent
\rowcolors{1}{white}{white}
\begin{tabularx}{\textwidth}{P}
  {
      \centering
      \begin{tikzpicture}
        \begin{axis}[
          axis lines = left,
          xlabel = \(t\),
          ylabel = {\(a(t)\)},
          legend pos=outer north east,
          ymajorgrids=true,
          xmajorgrids=true,
          grid style=dashed,
        ]

        \addplot [
          domain=-2:10,
          samples=100,
          color=red,
        ]
        {x - 4};
        \addlegendentry{\(x(t) = 2 \text{ m } + \left(2 \text{ m/s} \right) \cdot t\)}
        \end{axis}
    \end{tikzpicture}
  }
\end{tabularx}

\vspace{1em}
\noindent
Ecco che il punto in cui la derivata seconda cambia segno è un \textbf{punto di flesso}. ...continua...

\vspace{1em}
\noindent
\textbf{Esempio}: Si consideri la seguente funzione spostamento in funzione del tempo:
\[x(t) = A \cdot \cos(\omega \cdot t)\]
questa è l'equazione di oscillazione di un pendolo o di una molla. Per il calcolo della velocità è sufficiente calcolare la derivata prima, ovvero
\[v(t) = \frac{dx}{dt} = -\omega \cdot A \cdot \sin(\omega \cdot t)\]
e per l'accelerazione è sufficiente derivare nuomvemente la velocità
\[a(t) = \frac{dv}{dt} = -\omega^2 \cdot A \cdot \cos(\omega \cdot t) = -\omega^2 \cdot x(t)\]
Tale risultato ha senso e può essere interpretato anche graficamente, grazie al grafico di una molla: quando lo spostamento è positivo, l'accelerazione è negativa, e viceversa.

\vspace{1em}
\noindent
\textbf{Osservazione}: Quando la velocità è nulla, la posizione si mantiene costante e non cresce o decresce. Quando si ha un punto di salto della velocità si ha una situazione difficile da riprodurre fisicamente, in quanto si ha un crollo della velocità istantanea, come un impulso.\\
Si può utilizzare anche l'integrale per passare dalla velocità allo spostamento.

\vspace{1em}
\subsection{Moto uniformemente accelerato}
Nel moto uniformemente accelerato si ha che l'\textbf{accelerazione} è \textbf{costante}. È noto che
\[\frac{dv}{dt} = a \longrightarrow dv = a \cdot dt\]
questo è possibile farlo sia per una variazione $\Delta$, ma anche per una variazione infinitesimale. Dopo aver ottenuto $dv = a \cdot dt$ si può procedere all'integrazione
\[\int dv = \int a \cdot dt \longrightarrow v = at + c\]
La costante $c$ che compare nella formula ottenuta grazie alla risoluzione di una equazione differenziale è la cosiddettà \textbf{velocità inziale}: se $t=0$, infatti, $v = c = v_0$. L'equazione diviene
\[v(t) = v_0 + a \cdot t\]
Avendo ottenuto l'equazione della velocità, è opportuno ottenere l'equazione dello spostamento, integrando nuovamente, sempre partendo da
\[\frac{dx}{dt} = v \longrightarrow dx = v \cdot dt \longrightarrow \int dx = \int v \cdot dt = \int (v_0 + a \cdot t) dt\]
quindi si ottiene
\[x(t) = \int v_0 \cdot dt + \int a \cdot t \cdot dt + c = v_0 \cdot t + a \cdot \frac{t^2}{2} + c\]
ove $c = x_0 = x(t=0)$. Pertanto l'equazione dello spostamento è
\[x(t) = x_0 + v_0 \cdot t + \frac{1}{2} \cdot a \cdot t^2\]
che rappresenta una parabola, come esposto di seguito

\vspace{2em}
\noindent
\rowcolors{1}{white}{white}
\begin{tabularx}{\textwidth}{P}
  {
      \centering
      \begin{tikzpicture}
        \begin{axis}[
          axis lines = left,
          xlabel = \(t\),
          ylabel = {\(a(t)\)},
          legend pos=outer north east,
          ymajorgrids=true,
          xmajorgrids=true,
          grid style=dashed,
        ]

        \addplot [
          domain=-2:10,
          samples=100,
          color=red,
        ]
        {-(x-2)^2 + 5};
        \addlegendentry{\(x(t) = 2 \text{ m } + \left(2 \text{ m/s} \right) \cdot t\)}
        \end{axis}
    \end{tikzpicture}
  }
\end{tabularx}

\vspace{1em}
\noindent
In cui l'intersezione tra il grafico e l'asse $y$ è $x_0$, mentre la tangente in $(x_0,0)$ è proprio la velocità $v_0$. ... continua ... (35).

\vspace{1em}
\noindent
\textbf{Osservazione}: Si consideri la seguente equazione
\[v(t) = a \cdot t + v_0\]
e si provi ad eliminare $t$ da tale equazione, come segue
\[t = \frac{v - v_0}{a}\]
se ora si considera l'equazione seguente
\[x(t) = \frac{1}{2} \cdot a \cdot t^2 + v_0 \cdot t + x_0\]
e si sostituisce il $t$ calcolato in precedenza a tale equazione si ottiene
\[x(t) = \frac{1}{2} \cdot a \cdot \left(\frac{v - v_0}{a}\right)^2 + v_0 \cdot \left(\frac{v - v_0}{a}\right) + x_0 = \frac{1}{2} \cdot \frac{1}{a} \cdot (v^2 - 2 \cdot v \cdot v_0 + v_0^2) + \frac{1}{a} \cdot (v_0 \cdot v + v_0^2) + x_0\]
ovvero si ottiene
\[x = \frac{v^2}{2a} - \frac{v_0^2}{2a} + x_0\]
quindi
\[v^2 - v_0^2 = 2a \cdot (x - x_0)\]
la quale è \textbf{valida solamente per il moto uniformemente accelerato}.

\vspace{1em}
\subsubsection{Caduta libera dei gravi}
La \textbf{caduta libera} avviene con la \textbf{medesima accelerazione} per tutti i corpi (ovviamente, l'attrazione gravitazionale non è costante in tutto l'universo, ma se si considerano un punto sulla terra e distanze piccole e prossime a quella del raggio terrestre, si può, senza perdita di generalità, considerare un'accelerazione costante $g$).

\vspace{1em}
\noindent
\textbf{Osservazione}: Naturalmente sulla Luna non c'è aria, si è nel vuoto, per cui tutti i corpi vengono attratti dalla Luna solamente per la forza di attrazione gravitazionale, senza che tale moto sia influenzato dall'attrito dell'aria.

\vspace{1em}
\noindent
L'accelerazione gravitazionale può essere interpretata come segue:

\begin{tikzpicture}
  \draw [-stealth] (0,0) -- (0,-5);
\end{tikzpicture}

\vspace{1em}
\noindent
Ove, naturalmente si ha
\[\vec{a} = -g \cdot \hat{j} \hspace{0.5em} \text{con} \hspace{0.5em} 9.8 \text{ m/s}\]
Pertanto si ottengono le seguenti equazioni
\[v_y = -g \cdot t + v_{0y}\]
\[y = - \frac{1}{2} \cdot g \cdot t^2 + v_{0y} + y_0\]
Pe capire che altezza raggiungerà un corpo quando viene lanciato verso l'alto si deve osservare il grafico seguente

\vspace{2em}
\noindent
\rowcolors{1}{white}{white}
\begin{tabularx}{\textwidth}{P}
  {
      \centering
      \begin{tikzpicture}
        \begin{axis}[
          axis lines = left,
          xlabel = \(t\),
          ylabel = {\(y(t)\)},
          legend pos=outer north east,
          ymajorgrids=true,
          xmajorgrids=true,
          grid style=dashed,
        ]

        \addplot [
          domain=0:10,
          samples=100,
          color=red,
        ]
        {-(x-2)^2 + 5};
        \addlegendentry{\(x(t) = 2 \text{ m } + \left(2 \text{ m/s} \right) \cdot t\)}
        \end{axis}
    \end{tikzpicture}
  }
\end{tabularx}

\vspace{1em}
\noindent
Naturalmente si osserva che la velocità nel punto più alto è nulla, in quanto la tangente è verticale, mentre si conosce la velocità inziale $v_0$.\\
Per capire l'altezza, allora, si potrebbe calcolare il tempo impiegato per raggiungere il punto più alto e poi sostituire tale valore all'interno dell'equazione dello spostamento. Alternativamente, si potrebbe considerare la formula seguente
\[v_y^2 - v_{0y}^2 = -2 \cdot g \cdot (y - y_0)\]
e sostitutendo i valori si ha
\[0 - v_{y_0}^2 = 2 \cdot g \cdot (y_m - y_0)\]
\[y_m = y_0 + \frac{v_{0y}^2}{2g}\]

\vspace{1em}
\subsubsection{Moto dei proiettili}
Il moto dei proiettili ha un'importanza storica fondamentale: Aristotele, nel $340$ a.c. parlava di \textbf{moto \quotes{naturale} e \quotes{forzato}}: tuttavia, naturalmente, un oggetto non ha la propensione a muoversi, quindi non ha senso parlare di \emph{forzatura}.\\
Successivamente, Filipono ($490-570$) d.c. ha introdotto il concetto di \textbf{impeto}. Infine, \textbf{Galileo} ($1564-1642$) d.c., basandosi su osservazioni e misurazioni pratiche precedenti (invece che fornire una spiegazioni a priori), ha fornito una \textbf{spiegazione scentifica} a tale fenomeno.\\

% Tabella per le definizione di concetti, etc...
\vspace{1em}
\rowcolors{1}{black!5}{black!5}
\setlength{\tabcolsep}{14pt}
\renewcommand{\arraystretch}{2}
\noindent
\begin{tabularx}{\textwidth}{@{}|P|@{}}
    \hline
    {\textbf{LEGGE ORARIA DEL MOTO DEI PROIETTILI}}\\
    \parbox{\linewidth}{Per lo studio del moto dei proiettili si considera il vettore accelerazione
    \[\vec{a} = a_x \cdot \hat{i} + a_y \cdot \hat{j} = 0 \cdot \hat{i} - g \cdot \hat{j}\]
    Analogamente per la velocità si ha
    \[\vec{v} = v_x \cdot \hat{i} + v_y \cdot \hat{j} = v_{0x} \cdot \hat{i} - (v_{0y} - g \cdot t) \cdot \hat{j}\]
    Per quanto concerne lo spostamento si ha
    \[\vec{r} = x \cdot \hat{i} + y \cdot \hat{j} = (v_{0x} \cdot t + x_0) \cdot \hat{i} - (y_0 + v_{0y} \cdot t - \frac{1}{2} \cdot g \cdot t^2) \cdot \hat{j}\]
    \vspace{-1mm}}\\
    \hline
\end{tabularx}

\vspace{2em}
\noindent
Si consideri il seguente grafico della posizione di un proiettile:

\vspace{2em}
\noindent
\rowcolors{1}{white}{white}
\begin{tabularx}{\textwidth}{P}
  {
      \centering
      \begin{tikzpicture}
        \begin{axis}[
          axis lines = left,
          xlabel = \(x\),
          ylabel = {\(y\)},
          legend pos=outer north east,
          ymajorgrids=true,
          xmajorgrids=true,
          grid style=dashed,
        ]

        \addplot [
          domain=0:10,
          samples=100,
          color=red,
        ]
        {-(x-2)^2 + 5};
        \addlegendentry{\(x(t) = 2 \text{ m } + \left(2 \text{ m/s} \right) \cdot t\)}
        \end{axis}
    \end{tikzpicture}
  }
\end{tabularx}

\vspace{1em}
\noindent
Naturalmente si ha che
\[\vec{v_{0}} = v_{0x} \cdot \hat{i} + v_{0y} \cdot \hat{j}\]
\[\vec{r_{0}} = r_{0x} \cdot \hat{i} + r_{0y} \cdot \hat{j}\]






\newpage
\section{Dinamica}

\newpage
\section{Gravità}

\newpage
\section{Energia}

\newpage
\section{Moto dei sistemi}

\newpage
\section{Corpi rigidi}

\newpage
\section{Oscillazioni}

\newpage
\section{Solidi e fluidi}

\newpage
\section{Temperatura e calore}

\newpage
\section{Il primo principio della termodinamica}

\newpage
\section{Il secondo principio della termodinamica}


\end{document}
