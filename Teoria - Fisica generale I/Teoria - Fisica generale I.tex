\documentclass[a4paper]{extarticle}
\usepackage[utf8]{inputenc}
\usepackage[italian]{babel}
\selectlanguage{italian}
\usepackage[table]{xcolor}
\usepackage{xcolor}
\usepackage{circuitikz}
\usetikzlibrary{calc,patterns,angles,quotes}
\usetikzlibrary{positioning, circuits.logic.US}
\usetikzlibrary {shapes.gates.logic.US, shapes.gates.logic.IEC, calc}
\tikzset {branch/.style={fill, shape = circle, minimum size = 3pt, inner sep = 0pt}}
\usetikzlibrary{matrix,calc}
\usepackage{multirow}
\usepackage{float}
\usepackage{geometry}
\usepackage{pgfplots}
\usepackage{tabularx}
\usepackage{pgf-pie}
\usepackage{tikz}
\usepackage{amsmath}
\usepackage{amssymb}
\usepackage{color, soul}
\usepackage{fancyhdr}
\usepackage{graphicx}
\usepackage{subfig}
\graphicspath{ {./img/} }
\newtheorem{theorem}{Teorema}[section]
\newtheorem{corollary}{Corollario}[theorem]
\newtheorem{lemma}[theorem]{Lemma}

% Specifiche
\geometry{
 a4paper,
 top=20mm,
 left=30mm,
 right=30mm,
 bottom=30mm
}

\pagestyle{fancy}
\fancyhf{}
\fancyhead[LO]{\nouppercase{\leftmark}}
\fancyfoot[CE, CO]{\thepage}
\addtolength{\headheight}{1em}
\addtolength{\footskip}{-0.5em}

\newcommand{\quotes}[1]{``#1''}
\renewcommand\tabularxcolumn[1]{>{\vspace{\fill}}m{#1}<{\vspace{\fill}}}
\renewcommand\arraystretch{}
\newcolumntype{P}{>{\centering\arraybackslash}X}

\title{\textbf{Università di Trieste\\ \vspace{1em}
Laurea in ingegneria elettronica e informatica}}
\author{Enrico Piccin - Corso di Fisica generale I - Prof. Pierre Thibault}
\date{Anno Accademico 2021/2022 - 1 Marzo 2022}

\begin{document}

\vspace{-10mm}
\maketitle

\tableofcontents
\newpage

\noindent
\begin{center}
  1 Marzo 2022
\end{center}

\section{Introduzione}
La \textbf{Fisica} è lo studio della materia e delle sue interazioni. La \textbf{Fisica classica} è divisa in tre macroaree:
\begin{enumerate}
  \item Meccanica classica
  \item Termodinamica
  \item Elettromagnetismo
\end{enumerate}
La Fisica è organizzata in
\begin{itemize}
  \item \textbf{Leggi}: relazioni fra grandezze fisiche
  \item \textbf{Principi}: affermazioni generali da reputare vere
  \item \textbf{Modelli}: analogie o rappresentazioni pratiche su cui basare il proprio studio
  \item \textbf{Teoria}: insieme di leggi, principi e modelli
\end{itemize}

\vspace{1em}
\subsection{Metodo scientifico}
Il \textbf{metodo scientifico} si basa su \textbf{osservazioni} della realtà circostante, a cui seguono delle \textbf{ipotesi}, ossia delle possibili spiegazioni dei fonmeni osservati, basati sulle osservazioni precedentemente formulate.\\
Dopo aver esposto le proprie ipotesi, esse devono essere verificate, mediante degli \textbf{esperimenti}, a cui seguono delle \textbf{analisi} dei risultati sperimentali ottenuti. Il processo di analisi viene seguito da delle \textbf{conclusioni} che \quotes{concludono} il metodo scientifico.\\
Naturalmente tale circuito non è chiuso, in quanto ciascuna di queste fasi può essere ripetuta più e più volte. La parte più importante di tale \emph{metodo scientifico} è la dimostrazione, così come la verifica tramite \textbf{sperimentazioni} delle proprie ipotesi, in quanto le ipotesi devono essere \textbf{sempre verificate}. Tale processo permette di sviluppare leggi e teorie con un fondamento concreto e solido.

\vspace{1em}
\noindent
\textbf{Osservazione}: Si osservi che \textbf{verificare un'ipotesi} non significa dimostrare che un'ipotesi è vera, ma \textbf{verificare che un'ipotesi può essere contraddetta}, ovvero ci si deve assicurare che una \textbf{teoria deve essere \quotes{falsificabile}}, ossia che può essere dimostrato che essia sia falsa.

\newpage
\section{Unità e vettori}

\vspace{1em}
\subsection{Grandezza fisica}
Alla base della \textbf{Fisica} si pone il concetto di \textbf{grandezza fisica}. Non è facile, per esempio, definire che cosa sia il \emph{tempo}; tuttavia, la soluzione più immediata è quella che prevede di definire la misura del tempo come ciò che si riesce a misurare tramite, per esempio, un orologio.\\
Si parla, in tale caso, di \textbf{definizione operativa}:

% Tabella per le definizione di concetti, etc...
\vspace{1em}
\rowcolors{1}{black!5}{black!5}
\setlength{\tabcolsep}{14pt}
\renewcommand{\arraystretch}{2}
\noindent
\begin{tabularx}{\textwidth}{@{}|P|@{}}
    \hline
    {\textbf{DEFINIZIONE OPERATIVA}}\\
    \parbox{\linewidth}{Una grandezza fisica è definita solo dalle operazioni necessarie per misurarla.
    \vspace{3mm}}\\
    \hline
\end{tabularx}

\vspace{1em}
\noindent
Inoltre, le grandezze fisiche si esprimono in termini di un \textbf{campione}, il quale prende il nome di \textbf{unità}.\\
In Fisica, inoltre, si distinguono due diverse categorie di grandezze fisiche:
\begin{enumerate}
  \item Grandezze fisiche fondamentali
  \item Grandezze fisiche derivate
\end{enumerate}
Le \textbf{grandezze fisiche fondamentali} sono $3$:
\begin{enumerate}
  \item Tempo: il tempo presenta come unità il \textbf{secondo} (s) che, dal $1967$, è stato definito come
  \[9192631170 \text{ volte il periodo di oscillazione di una risonanza dell'atomo di Cesio } ^{133}C\]
  Prima di tale data, il secondo era definito come una suddivisione del giorno, ma tale definizione era imprecisa: la terra non ruota sempre con la stessa velocità.

  \item Lunghezza: la lunghezza presenta come unità il \textbf{metro} (m), il quale viene definito come
  \[\frac{1}{299 782 458} \text{ la distanza percorsa dalla luce in $1$ s}\]
  Prima di tale definizione, il metro era definito come \(\frac{1}{10000}\) la distanza tra equatore e polo.\\
  La nuova definizione, tuttavia, è più precisa, in quanto la velocità della luce è \textbf{costante}, fissata in quanto su tale costante si definisce il metro.

  \item Massa: la massa presenta come unità il \textbf{chilogrammo} (kg), il quale viene definito in funzione della \textbf{costante di Planck} ($h = 6.62607015 \times 10^{-34} \text{ kg} \text{ m}^{2} \text{ s}^{-1}$). Prima di tale definizione, il chilogrammo era definito con riferimento ad un campione presente a Parigi e su cui si faceva riferimento per ogni misura di massa.
\end{enumerate}

\noindent
Le grandezze fisiche fondamentali permettono, poi, di definire le grandezze fisiche derivate, quale il \textbf{Volume}, la \textbf{Forza}, etc.

\vspace{1em}
\subsection{Cifre significative e incertezza}
In Fisica, quando si effettuano delle misurazioni, deve essere sempre specificata la precisione e, dunque, l'incertezza. Infatti, \textbf{tutte le msiure hanno un livello di incertezza}.\\
Per esempio
\begin{flalign*}
  L & = 1.82 \pm 0.02 \text{ m}\\
  m & = 3.5 \pm 0.1 \text{ kg}
\end{flalign*}
Da notare che l'indicazione dell'incertezza è sempre (o quasi) data da una sola cifra: altrimenti si avrebbe incertezza nell'incertezza. L'indicazione dell'incertezza è la base della \textbf{fisica sperimentale}.\\
Nella pratica, tuttavia, l'indicazione dell'incertezza è ridondante e pesante. Per indicare il livello di precisione si ricorre alle cifre significative.\\
Per esempio
\begin{flalign*}
  L & = 1.82 \text{ m} = 1.82 \pm 0.01 \text{ m}\\
  m & = 3.5 \text{ kg} = 3.5 \pm 0.1 \text{ kg}
\end{flalign*}

\vspace{1em}
\subsubsection{Operazioni di base}
Per la gestione delle cifre significative nelle operazioni di calcolo è importante tenere a mente che
\begin{itemize}
  \item Moltiplicazione e Divisione: bisogna considerare come cifre significative del prodotto o del quoziente il più basso numero di cifre significative dei fattori o di dividendo e divisore.\\
  Per esempio
  \[1,1 \text{ m} \times 3.45 \text{ m} = 3.8 \text{ m}^2\]
  in quanto il più basso numero di cifre significative dei fattori è $1$.

  \item Addizione e Sottrazione: bisogna considerare come cifre significative della somma o differenza il più basso numero di decimali degli addendi o del minuendo e sottraendo.\\
  Per esempio
  \[1.1 \text{ m} - 12 \text{ cm} = 1.1 \text{ m} - 0.12 \text{ m} = 0.98 \text{ m} = 1.0 \text{ m}\]
  in quanto il più basso numero di decimali tra minuendo e sottraendo è $1$.
\end{itemize}

\vspace{1em}
\subsection{Ordini di grandezza}
Molto spesso, nelle stime è importante non tanto la precisione delle misure, ma l'ordine di grandezza delle stesse, in modo tale da effettuare un macroconfronto utile per delle valutazioni pratiche e veloci.\\
Lo scopo, quindi, dell'impiego degli ordini di grandezza è quello di effettuare dei calcoli veloci e, quindi, delle stime. Più precisamente:

% Tabella per le definizione di concetti, etc...
\vspace{1em}
\rowcolors{1}{black!5}{black!5}
\setlength{\tabcolsep}{14pt}
\renewcommand{\arraystretch}{2}
\noindent
\begin{tabularx}{\textwidth}{@{}|P|@{}}
    \hline
    {\textbf{ORDINE DI GRANDEZZA}}\\
    \parbox{\linewidth}{L'ordine di grandezza di una misura è la \textbf{potenza di $10$ più vicina}.
    \vspace{3mm}}\\
    \hline
\end{tabularx}

\vspace{1em}
\noindent
\textbf{Esempio}: Un ingegnere deve fabbricare un nuovo pacemaker. Si stimi quanti battiti di cuore deve fare senza malfunzionamento. Per effettuare tale stima è necessario conoscere la \textit{media dei battiti al secondo} e \textit{l'aspettativa di vita del soggetto}. Considerando, quindi, come media dei battiti $m_B = 1 \text{ battito}/\text{s}$ e come aspettativa di vita $a_V = 60 \text{ anni}$. La stima selectlanguage
\[m_B \times a_V \times \pi \times 10^7 \text{ s}/\text{anno} = 1 \text{ battito}/\text{s} \times 60 \text{ anni} \times \times 10^7 \text{ s}/\text{anno} = 2 \times 10^9 \text{ battiti}\]

\newpage
\noindent
\begin{center}
  2 Marzo 2022
\end{center}
Il metodo scientifico permette di \textbf{falsificare una teoria}, quindi non è vero che permette di validare una teoria senza ambiguità.

\vspace{1em}
\subsection{Analisi dimensionale}
Il concetto di \textbf{unità} è estremamente importante per parlare di \textbf{analisi dimensionale}. In particolare
\[A = B\]
non può essere valido e corretto formalmente se $A$ e $B$ hanno unità diverse. Questo è molto intuitivo per le grandezze fisiche fondamentali, ma quando si parla di grandezze derivate diventa un punto cruciale: tale concetto permette di validare anche delle possibili soluzioni di test.\\
Per esempio, l'unità di misura della costante di richiamo di una molla si può facilmente ricavare dalla formula della \emph{forza di richiamo}:
\[F = k \cdot x\]
Da cui è immediato capire che
\[\left[k\right] = \frac{\left[F\right]}{\left[x\right]} = \frac{\text{kg m s}^{-2}}{m} = \text{kg s}^{-2}\]

\vspace{1em}
\subsection{Scalari e vettori}
Di seguito si espone la definizione di \textbf{scalare}:

% Tabella per le definizione di concetti, etc...
\vspace{1em}
\rowcolors{1}{black!5}{black!5}
\setlength{\tabcolsep}{14pt}
\renewcommand{\arraystretch}{2}
\noindent
\begin{tabularx}{\textwidth}{@{}|P|@{}}
    \hline
    {\textbf{SCALARE}}\\
    \parbox{\linewidth}{Uno \textbf{scalare} è una grandezza specificata da un numero + unità.\\
    Per esempio la \emph{lunghezza}, la \emph{massa} o l'\emph{energia}.
    \vspace{3mm}}\\
    \hline
\end{tabularx}

\vspace{1em}
\noindent
Mentre un \textbf{vettore} è:

% Tabella per le definizione di concetti, etc...
\vspace{1em}
\rowcolors{1}{black!5}{black!5}
\setlength{\tabcolsep}{14pt}
\renewcommand{\arraystretch}{2}
\noindent
\begin{tabularx}{\textwidth}{@{}|P|@{}}
    \hline
    {\textbf{VETTORE}}\\
    \parbox{\linewidth}{Un \textbf{vettore} è  una quantità definita da un valore e una direzione (e un verso, che può essere implicito nella definizione di direzione).
    \vspace{3mm}}\\
    \hline
\end{tabularx}

\vspace{1em}
\noindent
Tale definizione, tuttavia, pur essendo molto intuitiva, non risulta particolarmente pratica. Si potrebbe anche considerare un vettore come una \textbf{quantità con più valori associati}, ovvero una \textbf{lista di numeri a cui conferiamo un significato}.\\
Per esempio, in algebra un vettore viene indicato come segue
\[\vec{v} = (1, 2, 3)\]
a cui la fisica attribuisce un significato preciso: $1$, $2$ e $3$ sono le componenti associate alle tre diverse dimensioni $x$, $y$ e $z$. Il vettore di cui sopra, allora, si può scrivere come
\[\vec{v} = (v_x, v_y, v_z)\]

\vspace{1em}
\noindent
\textbf{Osservazione}: Anche se tale definizione sembra identica alla definizione del \textbf{punto}, in realtà tale definizione è differente, in quanto
\begin{itemize}
  \item un punto non ha una lunghezza;
  \item non è possibile eseguire la somma di due punti, etc.
\end{itemize}

\vspace{1em}
\subsection{Prodotto con uno scalare}
Dato un vettore
\[\vec{v} = \left(v_x, v_y, v_z\right)\]
e si considera uno scalare $a \in \mathbb{R}$, allora
\[a \cdot \vec{v} = \left(a \cdot v_x, a \cdot v_y, a \cdot v_x\right)\]
in cui il vettore $a \cdot \vec{v}$ è un vettore che
\begin{itemize}
  \item presenta come lunghezza la lunghezza del vettore $\vec{v}$ moltiplicata per $\vert a \vert$;
  \item presenta come direzione la stessa direzione del vettore $\vec{v}$;
  \item presenta come verso lo stesso verso del vettore $\vec{v}$ se $a \geq 0$, mentre avrà verso opposto se $a \leq 0$.
\end{itemize}

\vspace{1em}
\subsection{Somma vettoriale}
Dati due vettori $\vec{u}$ e $\vec{v}$, la loro somma viene eseguita graficamente tramite la \textbf{regola del parallelogramma}, o il metodo \quotes{punta-coda}:

\begin{figure}[H]
  \centering
  \begin{tikzpicture}
    \draw [-stealth, thick, red]    (0,0) -- coordinate[midway](u1) (-3,2);
    \draw [-stealth, thick, blue]   (0,0) -- coordinate[midway](v)  (5,2);
    \draw [-stealth, thick, dashed] (5,2) -- coordinate[midway](u2) (2,4);
    \draw [-stealth, thick, orange] (0,0) -- coordinate[midway](r)  (2,4);
    \draw [thick, red]    (u1) node[above]{$\vec{u}$};
    \draw [thick, blue]   (v)  node[above]{$\vec{v}$};
    \draw [thick, dashed] (u2) node[above]{$\vec{u}$};
    \draw [thick, orange] (r)  node[left]{$\vec{r}$};
  \end{tikzpicture}
  \caption{Somma vettoriale con il metodo \quotes{punta-coda}}
  \label{fig:somma_vettoriale_metodo_punta_coda}
\end{figure}

\noindent
Se $\vec{u}$ e $\vec{v}$ sono espressi nello stesso sistema di riferimento, allora è chiaro che la loro somma sarà data \textbf{componente per componente}, ovvero
\[\vec{u} + \vec{v} = \left(u_x + v_x, u_y + v_y, u_z + v_z\right)\]

\vspace{1em}
\subsection{Versori}
Si definiscano tre versori come segue
\begin{flalign*}
  \hat{i} & = (1,0,0)\\
  \hat{j} & = (0,1,0)\\
  \hat{k} & = (0,0,1)
\end{flalign*}
Allora qualsiasi vettore può essere scritto come
\[\vec{v} = \left(v_x, v_y, v_z\right) = v_x \cdot \hat{i} + v_y \cdot \hat{j} + v_z \cdot \hat{k}\]
in cui, naturalmente, $v_x$, $v_y$ e $v_z$ sono le componenti di $\vec{v}$ in direzione $\hat{i}$, $\hat{j}$ e $\hat{k}$.\\
Naturalmente si scrive $\hat{i}$ e non $\vec{i}$ in quanto
\[\left \vert \hat{i} \right \vert = \left \vert \hat{j} \right \vert = \left \vert \hat{k} \right \vert = 1\]
essi, infatti, prendono il nome di \textbf{versori} o \textbf{vettori unità}.

\vspace{1em}
\subsection{Modulo e direzione}
Di seguito si espone la definizione di \textbf{modulo di un vettore}:

% Tabella per le definizione di concetti, etc...
\vspace{1em}
\rowcolors{1}{black!5}{black!5}
\setlength{\tabcolsep}{14pt}
\renewcommand{\arraystretch}{2}
\noindent
\begin{tabularx}{\textwidth}{@{}|P|@{}}
    \hline
    {\textbf{MODULO DI UN VETTORE}}\\
    \parbox{\linewidth}{Il modulo di un vettore è la sua \quotes{lunghezza geometrica} e si indica come segue
    \[v = \left \vert \vec{v} \right \vert\]
    È chiaro che il modulo può essere \textbf{positivo o nullo}, mai negativo. In termini di componenti il modulo si calcola come segue
    \[v = \sqrt{v_x^2 + v_y^2 + v_z^2}\]
    \vspace{3mm}}\\
    \hline
\end{tabularx}

\vspace{1em}
\noindent
Per esempio, si calcoli il modulo del vettore
\[\hat{n} = \frac{\vec{v}}{v}\]
ovviamente si procede come segue
\[\left \vert \frac{\vec{v}}{v} \right \vert = \frac{1}{\left \vert v \right \vert} \cdot \left \vert \vec{v} \right \vert = \frac{v}{v} = 1\]
Ecco che allora tale vettore è a tutti gli effetti un versore in direzione $\vec{v}$, in quanto di modulo $1$.\\
Questo fa capire come si possa definire un versore associato a qualunque vettore: basta dividere il vettore per il suo modulo.

\vspace{1em}
\noindent
Mentre di seguito si espone la definizione di \textbf{direzione di un vettore}:

% Tabella per le definizione di concetti, etc...
\vspace{1em}
\rowcolors{1}{black!5}{black!5}
\setlength{\tabcolsep}{14pt}
\renewcommand{\arraystretch}{2}
\noindent
\begin{tabularx}{\textwidth}{@{}|P|@{}}
    \hline
    {\textbf{DIREZIONE DI UN VETTORE}}\\
    \parbox{\linewidth}{La direzione di un vettore (e anche il suo verso) è definita, in due dimensioni, come l'angolo $\theta$ che il vettore descrive con il semiasse positivo delle ascisse.\\
    È immediato osservare che
    \begin{center}
      $\begin{array}{c}
        v_x = v \cdot \cos(\theta)\\
        v_y = v \cdot \sin(\theta)
      \end{array}$
    \end{center}
    e si può verificare che
    \[\left \vert \vec{v} \right \vert = \sqrt{v_x^2 + v_y^2} = \sqrt{v^2 \cdot \cos^2(\theta) + v^2 \cdot \sin^2(\theta)} = v \cdot \sqrt{\cos^2(\theta) + \sin^2(\theta)} = v\]
    \vspace{-1mm}}\\
    \hline
\end{tabularx}

\newpage
\noindent
\subsection{Prodotto scalare}
Di seguito si espone la definizione di \textbf{prodotto scalare}:

% Tabella per le definizione di concetti, etc...
\vspace{1em}
\rowcolors{1}{black!5}{black!5}
\setlength{\tabcolsep}{14pt}
\renewcommand{\arraystretch}{2}
\noindent
\begin{tabularx}{\textwidth}{@{}|P|@{}}
    \hline
    {\textbf{PRODOTTO SCALARE}}\\
    \parbox{\linewidth}{Il prodotto scalare tra due vettori $\vec{v}$ e $\vec{u}$, in termini di componenti si definisce come segue:
    \[\vec{v} \cdot \vec{u} = v_x \cdot u_x + v_y \cdot u_y + v_z \cdot u_z\]
    che è, naturalmente, uno scalare.\\
    Analogamente si può interpretare il prodotto scalare tra due vettori $\vec{v}$ e $\vec{u}$ come il prodotto dei moduli per il \textbf{coseno} dell'angolo $\theta$ compreso tra i vettori stessi, ovvero
    \[\vec{v} \cdot \vec{u} = v \cdot u \cdot \cos(\theta)\]
    \vspace{-1mm}}\\
    \hline
\end{tabularx}

\vspace{1em}
\noindent
\textbf{Osservazione}: Naturalmente, da tale definizione seguono delle importanti osservazioni:
\begin{itemize}
  \item \(\vec{v} \cdot \vec{v} = v_x^2 + v_y^2 + v_z^2 = \left \vert \vec{v} \right \vert ^2\)
  \item \(\hat{i} \cdot \hat{i} = \hat{j} \cdot \hat{j} = \hat{k} \cdot \hat{k} = 1\)
  \item \(\hat{i} \cdot \hat{j} = \hat{i} \cdot \hat{k} = \hat{j} \cdot \hat{k} = 0\). Questo significa che i due vettori considerati sono ortogonali, ovvero i versori $\hat{i}$, $\hat{j}$ e $\hat{k}$ sono a due a due ortogonali.
\end{itemize}
Si consideri, invece, l'esempio seguente:
\[\vec{v} \cdot \hat{i} = \left(v_x \cdot \hat{i} + v_y \cdot \hat{j} + v_z \cdot \hat{k} \right) \cdot \hat{i} = v_x \cdot \hat{i} \cdot \hat{i} + v_y \cdot \hat{i} \cdot \hat{j} + v_z \cdot \hat{i} \cdot \hat{k} = v_x\]
e questo significa che $\vec{v} \cdot \hat{i}$ è la \textbf{proiezione} del vettore $\vec{v}$ in direzione $\hat{i}$. Tale metodo è molto efficace per effettuare un cambio di base: se al posto dei versori $\hat{i}$, $\hat{j}$ e $\hat{k}$, che presuppongono l'origine del sistema di riferimento in $O = (0,0,0)$ si scegliessere degli altri versori, moltiplicando il vettore $\vec{v}$ per taluni versori si otterrebbero le componenti del nuovo vettore in una nuova base.

\vspace{1em}
\noindent
\textbf{Osservazione}: Se si considerando due vettori $\vec{c}$ e $\vec{d}$, allora il loro prodotto scalare può essere interpretato come segue
\[\vec{c} \cdot \vec{d} \cdot \frac{d}{d} = d \cdot \left(\vec{c} \cdot \frac{\vec{d}}{d}\right)\]
per cui, ricordando che
\[\hat{n} = \frac{\vec{d}}{d}\]
è un versore in direzione del vettore $\vec{d}$, allora il prodotto scalare tra $\vec{c}$ e $\vec{d}$ è proprio la proiezione del vettore $\vec{c}$ sul vettore $\vec{d}$, per quanto appena detto a proposito delle \textbf{proiezioni}, moltiplicata per il modulo del vettore $\vec{d}$.

\newpage
\noindent
\begin{center}
  3 Marzo 2022
\end{center}

\section{Cinematica}
La descrizione del moto di un corpo (approssimanto ad un punto) prende il nome di \textbf{cinematica} (mentre la ragione del moto viene studiata dalla \textbf{dinamica}). Com'è noto, inoltre, un vettore è una quantità con \textbf{modulo} e \textbf{direzione} (e \textbf{verso}). La descrizione di un vettore avviene tramite le sue componenti: in particolare, dato un versore $\hat{n}$, la componente di un vettore $\vec{v}$ in direzione del versore $\hat{n}$ è così definita
\[\vec{v} \cdot \hat{n}\]
Per esempio, la componente del vettore $\vec{v}$ lungo l'asse $x$ è
\[\vec{v} \cdot \hat{i} = v_x\]
Inoltre, il \textbf{prodotto scalare} tra due vettori $\vec{v}$ e $\vec{u}$ viene definito come:
\[\vec{v} \cdot \vec{u} = v_x \cdot u_x + v_y \cdot u_y \cdot v_z \cdot u_z = v \cdot z \cdot \cos(\theta)\]
ove $\theta$ è l'angolo compreso tra i due vettori $\vec{v}$ e $\vec{u}$.\\
Grazie a ciò è possibile definire il concetto di \textbf{Cinematica}:

% Tabella per le definizione di concetti, etc...
\vspace{1em}
\rowcolors{1}{black!5}{black!5}
\setlength{\tabcolsep}{14pt}
\renewcommand{\arraystretch}{2}
\noindent
\begin{tabularx}{\textwidth}{@{}|P|@{}}
    \hline
    {\textbf{CINEMATICA}}\\
    \parbox{\linewidth}{La cinematica è lo \textbf{studio del moto}, a differenza della \textbf{dinamica} che studia la \textbf{causa del moto} e della \textbf{statica} che studia l'\textbf{equilibrio meccanico}, ossia la \textbf{causa dell'immobilità}.
    \vspace{3mm}}\\
    \hline
\end{tabularx}

\vspace{1em}
\noindent
È chiaro che lo studio di un corpo complesso e non omogeneo è molto più elaborato dello studio di un solo \textbf{punto}. Pertanto, il primo passo per lo studio del moto è quello di studiare il comportamento di un modello standard a cui può essere ricondotto, tramite approssimazione, un altro corpo, a seconda della necessità.

\vspace{1em}
\subsection{Posizione e spostamento}
Dato un punto nello spazio, la sua posizione viene descritta tramite un \quotes{vettore} posizione $\vec{r}$, di cui è possibile calcolare la lunghezza $\left(\vec{r}\right)$, la quale, tuttavia, non ha molto significato dal momento che dipende dalla posizione dell'origine del sistema scelta: ovverosia dipende dalla posizione iniziale e, quindi, dal \textbf{sistema di riferimento adottato}:

\begin{figure}[H]
  \centering
  \begin{tikzpicture}
      \draw [-stealth] (0,0) -- (0,2);
      \draw [-stealth] (0,0) -- (1.5,-1);
      \draw [-stealth] (0,0) -- (-1.5,-1);
      \draw [-stealth, blue] (0,0) -- (1.5,1);
      \draw [blue] (0.75,0.8) node[]{$\vec{r}$};
  \end{tikzpicture}
  \caption{\quotes{Vettore} posizione}
  \label{fig:vettore_posizione}
\end{figure}

\vspace{1em}
\noindent
Conoscere il sistema di riferimento è fondamentale, in quanto in base a ciò possono essere effettuate diverse valutazioni che, naturalmente, variano a seconda del sistema di riferimento scelto: si pensi ed effettuare una misurazione adottando come sistema di riferimento un treno che si muove oppure un treno immobile, o ancora un treno che sta accelerando: si parla, in tale contesto, di un \textbf{sistema di riferimento non inerziale}.

% Tabella per le definizione di concetti, etc...
\vspace{1em}
\rowcolors{1}{black!5}{black!5}
\setlength{\tabcolsep}{14pt}
\renewcommand{\arraystretch}{2}
\noindent
\begin{tabularx}{\textwidth}{@{}|P|@{}}
    \hline
    {\textbf{SPOSTAMENTO}}\\
    \parbox{\linewidth}{Lo \textbf{spostamento}, invece, è proprio un vettore e, com'é intuibile, taluno è definito come la differenza tra due posizioni, ovvero
    \[\boxed{\Delta \vec{r} = \vec{r_2} - \vec{r_1}}\]
    di cui è possibile calcolare il modulo come segue
    \[\left \vert \Delta \vec{r} \right \vert = \left \vert \vec{r_2} - \vec{r_1} \right \vert = \sqrt{\left(x_2 - x_1\right)^2 + \left(y_2 - y_1\right)^2 + \left(z_2 - z_1\right)^2} = \text{ distanza}\]
    \vspace{1mm}}\\
    \hline
\end{tabularx}

\vspace{1em}
\noindent
Per esempio, la lunghezza sull'asse $x$ è
\[\Delta \vec{r} = \Delta x \cdot \hat{i}\]
di cui
\[\left \vert \Delta \vec{r} \right \vert = \left \vert x_2 - x_1 \right \vert\]

\vspace{1em}
\subsection{Posizione in funzione del tempo}
È particolarmente importante considerare la variazion della posizione nel tempo

\begin{figure}[H]
  \centering
  \begin{tikzpicture}
      \draw (0,0) node[circ]{} to[out=80,in=180] (2,1);
      \draw (2,1) to[out=0,in=-120] (4,2) node[circ]{};
      \draw (0,-0.3) node[]{$\vec{r}_i$};
      \draw (4,2.3) node[]{$\vec{r}_f$};
      \draw [-stealth] (1,0.91) -- (1.8,1.2);
      \draw (1.2,1.3) node[]{$\vec{r}(t)$};
  \end{tikzpicture}
  \caption{\quotes{Vettore} posizione in funzione del tempo}
  \label{fig:vettore_posizione_funzione_tempo}
\end{figure}

Sia data una funzione spostamento, definita in funzione del tempo $t$, quale
\[\vec{r}(t) = x(t) \cdot \hat{i} + y(t) \cdot \hat{j}\]
in cui
\begin{flalign*}
  x(t) & = 2 \text{ m } + \left(2 \text{ m/s} \right) \cdot t\\
  y(t) & = 0 \text{ m } + \left(4 \text{ m/s} \right) \cdot t
\end{flalign*}
Naturalmente si ha

\vspace{2em}
\noindent
\rowcolors{1}{white}{white}
\begin{tabularx}{\textwidth}{P}
  {
      \centering
      \begin{tikzpicture}
        \begin{axis}[
          grid=both,
          axis lines = middle,
          xlabel = \(t\),
          ylabel = {\(x(t)\)},
          legend pos=outer north east,
          ymajorgrids=true,
          xmajorgrids=true,
          grid style=dashed,
        ]

        \addplot [
          domain=-2:10,
          samples=100,
          color=red,
        ]
        {2 + 2*x};
        \addlegendentry{\(x(t) = 2 \text{ m } + \left(2 \text{ m/s} \right) \cdot t\)}
        \end{axis}
    \end{tikzpicture}
  }
\end{tabularx}

\vspace{2em}
\noindent
\rowcolors{1}{white}{white}
\begin{tabularx}{\textwidth}{P}
  {
      \centering
      \begin{tikzpicture}
          \begin{axis}[
            grid=both,
            axis lines = middle,
            xlabel = \(t\),
            ylabel = {\(y(t)\)},
            legend pos=outer north east,
            ymajorgrids=true,
            xmajorgrids=true,
            grid style=dashed,
          ]

          \addplot [
            domain=-2:10,
            samples=100,
            color=blue,
          ]
          {4*x};
          \addlegendentry{\(y(t) = 0 \text{ m } + \left(4 \text{ m/s} \right) \cdot t\)}
          \end{axis}
      \end{tikzpicture}
  }
\end{tabularx}

\vspace{2em}
\noindent
\rowcolors{1}{white}{white}
\begin{tabularx}{\textwidth}{P}
  {
      \centering
      \begin{tikzpicture}
        \begin{axis}[
          grid=both,
          axis lines = middle,
          xlabel = \(t\),
          ylabel = {\(r(t)\)},
          legend pos=outer north east,
          ymajorgrids=true,
          xmajorgrids=true,
          grid style=dashed,
        ]
      \addplot[
        domain=-2:10,
        samples=100,
        color=orange,
      ]
      ({2 + 2*x},
      {4*x});
      \addlegendentry{\(\vec{r}(t) = x(t) \cdot \hat{i} + y(t) \cdot \hat{j}\)}
      \end{axis}
      \end{tikzpicture}
        }
\end{tabularx}

\vspace{1em}
\subsection{Velocità}
La \textbf{velocità} si pone alla base della cinematica. In fisica la velocità si distingue in due tipolgie
\begin{itemize}
  \item Velocità istantanea
  \item Velocità media
\end{itemize}
Si consideri, a tal proposito, il seguente grafico spazio-tempo:

\vspace{2em}
\noindent
\rowcolors{1}{white}{white}
\begin{tabularx}{\textwidth}{P}
  {
      \centering
      \begin{tikzpicture}
        \begin{axis}[
          grid=both,
          axis lines = middle,
          xlabel = \(t\),
          ylabel = {\(x\)},
          legend pos=outer north east,
          ymajorgrids=true,
          xmajorgrids=true,
          grid style=dashed,
          xtick={50,120},
          xticklabels={$t_1$,$t_2$},
          ytick={76,166},
          yticklabels={$x_1$,$x_2$},
        ]
      \addplot[
        domain=0:200,
        samples=100,
        color=orange,
      ]
      {abs(x*(abs(sin(2*x)) + 0.5) + 2)};
      \addplot [color=red,mark=*] coordinates {(50,76)};
      \addplot [color=red,mark=*] coordinates {(120,166)};
      \addplot [color=blue] coordinates {(50,76)(120,166)};
      %\addlegendentry{\(\vec{r}(t) = x(t) \cdot \hat{i} + y(t) \cdot \hat{j}\)}
      \end{axis}
      \end{tikzpicture}
        }
\end{tabularx}

% Tabella per le definizione di concetti, etc...
\vspace{1em}
\rowcolors{1}{black!5}{black!5}
\setlength{\tabcolsep}{14pt}
\renewcommand{\arraystretch}{2}
\noindent
\begin{tabularx}{\textwidth}{@{}|P|@{}}
    \hline
    {\textbf{VELOCITÀ MEDIA}}\\
    \parbox{\linewidth}{Intuitivamente si ha che la velocità media è proprio il rapporto tra uno spostamento e il tempo impiegato per effettuarlo, ovvero
    \[\boxed{\left<v\right> = \frac{\Delta \vec{r}}{\Delta t} = \frac{x_2 - x_1}{t_2 - t_1}}\]
    che, graficamente, può essere interpretata come la pendenza (o coefficiente angolare), della congiungente i punti $(x_1,t_1)$ e $(x_2,t_2)$ nel grafico spazio/tempo.
    \vspace{3mm}}\\
    \hline
\end{tabularx}
\vspace{1em}

% Tabella per le definizione di concetti, etc...
\vspace{1em}
\rowcolors{1}{black!5}{black!5}
\setlength{\tabcolsep}{14pt}
\renewcommand{\arraystretch}{2}
\noindent
\begin{tabularx}{\textwidth}{@{}|P|@{}}
    \hline
    {\textbf{VELOCITÀ ISTANTANEA}}\\
    \parbox{\linewidth}{Mentre la velocità istantanea è, naturalmente, la derivata nel tempo del vettore posizione, ovvero
    \[\boxed{\vec{v}(t) = \frac{d\vec{r}}{dt} = \lim_{t \to 0} \frac{x - x_0}{t - t_0}}\]
    ovvero la retta tangente il grafico della funzione posizione nel punto $(x_0,t_0)$. Naturalmente essendo un vettore la velocità istantanea, è possibile descriverlo tramite \textbf{componenti} come segue:
    \[\vec{v} = \frac{d}{dt} \left(\vec{r}(t)\right) = \underbrace{\frac{dx}{dt} \cdot \hat{i}}_\text{$v_x$} + \underbrace{\frac{dy}{dt} \cdot \hat{j}}_\text{$v_y$} + \underbrace{\frac{dz}{dt} \cdot \hat{k}}_\text{$v_z$}\]
    in quanto $\hat{i}$, $\hat{j}$ e $\hat{k}$ non dipendono dal tempo (cosa che potrebbe accadere, comunque, in determinate circostanze).\\
    Il \textbf{modulo} della velocità si calcola come segue
    \[\left \vert \vec{v} \right \vert = v = \sqrt{v_x^2 + v_y^2 + v_z^2}\]
    \vspace{-1mm}}\\
    \hline
\end{tabularx}

\vspace{1em}
\noindent
\textbf{Esempio}: Si consideri uno spostamento verso l'alto tale per cui $x_1 = 0 \text{ m}$ e $x_2 = 12000 \text{ m}$ e $t_1 = 2600 \text{ s}$ e $t_2 = 4000 \text{ s}$. Allora si ha che
\[\left<v_z\right> = \frac{x_2 - x_1}{t_2 - t_1} = \frac{12000}{4000 - 2600} = 8.6 \text{ m/s} = 308.6 \text{ km/h}\]
Che è una velocità irrisoria; tuttavia, ciò non sorprende, in quanto è opportuno conoscere anche le altre componenti della velocità, ossia $v_x$ e $v_y$.

\vspace{1em}
\subsection{Accelerazione}
Di seguito si espone la definizione di \textbf{accelerazione}:

% Tabella per le definizione di concetti, etc...
\vspace{1em}
\rowcolors{1}{black!5}{black!5}
\setlength{\tabcolsep}{14pt}
\renewcommand{\arraystretch}{2}
\noindent
\begin{tabularx}{\textwidth}{@{}|P|@{}}
    \hline
    {\textbf{ACCELERAZIONE}}\\
    \parbox{\linewidth}{L'accelerazione viene definita come la derivata prima della velocità nel tempo, o la derivata seconda dello spostamento nel tempo:
    \[\vec{a} = \frac{d \vec{v}}{dt} = \frac{d^2 \vec{r}}{dt}\]
    \vspace{-1mm}}\\
    \hline
\end{tabularx}

\newpage
\noindent
\begin{center}
  7 Marzo 2022
\end{center}
Comè noto, l'accelerazione è la derivata prima della velocità in funzione del tempo, o la derivata seconda dello spostamento in funzione del tempo.\\
L'accelerazione è fondamentale in \textbf{meccanica}, in quanto grazie all'accelerazione è possibile definire il concetto di forza: l'accelerazione è la connessione tra cinematica e dimanica.

\vspace{2em}
\noindent
\rowcolors{1}{white}{white}
\begin{tabularx}{\textwidth}{P}
  {
      \centering
      \begin{tikzpicture}
        \begin{axis}[
          grid=both,
          axis lines = middle,
          xlabel = \(t\),
          ylabel = {\(x(t)\)},
          legend pos=outer north east,
          ymajorgrids=true,
          xmajorgrids=true,
          grid style=dashed,
        ]

        \addplot [
          domain=-2:10,
          samples=100,
          color=red,
        ]
        {(x-4)^3 + 500};
        % \addlegendentry{\(x(t) = 2 \text{ m } + \left(2 \text{ m/s} \right) \cdot t\)}
        \end{axis}
    \end{tikzpicture}
  }
\end{tabularx}

\vspace{1em}
\noindent
Avendo a disposizione il grafico che descrive la variazione della posizione nel tempo, è possibile ora studiarne l'andamento per poi descrivere il comportamento della velocità nel secondo grafico. È sufficiente, pertanto, osservare gli intervalli di crescenza e decrescenza e i punti in cui la derivata si annulla nulla:

\vspace{2em}
\noindent
\rowcolors{1}{white}{white}
\begin{tabularx}{\textwidth}{P}
  {
      \centering
      \begin{tikzpicture}
        \begin{axis}[
          grid=both,
          axis lines = middle,
          xlabel = \(t\),
          ylabel = {\(v(t)\)},
          legend pos=outer north east,
          ymajorgrids=true,
          xmajorgrids=true,
          grid style=dashed,
        ]

        \addplot [
          domain=-2:10,
          samples=100,
          color=red,
        ]
        {(x-4)^2+1000};
        % \addlegendentry{\(x(t) = 2 \text{ m } + \left(2 \text{ m/s} \right) \cdot t\)}
        \end{axis}
    \end{tikzpicture}
  }
\end{tabularx}

\vspace{1em}
\noindent
Ancora una volta, studiando l'andamento della velocità nel suo rispettivo grafico è ora possibile descrivere il grafico della derivata della velocità, ovvero dell'accelerazione, sempre analizzando gli intervalli di crescenza e decrescenza:

\vspace{2em}
\noindent
\rowcolors{1}{white}{white}
\begin{tabularx}{\textwidth}{P}
  {
      \centering
      \begin{tikzpicture}
        \begin{axis}[
          grid=both,
          axis lines = middle,
          xlabel = \(t\),
          ylabel = {\(a(t)\)},
          legend pos=outer north east,
          ymajorgrids=true,
          xmajorgrids=true,
          grid style=dashed,
        ]

        \addplot [
          domain=-2:10,
          samples=100,
          color=red,
        ]
        {x - 4};
        % \addlegendentry{\(x(t) = 2 \text{ m } + \left(2 \text{ m/s} \right) \cdot t\)}
        \end{axis}
    \end{tikzpicture}
  }
\end{tabularx}

\vspace{1em}
\noindent
Ecco che il punto in cui la derivata seconda (ovverosia l'accelerazine) cambia segno è un \textbf{punto di flesso}, ovvero il punto in cui il grafico dello spostamento cambia la propria concavità.

\vspace{1em}
\noindent
\textbf{Esempio}: Si consideri la seguente funzione spostamento in funzione del tempo:
\[x(t) = A \cdot \cos(\omega t)\]
questa è l'equazione di oscillazione di un pendolo o di una molla. Per il calcolo della velocità è sufficiente calcolare la derivata prima, ovvero
\[v(t) = \frac{dx}{dt} = -\omega \cdot A \cdot \sin(\omega t)\]
e per l'accelerazione è sufficiente derivare nuovamente la velocità
\[a(t) = \frac{dv}{dt} = -\omega^2 \cdot A \cdot \cos(\omega t) = -\omega^2 \cdot x(t)\]
Tale risultato ha senso e può essere interpretato anche graficamente, grazie al grafico di una molla: quando lo spostamento è positivo, l'accelerazione è negativa, e viceversa.

\begin{figure}[H]
  \centering
  \begin{tikzpicture}[every node/.style={draw,outer sep=0pt,inner sep=0pt,thick}, scale=2]
    \tikzstyle{spring}=[thick,decorate,decoration={aspect=0.5, segment length=1mm, amplitude=2mm,coil}]
    \draw[thick] (0,0) --(0,1);
    \draw[thick] (0,0) --(3,0) node[draw=none,xshift=5pt]{$x$};
    \node at(0,0.25) (a) [draw=none] {};
    \node at (2,0.25)(b) [minimum size=0.5cm,label=$\rightarrow$] {m};
    \draw [spring] (a) -- (b) node[draw=none,pos=.5,right=.25cm] {};
    \node at (2,-0.3)(c) [draw=none,yshift=5pt] {$x=0$};
    \draw[dashed] (b.south) -- (c.north);
  \end{tikzpicture}
  \caption{Fisica di una molla}
  \label{fig:fisica_molla}
\end{figure}

\vspace{1em}
\noindent
\textbf{Osservazione}: Quando la velocità è nulla, la posizione si mantiene costante e non cresce o decresce. Quando si ha un punto di salto della velocità si ha una situazione difficile da riprodurre fisicamente, in quanto si ha un crollo della velocità istantanea, come un impulso (si pensi anche al fatto che, per il teorema del limite della derivata, una funzione con un salto non può essere la derivata di una funzione derivabile).\\
Si può utilizzare anche l'\textbf{integrale} per passare dalla velocità allo spostamento.

\vspace{1em}
\subsection{Moto uniformemente accelerato}
Nel moto uniformemente accelerato si ha che l'\textbf{accelerazione} è \textbf{costante}. Sapenche l'accelerazione è la derivata prima della velocità nel tempo, si può scrivere:
\[\frac{dv}{dt} = a \longrightarrow dv = a \cdot dt\]
Questo, in particolare, è possibile farlo sia per una variazione $\Delta$, ma anche per una variazione infinitesimale $d$. Ciò che si sta facendo, in questo caso, è risolvere un'\textbf{equazione differenziale}. Pertanto, dopo aver ottenuto $dv = a \cdot dt$ si può procedere all'integrazione
\[\int dv = \int a \cdot dt \longrightarrow v = at + c\]
La costante $c$ che compare nella formula, ottenuta grazie alla risoluzione di una equazione differenziale, è la cosiddettà \textbf{velocità inziale}: se $t=0$, infatti, $v = c = v_0$. Quindi l'equazione diviene
\[\boxed{v(t) = v_0 + a \cdot t}\]
Avendo ottenuto l'equazione della velocità nel tempo, è opportuno ottenere l'equazione della posizione in funzione del tempo, integrando nuovamente, sempre partendo da
\[\frac{dx}{dt} = v \longrightarrow dx = v \cdot dt \longrightarrow \int dx = \int v \cdot dt = \int (v_0 + a \cdot t) dt\]
quindi si ottiene
\[x(t) = \int v_0 \cdot dt + \int a \cdot t \cdot dt + c = v_0 \cdot t + a \cdot \frac{t^2}{2} + c\]
ove $c = x_0 = x(t=0)$. Pertanto l'equazione della posizione in funzione del tempo è
\[\boxed{x(t) = x_0 + v_0 \cdot t + \frac{1}{2} \cdot a \cdot t^2}\]
che rappresenta l'equazione di una parabola, come illustrato nell'esempio seguente:

\vspace{2em}
\noindent
\rowcolors{1}{white}{white}
\begin{tabularx}{\textwidth}{P}
  {
      \centering
      \begin{tikzpicture}
        \begin{axis}[
          axis lines = left,
          xlabel = \(t\),
          ylabel = {\(x(t)\)},
          legend pos=outer north east,
          ymajorgrids=true,
          xmajorgrids=true,
          grid style=dashed,
          ytick={-11},
          yticklabels={$x_0$},
          ymax=15,
        ]

        \addplot [
          domain=-2:10,
          samples=100,
          color=red,
        ]
        {-(x-2)^2 + 5};
        \draw [-stealth, blue, thick] (axis cs:-2,-11) -- (axis cs:1,10);
        \draw [blue] (axis cs:-1.2,0) node[]{$\vec{v}_0$};
        \end{axis}
    \end{tikzpicture}
  }
\end{tabularx}

\vspace{1em}
\noindent
In cui, ovviamente, l'intersezione tra il grafico e l'asse $y$ è $x_0$, il vettore designato in blu, ossia la tangente in $(x_0,0)$, rappresenta la pendenza iniziale del grafico della posizione, ovvero la velocità iniziale $v_0$. Essndo il grafico di un moto uniformemente accelerato, è ovvio che la pendenza decresce in modo costante: prima la velocità sarà positiva, ma dcrescente, e poi continuerà a decrescere, ma con valori negativi.

\vspace{1em}
\noindent
\textbf{Osservazione}: Si consideri la seguente equazione
\[v(t) = a \cdot t + v_0\]
e si provi ad isolare $t$ da tale equazione, come segue
\[t = \frac{v - v_0}{a}\]
se ora si considera l'equazione seguente
\[x(t) = \frac{1}{2} \cdot a \cdot t^2 + v_0 \cdot t + x_0\]
e si sostituisce il $t$ calcolato in precedenza a tale equazione si ottiene
\[x(t) = \frac{1}{2} \cdot a \cdot \left(\frac{v - v_0}{a}\right)^2 + v_0 \cdot \left(\frac{v - v_0}{a}\right) + x_0 = \frac{1}{2} \cdot \frac{1}{a} \cdot (v^2 - 2 \cdot v \cdot v_0 + v_0^2) + \frac{1}{a} \cdot (v_0 \cdot v + v_0^2) + x_0\]
ovvero si ottiene
\[\boxed{x = \frac{v^2}{2a} - \frac{v_0^2}{2a} + x_0}\]
quindi
\[\boxed{v^2 - v_0^2 = 2a \cdot (x - x_0)}\]
la quale è \textbf{valida solamente per il moto uniformemente accelerato} ed è estremamente utile per conoscere lo spazio percorso da un corpo che si muove secondo queste legge oraria, senza conoscere il \textbf{tempo}.

\vspace{1em}
\subsubsection{Caduta libera dei gravi}
La \textbf{caduta libera} avviene con la \textbf{medesima accelerazione} per tutti i corpi (ovviamente, l'attrazione gravitazionale non è costante in tutto l'universo, ma se si considerano un punto sulla terra e distanze piccole e prossime a quella del raggio terrestre, si può, senza perdita di generalità, considerare un'accelerazione costante $g$).

\vspace{1em}
\noindent
\textbf{Osservazione}: Naturalmente sulla Luna non c'è aria, si è nel vuoto, per cui tutti i corpi vengono attratti dalla Luna solamente per la forza di attrazione gravitazionale, senza che tale moto sia influenzato dall'attrito dell'aria.

\vspace{1em}
\noindent
L'accelerazione gravitazionale può essere interpretata come segue:

\begin{figure}[H]
  \centering
  \begin{tikzpicture}
    \draw [-stealth] (0,0) -- (0,-5);
    \draw (2,-2.5) node[]{$\vec{a} = -g \cdot \hat{j} \hspace{0.5em} \text{con} \hspace{0.5em} 9.8 \text{ m/s}$};
  \end{tikzpicture}
  \caption{Visualizzazione grafica del moto di caduta libera}
  \label{fig:visualizzazione_grafica_caduta_libera}
\end{figure}


\vspace{1em}
\noindent
Ove, naturalmente si ha
\[\vec{a} = -g \cdot \hat{j} \hspace{0.5em} \text{con} \hspace{0.5em} 9.8 \text{ m/s}\]
Pertanto si ottengono le seguenti equazioni, a partire da quelle generali per un moto uniformemente accelerato. Per quanto riguarda la velocità di caduta si ha:
\[\boxed{v_y = -g \cdot t + v_{0y}}\]
mentre per quanto riguarda la posizione in funzione del tempo si ottiene
\[\boxed{y = - \frac{1}{2} \cdot g \cdot t^2 + v_{0y} + y_0}\]
Pe capire che altezza raggiungerà un corpo quando viene lanciato verso l'alto si deve osservare il grafico seguente

\vspace{2em}
\noindent
\rowcolors{1}{white}{white}
\begin{tabularx}{\textwidth}{P}
  {
      \centering
      \begin{tikzpicture}
        \begin{axis}[
          axis lines = left,
          xlabel = \(t\),
          ylabel = {\(x(t)\)},
          legend pos=outer north east,
          ymajorgrids=true,
          xmajorgrids=true,
          grid style=dashed,
          ytick={-11,5},
          yticklabels={$y_0$,$y_m$},
          ymax=15,
        ]

        \addplot [
          domain=-2:10,
          samples=100,
          color=red,
        ]
        {-(x-2)^2 + 5};
        \draw [thick, red, dashed] (axis cs:0,5) -- (axis cs:4,5);
        \draw [thick, red] (axis cs:2,9) node[]{$v_y=0$};
        \end{axis}
    \end{tikzpicture}
  }
\end{tabularx}

\vspace{1em}
\noindent
Naturalmente si osserva che la velocità nel punto più alto è nulla, in quanto la tangente è orizzontale, mentre si conosce la velocità inziale $v_0$.\\
Per capire l'altezza, allora, si potrebbe calcolare il tempo impiegato per raggiungere il punto più alto e poi sostituire tale valore all'interno dell'equazione dello spostamento.\\
 Alternativamente, si potrebbe considerare la formula seguente
\[v_y^2 - v_{y0}^2 = -2 \cdot g \cdot (y - y_0)\]
e sostitutendo i valori si ha
\[0 - v_{y0}^2 = 2 \cdot g \cdot (y_m - y_0)\]
Per cui l'altezza massima che un corpo raggiunge quando viene lanciato verso l'alto con velocità iniziale $v_{y0}$ e a partire da un'altezza $y_0$ è
\[\boxed{y_m = y_0 + \frac{v_{y0}^2}{2g}}\]

\vspace{1em}
\subsubsection{Moto dei proiettili}
Il moto dei proiettili ha un'importanza storica fondamentale: Aristotele, nel $340$ a.c. parlava di \textbf{moto \quotes{naturale} e \quotes{forzato}}: tuttavia, naturalmente, un oggetto fermo, privo di sollecitazioni, non ha la propensione a muoversi, quindi non ha senso parlare di \emph{forzatura}.\\
Successivamente, Filipono ($490-570$) d.c. ha introdotto il concetto di \textbf{impeto} e, infine, \textbf{Galileo} ($1564-1642$) d.c., basandosi su osservazioni e misurazioni pratiche precedenti (invece che fornire una spiegazioni a priori), ha fornito una \textbf{spiegazione scentifica} a tale fenomeno.

% Tabella per le definizione di concetti, etc...
\vspace{1em}
\rowcolors{1}{black!5}{black!5}
\setlength{\tabcolsep}{14pt}
\renewcommand{\arraystretch}{2}
\noindent
\begin{tabularx}{\textwidth}{@{}|P|@{}}
    \hline
    {\textbf{LEGGE ORARIA DEL MOTO DEI PROIETTILI}}\\
    \parbox{\linewidth}{Per lo studio del moto dei proiettili si considera il vettore accelerazione
    \[\boxed{\vec{a} = a_x \cdot \hat{i} + a_y \cdot \hat{j} = 0 \cdot \hat{i} - g \cdot \hat{j}}\]
    Analogamente per la velocità si ha
    \[\boxed{\vec{v} = v_x \cdot \hat{i} + v_y \cdot \hat{j} = v_{0x} \cdot \hat{i} - (v_{0y} - g t) \cdot \hat{j}}\]
    Per quanto concerne la posizione in funzione del tempo si ha
    \[\boxed{\vec{r} = x \cdot \hat{i} + y \cdot \hat{j} = (v_{0x} t + x_0) \cdot \hat{i} - (y_0 + v_{0y} t - \frac{1}{2} g t^2) \cdot \hat{j}}\]
    \vspace{-1mm}}\\
    \hline
\end{tabularx}

\vspace{2em}
\noindent
Si consideri il seguente grafico della posizione di un proiettile:

\vspace{2em}
\noindent
\rowcolors{1}{white}{white}
\begin{tabularx}{\textwidth}{P}
  {
      \centering
      \begin{tikzpicture}
        \begin{axis}[
          axis lines = left,
          xlabel = \(t\),
          ylabel = {\(x(t)\)},
          legend pos=outer north east,
          ymajorgrids=true,
          xmajorgrids=true,
          grid style=dashed,
          ytick={-11},
          yticklabels={$x_0$},
          ymax=15,
        ]

        \addplot [
          domain=-2:10,
          samples=100,
          color=red,
        ]
        {-(x-2)^2 + 5};
        \draw [-stealth, blue, thick] (axis cs:-2,-11) -- (axis cs:1,10);
        \draw [blue] (axis cs:-1.2,0) node[]{$\vec{v}_0$};
        \draw [blue] (axis cs:-1.6,-11.2) node[]{$\vec{r}_0$};
        \end{axis}
    \end{tikzpicture}
  }
\end{tabularx}

\vspace{1em}
\noindent
Naturalmente si ha che
\[\vec{v_{0}} = v_{x0} \cdot \hat{i} + v_{x0} \cdot \hat{j}\]
\[\vec{r_{0}} = x_0 \cdot \hat{i} + y_0 \cdot \hat{j}\]

\newpage

\noindent
\begin{center}
  8 Marzo 2022
\end{center}
Naturalmente il modulo di un vettore non dipende dal sistema di coordinate scelto.\\
Naturalmente è possibile avere
\[\left \vert \vec{A} + \vec{B} \right \vert = \left \vert \vec{A} - \vec{B} \right \vert\]
e per dimostrarne la veridicità basta cosnsiderare due vettori perpendicolari.\\
Ovviamente, un versore ha sempre modulo unitario per essere definito tale: in particolare, dato il vettore $\vec{v} = \hat{i} + \hat{j} + \hat{k}$, il versore
\[\hat{n} = \frac{\vec{v}}{v}\]
è proprio un versore in direzione $\hat{i} + \hat{j} + \hat{k}$.\\
La componente del vettore $\vec{v} = -3 \cdot \hat{i} + 5 \cdot \hat{j} + \hat{k}$ in direzione $\hat{n} = 0.6 \cdot \hat{j} - 0.8 \cdot \hat{k}$ è ovviamente
\[\vec{v} \cdot \hat{n} = -3 \cdot 0 + 5 \cdot 0.6 - 1 \cdot 0.8 = 2.2\]
È chiaro che in questo caso $\hat{n}$ era già un versore, altrimenti, se si avesse avuto un vetttore, si sarebbe dovuto calcolare il versore corrispondente dividendo per il suo modulo.

\vspace{1em}
\noindent
Nel moto dei proiettili è estremamente importante considerare l'\textbf{altezza massima} che esso raggiungerà, ma soprattutto la sua \textbf{gittata}, ovvero la distanza massima che esso raggiungerà.\\
Molto spesso, in questo caso, per lavorare è molto più convieniente operare con le coordinate polari $v_0$ e $\theta$ (ovvero con modulo e angolo), anziché con $v_{x0}$ e $v_{y0}$, sempre ricordando che
\[v = \left(
  \begin{array}{l}
    v_{x0} = v_0 \cdot \cos(\theta)\\
    v_{y0} = v_0 \cdot \sin(\theta)\\
  \end{array}
\right)\]
In questo caso, per calcolare l'altezza massima raggiunta è sufficiente considerare solamente la componente della velocità verticale, come per la caduta libera dei gravi, ovvero
\[y_m = y_0 + \frac{v_{y0}^2}{2g} = y_0 + \frac{v_0^2 \cdot \sin^2(\theta)}{2g}\]
Analogamente, per calcolare la gittata, ovvero la distanza orizzontale percorsa da un corpo lanciato in aria, si dovrà usare solo la componente della velocità orrizontale.

\vspace{2em}
\noindent
\rowcolors{1}{white}{white}
\begin{tabularx}{\textwidth}{P}
  {
      \centering
      \begin{tikzpicture}
        \begin{axis}[
          axis lines = left,
          xlabel = \(y\),
          ylabel = {\(x\)},
          legend pos=outer north east,
          ymajorgrids=true,
          xmajorgrids=true,
          grid style=dashed,
        ]

        \addplot [
          domain=0:10,
          samples=100,
          color=red,
        ]
        {-(x-2)^2 + 4};
        % \addlegendentry{\(x(t) = 2 \text{ m } + \left(2 \text{ m/s} \right) \cdot t\)}
        \end{axis}
    \end{tikzpicture}
  }
\end{tabularx}

\vspace{1em}
\noindent
Naturalmente, in questo caso, il calcolo della gittata $R$ si effettua come segue: è noto che
\[R = v_{x0} \cdot t_R\]
ove $t_R$ è la \textbf{durata del volo}. Però è noto che al tempo $t_R$ la coordinata $y$ è nulla, ovvero
\[y(t_R) = 0 = v_{y0} \cdot t_R - \frac{1}{2} \cdot g \cdot t_R^2\]
Dal momento che il tempo $t_R$ non è nullo è possibile dividere per $t$, ottenendo
\[v_{y0} - \frac{1}{2} \cdot g \cdot t_R = 0\]
da cui si evince che il tempo $t_R$ cercato è
\[t_R = \frac{2 \cdot v_{y0}}{g}\]
sostituendo, ora, il tempo trovato nella formula di cui sopra si ottiene
\[R = v_{x0} \cdot \frac{2 \cdot v_{y0}}{g} = 2 \cdot \frac{v_{x0} \cdot v_{y0}}{g}\]
ma ricordando come si calcolano le componenti $v_{x0}$ e $v_{y0}$ si ha
\[R = 2 \cdot \frac{v_0^2 \cdot \sin(\theta) \cdot \cos(\theta)}{g}\]
ed essendo $2 \cdot \sin(\theta) \cdot \cos(\theta) = \sin(2 \cdot \theta)$ si ottiene
\[\boxed{R = \frac{v_0^2 \cdot \sin(2\theta)}{g}}\]
Le unità in gioco sono
\[[v^2] = \frac{\text{m}^2}{\text{s}^2}\]
\[[g] = \frac{\text{m}}{\text{s}^2}\]
\[[R] = \frac{\dfrac{\text{m}^2}{\text{s}^2}}{\dfrac{\text{m}}{\text{s}^2}} = \text{m}\]

\vspace{1em}
\noindent
\textbf{Osservazione}: Si osservi che, naturalmente, quando $\theta=0^\circ$, si ha che $\sin(0) = 0$ e quindi $R=0$: ciò ha senso, in quanto lanciando orizzontalmente il corpo cade immediatamente.\\
Quando $\theta=90^\circ$ non si ha gittata, in quanto, lanciando verticalmente il corpo cada verticalmene.\\
Quando $\theta=45^\circ$ si ha la gittata massima, in quanto $\sin(90)=1$.

\vspace{1em}
\subsection{Moto in 2D e 3D}
Si consideri la seguente traiettoria

\begin{figure}[H]
  \centering
  \begin{tikzpicture}
      \draw (0,0) node[circ]{} to[out=80,in=180] (2,1);
      \draw (2,1) to[out=0,in=-120] (4,2) node[circ]{};
      \draw (0,-0.3) node[]{$\vec{r}_i$};
      \draw (4,2.3) node[]{$\vec{r}_f$};
      \draw [-stealth, red] (1,0.91) -- (1.8,1.2);
      \draw (1.2,1.3) node[]{$\vec{v}(t)$};
      \draw [-stealth, blue] (1,0.91) -- (1.2,0.4);
      \draw (0.85,0.6) node[]{$\vec{a}_\perp$};
  \end{tikzpicture}
  \caption{\quotes{Vettore} velocità in funzione del tempo}
  \label{fig:vettore_velocità_funzione_tempo}
\end{figure}

\vspace{1em}
\noindent
Naturalmente $\vec{v}$ è sempre parallelo alla tangente della curva $\vec{r}(t)$.\\
Si osservi, inoltre, che l'accelerazione deve sempre presentare una \textbf{componente lineare} $\vec{a}_\parallel$ (parallela a $\vec{v}$) e una \textbf{componente ortogonale} $\vec{a}_\perp$ (in direzione del cambiamento dell'orientazione della velocità), la quale è fondamentale: infatti, se ci fosse solo una componente lineare, la velocità aumenterebbe il proprio modulo, ma non direzione; ogni qualvolta si ha una variazione della direzione della velocità ci deve essere una componente ortogonale dell'accelerazione.

\vspace{1em}
\noindent
\subsection{Moto circolare uniforme}
Per moto circolare uniforme si intende un moto circolare in cui la \textbf{velocità angolare} si mantiene \textbf{costante}.\\

\begin{figure}[H]
  \centering
  \begin{tikzpicture}[>=Triangle]
    \shade [top color=white, bottom color=gray!50, middle color=white]
      (120:8/3) arc (120:190:8/3) node [black, near end, left] {$\omega$}
      -- (190:25/9) -- (200:15/6) -- (190:20/9) -- (190:7/3)
      arc (190:120:7/3) -- cycle;

    \foreach \i in {90, 210, 330}{
      \draw [->, thick, blue!50!cyan] (\i-65:2) arc (\i-65:\i+60:2);
      \tikzset{shift={(\i:2)}, rotate=\i+180}
      \draw [->, very thick, orange] (0,0) -- (1,0)
        node [black, near end, anchor=\i+90] {$\vec a$};
      \draw [->, very thick, green!50!black] (0,0) -- (0,-2)
        node [black, near end, anchor=\i+180] {$\vec v$};
      \fill circle [radius=1/10];
  }
  \end{tikzpicture}
  \caption{Moto circolare uniforme}
  \label{fig:moto_circolare_uniforme}
\end{figure}

\vspace{1em}
\noindent
Naturalmente, la \textbf{circonferenza} del cerchio è $C = 2\pi R$. Parlando di moto circolare uniforme è possibile introdurre il concetto di \textbf{periodo} $T$, ovvero il tempo impiegato a completare una circonferenza completa.\\
Pertanto, volendo conoscere la velocità del moto si ottiene
\[\boxed{v = \frac{2\pi R}{T} = \omega R}\]
ove $\omega$ prende il nome di \textbf{velocità angolare}, che ha una misura di $\text{RAD}/\text{s}$, per questo prende il nome di velocità angolare, in quanto ha la stessa unità di misura di una frequenza (visto che l'angolo non ha una propria vera misura).\\
Naturalmente, volendo conoscere l'accelerazione che agisce sul punto in movimento, si può immediatamente dire che, essendo la velocità costante, si dovrà solo considerare una componente ortogonale, in quanto se essa non ci fosse, il corpo si muoverebbe in linea retta, senza compiere una traiettoria circolare.\\
Pertanto, dal momento che $\left \vert \vec{v} \right \vert = v =$ costante, si ha che $\vec{a}_\parallel = 0$. L'accelerazione media, naturalmente, può essere calcolata come segue
\[\boxed{\left<\vec{a}\right> = \frac{\Delta \vec{v}}{\Delta t} = \frac{\vec{v}(t_2) - \vec{v}(t_1)}{t_2 - t_1}}\]
Anche graficamente appare evidente come l'accelerazione media sia un vettore ortogonale al vettore velocità e orientato verso il centro della circonferenza.\\

\begin{figure}[H]
  \centering
  \begin{tikzpicture}[scale=2]
    \node[minimum size=4cm,circle,draw,blue!50!cyan] (circle) {};
    \draw [thick] (0,0) -- coordinate[midway](m) (circle.80);
    \draw [thick] (0,0) -- (circle.55);
    \draw [thick, -stealth] (circle.80) -- (circle.55);
    \draw (circle.67.5) ++(0.2,0.2) node[]{$\Delta \vec r$};
    \draw (m) ++(-0.2,0) node[]{$R$};
    \begin{scope}
      \tikzset{shift={(30:2)}, rotate=300}
      \draw [-stealth] (circle.30) -- ++(0.5,0) node [black, near end, anchor=210] {$\vec v(t_1)$};
    \end{scope}
    \begin{scope}
      \tikzset{shift={(30:2)}, rotate=210}
      \draw [-stealth] (circle.300) -- ++(0.5,0) node [black, near end, anchor=90] {\hspace{0.5em} $\vec v(t_2)$};
    \end{scope}
    \begin{scope}
      \tikzset{shift={(30:2)}, rotate=210}
      \draw (circle.300) ++(0.5,0) coordinate(a);
      \tikzset{shift={(30:2)}, rotate=90}
      \draw [-stealth] (a) -- ++(-0.5,0) coordinate(b) node [black, near end, anchor=0] {$-\vec v(t_1)$\hspace{0.5em}};
      \draw [-stealth, orange] (circle.300) -- (b) node [orange, near end, anchor=270] {$\Delta \vec v$\hspace{0.5em}};
    \end{scope}
    \coordinate (i) at (circle.55);
    \coordinate (ctr) at (0,0);
    \coordinate (f) at (circle.80);
    \pic [draw=red, text=red, <->, "$\Delta \theta$", angle eccentricity=1.3, angle radius=1cm] {angle = i--ctr--f};
  \end{tikzpicture}
\end{figure}

\vspace{1em}
\noindent
Lo spostamento tra due punti, naturalmente, è $\Delta \vec{r}$ e i due triangoli che vengono così disegnati sono simili, per cui il rapporto tra i lati corrispondenti deve mantenersi costantte.\\
È facile, pertanto, vedere immediatamente come
\[\boxed{\frac{\left \vert \Delta \vec{r} \right \vert}{R} = \frac{\left \vert \Delta \vec{v} \right \vert}{v}}\]
ossia il rapporto tra lo spostamento e il raggio costante, così come la variazione di velocità e il modulo della velocità costante è uguale. Da ciò si evince che
\[a = \frac{\left \vert \Delta \vec{v} \right \vert}{\Delta t} = \frac{\left \vert \Delta \vec{v} \right \vert}{v} \cdot \frac{v}{\Delta t} = \frac{\left \vert \Delta \vec{r} \right \vert}{R} \cdot \frac{v}{\Delta t} = \frac{v^2}{R}\]
Ovvero si ha che, nel moto circolare uniforme, il modulo dell'accelerazione orientata verso il centro del cerchio è pari a
\[\boxed{a=\frac{v^2}{R}}\]

\vspace{1em}
\noindent
\textbf{Esempio}: Si consideri l'esempio seguente che riguarda un veicolo in movimento a velocità costante su una curva

\begin{figure}[H]
  \centering
  \begin{tikzpicture}
    \draw (0,0) -- (2,0) to[out=0,in=270] coordinate[midway](r) (4,2) -- (4,4);
    \draw (2,2) node[circ](start){} -- coordinate[midway](a) (r) (a) ++(0.3,0.3) node[]{$R$};
    \draw [dotted] (start) -- ++(2,0) node[at end, right]{$t_2$};
    \draw [dotted] (start) -- ++(0,-2) node[at end, below]{$t_1$};
    \draw [-stealth] (2.4,0.05) node[circ]{} -- (2.2,0.8);
    \draw (2.5,0.4) node[]{$\vec{a}$};
    \draw [-stealth, red] (2,0) -- (2,1);
    \draw [-stealth, red] (4,2) -- (3,2);
  \end{tikzpicture}
  \caption{Auto in movimento su una curva}
  \label{fig:auto_movimento_curva}
\end{figure}

\vspace{1em}
\noindent
e si considerino le componenti dell'accelerazione in $x$ e in $y$ in funzione del tempo.

\vspace{2em}
\noindent
\rowcolors{1}{white}{white}
\begin{tabularx}{\textwidth}{P}
  {
      \centering
      \begin{tikzpicture}
        \begin{axis}[
          grid=both,
          axis lines = middle,
          xlabel = \(t\),
          ylabel = {\(a_x\)},
          legend pos=outer north east,
          ymajorgrids=true,
          xmajorgrids=true,
          grid style=dashed,
          xtick={pi/2,pi},
          xticklabels={$t_1$,$t_2$},
          xmin=0,
          xmax=5,
          ymin=-2,
          ymax=2,
        ]

        \addplot [
          domain=pi/2:pi,
          samples=100,
          color=red,
          thick
        ]
        {cos(deg(x)};

        \addplot [
          domain=0:5,
          samples=100,
          color=red,
          dashed
        ]
        {cos(deg(x)};
        \draw [red, thick] (axis cs:0,0) -- (axis cs:pi/2,0);
        \draw [red, thick, dotted] (axis cs:pi,0) -- (axis cs:pi,-1);
        \draw [red, thick] (axis cs:pi,0) -- (axis cs:5,0);
        \end{axis}
    \end{tikzpicture}
  }
\end{tabularx}

\vspace{2em}
\noindent
\rowcolors{1}{white}{white}
\begin{tabularx}{\textwidth}{P}
  {
      \centering
      \begin{tikzpicture}
        \begin{axis}[
          grid=both,
          axis lines = middle,
          xlabel = \(t\),
          ylabel = {\(a_y\)},
          legend pos=outer north east,
          ymajorgrids=true,
          xmajorgrids=true,
          grid style=dashed,
          xtick={pi/2,pi},
          xticklabels={$t_1$,$t_2$},
          xmin=0,
          xmax=5,
          ymin=-2,
          ymax=2,
        ]

        \addplot [
          domain=pi/2:pi,
          samples=100,
          color=blue,
          thick
        ]
        {sin(deg(x))};

        \addplot [
          domain=0:5,
          samples=100,
          color=blue,
          dashed
        ]
        {sin(deg(x))};
        \draw [blue, thick] (axis cs:0,0) -- (axis cs:pi/2,0);
        \draw [blue, thick, dotted] (axis cs:pi/2,0) -- (axis cs:pi/2,1);
        \draw [blue, thick] (axis cs:pi,0) -- (axis cs:5,0);
        \end{axis}
    \end{tikzpicture}
  }
\end{tabularx}

\vspace{1em}
\noindent
per capire la natura delle curve appena disegnate, è sufficiente osservare la Figura \ref{fig:visualizzazione_angoli_coinvolti} seguente:

\begin{figure}[H]
  \centering
  \begin{tikzpicture}[scale=2]
    \draw (0,0) -- (2,0) to[out=0,in=270] coordinate[midway](r) (4,2) -- (4,4);
    \draw (2,2) node[circ](start){} -- coordinate[midway](a) (r) (a) ++(0.3,0.3) node[]{$R$};
    \draw [dotted] (start) -- ++(2,0) node[at end, right]{$t_2$};
    \draw [dotted] (start) -- ++(0,-2) node[at end, below]{$t_1$};
    \draw [dotted] (2.4,0.05) node[circ](b){} -- coordinate[midway](half) (2,2);
    \draw [-stealth] (b) -- (half);
    \draw (b) -- ++(1,0) coordinate(c);
    \draw (2.5,0.6) node[]{$\vec{a}$};
    \draw [-stealth, red] (2,0) -- (2,1);
    \draw [-stealth, red] (4,2) -- (3,2);

    \coordinate (f) at (2.2,0.8);
    \coordinate (ctr) at (b);
    \coordinate (i) at (c);
    \pic [draw=red, text=red, <->, "$\theta$", angle eccentricity=1.5] {angle = i--ctr--f};

    \coordinate (f1) at (half);
    \coordinate (ctr1) at (2,2);
    \coordinate (i1) at (2,1);
    \pic [draw=blue, text=blue, <->, "$\alpha$", angle eccentricity=1.2, angle radius=1.8cm] {angle = i1--ctr1--f1};
  \end{tikzpicture}
  \caption{Visualizzazione degli angoli coinvolti}
  \label{fig:visualizzazione_angoli_coinvolti}
\end{figure}

\vspace{1em}
\noindent
Si può facilmente capire come $\alpha$ sia l'angolo da sommare a $90^\circ$ per ottenere $\theta$, quindi
\[a_x=a \cdot \cos(\theta) = a \cdot \cos(\alpha + 90^\circ) = -a \cdot \sin(\alpha) = -a \cdot \sin(\omega t)\]
pertanto nel caso di $a_x$ si è considerato un ramo di $\sin(\omega t)$, mentre nel caso di $a_y$ si è considerato un ramo di $\cos(\omega t)$.

\newpage
\noindent
\begin{center}
  9 Marzo 2022
\end{center}
Il moto circolare uniforme è un moto semplice: la formula più importante da conoscere è il modulo dell'\textbf{accelereazione centripeta}, ovvero dell'accelerazione diretta ferso il centro della circonferenza.\\
Il raggio $R$ della circonferenza del moto è costante, mentre l'angolo $\theta$ descritto dal punto in movimento all'interno della circonferenza varia linearmente con il tempo, secondo la seguente legge
\[\boxed{\theta(t) = \frac{2\pi}{T} \cdot t = \omega t}\]
ove $\omega$ prende il nome di velocità angolare ed è definita come segue
\[\boxed{\omega = \frac{2\pi}{T}}\]
Per quando concerne la variazione della posizione nel tempo si ha
\[\boxed{\vec{r}(t) = x(t) \cdot \hat{i} + y(t) \cdot \hat{i} = R \cdot \cos(\omega t) \cdot \hat{i} + R \cdot \sin(\omega t) \cdot \hat{j}}\]
A partire da tale risultato si sarebbe potuto determinare il modulo dell'accelerazione, semplicemente procedendo per derivate successive, ottenendo dapprima
\[\boxed{\vec{v}(t) = - \omega R \cdot \sin(\omega t) \cdot \hat{i} + \omega R \cos(\omega t) \cdot \hat{i}}\]
e infine
\[\boxed{\vec{a}(t) = -\omega^2 \cdot \vec{r}(t)}\]
in cui è evidente come il vettore acceerazione é sempre parallelo al vettore posizione, ma con verso opposto: il vettore posizione è sempre diretto verso l'esterno, mentre il vettore acccelerazione è diretto verso il centro della circonferenza.\\
Se ora si procede al calcolo del modulo di tale vettore si ottiene
\[\boxed{\left \vert \vec{a} \right \vert = \omega^2 \cdot R = \frac{v^2}{R}}\]
dal momento che si ha
\[\boxed{v = \omega R}\]
In realtà anche $\omega$ è un vettore, in cui la sua direzione è l'asse di rotazione, mentre il modulo fornirà una stima della velocità alla quale si muove; tale risultato avrà una importante validità in seguito.

\vspace{1em}
\subsection{Moti relativi}
Si consideri il caso di un moto composto da più moti: un sasso che viene lasciato cadere sullo scafo di una barca in movimento.\\
Naturalmente, considerando la caduta di un sasso, fissando un tempo $t$, si ha che
\[\vec{v}_{PB} = -g t \cdot \hat{j}\]
in quanto si tratta di una caduta libera. Questo, tuttavia, osservando il moto dalla barca ($PB$ = punto-barca) in movimento. Se, invece, tale moto viene visto da terra ($PT$ = punto-terra), sarà dotato di due componenti, ovvero
\[\vec{v}_{PT} = \vec{v} + -g t \cdot \hat{j}\]
in cui la componente verticale è la stessa del moto precedente, mentre la componente orizzontale dipende dalla velocità della barca. Questo è proprio quello che ha fatto Galileo: osservare questo tipo di situazioni nella vita reale, fornendovi una spiegazione scientifica e definendo, in questo caso, il concetto di sistema di riferimento e di moto relativo.

\subsubsection{Caso generale}
Per capire come passare da un sistema di riferimento all'altro, è necessario considerare un caso generale, in cui come sistema di riferimento si assume quello definito da:
\begin{itemize}
  \item posizione dell'origine;
  \item assi (posizione e orientamento).
\end{itemize}

\vspace{2em}
\noindent
\rowcolors{1}{white}{white}
\begin{tabularx}{\textwidth}{P}
  {
      \centering
      \begin{tikzpicture}
        \begin{axis}[
          axis lines = left,
          xlabel = \(x_A\),
          ylabel = {\(y_A\)},
          legend pos=outer north east,
          ymajorgrids=true,
          xmajorgrids=true,
          grid style=dashed,
          ymax=10,
          xmax=10,
          ytick={3},
          xtick={3},
        ]

        \addplot [
          domain=0:2,
          samples=100,
          color=green,
        ]
        {x};
        \draw [-stealth] (axis cs:3,3) -- (axis cs:3,10);
        \draw [-stealth] (axis cs:3,3) -- (axis cs:10,3);
        \draw [-stealth, very thick, red] (axis cs:0,0) -- (axis cs:9,5);
        \draw [red] (axis cs:4.5,1.5) node[]{$\vec{r}_{PA}$};
        \draw [-stealth, very thick, blue] (axis cs:3,3) -- (axis cs:9,5);
        \draw [blue] (axis cs:5,4.5) node[]{$\vec{r}_{PB}$};
        \draw [-stealth, very thick, orange] (axis cs:0,0) -- (axis cs:3,3);
        \draw [orange] (axis cs:1,2) node[]{$\vec{r}_{BA}$};
        \node[label={[rotate=90]center:$y_B$}] at (axis cs:1.5,6.5) {};
        \node[label={center:$x_B$}] at (axis cs:6.5,1.5) {};
        \end{axis}
    \end{tikzpicture}
  }
\end{tabularx}

\vspace{1em}
\noindent
Per passare da un sistema di riferimento $A$ all'altro $B$ è necessario definire la posizione relativa tra i due sistemi di riferimento $A-B$.\\
Infatti, definendo un nuovo vettore $\vec{r}_{BA}$ è possibile scrivere la somma di vettori seguente
\[\vec{r}_{PA} = \vec{r}_{PB} + \vec{r}_{BA}\]
Ma non solo, è possibile anche calcolare la derivata nel tempo e considerare, quindi, le velocità
\[\vec{v}_{PA} = \vec{v}_{PB} + \vec{v}_{BA}\]
che corrisponde proprio al caso analizzato per la barca: la differenza tra i due sistemi di osservazione è proprio il moto della barca. Un caso ancora più importante è quello che prevede $\vec{v}_{BA}$ \textbf{costante}, per cui i due sistemi non si muovono l'uno rispetto all'altro e misurare l'accelerazione nell'uno o nell'altro non cambia, in quanto sarà la stessa. Derivando nuovamente nel tempo, infatti, si ottiene
\[\vec{a}_{PA} = \vec{a}_{PB} + \vec{a}_{BA}\]
ma essendo $\vec{v}_{BA}$ costante, ovviamente $\vec{a}_{BA} = 0$. Questo è il caso di un \textbf{sistema di riferimento inerziale}, ovvero di un sistema di riferimento che può muoversi ad una certa velocità, ma non può accelerare. Naturalmente l'accelerazione del sistema ierziale non è relativa, non è da definirsi rispetto ad un altro sistema di riferimento come in questo caso, ma necessita di una definizione molto più rigorosa fornita tramite le leggi della dinamica di Newton. Questo ragionamento, naturamente, si applica sia ad un caso in 2D, ma anche in 3D.

\vspace{1em}
\noindent
\textbf{Osservazione}: Si osservi, ovviamente, che nel moto circolare uniforme velocità e accelerazione.\\
Inoltre, si ha che
\[\frac{d \left \vert \vec{v} \right \vert}{dt} = 0\]
significa che la variazione del modulo della velocità nel tempo è nullo: pertanto, se non c'è variazione del modulo della velocità, si ha che la componnte parallela dell'accelerazione è nulla, ovvero $\vec{a}_\parallel = 0$.

\newpage
\section{Dinamica}
Alla base della dinamica vi sono i $3$ principi della dinamica di Newton, formulati da Newton all'interno del libro \textbf{Philosophiae Naturalis Principia Mathematica}.\\
Esse sono le seguenti:

% Tabella per le definizione di concetti, etc...
\vspace{1em}
\rowcolors{1}{black!5}{black!5}
\setlength{\tabcolsep}{14pt}
\renewcommand{\arraystretch}{2}
\noindent
\begin{tabularx}{\textwidth}{@{}|P|@{}}
    \hline
    {\textbf{LEGGI DELLA DINAMICA}}\\
    \parbox{\linewidth}{Le leggi della dinamica sono le seguenti:
    \begin{enumerate}
      \item \textbf{Prima legge}: \emph{Ciascun corpo persevera nel proprio stato di quiete o di moto rettilineo uniforme, eccetto che sia costretto a mutare quello stato da forze impresse.}
      \item \textbf{Seconda legge}: \emph{Il cambiamento di moto è proporzionale alla forza mmotrice impressa, ed avviene lungo la linea retta secondo la quale la forza è stata impressa.}
      \item \textbf{Terza legge}: \emph{Ad ogni azione corrisponde una reazione uguale e contraria: ossia le azioni di due corpi sono sempre uguai fra loro e dirette verso parti opposte.}
    \end{enumerate}
    \vspace{1mm}}\\
    \hline
\end{tabularx}
\vspace{1em}

\subsection{Massa}
Per parlare della dinamica, si devono introdurre due concetti fondamentali ed interconnessi; prima di tutto si fornisce la definizione di \textbf{massa}, la quale può essere definita in modi diversi a seconda della necessità:

% Tabella per le definizione di concetti, etc...
\vspace{1em}
\rowcolors{1}{black!5}{black!5}
\setlength{\tabcolsep}{14pt}
\renewcommand{\arraystretch}{2}
\noindent
\begin{tabularx}{\textwidth}{@{}|P|@{}}
    \hline
    {\textbf{MASSA INERZIALE}}\\
    \parbox{\linewidth}{La \textbf{massa inerziale} (da \textbf{inerzia}: propensione a non muoversi) viene definita come misura della resistenza alle variazioni di velocità.
    \vspace{3mm}}\\
    \hline
\end{tabularx}
\vspace{1em}

\noindent
che si adatta perfettamente alla seconda legge della dinamica, la quale afferma che l'accelerazione è proporzionale alla forza impressa ed è la massa a rappresentare la \textbf{costante di proporzionalità}: a parità di forza, più il corpo è massivo meno accelera, meno è massivo, più accelera.\\
Di seguito, invece, si definisce il concetto di \textbf{massa gravitazionale}:

% Tabella per le definizione di concetti, etc...
\vspace{1em}
\rowcolors{1}{black!5}{black!5}
\setlength{\tabcolsep}{14pt}
\renewcommand{\arraystretch}{2}
\noindent
\begin{tabularx}{\textwidth}{@{}|P|@{}}
    \hline
    {\textbf{MASSA GRAVITAZIONALE}}\\
    \parbox{\linewidth}{La \textbf{massa gravitazionale} è proporzionale al \textbf{peso}.
    \vspace{3mm}}\\
    \hline
\end{tabularx}
\vspace{1em}

\noindent
Non da ultimo si fornisce una definizione di massa che è approssimabile ad una quantificazione:

% Tabella per le definizione di concetti, etc...
\vspace{1em}
\rowcolors{1}{black!5}{black!5}
\setlength{\tabcolsep}{14pt}
\renewcommand{\arraystretch}{2}
\noindent
\begin{tabularx}{\textwidth}{@{}|P|@{}}
    \hline
    {\textbf{MASSA}}\\
    \parbox{\linewidth}{La \textbf{massa} viene definita come \textbf{quantità di materia} e la sua unità di misura è
    \[[m] = \text{kg}\]
    Inoltre la massa è \textbf{additiva}: dato un corpo, agglomerato compatto di due masse $m_1$ e $m_2$, la massa complessiva è
    \[m = m_1 + m_2\]
    \vspace{-1mm}}\\
    \hline
\end{tabularx}
\vspace{1em}

\newpage
\noindent
\subsection{Forza}
Di seguito si espone il signifiato fisico di \textbf{forza}:

% Tabella per le definizione di concetti, etc...
\vspace{1em}
\rowcolors{1}{black!5}{black!5}
\setlength{\tabcolsep}{14pt}
\renewcommand{\arraystretch}{2}
\noindent
\begin{tabularx}{\textwidth}{@{}|P|@{}}
    \hline
    {\textbf{FORZA}}\\
    \parbox{\linewidth}{Una \textbf{forza} è una spinta che produce un cambiamento di moto di un corpo.\\
    La forza è un \textbf{vettore} (con modulo, direzione e verso) la cui unità di misura è
    \[[F] = \text{N} = \frac{\text{kg m}}{\text{s}^2}\]
    ove N sta per Newton.
    \vspace{3mm}}\\
    \hline
\end{tabularx}

\vspace{1em}
\subsection{Principi della dinamica - Leggi di Newton}
Si descrivano, ora, le leggi di Newton in termini dei due concetti esposti, ossia massa e forza:

\begin{enumerate}
  \item Se la forza risultante che agisce su un corpo è nulla, ovvero
  \[\sum \vec{F} = 0\]
  allora l'accelerazione del corpo è nulla, cioé $\vec{a}=0$, ovvero
  \[\boxed{\sum \vec{F} = 0 \longrightarrow \vec{a}=0}\]
  Tale legge potrebbe sembrare superflua, in quanto un caso particolare della seconda: tuttavia, tale legge assolve al compito di definire un \textbf{sistema di riferimento inerziale}.

  \item La forza risultante su un corpo è direttamente proporzionale all'accelerazione del corpo stesso. L'accelerazione di un corpo, quindi, è proporzionale alla forza risultante
  \[\boxed{\sum \vec{F} = m \vec{a}}\]

  \item La forza esercitata da un corpo $a$ su un corpo $b$ è uguale in modulo e direzione, ma ha verso opposto alla forza esercitata da $b$ su $a$.\\
  Ovvero si ha che
  \[\boxed{\vec{F}_{ab} = -\vec{F}_{ba}}\]
  e ciò è sempre vero.
\end{enumerate}

\vspace{1em}
\noindent
Dopo aver definito tali leggi, è necessario definire diversi tipi di forze, distinguendole in base alla loro tipolgia.

\vspace{1em}
\noindent
\subsection{Forza peso}
Di seguito si espone il significato fisico di \textbf{forza peso}:

% Tabella per le definizione di concetti, etc...
\vspace{1em}
\rowcolors{1}{black!5}{black!5}
\setlength{\tabcolsep}{14pt}
\renewcommand{\arraystretch}{2}
\noindent
\begin{tabularx}{\textwidth}{@{}|P|@{}}
    \hline
    {\textbf{FORZA PESO}}\\
    \parbox{\linewidth}{La forza peso fiene designata con $\vec{F}_t$, ovvero la forza di attrazione esercitata dalla terra su un corpo di massa $m$. In particolare si ha
    \[\boxed{\vec{F}_t = m \vec{g}}\]
    in cui $\vec{g} = -9.8 \text{ m}/\text{s}^2 \cdot \hat{j}$ e prende il nome di \textbf{accelerazione gravitazionale} (o meglio, di \textbf{campo gravitazionale} sulla superficie della terra).
    \vspace{3mm}}\\
    \hline
\end{tabularx}

\vspace{1em}
Di seguito si espone una illustrazione della forza peso:

\vspace{1em}
\begin{figure}[H]
  \centering
  \begin{tikzpicture}
    \draw [fill = purple!30,draw = purple!50] (0,0) rectangle ++(2,1.2);
    \draw [-stealth] (1,0.6) node[circ]{} -- ++(0,-2) node [midway, below right] {$\vec{F}_t = m\vec{g}$};
    \draw (0.5,0.6) node[]{$m$};
  \end{tikzpicture}
  \caption{Forza peso}
  \label{fig:forza_peso}
\end{figure}

\vspace{1em}
\noindent
\textbf{Osservazione}: Si osservi che la formula seguente
\[\vec{F}_t = m \vec{g}\]
potrebbe rassomigliare la formula
\[\vec{F}_t = m \vec{a}\]
Tuttavia, i due concetti sono ben distinti, in quanto $\vec{F}_t = m \vec{g}$ è un caso particolare della \textbf{legge di gravitazione universale}.

\vspace{1em}
\noindent
\textbf{Osservazione}: Si osservi che anche nel caso della forza peso è presente il terzo principio della dinamica: infatti un corpo viene attratto verso il centro della terra ed esercita una forza sulla superficie terrestre, così come la terra esercita una forza uguale e contraria (solamente che è impercettibile, è sempre presente).

\vspace{1em}
\subsection{Forza normale}
Di seguito si espone un altro tipo di forza, una forza i contatto, che prende il nome di \textbf{forza normale}:

% Tabella per le definizione di concetti, etc...
\vspace{1em}
\rowcolors{1}{black!5}{black!5}
\setlength{\tabcolsep}{14pt}
\renewcommand{\arraystretch}{2}
\noindent
\begin{tabularx}{\textwidth}{@{}|P|@{}}
    \hline
    {\textbf{FORZA NORMALE}}\\
    \parbox{\linewidth}{La \textbf{forza normale} $\vec{F}_N$ è un caso di forza di contatto, definita come \textbf{spinta fornita da una superficie (o da un altro corpo)}: quando un oggetto è appoggiato su una superficie e non si muove (ovvero si ha che $\vec{v}=0$ e $\vec{a}=0$), alla \textbf{forza peso} si contrappone la \textbf{forza normale}, uguale e contraria alla forza peso.
    \vspace{3mm}}\\
    \hline
\end{tabularx}

\vspace{1em}
\noindent
Di seguito si una illustrazione della forza normale:

\vspace{1em}
\begin{figure}[H]
  \centering
  \begin{tikzpicture}
    \draw [fill = purple!30,draw = purple!50] (0,0) rectangle ++(2,1.2);
    \draw (-1,0) -- (3,0);
    \foreach \i in {-11,-9,...,27} {
      \draw (\i / 10,-0.3) -- (\i / 10 + 0.3,0);
    }
    \draw [-stealth] (1.5,0) node[circ]{} -- ++(0,2) node [at end, right] {$\vec{F}_N$};
    \draw (1.5,0.2) -- ++(0.2,0) -- ++ (0,-0.2);
    \draw [-stealth] (1,0.6) node[circ]{} -- ++(0,-2) node [midway, below right] {$\vec{F}_t$};
    \draw (0.3,0.6) node[]{$m$};
    \draw (-1,1) node[]{$\vec{a}=0$};
  \end{tikzpicture}
  \caption{Forza normale}
  \label{fig:forza_normale}
\end{figure}

\vspace{1em}
\noindent
Una caratteristica fondamentale della forza normale è che essa è sempre \textbf{ortogonale alla superficie} su cui poggia l'oggetto. Se si osserva che il corpo presenta $\vec{a}=0$, significa che
\[\vec{F}_t + \vec{F}_N\ = 0 = m \vec{a} \longrightarrow \vec{F}_N = - \vec{F}_t\]
Mentre la forza peso presenta un modulo preciso, calcolabile tramite la legge di gravitazione universale, la forza normale, invece, adatta la propria intensità al corpo appoggiato sulla superficie: fintantoché la superficie resiste, il modulo della forza normale coincide con quello della forza peso; se il corpo è eccessivamente massiccio, la superficie si rompe.

\vspace{1em}
\noindent
\textbf{Osservazione}: Una forza presenta sempre un \textbf{punto di applicazione} che, graficamente, è rappresentata dalla \quotes{coda del vettore}:
\begin{enumerate}
  \item Nel caso della forza peso, il punto di applicazione è sempre dato del \textbf{centro di massa} del corpo stesso;
  \item Nel caso della forza normale, il punto di applicazione è la superficie di contatto (anche se vi sono molti punti di applicazione vista l'irregolarità della superficie stessa).
\end{enumerate}
Tuttavia è sempre possibile eseguire la somma di forze per conoscerne la risultante.

\vspace{1em}
\noindent
\textbf{Osservazione}: Quando si deve eseguire la rappresentazione grafica delle forze è necessario introdurre il concetto di \textbf{diagramma di corpo libero}:
\begin{itemize}
  \item ogni corpo è rappresentato da un \textbf{punto} (per cui il punto di applicazione delle forze sul corpo è proprio rappresentato dal punto stesso);
  \item comporta \textbf{solo le forze} che sono applicate sul corpo, e ciò diviene particolarmente utile quando bisogna considerare un sistema di più corpi interagenti.
\end{itemize}

\vspace{1em}
\noindent
\textbf{Esempio}: Si considerino due corpi poggiati uno sopra l'altro e stanti su una superficie fissa. Naturalmente su tali corpi agiscono due forze peso distinte. Inoltre la superficie di contatto tra i due corpi permette di indivduare due forze normali: uno del primo corpo sul secondo e una del secondo corpo sul primo. Infine vi è la forza di contatto dei due corpi con la superficie su cui poggiano, come mostrato di seguito:

\vspace{1em}
\begin{figure}[H]
  \centering
  \begin{tikzpicture}[scale=2]
    \draw [fill = purple!30,draw = purple!50] (0,0) rectangle ++(2,1.2);
    \draw [fill = red!30,draw = red!50] (-0.5,1.2) rectangle ++(3,1.2);
    \draw (-1,0) -- (3,0);
    \foreach \i in {-11,-9,...,27} {
      \draw (\i / 10,-0.3) -- (\i / 10 + 0.3,0);
    }
    \draw [-stealth] (1.2,0) node[circ]{} -- ++(0,2) node [at end, right] {$\vec{F}_N$};
    \draw (1.2,0.2) -- ++(0.2,0) -- ++ (0,-0.2);
    \draw [-stealth] (1,0.6) node[circ]{} -- ++(0,-2) node [midway, below right] {$\vec{F}_{t1}$};
    \draw [-stealth] (1.7,1.8) node[circ]{} -- ++(0,-2) node [midway, right] {$\vec{F}_{t2}$};
    \draw [-stealth] (0.5,1.2) node[circ]{} -- ++(0,-1) node [midway, right] {$\vec{F}_{N1}$};
    \draw [-stealth] (0.5,1.2) node[circ]{} -- ++(0,1) node [midway, right] {$\vec{F}_{N2}$};
    \draw (0.3,0.6) node[]{$m_1$};
    \draw (-0.1,1.8) node[]{$m_2$};
  \end{tikzpicture}
  \caption{Forza normale di due corpi a contatto}
  \label{fig:forza_normale_corpi_contatto}
\end{figure}

\vspace{1em}
\noindent
Dopo aver effettuato la raffigurazione, si procede alla realizzazione del \textbf{diagramma a corpo libero} di ciascuno dei due corpi, come mostrato di seguito:

\vspace{1em}
\begin{figure}[H]
  \centering
  \begin{tikzpicture}[scale=1]
    \draw [-stealth] (0,0) node[circ]{} -- ++(0,2) node [at end, right] {$\vec{F}_N$};
    \draw [-stealth] (0.1,0) -- ++(0,-1) node [midway, right] {$\vec{F}_{t1}$};
    \draw [-stealth] (-0.1,0) -- ++(0,-1) node [midway, left] {$\vec{F}_{N1}$};
  \end{tikzpicture}
  \hspace{2em}
  \begin{tikzpicture}[scale=1]
    \draw [-stealth] (0,0) node[circ]{} -- ++(0,1.5) node [at end, right] {$\vec{F}_{N2}$};
    \draw [-stealth] (0,0) -- ++(0,-1.5) node [midway, right] {$\vec{F}_{t2}$};
  \end{tikzpicture}
  \caption{Diagramma a corpo libero di due corpi a contatto}
  \label{fig:diagramma_corpo_libero_due_corpi_contatto}
\end{figure}

\vspace{1em}
\noindent
Naturalmente in questo caso si possono determinare direttamente le forze coinvolte
\begin{flalign*}
  \vec{F}_{t1} & = -g m_1 \cdot \hat{j} = m_1 \vec{g}\\
  \vec{F}_{t2} & = -g m_2 \cdot \hat{j} = m_2 \vec{g}\\
\end{flalign*}
Applicando la \textbf{$\boldsymbol{2^a}$ legge della dinamica} si perviene al risultato seguente
\begin{flalign*}
  m_1 \vec{a}_1 & = m_1 \vec{g} = \sum \vec{F} = \vec{F}_N + \vec{F}_{N1} + \vec{F}_{t1} = 0\\
  m_2 \vec{a}_2 & = m_2 \vec{g} = \sum \vec{F} = \vec{F}_{N2} + \vec{F}_{t2} = 0
\end{flalign*}
Questo, in quanto l'accelerazione è nulla, un dato noto dal problema. Applicando, ora, la \textbf{$\boldsymbol{3^a}$ legge della dinamica} si perviene al risultato seguente:
\[\vec{F}_{N1} = - \vec{F}_{N2}\]
Da ciò si può concludere il problema, andando a determinare
\begin{flalign*}
  \vec{F}_{N2} & = - \vec{F}_{t2} = g m_2 \cdot \hat{j}\\
  \vec{F}_{N1} & = - \vec{F}_{N2} = \vec{F}_{t2} = - g m_2 \cdot \hat{j}\\
  \vec{F}_{N} & = - \vec{F}_{N1} - \vec{F}_{t1} = g m_2 \cdot \hat{j} + g m_1 \cdot \hat{j} = g \cdot (m_1 + m_2) \cdot \hat{j}\\
\end{flalign*}

\newpage
\noindent
\begin{center}
  10 Marzo 2022
\end{center}
\subsection{Forza di tensione}
Di seguito si espone il significato fisico della \textbf{forza di tensione} che, nel suo comportamente, non è dissimile dalla forza normale:

% Tabella per le definizione di concetti, etc...
\vspace{1em}
\rowcolors{1}{black!5}{black!5}
\setlength{\tabcolsep}{14pt}
\renewcommand{\arraystretch}{2}
\noindent
\begin{tabularx}{\textwidth}{@{}|P|@{}}
    \hline
    {\textbf{FORZA DI TENSIONE}}\\
    \parbox{\linewidth}{La forza di tensione è la forza esercitata, per esempio, da un cavo o una fune, come mostrato di seguito, in cui la forza di tensione va a cancellare la forza peso del corpo appeso alla fune. È importante notare che $\vec{F}_T$ è sempre \textbf{parallela alla direzione della corda} stessa.\vspace{3mm}}\\
    \hline
\end{tabularx}

\vspace{1em}
\noindent
Di seguito si espone una illustrazione della forza di tensione:

\vspace{1em}
\begin{figure}[H]
  \centering
  \begin{tikzpicture}[scale=1]
    \draw [fill = purple!30,draw = purple!50] (0,-3) rectangle ++(2,1.2);
    \draw (-1,-0.3) -- (3,-0.3);
    \foreach \i in {-10,-8,...,26} {
      \draw (\i / 10,-0.3) -- (\i / 10 + 0.3,0);
    }
    \draw (1,-1.8) -- (1,-0.3);
    \draw [-stealth, red] (1,-1.8) -- node[midway, left, red]{$\vec{F}_T$} (1,-0.75);
    \draw [-stealth] (1,-2.4) node[circ]{} -- ++(0,-2) node [midway, below right] {$\vec{F}_{t}$};
    \draw (0.5,-2.4) node[]{$m$};
  \end{tikzpicture}
  \caption{Forza di tensione di un corpo sospeso}
  \label{fig:forza_tensione_corpo_sospeso}
\end{figure}

\vspace{1em}
\noindent
\textbf{Esempio}: Si consideri l'esempio seguente, in cui vi è un corpo che rimane sospeso nel vuoto da due funi che descrivono con il sofftto due angoli, rispettivamente $\theta_1$ e $\theta_2$. Si determinino le tensioni sulle due corde.

\vspace{1em}
\begin{figure}[H]
  \centering
  \begin{tikzpicture}[scale=1]
    \draw [fill = purple!30,draw = purple!50] (0,-3) rectangle ++(2,1.2);
    \draw (-1,-0.3) -- (3,-0.3);
    \foreach \i in {-10,-8,...,26} {
      \draw (\i / 10,-0.3) -- (\i / 10 + 0.3,0);
    }
    \draw (0.5,-1.8) -- coordinate[midway](a) node[midway, below left, red]{$\vec{F}_{T1}$} (-0.5,-0.3);
    \draw (1.5,-1.8) -- coordinate[midway](b) node[midway, below right, red]{$\vec{F}_{T2}$} (2.3,-0.3);
    \draw [-stealth, red] (0.5,-1.8) -- (a);
    \draw [-stealth, red] (1.5,-1.8) -- (b);
    \coordinate (O1) at (-0.5,-0.3);
    \coordinate (O2) at (2.3,-0.3);
    \draw [draw = orange] (O1) ++(.8,0) arc (0:-55:0.8)
    	node [pos=.4, left] {$\theta_1$};
    \draw [draw = violet] (O2) ++(-.8,0) arc (180:242:0.8)
    	node [pos=.4, left] {$\theta_2$};
    \draw [-stealth] (1,-2.4) node[circ]{} -- ++(0,-2) node [midway, below right] {$\vec{F}_{t}$};
    \draw (0.5,-2.4) node[]{$m$};
  \end{tikzpicture}
  \caption{Forza di tensione di un corpo sospeso da due corde}
  \label{fig:forza_tensione_corpo_sospeso_due_corde}
\end{figure}

\noindent
Dopo aver realizzato una figura illustrativa, bisogna sempre procedere alla raffigurazione del \textbf{diagramma a corpo libero}, come mostrato di seguito:

\vspace{1em}
\begin{figure}[H]
  \centering
  \begin{tikzpicture}[scale=1.5]
    \draw [-stealth, red] (0,0) node[circ]{} -- node[midway, below left, red]{$\vec{F}_{T1}$} (-0.5,0.75);
    \draw [-stealth, red] (0,0) -- node[midway, below right, red]{$\vec{F}_{T2}$} (0.4,0.75);
    \draw [-stealth] (0,0) -- node[midway, right]{$\vec{F}_{t1}$} (0,-1);
  \end{tikzpicture}
  \caption{Diagramma a corpo libero di un corpo sospeso da due corde}
  \label{fig:diagramma_corpo_libero_corpo_sospeso_due_corde}
\end{figure}

\vspace{1em}
\noindent
Applicando, ora, la \textbf{$\boldsymbol{2^a}$ legge della dinamica} si perviene al risultato seguente:
\[\sum \vec{F} = m \vec{a} = 0\]
essendo l'accelerazione nulla. Pertanto si ha che
\[\vec{F}_{T1} + \vec{F}_{T2} + \vec{F}_t = 0\]
Sarà ora sufficiente decomporre tale equazione vettoriale nelle sue due componenti ($x$ e $y$), come mostrato di seguito
\begin{flalign*}
    x & = - F_{T1} \cdot \cos(\theta_1) + F_{T2} \cdot \cos(\theta_2) + 0 = 0\\
    y & = F_{T1} \cdot \sin(\theta_1) + F_{T2} \cdot \sin(\theta_2) - mg = 0
\end{flalign*}
Dalla prima equazione si ha che
\[F_{T1} = F_{T2} \cdot \frac{\cos(\theta_2)}{\cos(\theta_1)}\]
che, sostituita nella seconda ecquazione, permette di ottenere
\[F_{T2} \cdot \frac{\cos(\theta_2)}{\cos(\theta_1)} \cdot \sin(\theta_1) + F_{T2} \cdot \sin(\theta_2)= mg\]
che può essere riscritto come segue
\[F_{T2} \cdot \cos(\theta_2) \cdot \left[ \frac{\sin(\theta_1)}{\cos(\theta_1)} + \frac{\sin(\theta_2)}{\cos(\theta_2)} \right] = mg\]
Pertanto si ha che
\[F_{T2} = \frac{mg}{\cos(\theta_2) \cdot \left(\tan(\theta_1) + \tan(\theta_2)\right)} \hspace{1em} \text{e} \hspace{1em} F_{T1} = \frac{mg}{\cos(\theta_1) \cdot \left(\tan(\theta_1) + \tan(\theta_2)\right)}\]

\vspace{2em}
\noindent
\textbf{Esempio}: Si consideri un quadricottero, ovverosia un drone con $4$ eliche e rotori, ciascuna capace do fornire una forza propulsiva verso l'alto di eguale modulo.\\
Considerando il drone stazionario si ha, per la \textbf{$\boldsymbol{2^a}$ legge della dinamica}, la seguente eguaglianza
\[4 \cdot \vec{F}_{s} + \vec{F}_t = 0\]
questo significa che ciascun rotore deve essere in grado di sviluppare una forza propulsiva verso l'alto pari a un quarto del peso del drone; quando, invece, si ha uno sbilanciamento delle forze dei rotori si ottiene un'inclinazione del drone nella direzione delle forze di minore intensità (quello che viene chiamato \textbf{momento di forza}).\\
Pertanto si può concludere che
\[\vec{F}_{s} = - \frac{1}{4} \cdot \vec{F}_t\]

\newpage
\noindent
\textbf{Esempio}: Si consideri una massa su un piano inclinato, come mostrato di seguito (considerando ininfluente l'attrito tra la massa e la superficie del piano):

\vspace{1em}
\noindent
\begin{figure}[H]
  \centering
  \newcommand{\ang}{30}

  \begin{tikzpicture} [font = \small, scale=1.5]

  % triangle:
  \draw [draw = orange, fill = orange!15] (0,0) coordinate (O) -- (\ang:6)
  	coordinate [pos=.45] (M) |- coordinate (B) (O);

  % angles:
  \draw [draw = orange] (O) ++(.8,0) arc (0:\ang:0.8)
  	node [pos=.4, left] {$\theta$};
  \draw [draw = orange] (B) rectangle ++(-0.3,0.3);

  \begin{scope} [-latex,rotate=\ang]

  % Object (rectangle)
  \draw [fill = purple!30,
  	draw = purple!50] (M) rectangle ++ (1,.6);

  % Weight Force and its projections
  \draw [dashed] (M) ++ (.5,.3) coordinate (MM) -- ++ (0,-1.29)
  	node [very near end, right] {$\vec{F}_t \cdot \cos{\theta}$};

  \draw [dashed] (MM) -- ++ (-0.75,0)
  	node [very near end, left] {$\vec{F}_t \cdot \sin{\theta}$};

  \draw (MM) -- ++ (-\ang-90:1.5)
  	node [very near end,below left ] {$\vec{F}_t$};

  % Normal Force
  \draw (MM) -- ++ (0,1.29)
  node [very near end, right] {$\vec{F}_N$};
  \end{scope}
  \end{tikzpicture}
  \caption{Piano inclinato}
  \label{fig:piano_inclinato}
\end{figure}

\vspace{1em}
\noindent
A cui segue il diagramma a corpo libero seguente:

\begin{figure}[H]
  \newcommand{\ang}{30}
  \vspace{-5em}
  \hspace{15em}
  \begin{tikzpicture} [font = \small, scale=1.5]
  % triangle:
  \draw (0,0) coordinate (O)  (\ang:6)
  	coordinate [pos=.45] (M) coordinate (B) (O);

  \begin{scope} [-latex,rotate=\ang]
  % Weight Force and its projections
  \draw [dashed] (M) ++ (.5,.3) coordinate (MM) -- ++ (0,-1.29)
  	node [very near end, right] {$\vec{F}_t \cdot \cos{\theta}$};

  \draw [dashed] (MM) -- ++ (-0.75,0)
  	node [very near end, left] {$\vec{F}_t \cdot \sin{\theta}$};

  \draw (MM) -- ++ (-\ang-90:1.5)
  	node [very near end,below left ] {$\vec{F}_t$};

  % Normal Force
  \draw (MM) -- ++ (0,1.29)
  node [very near end, right] {$\vec{F}_N$};
  \end{scope}
  \end{tikzpicture}
  \caption{Diagramma a corpo libero di un piano inclinato}
  \label{fig:diagramma_corpo_libero_piano_inclinato}
\end{figure}

\vspace{1em}
\noindent
Dopo aver disegnato anche il diagramma a corpo libero si può procedere a ragionare con la \textbf{$\boldsymbol{2^a}$ legge della dinamica}, ottenendo
\[m \cdot \vec{a} = \vec{F}_N + \vec{F}_t\]
e scomponendo tale equazione nelle sue componenti si ottiene
\begin{flalign*}
  m a_x & = - F_N \cdot \sin(\theta)\\
  m a_y & = F_N \cdot \cos(\theta) - mg
\end{flalign*}
Volendo conoscere l'accelerazione del corpo, si osserva che sussiste il seguente vincolo geometrico:
\[\frac{a_y}{a_x} = \tan(\theta) \longrightarrow a_y = \frac{\sin(\theta)}{\cos(\theta)} \cdot a_x\]
da cui
\[a_y \cdot \cos(\theta) = a_x \cdot \sin(\theta) \longrightarrow a_x = a_y \cdot \frac{\cos(\theta)}{\sin(\theta)}\]
Pertanto, procedendo dalla prima equazione si ottiene
\[F_N = -\frac{m a_x}{\sin(\theta)}\]
e quindi, nella seconda equazione si ottiene
\[m a_x \cdot \frac{\sin(\theta)}{\cos(\theta)} = -m a_x \cdot \frac{\cos(\theta)}{\sin(\theta)} - mg \longrightarrow a_x = - g \cdot \sin(\theta) \cdot \cos(\theta)\]
Sfruttando il vincolo geometrico precedente si ottiene anche che
\[a_y = -g \cdot \sin^2(\theta)\]
Avendo determinato ciò è possibile calcolare il modulo dell'accelerazione come segue
\[a = \sqrt{g^2 \cdot \sin^2(\theta) \cdot \cos^2(\theta) + g^2 \cdot \sin^4(\theta)} = g \cdot \sin(\theta) \cdot \sqrt{\cos^2(\theta) + \sin^2(\theta)} = g \cdot \sin(\theta)\]

\vspace{1em}
\noindent
\textbf{Esempio}: Si consideri il caso del piano inclinato precedente, procedendo, ora, alla rotazione del sistema di riferimento dell'angolo $\theta$, dimodoché lo spostamento avvenga lungo l'asse $x$ soltanto, e non vi sia, conseguentemente, acceerazione lungo l'asse $y$. In base a questo nuovo sistema di riferimento si ottiene la seguente decomposizione
\begin{flalign*}
  x & : m a_x = -m g \cdot \sin(\theta)\\
  y & : m a_y = -m g \cdot \cos(\theta) + F_N = 0
\end{flalign*}
per cui si ha che
\[a_x = -g \cdot \sin(\theta)\]

\vspace{1em}
\subsection{Forza di attrito}
Di seguito si espone la definizione generale di \textbf{forza di attrito}:

% Tabella per le definizione di concetti, etc...
\vspace{1em}
\rowcolors{1}{black!5}{black!5}
\setlength{\tabcolsep}{14pt}
\renewcommand{\arraystretch}{2}
\noindent
\begin{tabularx}{\textwidth}{@{}|P|@{}}
    \hline
    {\textbf{FORZA DI ATTRITO}}\\
    \parbox{\linewidth}{La \textbf{forza di attrito} è una forza di contatto, esattamente come la forza normale: quest'ultima è sempre perpendicolare alla superficie e il suo modulo è tale che vincola il moto, al fine di contrastare la forza nell'altra direzione.\\
    La \textbf{forza di attrito}, come la forza normale, è una forza che presenta come punto di applicazione la superficie di contatto con il corpo. Inoltre, la forza di attrito
    \begin{itemize}
      \item è sempre \textbf{parallela alla superficie};
      \item è di \textbf{modulo proporzionale} a $\left \vert F_N \right \vert$.
    \end{itemize}
    Quest'ultima osservazione non è ovvia, in quanto bisogna osservare il comportamento delle particelle a livello microscopico.\vspace{3mm}}\\
    \hline
\end{tabularx}

\vspace{1em}
\noindent
Si consideri l'illustrazione seguente, in cui si espongono le forze di attrito agenti:

\vspace{1em}
\begin{figure}[H]
  \centering
  \begin{tikzpicture}[scale=1]
    \tikzset{rotate=45} {
      \draw [fill = purple!30,draw = purple!50] (0,0) rectangle ++(2,1.2);
      \draw (-1,0) -- (3,0);
      \foreach \i in {-10,-8,...,26} {
        \draw (\i / 10,-0.3) -- (\i / 10 + 0.3,0);
      }
    }
    \draw (1,0.3) -- ++(0.3,0) -- ++(0,-0.3);
    \draw [-stealth] (1,0) -- node[midway, left]{$\vec{F}_N$} (1,3) ;
    \draw [-stealth] (1,0.6) node[circ]{} -- ++(-2,-2) node [midway, below right] {$\vec{F}_{t}$};
    \draw (1.5,0.6) node[]{$m$};
    \draw [-stealth, red] (2,0) node[circ]{} -- ++(2,0) node [midway, above left] {$\vec{F}_{s/k}$};
  \end{tikzpicture}
  \caption{Forza di attrito}
  \label{fig:forza_attrito}
\end{figure}

\vspace{1em}
\noindent
La forza di attrito si distingue in due diverse tipologie:
\begin{enumerate}
  \item L'\textbf{attrito cinetico}, si ha quando il moto relativo tra le superfici in contatto non è nullo, ovvero si ha movimento con $v \neq 0$.\\
  Si ha che
  \[\boxed{F_k = \mu_k \cdot F_N}\]
  in cui $\boldsymbol{\mu_k}$ prende il nome di \textbf{coefficiente di attrito cinetico} (dall'inglese $k$, di \quotes{kinetic}), il quale, per ovvie ragioni, è \textbf{adimensionale}.

  \item L'\textbf{attrito statico}, si ha quando la velocità relativa tra le superfici di contatto è nulla, ovvero non si ha movimento, $v = 0$.\\
  Si ha che
  \[\boxed{F_s \leq \mu_s \cdot F_N}\]
  ovvero il modulo $F_s$ aumenta affinché la risultante delle forze interagenti (e quindi la risultante) sia $0$. Ovviamente, quando la forza che agisce aumenta il proprio modulo fino a superare la forza di attrito statico, il corpo inizia a muoversi e l'attrito si tramuta, a seguito del moto, in attrito cinetico, come si vede di seguito:

  \vspace{2em}
  \noindent
  \rowcolors{1}{white}{white}
  \begin{tabularx}{\textwidth}{P}
    {
        \centering
        \begin{tikzpicture}
          \begin{axis}[
            axis lines = left,
            xlabel = \(F_a\),
            ylabel = {\(F_{a,k}\)},
            legend pos=outer north east,
            ymax=4,
            xmax=10,
            xtick={3},
            xticklabels={$\mu_s \cdot F_N$},
          ]

          \addplot [
            domain=0:3,
            samples=100,
            color=black,
          ]
          {x};
          \draw (axis cs:3,2) -- (axis cs:10,2);
          \draw [dashed] (axis cs:3,0) -- (axis cs:3,3);
          \end{axis}
      \end{tikzpicture}
    }
  \end{tabularx}

  \vspace{1em}
  \noindent
  Il coefficiente moltiplicativo $\boldsymbol{\mu_s}$ prende il nome di \textbf{coefficiente di attrito statico} e generalmente si ha che
  \[\boxed{\mu_k < \mu_s}\]
\end{enumerate}

\newpage
\noindent
\begin{center}
  14 Marzo 2022
\end{center}
Si osservi che quando una massa rimane sospesa tramite due cavi che formano con la superficie di collegamento un angolo di $90^\circ$ ciascuno, non entrano in gioco solamente le forze, ma risulta fondamentale anche conoscere il concetto di \textbf{momento di forza} e di \textbf{punto di applicazione}: se il punto di applicazione delle forze di tensione è lo stesso sul corpo sospeso, basta una leggerissima differenza di lunghezza dei cavi per avere uno sbilanciamento significativo delle forze di tensione.

\vspace{1em}
\noindent
\textbf{Osservazione}: L'attrito statico è come se non fosse un attrito, in quanto non si ha movimento: è come una forza normale che si oppone al tentativo di spostamento.\\
L'attrito cinetico, invece, è un vero e proprio attrito che si oppone al moto tramite la dissipazione di potenza.

\vspace{1em}
\noindent
\textbf{Esempio}: Si consideri una vettura che, in movimento, procede a frenare e a rallentare fino a fermarsi: il fenomeno che si sta studiando è l'attrito tra le ruote e la strada. in particolare la vettura richiede $70$ m per fermarsi, partendo da una velocità di $100$ km/h (a causa di uno slittamento delle gomme sull'asfalto).\\
A partire da questi dati, si determini il \textbf{coefficiente di attrito cinetico}; si proceda alla realizzazione di un modello grafico del problema:

\[raffigurazione\]

È molto importante osservare che la forza normale è essenziale per il calcolo della forza di attrito cinetico, in quanto da essa dipende il suo modulo. Si realizzi, ora, il diagramma a corpo libero:

\vspace{1em}
\begin{figure}[H]
  \centering
  \begin{tikzpicture}[scale=1]
    \draw [-stealth] (0,0) node[circ]{} -- ++(0,1) node [at end, right] {$\vec{F}_N$};
    \draw [-stealth] (0,0) -- ++(0,-1) node [midway, right] {$\vec{F}_{t}$};
    \draw [-stealth] (0,0) -- ++(-1,0) node [midway, above] {$\vec{F}_{k}$};
  \end{tikzpicture}
  \caption{Diagramma a corpo libero di una vettura in rallentamento}
  \label{fig:diagramma_corpo_libero_vettura_rallentamento}
\end{figure}

\vspace{1em}
\noindent
Per procedere si richiami la $2^a$ legge della dinamica e si scriva
\[\sum \vec{F} = m \vec{a} = m \cdot \left(a_x \cdot \hat{i} + 0 \cdot \hat{j} \right)\]
È noto, inoltre, che
\[\vec{F}_t + \vec{F}_N = 0 \longrightarrow \vec{F}_t = - \vec{F}_N\]
Per cui si ottiene che
\[\vec{F}_k = m a_x \cdot \hat{i} \longrightarrow a_x = -\frac{F_k}{m}\]
in cui, per convenzione, si pone la componente orizziontale negativa.\\
Inoltre è possibile calcolare anche l'accelerazione con cui la macchina rallenta, impiegando la seguente formula del moto uniformemente accekerato seguente:
\[v^2 - v_0^2 = 2a \cdot (x - x_0)\]
che è possibile applicare al contesto in quanto si parla di moto uniformemente accelerato, giacché l'accelerazione è causata dalla forza di attrito, che è costante in quanto prodotto tra un coefficiente e la forza normale, la quale è costante in quanto si oppone al peso che è costante.\\
Da tale formula si ottiene che
\[a = \frac{1}{2} \cdot \frac{v^2-v_0^2}{x-x_0} = -\frac{1}{2} \cdot \frac{v_0^2}{2d}\]
Da ciò segue che, ovviamente
\[F_k = \mu_k F_N\]
in cui ovviamente
\[F_N = F_t = mg\]
per cui si evince che
\[F_k = \mu_k \cdot m g \longrightarrow a_x = - \frac{F_k}{m} = -\mu_k g\]
da cui
\[\mu_k = \frac{v_0^2}{2 g d} = 0.56\]

\vspace{1em}
\noindent
\textbf{Osservazione}: Si osservi che la $2^a$ legge di Newton è applicabile in un sistema di riferimento inerziale, ovvero in un sistema di riferimento che non ha propensione a muoversi.\\
Il tempo di volo di un proiettile è
\[\boxed{t = \frac{2 v_0 \cdot \sin(\theta)}{g}}\]

\vspace{1em}
\noindent
\textbf{Esempio}: Si consideri un piano su cui poggiano tre masse, l'una collegata all'altra, come mostrato di seguito
\[raffigurazione\]
Tali masse vengono trainate con una forza $F = 200$ N, mentre le massse sono $m_A = 30$ kg, $m_B = 50$ kg e $m_C = 20$ kg; inoltre è noto che il coefficiente di attrito cinetic è $\mu_k = 0,1$. Si determini, allora
\begin{itemize}
  \item l'accelerazione dell'intero sistema;
  \item la tensione delle corde $A-B$ e $B-C$.
\end{itemize}
Si realizzi il diagramma a corpo libero del sistema oggetto di studio

\vspace{1em}
\begin{figure}[H]
  \centering
  \begin{tikzpicture}[scale=1.5]
    \draw [-stealth] (0,0) node[circ]{} -- ++(0,1) node [at end, right] {$\vec{F}_{NA}$};
    \draw [-stealth] (0,0) -- ++(0,-1) node [midway, right] {$\vec{F}_{tA}$};
    \draw [-stealth] (0,0) -- ++(-1,0) node [midway, above] {$\vec{F}_{kA}$};
    \draw [-stealth] (0,0) -- ++(1,0) node [midway, above] {$\vec{F}_{TA}$};
  \end{tikzpicture}
  \hspace{2em}
  \begin{tikzpicture}[scale=1.5]
    \draw [-stealth] (0,0) node[circ]{} -- ++(0,1) node [at end, right] {$\vec{F}_{NB}$};
    \draw [-stealth] (0,0) -- ++(0,-1) node [midway, right] {$\vec{F}_{tB}$};
    \draw [-stealth] (0,0.1) -- ++(-1,0) node [midway, above] {$\vec{F}_{kB}$};
    \draw [-stealth] (0,-0.1) -- ++(-1,0) node [midway, below] {$\vec{F}_{AB}$};
    \draw [-stealth] (0,0) -- ++(1,0) node [midway, above] {$\vec{F}_{TB}$};
  \end{tikzpicture}
  \hspace{2em}
  \begin{tikzpicture}[scale=1.5]
    \draw [-stealth] (0,0) node[circ]{} -- ++(0,1) node [at end, right] {$\vec{F}_{NC}$};
    \draw [-stealth] (0,0) -- ++(0,-1) node [midway, right] {$\vec{F}_{tC}$};
    \draw [-stealth] (0,0.1) -- ++(-1,0) node [midway, above] {$\vec{F}_{kC}$};
    \draw [-stealth] (0,-0.1) -- ++(-1,0) node [midway, below] {$\vec{F}_{BC}$};
    \draw [-stealth] (0,0) -- ++(1,0) node [midway, above] {$\vec{F}$};
  \end{tikzpicture}
  \caption{Diagramma a corpo libero di $3$ masse trainate}
  \label{fig:diagramma_corpo_libero_3_masse_trainate}
\end{figure}

\vspace{1em}
\noindent
Per la risoluzione del primo quesito, si può sfruttare la \textbf{proprietà additiva} della massa, essendo i tre corpi omogenei e costituiti dalla stessa sostanza. Pertanto l'assieme $ABC$ si comporta come un unico corpo, del quale la massa complessiva è
\[m = m_A + m_B + m_C = 30 \text{ kg} + 50 \text{ kg} + 20 \text{ kg} = 100 \text{ kg}\]
Pertanto si può realizzare un nuovo diagramam a corpo libero, mostrato di seguito:

\vspace{1em}
\begin{figure}[H]
  \centering
  \begin{tikzpicture}[scale=1.5]
    \draw [-stealth] (0,0) node[circ]{} -- ++(0,1) node [at end, right] {$\vec{F}_{N}$};
    \draw [-stealth] (0,0) -- ++(0,-1) node [midway, right] {$\vec{F}_{t}$};
    \draw [-stealth] (0,0) -- ++(-1,0) node [midway, above] {$\vec{F}_{k}$};
    \draw [-stealth] (0,0) -- ++(1,0) node [midway, above] {$\vec{F}$};
  \end{tikzpicture}
  \caption{Diagramma a corpo libero di un'unica massa}
  \label{fig:diagramma_corpo_libero_unica_massa}
\end{figure}

\vspace{1em}
\noindent
Pertanto si ottiene che, per la $2^a$ legge della dinamica:
\[\sum \vec{F} = m \vec{a}\]
Naturalmente le forze possono essere scomposte nei loro rispettivi componenti, per cui si ottiene
\[
  y : \left\{
  \begin{array}{l}
    \vec{F}_N = - \vec{F}_t\\
    F_N = mg
  \end{array}
  \right.
\]
mentre si ottiene che
\[
  x : \left\{
  \begin{array}{l}
    m a_x = F - F_k\\
    F_k = \mu_k \cdot F_N
  \end{array}
  \right.
\]
da cui
\[a_x = \frac{F}{m} - \mu_k \cdot g = 2.0 \text{ m/s}^2 - 0.98 \text{ m/s}^2 = 1.0 \text{ m/s}^2\]
Per la risoluzione del secondo quesito, si determini dapprima la tensione della corda $A-B$, ovvero $\vec{F}_{BA}$. Dalla $2^a$ legge della dinamica si ottiene
\[\vec{F}_{NA} + \vec{F}_{tA} + \vec{F}_{BA} + \vec{F}_{kA} = m_A \cdot \vec{a}_A\]
Naturalmente si ha che, scomponedo tale equazione nelle sue componenti $x$ e $y$ si ottiene
\[
  \left\{
  \begin{array}{l}
    F_{BA x} - \mu_k m_A g = m_A a_{A x}\\
    F_N + F_t = m_A a_{A y} = 0 \longrightarrow F_N = - F_t
  \end{array}
  \right.
\]
Si può procedere, ora, al calcolo di $F_{BA x}$ come segue
\[F_{BA x} = m_A \cdot (a_{Ax} + \mu_k g) = m_A \cdot \left[\right]\]


\newpage
\section{Gravità}

\newpage
\section{Energia}

\newpage
\section{Moto dei sistemi}

\newpage
\section{Corpi rigidi}

\newpage
\section{Oscillazioni}

\newpage
\section{Solidi e fluidi}

\newpage
\section{Temperatura e calore}

\newpage
\section{Il primo principio della termodinamica}

\newpage
\section{Il secondo principio della termodinamica}


\end{document}
